%% Generated by Sphinx.
\def\sphinxdocclass{jupyterBook}
\documentclass[letterpaper,10pt,english]{jupyterBook}
\ifdefined\pdfpxdimen
   \let\sphinxpxdimen\pdfpxdimen\else\newdimen\sphinxpxdimen
\fi \sphinxpxdimen=.75bp\relax
\ifdefined\pdfimageresolution
    \pdfimageresolution= \numexpr \dimexpr1in\relax/\sphinxpxdimen\relax
\fi
%% let collapsible pdf bookmarks panel have high depth per default
\PassOptionsToPackage{bookmarksdepth=5}{hyperref}
%% turn off hyperref patch of \index as sphinx.xdy xindy module takes care of
%% suitable \hyperpage mark-up, working around hyperref-xindy incompatibility
\PassOptionsToPackage{hyperindex=false}{hyperref}
%% memoir class requires extra handling
\makeatletter\@ifclassloaded{memoir}
{\ifdefined\memhyperindexfalse\memhyperindexfalse\fi}{}\makeatother

\PassOptionsToPackage{warn}{textcomp}

\catcode`^^^^00a0\active\protected\def^^^^00a0{\leavevmode\nobreak\ }
\usepackage{cmap}
\usepackage{fontspec}
\defaultfontfeatures[\rmfamily,\sffamily,\ttfamily]{}
\usepackage{amsmath,amssymb,amstext}
\usepackage{polyglossia}
\setmainlanguage{english}



\setmainfont{FreeSerif}[
  Extension      = .otf,
  UprightFont    = *,
  ItalicFont     = *Italic,
  BoldFont       = *Bold,
  BoldItalicFont = *BoldItalic
]
\setsansfont{FreeSans}[
  Extension      = .otf,
  UprightFont    = *,
  ItalicFont     = *Oblique,
  BoldFont       = *Bold,
  BoldItalicFont = *BoldOblique,
]
\setmonofont{FreeMono}[
  Extension      = .otf,
  UprightFont    = *,
  ItalicFont     = *Oblique,
  BoldFont       = *Bold,
  BoldItalicFont = *BoldOblique,
]



\usepackage[Bjarne]{fncychap}
\usepackage[,numfigreset=1,mathnumfig]{sphinx}

\fvset{fontsize=\small}
\usepackage{geometry}


% Include hyperref last.
\usepackage{hyperref}
% Fix anchor placement for figures with captions.
\usepackage{hypcap}% it must be loaded after hyperref.
% Set up styles of URL: it should be placed after hyperref.
\urlstyle{same}

\addto\captionsenglish{\renewcommand{\contentsname}{Meccanica di Newton}}

\usepackage{sphinxmessages}



        % Start of preamble defined in sphinx-jupyterbook-latex %
         \usepackage[Latin,Greek]{ucharclasses}
        \usepackage{unicode-math}
        % fixing title of the toc
        \addto\captionsenglish{\renewcommand{\contentsname}{Contents}}
        \hypersetup{
            pdfencoding=auto,
            psdextra
        }
        % End of preamble defined in sphinx-jupyterbook-latex %
        

\title{Meccanica classica}
\date{Nov 09, 2024}
\release{}
\author{basics}
\newcommand{\sphinxlogo}{\vbox{}}
\renewcommand{\releasename}{}
\makeindex
\begin{document}

\pagestyle{empty}
\sphinxmaketitle
\pagestyle{plain}
\sphinxtableofcontents
\pagestyle{normal}
\phantomsection\label{\detokenize{intro::doc}}


\sphinxAtStartPar
\sphinxstylestrong{Introduzione.}
\begin{itemize}
\item {} 
\sphinxAtStartPar
la meccanica si occupa della descrizione del moto dei sistemi e delle sue cause.

\item {} 
\sphinxAtStartPar
lo sviluppo della teoria può essere riassunto in due parti: la prima formulazione di Newton nel XVII secolo, accompagnata allo sviluppo del calcolo infinitesimale; le riformulazioni “geometriche” di Lagrange e Hamilton, con lo sviluppo del calcolo delle variazioni;

\item {} 
\sphinxAtStartPar
i \sphinxstyleemphasis{Principi} di Newton (\sphinxstyleemphasis{Philosophiae Naturalis Mathematica Principia}) concentono i concetti e i modelli necessari per formulare i principi della meccanica e sviluppare la teoria: partendo dalle definizioni di spazio, tempo (per lui assoluti), massa (per Lavoisier, costante), quantità di moto e forza, Newton formula i principi della meccanica classica; gli strumenti matematici del \sphinxstylestrong{calcolo infinitesimale} da lui congeniati, permettono a Newton di sviluppare la teoria;

\item {} 
\sphinxAtStartPar
in questo lavoro, Newton formula la \sphinxstylestrong{legge di gravitazione universale}, riconoscendo la gravita come causa unica del moto dei corpi celesti e della caduta dei gravi sulla Terra. Questa legge, insieme alle equazioni della meccanica sviluppate gli permettono di dimostrare le leggi di Keplero

\item {} 
\sphinxAtStartPar
nel secolo successivo a Parigi (\sphinxstylestrong{todo} riferimenti storici?), l’opera di D’Alembert e, soprattutto, Lagrange permette una riformulazione “geometrica” (\sphinxstylestrong{todo} spiegare) delle leggi della meccanica, usando gli strumenti del \sphinxstylestrong{calcolo della variazioni}; qualche anno dopo segue la riformulazione di Hamitlon; nonostante queste riformulazioni forniscano una teoria equivalente alla teoria di Newton, e non permettano di risolvere nuovi problemi in meccanica classica, esse forniscono un approccio moderno e comodo, che semplifica la scrittura delle equazioni che governano il moto dei sistemi, e che viene usato come base nella fisica moderna per l’approccio alle nuove discipline, come la meccanica quantistica.

\end{itemize}

\sphinxAtStartPar
\sphinxstylestrong{Argomenti}
\begin{itemize}
\item {} 
\sphinxAtStartPar
Meccanica di Newton:
\begin{itemize}
\item {} 
\sphinxAtStartPar
cinematica

\item {} 
\sphinxAtStartPar
azioni

\item {} 
\sphinxAtStartPar
inerzia

\item {} 
\sphinxAtStartPar
dinamica

\end{itemize}

\item {} 
\sphinxAtStartPar
Meccanica analitica
\begin{itemize}
\item {} 
\sphinxAtStartPar
formualzione di Lagrange

\item {} 
\sphinxAtStartPar
formulazione di Hamilton

\end{itemize}

\end{itemize}

\sphinxAtStartPar
\sphinxstylestrong{Pre\sphinxhyphen{}requisiti}: … 
\sphinxstylestrong{Livello}: università 
\sphinxstylestrong{Altro…}:



\sphinxstepscope


\part{Meccanica di Newton}

\sphinxstepscope

\begin{sphinxuseclass}{sd-container-fluid}
\begin{sphinxuseclass}{sd-sphinx-override}
\begin{sphinxuseclass}{sd-p-0}
\begin{sphinxuseclass}{sd-mt-2}
\begin{sphinxuseclass}{sd-mb-4}
\begin{sphinxuseclass}{sd-row}
\begin{sphinxuseclass}{sd-row-cols-2}
\begin{sphinxuseclass}{sd-gx-2}
\begin{sphinxuseclass}{sd-gy-1}
\begin{sphinxuseclass}{sd-col}
\begin{sphinxuseclass}{sd-d-flex-row}
\begin{sphinxuseclass}{sd-align-minor-center}
\begin{sphinxuseclass}{sd-container-fluid}
\begin{sphinxuseclass}{sd-sphinx-override}
\begin{sphinxuseclass}{sd-row}
\begin{sphinxuseclass}{sd-row-cols-2}
\begin{sphinxuseclass}{sd-row-cols-xs-2}
\begin{sphinxuseclass}{sd-row-cols-sm-3}
\begin{sphinxuseclass}{sd-row-cols-md-3}
\begin{sphinxuseclass}{sd-row-cols-lg-3}
\begin{sphinxuseclass}{sd-gx-3}
\begin{sphinxuseclass}{sd-gy-1}
\begin{sphinxuseclass}{sd-col}
\begin{sphinxuseclass}{sd-col-auto}
\begin{sphinxuseclass}{sd-d-flex-row}
\begin{sphinxuseclass}{sd-align-minor-center}
\sphinxAtStartPar
basics

\end{sphinxuseclass}
\end{sphinxuseclass}
\end{sphinxuseclass}
\end{sphinxuseclass}
\begin{sphinxuseclass}{sd-col}
\begin{sphinxuseclass}{sd-col-auto}
\begin{sphinxuseclass}{sd-d-flex-row}
\begin{sphinxuseclass}{sd-align-minor-center}
\sphinxAtStartPar
Nov 09, 2024

\end{sphinxuseclass}
\end{sphinxuseclass}
\end{sphinxuseclass}
\end{sphinxuseclass}
\begin{sphinxuseclass}{sd-col}
\begin{sphinxuseclass}{sd-col-auto}
\begin{sphinxuseclass}{sd-d-flex-row}
\begin{sphinxuseclass}{sd-align-minor-center}
\sphinxAtStartPar
0 min read

\end{sphinxuseclass}
\end{sphinxuseclass}
\end{sphinxuseclass}
\end{sphinxuseclass}
\end{sphinxuseclass}
\end{sphinxuseclass}
\end{sphinxuseclass}
\end{sphinxuseclass}
\end{sphinxuseclass}
\end{sphinxuseclass}
\end{sphinxuseclass}
\end{sphinxuseclass}
\end{sphinxuseclass}
\end{sphinxuseclass}
\end{sphinxuseclass}
\end{sphinxuseclass}
\end{sphinxuseclass}
\end{sphinxuseclass}
\end{sphinxuseclass}
\end{sphinxuseclass}
\end{sphinxuseclass}
\end{sphinxuseclass}
\end{sphinxuseclass}
\end{sphinxuseclass}
\end{sphinxuseclass}
\end{sphinxuseclass}

\chapter{Cinematica}
\label{\detokenize{ch/kinematics:cinematica}}\label{\detokenize{ch/kinematics:classical-mechanics-kinematics}}\label{\detokenize{ch/kinematics::doc}}


\sphinxstepscope


\section{Cinematica del punto}
\label{\detokenize{ch/kinematics-point:cinematica-del-punto}}\label{\detokenize{ch/kinematics-point:classical-mechanics-kinematics-point}}\label{\detokenize{ch/kinematics-point::doc}}
\sphinxAtStartPar
La cinematica di un punto generalmente si riduce alla determinazione della posizione, della velocità e dell’accelerazione del punto.


\subsection{Posizione}
\label{\detokenize{ch/kinematics-point:posizione}}
\sphinxAtStartPar
La posizione di un punto nello spazio euclideo \(E^3\) rispetto a un sistema di riferimento \(I\) è identificata dal raggio vettore tra l’origine \(O_I\) del sistema di riferimento e il punto stesso,
\begin{equation*}
\begin{split}\mathbf{r}_P = (P - O) \ .\end{split}
\end{equation*}
\sphinxAtStartPar
La posizione è identificata da una quantità vettoriale. Per identificare il punto si possono scegliere diversi sistemi di coordinate, ma la posizione del punto non è influenzata dalla scelta delle coordinate.

\sphinxAtStartPar
Ad esempio, usando un sistema di coordinate Cartesiane associate alla base \(\{\hat{\mathbf{x}}_I, \hat{\mathbf{y}}_I, \hat{\mathbf{z}}_I \}\), indipendente dal tempo, la posizione del punto P può essere scritta come combinazione lineare dei vettori della base,
\begin{equation*}
\begin{split}\mathbf{r}_P(t) = x_P(t) \hat{\mathbf{x}}_I + y_P(t) \hat{\mathbf{y}}_I + z_P(t) \hat{\mathbf{z}}_I \ .\end{split}
\end{equation*}

\subsection{Velocità}
\label{\detokenize{ch/kinematics-point:velocita}}
\sphinxAtStartPar
La velocità del punto \(P\) rispetto al sistema di riferimento \(I\) è la derivata rispetto al tempo della posizione del punto,
\begin{equation*}
\begin{split}\mathbf{v}_P(t) = \dot{\mathbf{r}}_P(t) \ .\end{split}
\end{equation*}
\sphinxAtStartPar
Usando il sistema di riferimento cartesiano, e che i vettori della base sono costanti, la velocità può essere scritta come
\begin{equation*}
\begin{split}\mathbf{v}_P(t) = \dot{\mathbf{r}}_P(t) = \dot{x}_P(t) \hat{\mathbf{x}}_I + \dot{y}_P(t) \hat{\mathbf{y}}_I + \dot{z}_P(t) \hat{\mathbf{z}}_I \ .\end{split}
\end{equation*}

\subsection{Accelerazione}
\label{\detokenize{ch/kinematics-point:accelerazione}}
\sphinxAtStartPar
L’accelerazione del punto \(P\) rispetto al sistema di riferimento \(I\) è la derivata rispetto al tempo della velocità del punto, la derivata seconda della posizione
\begin{equation*}
\begin{split}\mathbf{a}_P(t) = \dot{\mathbf{v}}_P(t) = \ddot{\mathbf{r}}_P(t) \ .\end{split}
\end{equation*}
\sphinxAtStartPar
Usando il sistema di riferimento cartesiano, e che i vettori della base sono costanti, l’accelerazione può essere scritta come
\begin{equation*}
\begin{split}\begin{aligned}
\mathbf{a}_P(t) & = \ddot{\mathbf{r}}_P(t) = \ddot{x}_P(t) \hat{\mathbf{x}}_I + \ddot{y}_P(t) \hat{\mathbf{y}}_I + \ddot{z}_P(t) \hat{\mathbf{z}}_I \\
                & =  \dot{\mathbf{v}}_P(t) =  \dot{v}_{x,P}(t) \hat{\mathbf{x}}_I + \dot{v}_{y,P}(t) \hat{\mathbf{y}}_I + \dot{v}_{z,P}(t) \hat{\mathbf{z}}_I \ .
\end{aligned}\end{split}
\end{equation*}

\subsection{Parametrizzazione della traiettoria, coordinata naturale e terna di Frenet}
\label{\detokenize{ch/kinematics-point:parametrizzazione-della-traiettoria-coordinata-naturale-e-terna-di-frenet}}
\sphinxAtStartPar
Usando i concetti della geometria delle curve, è possibile definire la coordinata naturale \(s\) lungo la curva, tale che il vettore unitario tangente alla curva coincide con la derivata della posizione rispetto alla coordinata \(s\),
\begin{equation*}
\begin{split}\hat{\mathbf{t}}(s) = \dfrac{d \mathbf{r}}{d s} \ .\end{split}
\end{equation*}
\sphinxAtStartPar
La derivata del versore tangente è ortogonale ad esso, punta verso il centro del cerchio osculatore, il cui raggio \(R\) è definito come il raggio di curvatura della curva nel punto; la curvatura è definita come l’inverso del raggio di curvatura, \(\kappa = \frac{1}{R}\); il valore assoluto della derivata del versore tangente rispetto alla coordinata \(s\) è uguale alla curvatura,
\begin{equation*}
\begin{split}\kappa(s) \hat{\mathbf{n}} := \frac{d \hat{\mathbf{t}}}{ d s } =  \frac{d^2 \mathbf{r}}{ d s^2 } \ .\end{split}
\end{equation*}
\sphinxAtStartPar
Il versore binormale che forma la terna di Frenet è definito come il prodotto vettore tra \(\hat{\mathbf{t}}\) e \(\hat{\mathbf{n}}\),
\begin{equation*}
\begin{split}\hat{\mathbf{b}}(s) := \hat{\mathbf{t}}(s) \times \hat{\mathbf{n}}(s) \ .\end{split}
\end{equation*}
\sphinxAtStartPar
Usando le regole per la derivazione di funzioni composte, si può scrivere la velocità come
\begin{equation*}
\begin{split}\mathbf{v}_P(t) = \dfrac{d}{dt} \mathbf{r}_P = \dfrac{d s}{d t} \dfrac{d}{ds} \mathbf{r}_P = v_P \hat{\mathbf{t}} \ ,\end{split}
\end{equation*}
\sphinxAtStartPar
essendo \(v_P\) il modulo della velocità, sempre tangente alla traiettoria.

\sphinxAtStartPar
Derivando una seconda volta in tempo, si ottiene l’espressione dell’accelerazione,
\begin{equation*}
\begin{split}\begin{aligned}
  \mathbf{a}_P(t) & = \dfrac{d}{dt} \mathbf{v}_P(t) = \\
                  & = \dfrac{d}{dt} \left( v_P \hat{\mathbf{t}} \right) = \\
                  & = \dfrac{d}{dt} v_P \hat{\mathbf{t}} + v_P \dfrac{ds}{dt} \dfrac{d}{ds} \hat{\mathbf{t}}= \\
                  & = a_P \hat{\mathbf{t}} + \kappa \, v^2_P \hat{\mathbf{n}}  \ ,
\end{aligned}\end{split}
\end{equation*}
\sphinxAtStartPar
che può essere scritta come la somma de:
\begin{itemize}
\item {} 
\sphinxAtStartPar
l’accelerazione tangenziale lungo la curva, che è causa della variazione del modulo della velocità

\item {} 
\sphinxAtStartPar
l’accelerazione in direzione normale ad essa, l’accelerazione centripeta, che fa cambiare direzione al direttore velocità.

\end{itemize}

\sphinxAtStartPar
\sphinxstylestrong{todo.} Mostrare queste ultime due affermazioni, calcolando la derivata di \(|\mathbf{v}|\)…

\sphinxstepscope

\begin{sphinxuseclass}{sd-container-fluid}
\begin{sphinxuseclass}{sd-sphinx-override}
\begin{sphinxuseclass}{sd-p-0}
\begin{sphinxuseclass}{sd-mt-2}
\begin{sphinxuseclass}{sd-mb-4}
\begin{sphinxuseclass}{sd-row}
\begin{sphinxuseclass}{sd-row-cols-2}
\begin{sphinxuseclass}{sd-gx-2}
\begin{sphinxuseclass}{sd-gy-1}
\begin{sphinxuseclass}{sd-col}
\begin{sphinxuseclass}{sd-d-flex-row}
\begin{sphinxuseclass}{sd-align-minor-center}
\begin{sphinxuseclass}{sd-container-fluid}
\begin{sphinxuseclass}{sd-sphinx-override}
\begin{sphinxuseclass}{sd-row}
\begin{sphinxuseclass}{sd-row-cols-2}
\begin{sphinxuseclass}{sd-row-cols-xs-2}
\begin{sphinxuseclass}{sd-row-cols-sm-3}
\begin{sphinxuseclass}{sd-row-cols-md-3}
\begin{sphinxuseclass}{sd-row-cols-lg-3}
\begin{sphinxuseclass}{sd-gx-3}
\begin{sphinxuseclass}{sd-gy-1}
\begin{sphinxuseclass}{sd-col}
\begin{sphinxuseclass}{sd-col-auto}
\begin{sphinxuseclass}{sd-d-flex-row}
\begin{sphinxuseclass}{sd-align-minor-center}
\sphinxAtStartPar
basics

\end{sphinxuseclass}
\end{sphinxuseclass}
\end{sphinxuseclass}
\end{sphinxuseclass}
\begin{sphinxuseclass}{sd-col}
\begin{sphinxuseclass}{sd-col-auto}
\begin{sphinxuseclass}{sd-d-flex-row}
\begin{sphinxuseclass}{sd-align-minor-center}
\sphinxAtStartPar
Nov 09, 2024

\end{sphinxuseclass}
\end{sphinxuseclass}
\end{sphinxuseclass}
\end{sphinxuseclass}
\begin{sphinxuseclass}{sd-col}
\begin{sphinxuseclass}{sd-col-auto}
\begin{sphinxuseclass}{sd-d-flex-row}
\begin{sphinxuseclass}{sd-align-minor-center}
\sphinxAtStartPar
1 min read

\end{sphinxuseclass}
\end{sphinxuseclass}
\end{sphinxuseclass}
\end{sphinxuseclass}
\end{sphinxuseclass}
\end{sphinxuseclass}
\end{sphinxuseclass}
\end{sphinxuseclass}
\end{sphinxuseclass}
\end{sphinxuseclass}
\end{sphinxuseclass}
\end{sphinxuseclass}
\end{sphinxuseclass}
\end{sphinxuseclass}
\end{sphinxuseclass}
\end{sphinxuseclass}
\end{sphinxuseclass}
\end{sphinxuseclass}
\end{sphinxuseclass}
\end{sphinxuseclass}
\end{sphinxuseclass}
\end{sphinxuseclass}
\end{sphinxuseclass}
\end{sphinxuseclass}
\end{sphinxuseclass}
\end{sphinxuseclass}

\section{Cinematica del corpo rigido}
\label{\detokenize{ch/kinematics-rigid:cinematica-del-corpo-rigido}}\label{\detokenize{ch/kinematics-rigid:classical-mechanics-kinematics-rigid-body}}\label{\detokenize{ch/kinematics-rigid::doc}}
\sphinxAtStartPar
I punti di corpi estesi che compiono un atto di moto rigido (\sphinxstylestrong{todo} definizione di atto di moto? E’ utile?) mantengono costante la propria distanza. Poiché questa condizione vale per ogni coppia di punti, vengono mantenuti costanti anche gli angoli tra segmenti che congiungono punti materiali, cioè appartenenti al copo e che si muovono con esso.

\sphinxAtStartPar
Affinché queste condizioni siano soddisfatte, un vettore materiale si trasforma con una rotazione. Il moto di un corpo rigido può quindi essere descritto come la composizione di un moto di traslazione di un punto \(P\) e la rotazione del corpo attorno ad esso.
\begin{equation*}
\begin{split}\mathbf{r}_P - \mathbf{r}_Q = \mathbb{R} \cdot (\mathbf{r}_P - \mathbf{r}_Q)_0 \ ,\end{split}
\end{equation*}
\sphinxAtStartPar
dove \((\mathbf{r}_P - \mathbf{r}_Q)_0\) rappresenta il vettore materiale nella posizione di riferimento, ed \(\mathbb{R}\) è il tensore di rotazione che rappresenta la rotazione del corpo rigido dalla posizione di riferimento alla posizione corrente. La posizione di ogni punto materiale \(P\) può essere scritta come la posizione di un punto materiale di riferimento \(Q\) e il vettore di riferimento \((\mathbf{r}_P - \mathbf{r}_Q)_0\) ruotato.


\subsection{Tensore di rotazione}
\label{\detokenize{ch/kinematics-rigid:tensore-di-rotazione}}
\sphinxAtStartPar
\sphinxstylestrong{Proprietà}
\begin{itemize}
\item {} 
\sphinxAtStartPar
definizione con due sistemi di riferimento cartesiani in moto relativo. Dati due basi ortonormali, \(\{\hat{\mathbf{e}}^0_i\}\), \(\{\hat{\mathbf{e}}^1_j\}\), è possibile scrivere i vettori di una base come

\end{itemize}
\begin{equation*}
\begin{split}\begin{aligned}
  \hat{\mathbf{e}}^1_j & = (\hat{\mathbf{e}}^1_j \cdot \hat{\mathbf{e}}^0_i ) \hat{\mathbf{e}}^0_i = \\
                       & = (\hat{\mathbf{e}}^0_i \cdot \hat{\mathbf{e}}^1_k ) \hat{\mathbf{e}}^0_i \otimes \hat{\mathbf{e}}^0_k \cdot \hat{\mathbf{e}}^0_k = \\
                       & = R^{0\rightarrow 1}_{ik} \hat{\mathbf{e}}^0_i \otimes \hat{\mathbf{e}}^0_k \cdot \hat{\mathbf{e}}^0_j = \\
                       & = \mathbb{R}^{0 \rightarrow 1} \cdot \hat{\mathbf{e}}^0_j = \\
\end{aligned}\end{split}
\end{equation*}\begin{itemize}
\item {} 
\sphinxAtStartPar
unitarietà

\end{itemize}
\begin{equation*}
\begin{split}\mathbb{R}^T \cdot \mathbb{R}  = \mathbb{I}\end{split}
\end{equation*}\begin{equation*}
\begin{split}\mathbb{R} \cdot \mathbb{R}^T  = \mathbb{I}\end{split}
\end{equation*}\begin{itemize}
\item {} 
\sphinxAtStartPar
derivata del tensore di rotazione e definizione del vettore velocità angolare

\end{itemize}
\begin{equation*}
\begin{split}\mathbb{0} = \dot{\mathbb{R}} \cdot \mathbb{R}^T + {\mathbb{R}} \cdot \dot{\mathbb{R}}^T \qquad \rightarrow \qquad
\dot{\mathbb{R}} \cdot \mathbb{R}^T = - \mathbb{R} \cdot \dot{\mathbb{R}}^T =: \omega_{\times}\end{split}
\end{equation*}\begin{equation*}
\begin{split}\dot{\mathbb{R}} = \omega_{\times} \cdot \mathbb{R}\end{split}
\end{equation*}\begin{itemize}
\item {} 
\sphinxAtStartPar
composizione di rotazioni (qui o nella cinematica relativa? o in un’appendice apposita per le rotazioni?)

\end{itemize}


\subsection{Velocità}
\label{\detokenize{ch/kinematics-rigid:velocita}}
\sphinxAtStartPar
La derivata nel tempo della legge dell’atto di moto rigido permette di ricavare la relazione tra le velocità di due punti materiali di un corpo rigido,
\begin{equation*}
\begin{split}\begin{aligned}
  \mathbf{v}_P - \mathbf{v}_Q & = \frac{d}{dt} \left( \mathbb{R} \cdot (\mathbf{r}_P - \mathbf{r}_Q)_0 \right) = \\
                              & = \dot{\mathbb{R}} \cdot (\mathbf{r}_P - \mathbf{r}_Q)_0 = \\
                              & = \symbf{\omega}_{\times} \cdot \mathbb{R} \cdot (\mathbf{r}_P - \mathbf{r}_Q)_0 = \\
                              & = \symbf{\omega} \times ( \mathbf{r}_P - \mathbf{r}_Q ) 
\end{aligned}\end{split}
\end{equation*}

\subsection{Accelerazione}
\label{\detokenize{ch/kinematics-rigid:accelerazione}}
\sphinxAtStartPar
Derivando nuovamente nel tempo, si trova la relazione tra l’accelerazione di una coppia di punti materiali del corpo rigido,
\begin{equation*}
\begin{split}\begin{aligned}
  \mathbf{a}_P - \mathbf{a}_Q & = \frac{d}{dt} \left( \symbf{\omega} \times ( \mathbf{r}_P - \mathbf{r}_Q )  \right) = \\
                              & = \frac{d \symbf{\omega}}{dt} \times ( \mathbf{r}_P - \mathbf{r}_Q ) + \symbf{\omega} \times \dfrac{d}{dt}( \mathbf{r}_P - \mathbf{r}_Q ) = \\
                              & = \symbf{\alpha} \times ( \mathbf{r}_P - \mathbf{r}_Q ) + \symbf{\omega} \times \left ( \symbf{\omega} \times ( \mathbf{r}_P - \mathbf{r}_Q ) \right) = \\
                              & = \symbf{\alpha} \times ( \mathbf{r}_P - \mathbf{r}_Q ) + \symbf{\omega}_{\times} \cdot  \symbf{\omega}_{\times} \cdot ( \mathbf{r}_P - \mathbf{r}_Q ) \ . 
\end{aligned}\end{split}
\end{equation*}
\sphinxstepscope

\begin{sphinxuseclass}{sd-container-fluid}
\begin{sphinxuseclass}{sd-sphinx-override}
\begin{sphinxuseclass}{sd-p-0}
\begin{sphinxuseclass}{sd-mt-2}
\begin{sphinxuseclass}{sd-mb-4}
\begin{sphinxuseclass}{sd-row}
\begin{sphinxuseclass}{sd-row-cols-2}
\begin{sphinxuseclass}{sd-gx-2}
\begin{sphinxuseclass}{sd-gy-1}
\begin{sphinxuseclass}{sd-col}
\begin{sphinxuseclass}{sd-d-flex-row}
\begin{sphinxuseclass}{sd-align-minor-center}
\begin{sphinxuseclass}{sd-container-fluid}
\begin{sphinxuseclass}{sd-sphinx-override}
\begin{sphinxuseclass}{sd-row}
\begin{sphinxuseclass}{sd-row-cols-2}
\begin{sphinxuseclass}{sd-row-cols-xs-2}
\begin{sphinxuseclass}{sd-row-cols-sm-3}
\begin{sphinxuseclass}{sd-row-cols-md-3}
\begin{sphinxuseclass}{sd-row-cols-lg-3}
\begin{sphinxuseclass}{sd-gx-3}
\begin{sphinxuseclass}{sd-gy-1}
\begin{sphinxuseclass}{sd-col}
\begin{sphinxuseclass}{sd-col-auto}
\begin{sphinxuseclass}{sd-d-flex-row}
\begin{sphinxuseclass}{sd-align-minor-center}
\sphinxAtStartPar
basics

\end{sphinxuseclass}
\end{sphinxuseclass}
\end{sphinxuseclass}
\end{sphinxuseclass}
\begin{sphinxuseclass}{sd-col}
\begin{sphinxuseclass}{sd-col-auto}
\begin{sphinxuseclass}{sd-d-flex-row}
\begin{sphinxuseclass}{sd-align-minor-center}
\sphinxAtStartPar
Nov 09, 2024

\end{sphinxuseclass}
\end{sphinxuseclass}
\end{sphinxuseclass}
\end{sphinxuseclass}
\begin{sphinxuseclass}{sd-col}
\begin{sphinxuseclass}{sd-col-auto}
\begin{sphinxuseclass}{sd-d-flex-row}
\begin{sphinxuseclass}{sd-align-minor-center}
\sphinxAtStartPar
0 min read

\end{sphinxuseclass}
\end{sphinxuseclass}
\end{sphinxuseclass}
\end{sphinxuseclass}
\end{sphinxuseclass}
\end{sphinxuseclass}
\end{sphinxuseclass}
\end{sphinxuseclass}
\end{sphinxuseclass}
\end{sphinxuseclass}
\end{sphinxuseclass}
\end{sphinxuseclass}
\end{sphinxuseclass}
\end{sphinxuseclass}
\end{sphinxuseclass}
\end{sphinxuseclass}
\end{sphinxuseclass}
\end{sphinxuseclass}
\end{sphinxuseclass}
\end{sphinxuseclass}
\end{sphinxuseclass}
\end{sphinxuseclass}
\end{sphinxuseclass}
\end{sphinxuseclass}
\end{sphinxuseclass}
\end{sphinxuseclass}

\section{Cinematica di mezzi continui}
\label{\detokenize{ch/kinematics-continuum:cinematica-di-mezzi-continui}}\label{\detokenize{ch/kinematics-continuum:classical-mechanics-kinematics-contiuum}}\label{\detokenize{ch/kinematics-continuum::doc}}
\sphinxstepscope

\begin{sphinxuseclass}{sd-container-fluid}
\begin{sphinxuseclass}{sd-sphinx-override}
\begin{sphinxuseclass}{sd-p-0}
\begin{sphinxuseclass}{sd-mt-2}
\begin{sphinxuseclass}{sd-mb-4}
\begin{sphinxuseclass}{sd-row}
\begin{sphinxuseclass}{sd-row-cols-2}
\begin{sphinxuseclass}{sd-gx-2}
\begin{sphinxuseclass}{sd-gy-1}
\begin{sphinxuseclass}{sd-col}
\begin{sphinxuseclass}{sd-d-flex-row}
\begin{sphinxuseclass}{sd-align-minor-center}
\begin{sphinxuseclass}{sd-container-fluid}
\begin{sphinxuseclass}{sd-sphinx-override}
\begin{sphinxuseclass}{sd-row}
\begin{sphinxuseclass}{sd-row-cols-2}
\begin{sphinxuseclass}{sd-row-cols-xs-2}
\begin{sphinxuseclass}{sd-row-cols-sm-3}
\begin{sphinxuseclass}{sd-row-cols-md-3}
\begin{sphinxuseclass}{sd-row-cols-lg-3}
\begin{sphinxuseclass}{sd-gx-3}
\begin{sphinxuseclass}{sd-gy-1}
\begin{sphinxuseclass}{sd-col}
\begin{sphinxuseclass}{sd-col-auto}
\begin{sphinxuseclass}{sd-d-flex-row}
\begin{sphinxuseclass}{sd-align-minor-center}
\sphinxAtStartPar
basics

\end{sphinxuseclass}
\end{sphinxuseclass}
\end{sphinxuseclass}
\end{sphinxuseclass}
\begin{sphinxuseclass}{sd-col}
\begin{sphinxuseclass}{sd-col-auto}
\begin{sphinxuseclass}{sd-d-flex-row}
\begin{sphinxuseclass}{sd-align-minor-center}
\sphinxAtStartPar
Nov 09, 2024

\end{sphinxuseclass}
\end{sphinxuseclass}
\end{sphinxuseclass}
\end{sphinxuseclass}
\begin{sphinxuseclass}{sd-col}
\begin{sphinxuseclass}{sd-col-auto}
\begin{sphinxuseclass}{sd-d-flex-row}
\begin{sphinxuseclass}{sd-align-minor-center}
\sphinxAtStartPar
0 min read

\end{sphinxuseclass}
\end{sphinxuseclass}
\end{sphinxuseclass}
\end{sphinxuseclass}
\end{sphinxuseclass}
\end{sphinxuseclass}
\end{sphinxuseclass}
\end{sphinxuseclass}
\end{sphinxuseclass}
\end{sphinxuseclass}
\end{sphinxuseclass}
\end{sphinxuseclass}
\end{sphinxuseclass}
\end{sphinxuseclass}
\end{sphinxuseclass}
\end{sphinxuseclass}
\end{sphinxuseclass}
\end{sphinxuseclass}
\end{sphinxuseclass}
\end{sphinxuseclass}
\end{sphinxuseclass}
\end{sphinxuseclass}
\end{sphinxuseclass}
\end{sphinxuseclass}
\end{sphinxuseclass}
\end{sphinxuseclass}

\section{Cinematica relativa}
\label{\detokenize{ch/kinematics-relative:cinematica-relativa}}\label{\detokenize{ch/kinematics-relative:classical-mechanics-kinematics-relative}}\label{\detokenize{ch/kinematics-relative::doc}}
\sphinxstepscope

\begin{sphinxuseclass}{sd-container-fluid}
\begin{sphinxuseclass}{sd-sphinx-override}
\begin{sphinxuseclass}{sd-p-0}
\begin{sphinxuseclass}{sd-mt-2}
\begin{sphinxuseclass}{sd-mb-4}
\begin{sphinxuseclass}{sd-row}
\begin{sphinxuseclass}{sd-row-cols-2}
\begin{sphinxuseclass}{sd-gx-2}
\begin{sphinxuseclass}{sd-gy-1}
\begin{sphinxuseclass}{sd-col}
\begin{sphinxuseclass}{sd-d-flex-row}
\begin{sphinxuseclass}{sd-align-minor-center}
\begin{sphinxuseclass}{sd-container-fluid}
\begin{sphinxuseclass}{sd-sphinx-override}
\begin{sphinxuseclass}{sd-row}
\begin{sphinxuseclass}{sd-row-cols-2}
\begin{sphinxuseclass}{sd-row-cols-xs-2}
\begin{sphinxuseclass}{sd-row-cols-sm-3}
\begin{sphinxuseclass}{sd-row-cols-md-3}
\begin{sphinxuseclass}{sd-row-cols-lg-3}
\begin{sphinxuseclass}{sd-gx-3}
\begin{sphinxuseclass}{sd-gy-1}
\begin{sphinxuseclass}{sd-col}
\begin{sphinxuseclass}{sd-col-auto}
\begin{sphinxuseclass}{sd-d-flex-row}
\begin{sphinxuseclass}{sd-align-minor-center}
\sphinxAtStartPar
basics

\end{sphinxuseclass}
\end{sphinxuseclass}
\end{sphinxuseclass}
\end{sphinxuseclass}
\begin{sphinxuseclass}{sd-col}
\begin{sphinxuseclass}{sd-col-auto}
\begin{sphinxuseclass}{sd-d-flex-row}
\begin{sphinxuseclass}{sd-align-minor-center}
\sphinxAtStartPar
Nov 09, 2024

\end{sphinxuseclass}
\end{sphinxuseclass}
\end{sphinxuseclass}
\end{sphinxuseclass}
\begin{sphinxuseclass}{sd-col}
\begin{sphinxuseclass}{sd-col-auto}
\begin{sphinxuseclass}{sd-d-flex-row}
\begin{sphinxuseclass}{sd-align-minor-center}
\sphinxAtStartPar
0 min read

\end{sphinxuseclass}
\end{sphinxuseclass}
\end{sphinxuseclass}
\end{sphinxuseclass}
\end{sphinxuseclass}
\end{sphinxuseclass}
\end{sphinxuseclass}
\end{sphinxuseclass}
\end{sphinxuseclass}
\end{sphinxuseclass}
\end{sphinxuseclass}
\end{sphinxuseclass}
\end{sphinxuseclass}
\end{sphinxuseclass}
\end{sphinxuseclass}
\end{sphinxuseclass}
\end{sphinxuseclass}
\end{sphinxuseclass}
\end{sphinxuseclass}
\end{sphinxuseclass}
\end{sphinxuseclass}
\end{sphinxuseclass}
\end{sphinxuseclass}
\end{sphinxuseclass}
\end{sphinxuseclass}
\end{sphinxuseclass}

\chapter{Azioni}
\label{\detokenize{ch/actions:azioni}}\label{\detokenize{ch/actions:classical-mechanics-actions}}\label{\detokenize{ch/actions::doc}}
\sphinxAtStartPar
\sphinxstylestrong{Cos’è una forza?}
Newton lega il concetto di azione, che comprende i concetti di forza e momento, alle cause che distinguono il \sphinxstyleemphasis{moto vero} dal \sphinxstyleemphasis{moto relativo}: per Newton, queste cause sono le forze.

\sphinxAtStartPar
\sphinxstylestrong{Argomenti.}
\begin{itemize}
\item {} 
\sphinxAtStartPar
Forza

\item {} 
\sphinxAtStartPar
Momento di una forza e coppia di forze

\item {} 
\sphinxAtStartPar
Azioni distribuite: cenni su sforzo e pressione

\item {} 
\sphinxAtStartPar
Lavoro e potenza

\item {} 
\sphinxAtStartPar
Azioni conservative

\item {} 
\sphinxAtStartPar
\sphinxstylestrong{Reazioni vincolari}

\item {} 
\sphinxAtStartPar
Esempi di forze:
\begin{itemize}
\item {} 
\sphinxAtStartPar
legge di attrazione di gravità generale

\item {} 
\sphinxAtStartPar
forza di gravità nei pressi della superficie terrestre

\item {} 
\sphinxAtStartPar
legge costitutiva elastica della molla e molle senza massa

\item {} 
\sphinxAtStartPar
attrito

\item {} 
\sphinxAtStartPar
legge costitutiva di smorzatori lineari

\item {} 
\sphinxAtStartPar
reazioni vincolari

\end{itemize}

\end{itemize}

\sphinxstepscope

\begin{sphinxuseclass}{sd-container-fluid}
\begin{sphinxuseclass}{sd-sphinx-override}
\begin{sphinxuseclass}{sd-p-0}
\begin{sphinxuseclass}{sd-mt-2}
\begin{sphinxuseclass}{sd-mb-4}
\begin{sphinxuseclass}{sd-row}
\begin{sphinxuseclass}{sd-row-cols-2}
\begin{sphinxuseclass}{sd-gx-2}
\begin{sphinxuseclass}{sd-gy-1}
\begin{sphinxuseclass}{sd-col}
\begin{sphinxuseclass}{sd-d-flex-row}
\begin{sphinxuseclass}{sd-align-minor-center}
\begin{sphinxuseclass}{sd-container-fluid}
\begin{sphinxuseclass}{sd-sphinx-override}
\begin{sphinxuseclass}{sd-row}
\begin{sphinxuseclass}{sd-row-cols-2}
\begin{sphinxuseclass}{sd-row-cols-xs-2}
\begin{sphinxuseclass}{sd-row-cols-sm-3}
\begin{sphinxuseclass}{sd-row-cols-md-3}
\begin{sphinxuseclass}{sd-row-cols-lg-3}
\begin{sphinxuseclass}{sd-gx-3}
\begin{sphinxuseclass}{sd-gy-1}
\begin{sphinxuseclass}{sd-col}
\begin{sphinxuseclass}{sd-col-auto}
\begin{sphinxuseclass}{sd-d-flex-row}
\begin{sphinxuseclass}{sd-align-minor-center}
\sphinxAtStartPar
basics

\end{sphinxuseclass}
\end{sphinxuseclass}
\end{sphinxuseclass}
\end{sphinxuseclass}
\begin{sphinxuseclass}{sd-col}
\begin{sphinxuseclass}{sd-col-auto}
\begin{sphinxuseclass}{sd-d-flex-row}
\begin{sphinxuseclass}{sd-align-minor-center}
\sphinxAtStartPar
Nov 09, 2024

\end{sphinxuseclass}
\end{sphinxuseclass}
\end{sphinxuseclass}
\end{sphinxuseclass}
\begin{sphinxuseclass}{sd-col}
\begin{sphinxuseclass}{sd-col-auto}
\begin{sphinxuseclass}{sd-d-flex-row}
\begin{sphinxuseclass}{sd-align-minor-center}
\sphinxAtStartPar
0 min read

\end{sphinxuseclass}
\end{sphinxuseclass}
\end{sphinxuseclass}
\end{sphinxuseclass}
\end{sphinxuseclass}
\end{sphinxuseclass}
\end{sphinxuseclass}
\end{sphinxuseclass}
\end{sphinxuseclass}
\end{sphinxuseclass}
\end{sphinxuseclass}
\end{sphinxuseclass}
\end{sphinxuseclass}
\end{sphinxuseclass}
\end{sphinxuseclass}
\end{sphinxuseclass}
\end{sphinxuseclass}
\end{sphinxuseclass}
\end{sphinxuseclass}
\end{sphinxuseclass}
\end{sphinxuseclass}
\end{sphinxuseclass}
\end{sphinxuseclass}
\end{sphinxuseclass}
\end{sphinxuseclass}
\end{sphinxuseclass}

\section{Forze, momenti, forze distribuite}
\label{\detokenize{ch/actions-types:forze-momenti-forze-distribuite}}\label{\detokenize{ch/actions-types:classical-mechanics-actions-density}}\label{\detokenize{ch/actions-types::doc}}
\sphinxAtStartPar
\sphinxstylestrong{Forza.} Grandezza vettoriale, caratterizzata da intensità, direzione, verso, e punto di applicazione.

\sphinxAtStartPar
\sphinxstylestrong{Momento.} Il momento di una forza rispetto a un polo \(H\)
\begin{equation*}
\begin{split}\mathbf{M}_H = (\mathbf{r}_P - \mathbf{r}_H) \times \mathbf{F} \ .\end{split}
\end{equation*}
\sphinxAtStartPar
\sphinxstylestrong{Forze distribuite.} \sphinxstylestrong{todo}

\sphinxstepscope

\begin{sphinxuseclass}{sd-container-fluid}
\begin{sphinxuseclass}{sd-sphinx-override}
\begin{sphinxuseclass}{sd-p-0}
\begin{sphinxuseclass}{sd-mt-2}
\begin{sphinxuseclass}{sd-mb-4}
\begin{sphinxuseclass}{sd-row}
\begin{sphinxuseclass}{sd-row-cols-2}
\begin{sphinxuseclass}{sd-gx-2}
\begin{sphinxuseclass}{sd-gy-1}
\begin{sphinxuseclass}{sd-col}
\begin{sphinxuseclass}{sd-d-flex-row}
\begin{sphinxuseclass}{sd-align-minor-center}
\begin{sphinxuseclass}{sd-container-fluid}
\begin{sphinxuseclass}{sd-sphinx-override}
\begin{sphinxuseclass}{sd-row}
\begin{sphinxuseclass}{sd-row-cols-2}
\begin{sphinxuseclass}{sd-row-cols-xs-2}
\begin{sphinxuseclass}{sd-row-cols-sm-3}
\begin{sphinxuseclass}{sd-row-cols-md-3}
\begin{sphinxuseclass}{sd-row-cols-lg-3}
\begin{sphinxuseclass}{sd-gx-3}
\begin{sphinxuseclass}{sd-gy-1}
\begin{sphinxuseclass}{sd-col}
\begin{sphinxuseclass}{sd-col-auto}
\begin{sphinxuseclass}{sd-d-flex-row}
\begin{sphinxuseclass}{sd-align-minor-center}
\sphinxAtStartPar
basics

\end{sphinxuseclass}
\end{sphinxuseclass}
\end{sphinxuseclass}
\end{sphinxuseclass}
\begin{sphinxuseclass}{sd-col}
\begin{sphinxuseclass}{sd-col-auto}
\begin{sphinxuseclass}{sd-d-flex-row}
\begin{sphinxuseclass}{sd-align-minor-center}
\sphinxAtStartPar
Nov 09, 2024

\end{sphinxuseclass}
\end{sphinxuseclass}
\end{sphinxuseclass}
\end{sphinxuseclass}
\begin{sphinxuseclass}{sd-col}
\begin{sphinxuseclass}{sd-col-auto}
\begin{sphinxuseclass}{sd-d-flex-row}
\begin{sphinxuseclass}{sd-align-minor-center}
\sphinxAtStartPar
1 min read

\end{sphinxuseclass}
\end{sphinxuseclass}
\end{sphinxuseclass}
\end{sphinxuseclass}
\end{sphinxuseclass}
\end{sphinxuseclass}
\end{sphinxuseclass}
\end{sphinxuseclass}
\end{sphinxuseclass}
\end{sphinxuseclass}
\end{sphinxuseclass}
\end{sphinxuseclass}
\end{sphinxuseclass}
\end{sphinxuseclass}
\end{sphinxuseclass}
\end{sphinxuseclass}
\end{sphinxuseclass}
\end{sphinxuseclass}
\end{sphinxuseclass}
\end{sphinxuseclass}
\end{sphinxuseclass}
\end{sphinxuseclass}
\end{sphinxuseclass}
\end{sphinxuseclass}
\end{sphinxuseclass}
\end{sphinxuseclass}

\section{Lavoro e potenza}
\label{\detokenize{ch/actions-work-power:lavoro-e-potenza}}\label{\detokenize{ch/actions-work-power:classical-mechanics-actions-work-power}}\label{\detokenize{ch/actions-work-power::doc}}

\subsection{Lavoro e potenza di una forza}
\label{\detokenize{ch/actions-work-power:lavoro-e-potenza-di-una-forza}}
\sphinxAtStartPar
\sphinxstylestrong{Lavoro.} Il lavoro elementare di una forza \(\mathbf{F}\) applicata in un punto di applicazione \(\mathbf{r}_P\) che subisce uno spostamento \(d\mathbf{r}\) viene definito come
\begin{equation*}
\begin{split}\delta L = \mathbf{F} \cdot d \mathbf{r} \ .\end{split}
\end{equation*}
\sphinxAtStartPar
Per uno spostamento finito del punto \(P\) dal punto \(\mathbf{r}_A\) al punto \(\mathbf{r}_B\) lungo la linea \(\ell_{AB}\), è necessario sommare i contributi elementari (e quindi integrare) lungo il percorso \(\ell_{AB}\)
\begin{equation*}
\begin{split}L = \int_{\ell_{AB}} \delta L = \int_{\ell_{AB}} \mathbf{F} \cdot d \mathbf{r}_P \ .\end{split}
\end{equation*}
\sphinxAtStartPar
\sphinxstylestrong{todo} In generale il lavoro di una forza o di un campo di forze \sphinxstylestrong{dipende dal percorso di integrazione} \(\ell_{AB}\). Sarebbe meglio usare \(\delta L\) per ricordare questa proprietà del lavoro, che quindi non è un \sphinxstylestrong{differenziale esatto}.

\sphinxAtStartPar
\sphinxstylestrong{Potenza.} La potenza di una forza \(\mathbf{F}\) applicata in un punto di applicazione \(\mathbf{r}_P\) che ha velocità \(\mathbf{v}_P\) viene definita come
\begin{equation*}
\begin{split}P = \mathbf{F} \cdot \mathbf{v}_P \ ,\end{split}
\end{equation*}
\sphinxAtStartPar
cioè come la derivata nel tempo del lavoro,
\begin{equation*}
\begin{split}\dfrac{dL}{dt} = \mathbf{F} \cdot \dfrac{d \mathbf{r}_P}{d t} = \mathbf{F} \cdot \mathbf{v}_P = P \ .\end{split}
\end{equation*}

\subsection{Lavoro e potenza di una coppia di forze}
\label{\detokenize{ch/actions-work-power:lavoro-e-potenza-di-una-coppia-di-forze}}
\sphinxAtStartPar
Una coppia di forze è definita come il momento generato da due forze con uguale intensità e verso opposto, \(\mathbf{F}_1 = - \mathbf{F}_2\),
\begin{equation*}
\begin{split}\mathbf{C} = (\mathbf{r}_1 - \mathbf{r}_2) \times \mathbf{F}_1 \ .\end{split}
\end{equation*}
\sphinxAtStartPar
\sphinxstylestrong{Potenza.} La potenza di una coppia di forze è la somma della potenza generata da entrambe le forze,
\begin{equation*}
\begin{split}P = P_1 + P_2 = \mathbf{F}_1 \cdot \mathbf{v}_1 + \mathbf{F}_2 \cdot \mathbf{v}_2 = \mathbf{F}_1 \cdot (\mathbf{v}_1 - \mathbf{v}_2) \ .\end{split}
\end{equation*}
\sphinxAtStartPar
Se la coppia di forze è applicata a una coppia di punti che descrivono un atto di moto rigido,
\begin{equation*}
\begin{split}\mathbf{v}_1 - \mathbf{v}_2 = \symbf{\omega} \times (\mathbf{r}_1 - \mathbf{r}_2) \ ,\end{split}
\end{equation*}
\sphinxAtStartPar
si può riscrivere la potenza della coppia di forze come
\begin{equation*}
\begin{split}\begin{aligned}
P & = \mathbf{F}_1 \cdot \symbf{\omega} \times (\mathbf{r}_1 - \mathbf{r}_2) = \\
  & = \symbf{\omega} \cdot ( \mathbf{r}_1 - \mathbf{r}_2 ) \times \mathbf{F}_1 = \\
  & = \symbf{\omega} \cdot \mathbf{C}
\end {aligned}\end{split}
\end{equation*}
\sphinxstepscope

\begin{sphinxuseclass}{sd-container-fluid}
\begin{sphinxuseclass}{sd-sphinx-override}
\begin{sphinxuseclass}{sd-p-0}
\begin{sphinxuseclass}{sd-mt-2}
\begin{sphinxuseclass}{sd-mb-4}
\begin{sphinxuseclass}{sd-row}
\begin{sphinxuseclass}{sd-row-cols-2}
\begin{sphinxuseclass}{sd-gx-2}
\begin{sphinxuseclass}{sd-gy-1}
\begin{sphinxuseclass}{sd-col}
\begin{sphinxuseclass}{sd-d-flex-row}
\begin{sphinxuseclass}{sd-align-minor-center}
\begin{sphinxuseclass}{sd-container-fluid}
\begin{sphinxuseclass}{sd-sphinx-override}
\begin{sphinxuseclass}{sd-row}
\begin{sphinxuseclass}{sd-row-cols-2}
\begin{sphinxuseclass}{sd-row-cols-xs-2}
\begin{sphinxuseclass}{sd-row-cols-sm-3}
\begin{sphinxuseclass}{sd-row-cols-md-3}
\begin{sphinxuseclass}{sd-row-cols-lg-3}
\begin{sphinxuseclass}{sd-gx-3}
\begin{sphinxuseclass}{sd-gy-1}
\begin{sphinxuseclass}{sd-col}
\begin{sphinxuseclass}{sd-col-auto}
\begin{sphinxuseclass}{sd-d-flex-row}
\begin{sphinxuseclass}{sd-align-minor-center}
\sphinxAtStartPar
basics

\end{sphinxuseclass}
\end{sphinxuseclass}
\end{sphinxuseclass}
\end{sphinxuseclass}
\begin{sphinxuseclass}{sd-col}
\begin{sphinxuseclass}{sd-col-auto}
\begin{sphinxuseclass}{sd-d-flex-row}
\begin{sphinxuseclass}{sd-align-minor-center}
\sphinxAtStartPar
Nov 09, 2024

\end{sphinxuseclass}
\end{sphinxuseclass}
\end{sphinxuseclass}
\end{sphinxuseclass}
\begin{sphinxuseclass}{sd-col}
\begin{sphinxuseclass}{sd-col-auto}
\begin{sphinxuseclass}{sd-d-flex-row}
\begin{sphinxuseclass}{sd-align-minor-center}
\sphinxAtStartPar
1 min read

\end{sphinxuseclass}
\end{sphinxuseclass}
\end{sphinxuseclass}
\end{sphinxuseclass}
\end{sphinxuseclass}
\end{sphinxuseclass}
\end{sphinxuseclass}
\end{sphinxuseclass}
\end{sphinxuseclass}
\end{sphinxuseclass}
\end{sphinxuseclass}
\end{sphinxuseclass}
\end{sphinxuseclass}
\end{sphinxuseclass}
\end{sphinxuseclass}
\end{sphinxuseclass}
\end{sphinxuseclass}
\end{sphinxuseclass}
\end{sphinxuseclass}
\end{sphinxuseclass}
\end{sphinxuseclass}
\end{sphinxuseclass}
\end{sphinxuseclass}
\end{sphinxuseclass}
\end{sphinxuseclass}
\end{sphinxuseclass}

\section{Azioni conservative}
\label{\detokenize{ch/actions-conservative:azioni-conservative}}\label{\detokenize{ch/actions-conservative:classical-mechanics-actions-conservative}}\label{\detokenize{ch/actions-conservative::doc}}
\sphinxAtStartPar
In generale, il lavoro di una forza o di un campo di forze dipende dalla traiettoria \(\ell_{AB}\) del punto di applicazione tra i punti iniziale e finale \(\mathbf{r}_A\), \(\mathbf{r}_B\).

\sphinxAtStartPar
In alcuni casi, il lavoro è indipendente dal percorso \(\ell_{AB}\), ma dipende solo dai suoi punti estremi. Se questo è vero per tutti i percorsi, il lavoro è svolto da una \sphinxstylestrong{forza conservativa}.

\sphinxAtStartPar
In questo caso, l’integrale di linea può essere riscritto senza indicare esplicitamente la dipendenza dal percorso \(\ell_{AB}\), ma indicandone solo gli estremi
\begin{equation*}
\begin{split}L = \int_{\mathbf{r}_A}^{\mathbf{r}_B} \mathbf{F} \cdot d\mathbf{r} \ ,\end{split}
\end{equation*}
\sphinxAtStartPar
e il suo valore può essere calcolato come differenza di una funzione scalare della variabile spaziale \(U(\mathbf{r}) = -V(\mathbf{r})\) valutata nei due estremi del percorso,
\begin{equation*}
\begin{split}L_{AB} = U(\mathbf{r}_B) - U(\mathbf{r}_A) = - V(\mathbf{r}_B) + V(\mathbf{r}_A) .\end{split}
\end{equation*}
\sphinxAtStartPar
o in forma differenziale,
\begin{equation*}
\begin{split}\delta L = dU = - dV \ .\end{split}
\end{equation*}
\sphinxAtStartPar
Sotto l’ipotesi di sufficiente regolarità (\sphinxstylestrong{todo}), e confrontando le espressioni del lavoro infinitesimale di un campo di forze \(\mathbf{F}(\mathbf{r})\),
\begin{equation*}
\begin{split}\delta L = dU = \mathbf{F} \cdot \mathbf{r} \ ,\end{split}
\end{equation*}
\sphinxAtStartPar
si può scrivere il campo di forze come il gradiente della funzione \(U\),
\begin{equation*}
\begin{split}\mathbf{F}(\mathbf{r}) = \nabla U(\mathbf{r}) = - \nabla V(\mathbf{r} )\ .\end{split}
\end{equation*}
\sphinxAtStartPar
Per le proprietà dei campi e degli operatori vettoriali, se un campo vettoriale può essere scritto come gradiente di una funzione scalare, il suo rotore è nullo,
\begin{equation*}
\begin{split}\nabla \times \mathbf{F}(\mathbf{r}) = \mathbf{0} \ .\end{split}
\end{equation*}
\sphinxstepscope

\begin{sphinxuseclass}{sd-container-fluid}
\begin{sphinxuseclass}{sd-sphinx-override}
\begin{sphinxuseclass}{sd-p-0}
\begin{sphinxuseclass}{sd-mt-2}
\begin{sphinxuseclass}{sd-mb-4}
\begin{sphinxuseclass}{sd-row}
\begin{sphinxuseclass}{sd-row-cols-2}
\begin{sphinxuseclass}{sd-gx-2}
\begin{sphinxuseclass}{sd-gy-1}
\begin{sphinxuseclass}{sd-col}
\begin{sphinxuseclass}{sd-d-flex-row}
\begin{sphinxuseclass}{sd-align-minor-center}
\begin{sphinxuseclass}{sd-container-fluid}
\begin{sphinxuseclass}{sd-sphinx-override}
\begin{sphinxuseclass}{sd-row}
\begin{sphinxuseclass}{sd-row-cols-2}
\begin{sphinxuseclass}{sd-row-cols-xs-2}
\begin{sphinxuseclass}{sd-row-cols-sm-3}
\begin{sphinxuseclass}{sd-row-cols-md-3}
\begin{sphinxuseclass}{sd-row-cols-lg-3}
\begin{sphinxuseclass}{sd-gx-3}
\begin{sphinxuseclass}{sd-gy-1}
\begin{sphinxuseclass}{sd-col}
\begin{sphinxuseclass}{sd-col-auto}
\begin{sphinxuseclass}{sd-d-flex-row}
\begin{sphinxuseclass}{sd-align-minor-center}
\sphinxAtStartPar
basics

\end{sphinxuseclass}
\end{sphinxuseclass}
\end{sphinxuseclass}
\end{sphinxuseclass}
\begin{sphinxuseclass}{sd-col}
\begin{sphinxuseclass}{sd-col-auto}
\begin{sphinxuseclass}{sd-d-flex-row}
\begin{sphinxuseclass}{sd-align-minor-center}
\sphinxAtStartPar
Nov 09, 2024

\end{sphinxuseclass}
\end{sphinxuseclass}
\end{sphinxuseclass}
\end{sphinxuseclass}
\begin{sphinxuseclass}{sd-col}
\begin{sphinxuseclass}{sd-col-auto}
\begin{sphinxuseclass}{sd-d-flex-row}
\begin{sphinxuseclass}{sd-align-minor-center}
\sphinxAtStartPar
1 min read

\end{sphinxuseclass}
\end{sphinxuseclass}
\end{sphinxuseclass}
\end{sphinxuseclass}
\end{sphinxuseclass}
\end{sphinxuseclass}
\end{sphinxuseclass}
\end{sphinxuseclass}
\end{sphinxuseclass}
\end{sphinxuseclass}
\end{sphinxuseclass}
\end{sphinxuseclass}
\end{sphinxuseclass}
\end{sphinxuseclass}
\end{sphinxuseclass}
\end{sphinxuseclass}
\end{sphinxuseclass}
\end{sphinxuseclass}
\end{sphinxuseclass}
\end{sphinxuseclass}
\end{sphinxuseclass}
\end{sphinxuseclass}
\end{sphinxuseclass}
\end{sphinxuseclass}
\end{sphinxuseclass}
\end{sphinxuseclass}

\section{Reazioni vincolari}
\label{\detokenize{ch/actions-reactions:reazioni-vincolari}}\label{\detokenize{ch/actions-reactions:classical-mechanics-actions-reactions}}\label{\detokenize{ch/actions-reactions::doc}}
\sphinxAtStartPar
I vincoli cinematici agiscono su un sistema limitandone la possibilità di movimento, esercitando forze e momenti su di esso, definibili come reazioni vincolari.

\sphinxAtStartPar
In generale, in corrispondenza di un vincolo \sphinxstylestrong{ideale} (\sphinxstylestrong{todo} dare definizione di vincolo ideale, e trattare/accennare/rimandare all’attrito) nasce un’azione vincolare corrispondente a ogni grado di libertà vincolato: così ad esempio il vincolo della traslazione di un punto in una direzione ha come reazione corrispondente una forza in quella direzione; il vincolo della rotazione attorno a un asse ha come azione corrispondente un momento allineato con quell’asse.

\sphinxAtStartPar
Queste condizioni possono essere ricavate dalle equazioni della dinamica per sistemi senza massa, come spesso considerato nel modello ideale dei vincoli.


\subsection{Esempi 2D}
\label{\detokenize{ch/actions-reactions:esempi-2d}}\begin{itemize}
\item {} 
\sphinxAtStartPar
Incastro

\item {} 
\sphinxAtStartPar
Cerniera

\item {} 
\sphinxAtStartPar
Pattino

\item {} 
\sphinxAtStartPar
Carrello

\item {} 
\sphinxAtStartPar
Manicotto

\end{itemize}


\subsection{Esempi 3D}
\label{\detokenize{ch/actions-reactions:esempi-3d}}\begin{itemize}
\item {} 
\sphinxAtStartPar
Incastro

\item {} 
\sphinxAtStartPar
Cerniera sferica

\item {} 
\sphinxAtStartPar
Cerniera cilindrica

\item {} 
\sphinxAtStartPar
Pattino

\item {} 
\sphinxAtStartPar
Carrello

\item {} 
\sphinxAtStartPar
Manicotto

\end{itemize}

\sphinxstepscope

\begin{sphinxuseclass}{sd-container-fluid}
\begin{sphinxuseclass}{sd-sphinx-override}
\begin{sphinxuseclass}{sd-p-0}
\begin{sphinxuseclass}{sd-mt-2}
\begin{sphinxuseclass}{sd-mb-4}
\begin{sphinxuseclass}{sd-row}
\begin{sphinxuseclass}{sd-row-cols-2}
\begin{sphinxuseclass}{sd-gx-2}
\begin{sphinxuseclass}{sd-gy-1}
\begin{sphinxuseclass}{sd-col}
\begin{sphinxuseclass}{sd-d-flex-row}
\begin{sphinxuseclass}{sd-align-minor-center}
\begin{sphinxuseclass}{sd-container-fluid}
\begin{sphinxuseclass}{sd-sphinx-override}
\begin{sphinxuseclass}{sd-row}
\begin{sphinxuseclass}{sd-row-cols-2}
\begin{sphinxuseclass}{sd-row-cols-xs-2}
\begin{sphinxuseclass}{sd-row-cols-sm-3}
\begin{sphinxuseclass}{sd-row-cols-md-3}
\begin{sphinxuseclass}{sd-row-cols-lg-3}
\begin{sphinxuseclass}{sd-gx-3}
\begin{sphinxuseclass}{sd-gy-1}
\begin{sphinxuseclass}{sd-col}
\begin{sphinxuseclass}{sd-col-auto}
\begin{sphinxuseclass}{sd-d-flex-row}
\begin{sphinxuseclass}{sd-align-minor-center}
\sphinxAtStartPar
basics

\end{sphinxuseclass}
\end{sphinxuseclass}
\end{sphinxuseclass}
\end{sphinxuseclass}
\begin{sphinxuseclass}{sd-col}
\begin{sphinxuseclass}{sd-col-auto}
\begin{sphinxuseclass}{sd-d-flex-row}
\begin{sphinxuseclass}{sd-align-minor-center}
\sphinxAtStartPar
Nov 09, 2024

\end{sphinxuseclass}
\end{sphinxuseclass}
\end{sphinxuseclass}
\end{sphinxuseclass}
\begin{sphinxuseclass}{sd-col}
\begin{sphinxuseclass}{sd-col-auto}
\begin{sphinxuseclass}{sd-d-flex-row}
\begin{sphinxuseclass}{sd-align-minor-center}
\sphinxAtStartPar
0 min read

\end{sphinxuseclass}
\end{sphinxuseclass}
\end{sphinxuseclass}
\end{sphinxuseclass}
\end{sphinxuseclass}
\end{sphinxuseclass}
\end{sphinxuseclass}
\end{sphinxuseclass}
\end{sphinxuseclass}
\end{sphinxuseclass}
\end{sphinxuseclass}
\end{sphinxuseclass}
\end{sphinxuseclass}
\end{sphinxuseclass}
\end{sphinxuseclass}
\end{sphinxuseclass}
\end{sphinxuseclass}
\end{sphinxuseclass}
\end{sphinxuseclass}
\end{sphinxuseclass}
\end{sphinxuseclass}
\end{sphinxuseclass}
\end{sphinxuseclass}
\end{sphinxuseclass}
\end{sphinxuseclass}
\end{sphinxuseclass}

\section{Esempi}
\label{\detokenize{ch/actions-examples:esempi}}\label{\detokenize{ch/actions-examples:classical-mechanics-actions-examples}}\label{\detokenize{ch/actions-examples::doc}}
\sphinxstepscope


\chapter{Inertia}
\label{\detokenize{ch/inertia:inertia}}\label{\detokenize{ch/inertia:classical-mechanics-inertia}}\label{\detokenize{ch/inertia::doc}}
\sphinxstepscope

\begin{sphinxuseclass}{sd-container-fluid}
\begin{sphinxuseclass}{sd-sphinx-override}
\begin{sphinxuseclass}{sd-p-0}
\begin{sphinxuseclass}{sd-mt-2}
\begin{sphinxuseclass}{sd-mb-4}
\begin{sphinxuseclass}{sd-row}
\begin{sphinxuseclass}{sd-row-cols-2}
\begin{sphinxuseclass}{sd-gx-2}
\begin{sphinxuseclass}{sd-gy-1}
\begin{sphinxuseclass}{sd-col}
\begin{sphinxuseclass}{sd-d-flex-row}
\begin{sphinxuseclass}{sd-align-minor-center}
\begin{sphinxuseclass}{sd-container-fluid}
\begin{sphinxuseclass}{sd-sphinx-override}
\begin{sphinxuseclass}{sd-row}
\begin{sphinxuseclass}{sd-row-cols-2}
\begin{sphinxuseclass}{sd-row-cols-xs-2}
\begin{sphinxuseclass}{sd-row-cols-sm-3}
\begin{sphinxuseclass}{sd-row-cols-md-3}
\begin{sphinxuseclass}{sd-row-cols-lg-3}
\begin{sphinxuseclass}{sd-gx-3}
\begin{sphinxuseclass}{sd-gy-1}
\begin{sphinxuseclass}{sd-col}
\begin{sphinxuseclass}{sd-col-auto}
\begin{sphinxuseclass}{sd-d-flex-row}
\begin{sphinxuseclass}{sd-align-minor-center}
\sphinxAtStartPar
basics

\end{sphinxuseclass}
\end{sphinxuseclass}
\end{sphinxuseclass}
\end{sphinxuseclass}
\begin{sphinxuseclass}{sd-col}
\begin{sphinxuseclass}{sd-col-auto}
\begin{sphinxuseclass}{sd-d-flex-row}
\begin{sphinxuseclass}{sd-align-minor-center}
\sphinxAtStartPar
Nov 09, 2024

\end{sphinxuseclass}
\end{sphinxuseclass}
\end{sphinxuseclass}
\end{sphinxuseclass}
\begin{sphinxuseclass}{sd-col}
\begin{sphinxuseclass}{sd-col-auto}
\begin{sphinxuseclass}{sd-d-flex-row}
\begin{sphinxuseclass}{sd-align-minor-center}
\sphinxAtStartPar
1 min read

\end{sphinxuseclass}
\end{sphinxuseclass}
\end{sphinxuseclass}
\end{sphinxuseclass}
\end{sphinxuseclass}
\end{sphinxuseclass}
\end{sphinxuseclass}
\end{sphinxuseclass}
\end{sphinxuseclass}
\end{sphinxuseclass}
\end{sphinxuseclass}
\end{sphinxuseclass}
\end{sphinxuseclass}
\end{sphinxuseclass}
\end{sphinxuseclass}
\end{sphinxuseclass}
\end{sphinxuseclass}
\end{sphinxuseclass}
\end{sphinxuseclass}
\end{sphinxuseclass}
\end{sphinxuseclass}
\end{sphinxuseclass}
\end{sphinxuseclass}
\end{sphinxuseclass}
\end{sphinxuseclass}
\end{sphinxuseclass}

\chapter{Dinamica}
\label{\detokenize{ch/dynamics:dinamica}}\label{\detokenize{ch/dynamics:classical-mechanics-dynamics}}\label{\detokenize{ch/dynamics::doc}}
\sphinxAtStartPar
La dinamica fornisce il legame tra il moto di un corpo e le azioni causa del moto stesso.

\sphinxAtStartPar
I principi della dinamica di Newton e le equazioni cardinali della dinamica sono le leggi fisiche che governano il moto dei sistemi meccanici. Queste leggi fisiche vengono formulate nei termini di alcune grandezze fisiche, come la quantità di moto, il momento della quantità di moto o l’energia cinetica del sistema. Queste grandezze dinamiche hanno la proprietà di essere additive (per definizione) e rendono particolarmente facile la scrittura di una forma generale delle equazioni che governano il moto, e che si riducono a una relazione tra la derivate nel tempo di queste grandezze dinamiche e le cause della loro variazione. In assenza di cause nette, si ottengono i princìpi di conservazione.

\sphinxstepscope

\begin{sphinxuseclass}{sd-container-fluid}
\begin{sphinxuseclass}{sd-sphinx-override}
\begin{sphinxuseclass}{sd-p-0}
\begin{sphinxuseclass}{sd-mt-2}
\begin{sphinxuseclass}{sd-mb-4}
\begin{sphinxuseclass}{sd-row}
\begin{sphinxuseclass}{sd-row-cols-2}
\begin{sphinxuseclass}{sd-gx-2}
\begin{sphinxuseclass}{sd-gy-1}
\begin{sphinxuseclass}{sd-col}
\begin{sphinxuseclass}{sd-d-flex-row}
\begin{sphinxuseclass}{sd-align-minor-center}
\begin{sphinxuseclass}{sd-container-fluid}
\begin{sphinxuseclass}{sd-sphinx-override}
\begin{sphinxuseclass}{sd-row}
\begin{sphinxuseclass}{sd-row-cols-2}
\begin{sphinxuseclass}{sd-row-cols-xs-2}
\begin{sphinxuseclass}{sd-row-cols-sm-3}
\begin{sphinxuseclass}{sd-row-cols-md-3}
\begin{sphinxuseclass}{sd-row-cols-lg-3}
\begin{sphinxuseclass}{sd-gx-3}
\begin{sphinxuseclass}{sd-gy-1}
\begin{sphinxuseclass}{sd-col}
\begin{sphinxuseclass}{sd-col-auto}
\begin{sphinxuseclass}{sd-d-flex-row}
\begin{sphinxuseclass}{sd-align-minor-center}
\sphinxAtStartPar
basics

\end{sphinxuseclass}
\end{sphinxuseclass}
\end{sphinxuseclass}
\end{sphinxuseclass}
\begin{sphinxuseclass}{sd-col}
\begin{sphinxuseclass}{sd-col-auto}
\begin{sphinxuseclass}{sd-d-flex-row}
\begin{sphinxuseclass}{sd-align-minor-center}
\sphinxAtStartPar
Nov 09, 2024

\end{sphinxuseclass}
\end{sphinxuseclass}
\end{sphinxuseclass}
\end{sphinxuseclass}
\begin{sphinxuseclass}{sd-col}
\begin{sphinxuseclass}{sd-col-auto}
\begin{sphinxuseclass}{sd-d-flex-row}
\begin{sphinxuseclass}{sd-align-minor-center}
\sphinxAtStartPar
1 min read

\end{sphinxuseclass}
\end{sphinxuseclass}
\end{sphinxuseclass}
\end{sphinxuseclass}
\end{sphinxuseclass}
\end{sphinxuseclass}
\end{sphinxuseclass}
\end{sphinxuseclass}
\end{sphinxuseclass}
\end{sphinxuseclass}
\end{sphinxuseclass}
\end{sphinxuseclass}
\end{sphinxuseclass}
\end{sphinxuseclass}
\end{sphinxuseclass}
\end{sphinxuseclass}
\end{sphinxuseclass}
\end{sphinxuseclass}
\end{sphinxuseclass}
\end{sphinxuseclass}
\end{sphinxuseclass}
\end{sphinxuseclass}
\end{sphinxuseclass}
\end{sphinxuseclass}
\end{sphinxuseclass}
\end{sphinxuseclass}

\section{Principi della dinamica}
\label{\detokenize{ch/dynamics-principles:principi-della-dinamica}}\label{\detokenize{ch/dynamics-principles:classical-mechanics-dynamics-principles}}\label{\detokenize{ch/dynamics-principles::doc}}
\sphinxAtStartPar
\sphinxstylestrong{Primo principio della dinamica.}
Un corpo (o meglio, il baricentro di un corpo) sul quale agisce una forza netta nulla, persevera nel suo stato di quiete o di moto rettilineo uniforme rispetto a un sistema di riferimento inerziale.

\sphinxAtStartPar
\sphinxstylestrong{Secondo principio della dinamica.} Rispetto a un sistema di riferimento inerziale, la variazione della quantità di moto di un sistema è uguale all’impulso delle forze esterne agenti su di esso,
\begin{equation*}
\begin{split}\Delta \mathbf{Q} = \mathbf{I}^e \ .\end{split}
\end{equation*}
\sphinxAtStartPar
Nel caso di moto regolare, in cui la quantità di moto può essere rappresentata da una funzione continua e differenziabile in funzione del tempo, si può scrivere il secondo principio della dinamica in forma differenziale,
\begin{equation*}
\begin{split}\dot{\mathbf{Q}} = \mathbf{R}^e \ ,\end{split}
\end{equation*}
\sphinxAtStartPar
essendo la risultante delle forze esterne, \(\mathbf{R}^e = \frac{d \mathbf{I}^e}{dt}\), la derivata nel tempo dell’impulso.

\sphinxAtStartPar
\sphinxstylestrong{Terzo principio della dinamica.} Se un sistema \(i\) esercita su un sistema \(j\) una forza \(\mathbf{F}_{ji}\), allora il sistema \(j\) esercita sul sistema \(i\) una forza \(\mathbf{F}_{ij}\) “uguale e contraria”, con modulo uguale e verso opposto,
\begin{equation*}
\begin{split}\mathbf{F}_{ij} = - \mathbf{F}_{ji} \ .\end{split}
\end{equation*}
\sphinxstepscope

\begin{sphinxuseclass}{sd-container-fluid}
\begin{sphinxuseclass}{sd-sphinx-override}
\begin{sphinxuseclass}{sd-p-0}
\begin{sphinxuseclass}{sd-mt-2}
\begin{sphinxuseclass}{sd-mb-4}
\begin{sphinxuseclass}{sd-row}
\begin{sphinxuseclass}{sd-row-cols-2}
\begin{sphinxuseclass}{sd-gx-2}
\begin{sphinxuseclass}{sd-gy-1}
\begin{sphinxuseclass}{sd-col}
\begin{sphinxuseclass}{sd-d-flex-row}
\begin{sphinxuseclass}{sd-align-minor-center}
\begin{sphinxuseclass}{sd-container-fluid}
\begin{sphinxuseclass}{sd-sphinx-override}
\begin{sphinxuseclass}{sd-row}
\begin{sphinxuseclass}{sd-row-cols-2}
\begin{sphinxuseclass}{sd-row-cols-xs-2}
\begin{sphinxuseclass}{sd-row-cols-sm-3}
\begin{sphinxuseclass}{sd-row-cols-md-3}
\begin{sphinxuseclass}{sd-row-cols-lg-3}
\begin{sphinxuseclass}{sd-gx-3}
\begin{sphinxuseclass}{sd-gy-1}
\begin{sphinxuseclass}{sd-col}
\begin{sphinxuseclass}{sd-col-auto}
\begin{sphinxuseclass}{sd-d-flex-row}
\begin{sphinxuseclass}{sd-align-minor-center}
\sphinxAtStartPar
basics

\end{sphinxuseclass}
\end{sphinxuseclass}
\end{sphinxuseclass}
\end{sphinxuseclass}
\begin{sphinxuseclass}{sd-col}
\begin{sphinxuseclass}{sd-col-auto}
\begin{sphinxuseclass}{sd-d-flex-row}
\begin{sphinxuseclass}{sd-align-minor-center}
\sphinxAtStartPar
Nov 09, 2024

\end{sphinxuseclass}
\end{sphinxuseclass}
\end{sphinxuseclass}
\end{sphinxuseclass}
\begin{sphinxuseclass}{sd-col}
\begin{sphinxuseclass}{sd-col-auto}
\begin{sphinxuseclass}{sd-d-flex-row}
\begin{sphinxuseclass}{sd-align-minor-center}
\sphinxAtStartPar
2 min read

\end{sphinxuseclass}
\end{sphinxuseclass}
\end{sphinxuseclass}
\end{sphinxuseclass}
\end{sphinxuseclass}
\end{sphinxuseclass}
\end{sphinxuseclass}
\end{sphinxuseclass}
\end{sphinxuseclass}
\end{sphinxuseclass}
\end{sphinxuseclass}
\end{sphinxuseclass}
\end{sphinxuseclass}
\end{sphinxuseclass}
\end{sphinxuseclass}
\end{sphinxuseclass}
\end{sphinxuseclass}
\end{sphinxuseclass}
\end{sphinxuseclass}
\end{sphinxuseclass}
\end{sphinxuseclass}
\end{sphinxuseclass}
\end{sphinxuseclass}
\end{sphinxuseclass}
\end{sphinxuseclass}
\end{sphinxuseclass}

\section{Equazioni cardinali della dinamica}
\label{\detokenize{ch/dynamics-eom:equazioni-cardinali-della-dinamica}}\label{\detokenize{ch/dynamics-eom:classical-mechanics-dynamics-eom}}\label{\detokenize{ch/dynamics-eom::doc}}
\sphinxAtStartPar
Utilizzando i concetti di quantità di moto, momento della quantità di moto ed energia cinetica di un sistema, si possono scrivere le 3 equazioni cardinali della dinamica in una forma valida per \sphinxstylestrong{ogni sistema chiuso}. Nel caso siano soddisfatte alcune condizioni, e solo in questo caso, le equazioni cardinali della dinamica rappresentano dei principi di conservazione delle quantità dinamiche: osservando le espressioni delle equazioni cardinali, è facile intuire che la condizione da soddisfare per otenere un principio di conservazione è l’annullamento di tutti i termini ad eccezione della derivata temporale della quantità conservata.


\subsection{Equazioni cardinali}
\label{\detokenize{ch/dynamics-eom:equazioni-cardinali}}
\sphinxAtStartPar
\sphinxstylestrong{Bilancio della quantità di moto.} La derivata nel tempo della quantità di moto è uguale alla risultante delle forze esterne,
\begin{equation}\label{equation:ch/dynamics-eom:principle:q}
\begin{split}\dot{\mathbf{Q}} = \mathbf{R}^e \ .\end{split}
\end{equation}
\sphinxAtStartPar
\sphinxstylestrong{Bilancio del momento della quantità di moto rispetto a un polo \(H\).} La derivata nel tempo del momento della quantità di moto rispetto a un punto \(H\), a meno di “un termine di trasporto” è uguale alla risultate dei momenti esterni rispetto al polo \(H\)
\begin{equation}\label{equation:ch/dynamics-eom:principle:l}
\begin{split}\dot{\mathbf{L}}_H + \dot{\mathbf{x}}_H \times \mathbf{Q} = \mathbf{M}_H^e \ .\end{split}
\end{equation}
\sphinxAtStartPar
\sphinxstylestrong{Bilancio dell’energia cinetica.} La derivata nel tempo dell’energia cinetica è uguale alla potenza totale agente sul sistema, risultato della somma della potenza delle azioni esterne e della potenza delle azioni interne al sistema
\begin{equation}\label{equation:ch/dynamics-eom:principle:k}
\begin{split}\dot{K} = P^{tot} = P^e + P^i\end{split}
\end{equation}

\subsection{Principi di conservazione}
\label{\detokenize{ch/dynamics-eom:principi-di-conservazione}}
\sphinxAtStartPar
\sphinxstylestrong{Conservazione della quantità di moto, in presenza di forze esterne con risultante nulla.} Se la risultante delle forze esterne è nulla, \(\mathbf{R}^e = \mathbf{0}\), dal bilancio della quantità di moto segue immediatamente
\begin{equation*}
\begin{split}\dot{\mathbf{Q}} = \mathbf{0} \qquad \rightarrow \qquad \mathbf{Q} = \bar{\mathbf{Q}} = \text{cost.}\end{split}
\end{equation*}
\sphinxAtStartPar
\sphinxstylestrong{Conservazione del momento della quantità di moto, in presenza di momenti esterni con risultante nulla.} Se la scelta del polo \(H\) rende nullo il termine di trasporto, \(\dot{\mathbf{r}}_H \times \mathbf{Q} = \mathbf{0}\), se la risultante dei momenti esterni è nulla, \(\mathbf{M}^e_H = \mathbf{0}\), dal bilancio del momento della quantità di moto segue immediatamente
\begin{equation*}
\begin{split}\dot{\mathbf{L}}_H = \mathbf{0} \qquad \rightarrow \qquad \mathbf{L}_H = \bar{\mathbf{L}}_H = \text{cost.}\end{split}
\end{equation*}
\sphinxAtStartPar
\sphinxstylestrong{Conservazione dell’energia cinetica, in presenza di potenza totale nulla.} Se la potenza totale delle azioni sul sistema è nulla, \(P^{tot} = 0\), dal bilancio dell’energia cinetica segue immediatamente
\begin{equation*}
\begin{split}\dot{K} = 0 \qquad \rightarrow \qquad K = \bar{K} = \text{cost.}\end{split}
\end{equation*}
\sphinxAtStartPar
\sphinxstylestrong{Conservazione dell’energia meccanica, in assenza di azioni non conservative.} Ai tre principi di conservazione direttamente ottenibili dalle equazioni cardinali, si aggiunge il principio della conservazione dell’energia meccanica, somma dell’energia cinetica e dell’energia potenziale del sistema,
\begin{equation*}
\begin{split}E^{mech} = K + V \ ,\end{split}
\end{equation*}
\sphinxAtStartPar
in assenza di azioni non conservative. Se non ci sono forze non conervative, la potenza delle azioni agenti sul sistema può essere scritta come l’opposto della derivata nel tempo dell’energia potenziale del sistema,
\begin{equation*}
\begin{split}P^{tot} = -\dot{V}\end{split}
\end{equation*}
\sphinxAtStartPar
Dal bilancio dell’energia cinetica segue
\begin{equation*}
\begin{split}\dot{K} = - \dot{V} \qquad \rightarrow \qquad \dfrac{d}{dt}(K+V) = 0 \qquad \rightarrow \qquad \dot{E}^{mech = 0} \qquad \rightarrow \qquad E^{mech} = \bar{E}^{mech} = \text{const.}\end{split}
\end{equation*}
\sphinxstepscope

\begin{sphinxuseclass}{sd-container-fluid}
\begin{sphinxuseclass}{sd-sphinx-override}
\begin{sphinxuseclass}{sd-p-0}
\begin{sphinxuseclass}{sd-mt-2}
\begin{sphinxuseclass}{sd-mb-4}
\begin{sphinxuseclass}{sd-row}
\begin{sphinxuseclass}{sd-row-cols-2}
\begin{sphinxuseclass}{sd-gx-2}
\begin{sphinxuseclass}{sd-gy-1}
\begin{sphinxuseclass}{sd-col}
\begin{sphinxuseclass}{sd-d-flex-row}
\begin{sphinxuseclass}{sd-align-minor-center}
\begin{sphinxuseclass}{sd-container-fluid}
\begin{sphinxuseclass}{sd-sphinx-override}
\begin{sphinxuseclass}{sd-row}
\begin{sphinxuseclass}{sd-row-cols-2}
\begin{sphinxuseclass}{sd-row-cols-xs-2}
\begin{sphinxuseclass}{sd-row-cols-sm-3}
\begin{sphinxuseclass}{sd-row-cols-md-3}
\begin{sphinxuseclass}{sd-row-cols-lg-3}
\begin{sphinxuseclass}{sd-gx-3}
\begin{sphinxuseclass}{sd-gy-1}
\begin{sphinxuseclass}{sd-col}
\begin{sphinxuseclass}{sd-col-auto}
\begin{sphinxuseclass}{sd-d-flex-row}
\begin{sphinxuseclass}{sd-align-minor-center}
\sphinxAtStartPar
basics

\end{sphinxuseclass}
\end{sphinxuseclass}
\end{sphinxuseclass}
\end{sphinxuseclass}
\begin{sphinxuseclass}{sd-col}
\begin{sphinxuseclass}{sd-col-auto}
\begin{sphinxuseclass}{sd-d-flex-row}
\begin{sphinxuseclass}{sd-align-minor-center}
\sphinxAtStartPar
Nov 09, 2024

\end{sphinxuseclass}
\end{sphinxuseclass}
\end{sphinxuseclass}
\end{sphinxuseclass}
\begin{sphinxuseclass}{sd-col}
\begin{sphinxuseclass}{sd-col-auto}
\begin{sphinxuseclass}{sd-d-flex-row}
\begin{sphinxuseclass}{sd-align-minor-center}
\sphinxAtStartPar
0 min read

\end{sphinxuseclass}
\end{sphinxuseclass}
\end{sphinxuseclass}
\end{sphinxuseclass}
\end{sphinxuseclass}
\end{sphinxuseclass}
\end{sphinxuseclass}
\end{sphinxuseclass}
\end{sphinxuseclass}
\end{sphinxuseclass}
\end{sphinxuseclass}
\end{sphinxuseclass}
\end{sphinxuseclass}
\end{sphinxuseclass}
\end{sphinxuseclass}
\end{sphinxuseclass}
\end{sphinxuseclass}
\end{sphinxuseclass}
\end{sphinxuseclass}
\end{sphinxuseclass}
\end{sphinxuseclass}
\end{sphinxuseclass}
\end{sphinxuseclass}
\end{sphinxuseclass}
\end{sphinxuseclass}
\end{sphinxuseclass}

\subsection{Equazioni cardinali della dinamica: punto materiale}
\label{\detokenize{ch/dynamics-eom-point:equazioni-cardinali-della-dinamica-punto-materiale}}\label{\detokenize{ch/dynamics-eom-point:classical-mechanics-dynamics-eom-point}}\label{\detokenize{ch/dynamics-eom-point::doc}}
\sphinxAtStartPar
\sphinxstylestrong{Quantità dinamiche.}
\begin{equation*}
\begin{split}\begin{aligned}
  \mathbf{Q}_P & := m_P \mathbf{v}_P \\
  \mathbf{L}_{P,H} & := (\mathbf{r}_P - \mathbf{r}_H) \times \mathbf{Q} = m_P (\mathbf{r}_P - \mathbf{r}_H) \times \mathbf{v}_P \\
  K & := \frac{1}{2} m_P \mathbf{v}_P \cdot \mathbf{v}_P = \frac{1}{2} m_P |\mathbf{v}_P|^2
\end{aligned}\end{split}
\end{equation*}
\sphinxAtStartPar
\sphinxstylestrong{Bilancio della quantità di moto.} Il bilancio della quantità di moto di un punto materiale \(P\), \(\mathbf{Q}_P = m \mathbf{v}_P\) segue direttamente dal secondo principio della dinamica di Newton,
\begin{equation*}
\begin{split}\dot{\mathbf{Q}}_P = \mathbf{R}^e_P\end{split}
\end{equation*}
\sphinxAtStartPar
\sphinxstylestrong{Bilancio del momento della quantità di moto.} La derivata nel tempo del momento della quantità di moto viene calcolata usando la regola del prodotto,
\begin{equation*}
\begin{split}\begin{aligned}
\dot{\mathbf{L}}_{P,H} & = \dfrac{d}{dt} \left[ m_P (\mathbf{r}_P - \mathbf{r}_H) \times \mathbf{v}_P \right] = \\
& = m \left[ ( \dot{\mathbf{r}}_P - \dot{\mathbf{r}}_H ) \times \mathbf{v}_P + m_P (\mathbf{r}_P - \mathbf{r}_H) \times \dot{\mathbf{v}}_P \right] = \\
& = - m_P \dot{\mathbf{r}}_H \times \mathbf{v}_P + m_P (\mathbf{r}_P - \mathbf{r}_H) \times \dot{\mathbf{v}}_P = \\
& = - \dot{\mathbf{r}}_H \times \mathbf{Q} + \mathbf{M}_H^{ext} \ .
\end{aligned}\end{split}
\end{equation*}
\sphinxAtStartPar
\sphinxstylestrong{Bilancio dell’energia cinetica.}
\begin{equation*}
\begin{split}\begin{aligned}
\dot{K}_{P} & = \dfrac{d}{dt} \left( \frac{1}{2} m_P \mathbf{v}_P \cdot \mathbf{v}_P \right) = \\
            & = m_P \dot{\mathbf{v}}_P \cdot \mathbf{v}_P = \\
            & = \mathbf{R}^e \cdot \mathbf{v}_P = \\
            & = \mathbf{R}^{tot} \cdot \mathbf{v}_P = P^{tot} \ .
\end{aligned}\end{split}
\end{equation*}
\sphinxstepscope

\begin{sphinxuseclass}{sd-container-fluid}
\begin{sphinxuseclass}{sd-sphinx-override}
\begin{sphinxuseclass}{sd-p-0}
\begin{sphinxuseclass}{sd-mt-2}
\begin{sphinxuseclass}{sd-mb-4}
\begin{sphinxuseclass}{sd-row}
\begin{sphinxuseclass}{sd-row-cols-2}
\begin{sphinxuseclass}{sd-gx-2}
\begin{sphinxuseclass}{sd-gy-1}
\begin{sphinxuseclass}{sd-col}
\begin{sphinxuseclass}{sd-d-flex-row}
\begin{sphinxuseclass}{sd-align-minor-center}
\begin{sphinxuseclass}{sd-container-fluid}
\begin{sphinxuseclass}{sd-sphinx-override}
\begin{sphinxuseclass}{sd-row}
\begin{sphinxuseclass}{sd-row-cols-2}
\begin{sphinxuseclass}{sd-row-cols-xs-2}
\begin{sphinxuseclass}{sd-row-cols-sm-3}
\begin{sphinxuseclass}{sd-row-cols-md-3}
\begin{sphinxuseclass}{sd-row-cols-lg-3}
\begin{sphinxuseclass}{sd-gx-3}
\begin{sphinxuseclass}{sd-gy-1}
\begin{sphinxuseclass}{sd-col}
\begin{sphinxuseclass}{sd-col-auto}
\begin{sphinxuseclass}{sd-d-flex-row}
\begin{sphinxuseclass}{sd-align-minor-center}
\sphinxAtStartPar
basics

\end{sphinxuseclass}
\end{sphinxuseclass}
\end{sphinxuseclass}
\end{sphinxuseclass}
\begin{sphinxuseclass}{sd-col}
\begin{sphinxuseclass}{sd-col-auto}
\begin{sphinxuseclass}{sd-d-flex-row}
\begin{sphinxuseclass}{sd-align-minor-center}
\sphinxAtStartPar
Nov 09, 2024

\end{sphinxuseclass}
\end{sphinxuseclass}
\end{sphinxuseclass}
\end{sphinxuseclass}
\begin{sphinxuseclass}{sd-col}
\begin{sphinxuseclass}{sd-col-auto}
\begin{sphinxuseclass}{sd-d-flex-row}
\begin{sphinxuseclass}{sd-align-minor-center}
\sphinxAtStartPar
2 min read

\end{sphinxuseclass}
\end{sphinxuseclass}
\end{sphinxuseclass}
\end{sphinxuseclass}
\end{sphinxuseclass}
\end{sphinxuseclass}
\end{sphinxuseclass}
\end{sphinxuseclass}
\end{sphinxuseclass}
\end{sphinxuseclass}
\end{sphinxuseclass}
\end{sphinxuseclass}
\end{sphinxuseclass}
\end{sphinxuseclass}
\end{sphinxuseclass}
\end{sphinxuseclass}
\end{sphinxuseclass}
\end{sphinxuseclass}
\end{sphinxuseclass}
\end{sphinxuseclass}
\end{sphinxuseclass}
\end{sphinxuseclass}
\end{sphinxuseclass}
\end{sphinxuseclass}
\end{sphinxuseclass}
\end{sphinxuseclass}

\subsection{Equazioni cardinali della dinamica: sistemi discreti di punti materiali}
\label{\detokenize{ch/dynamics-eom-points:equazioni-cardinali-della-dinamica-sistemi-discreti-di-punti-materiali}}\label{\detokenize{ch/dynamics-eom-points:classical-mechanics-dynamics-eom-points}}\label{\detokenize{ch/dynamics-eom-points::doc}}
\sphinxAtStartPar
Partendo dalle equazioni dinamiche per un punto, si calcolano le equazioni dinamiche per un sistema di punti, sfruttando il terzo principio della dinamica di azione/reazione. Lo sviluppo delle equazioni permette di comprendere che la natura additiva delle grandezze dinamiche (quantità di moto, momento della quantità di moto, energia cinetica) segue direttamente dalla loro definizione.

\sphinxAtStartPar
\sphinxstylestrong{Bilancio della quantità di moto.}
E’ possibile scrivere il bilancio della quantità di moto per ogni punto \(i\) del sistema, scrivendo la risultante delle forze esterne agente sul punto come la somma delle forze esterne all’intero sistema agenti sul punto e le forze interne scambiate con gli altri punti del sistema,
\begin{equation*}
\begin{split}\mathbf{R}_i^{ext,i} = \mathbf{F}_i^{ext} + \sum_{j \ne i} \mathbf{F}_{ij} \ .\end{split}
\end{equation*}
\sphinxAtStartPar
L’equazione di bilancio per la \(i\)\sphinxhyphen{}esima massa diventa quindi
\begin{equation*}
\begin{split}\dot{\mathbf{Q}}_i = \mathbf{R}_i^{ext,i} = \mathbf{F}_i^{ext} + \sum_{j \ne i} \mathbf{F}_{ij} \ .\end{split}
\end{equation*}
\sphinxAtStartPar
Sommando le equazioni di bilancio di tutte le masse, si ottiene
\begin{equation*}
\begin{split}\begin{aligned}
\sum_{i} \dot{\mathbf{Q}}_i & = \sum_i \mathbf{F}_{i}^{ext} + \sum_i \sum_{j \ne i} \mathbf{F}_{ij} = \\
                            & = \sum_i \mathbf{F}_{i}^{ext} + \sum_{\{i,j\}} \underbrace{\left( \mathbf{F}_{ij} + \mathbf{F}_{ji} \right)}_{=\mathbf{0}} 
\end{aligned}\end{split}
\end{equation*}
\sphinxAtStartPar
e definendo la quantità di moto di un sistema come la somma delle quantità di moto delle sue parti e la risultante delle forze esterne come somma delle forze esterne agenti sulle parti del sistema,
\begin{equation*}
\begin{split}\mathbf{Q} := \sum_i \mathbf{Q}\end{split}
\end{equation*}\begin{equation*}
\begin{split}\mathbf{R}^e := \sum_i \mathbf{F}_i^{ext}\end{split}
\end{equation*}
\sphinxAtStartPar
si ritrova la forma generale del bilancio della quantità di moto,
\begin{equation*}
\begin{split}\dot{\mathbf{Q}} = \mathbf{R}^e \ .\end{split}
\end{equation*}
\sphinxAtStartPar
\sphinxstylestrong{Bilancio del momento della quantità di moto.}
E’ possibile scrivere il bilancio del momento della quantità di moto per ogni punto \(i\) del sistema, scrivendo la risultante dei momenti esterni agente sul punto come la somma dei momenti esterni all’intero sistema agenti sul punto e i momenti interni scambiati con gli altri punti del sistema,
\begin{equation*}
\begin{split}\mathbf{M}_{H,i}^{ext,i} = \mathbf{M}_{H,i}^{ext} + \sum_{j \ne i} \mathbf{M}_{H,ij} \ .\end{split}
\end{equation*}
\sphinxAtStartPar
Nel caso le parti del sistema interagiscano tramite forze, il momento rispetto al polo \(H\) generato dalla massa \(j\) sulla massa \(i\) vale
\begin{equation*}
\begin{split}\mathbf{M}_{H,ij} = (\mathbf{r}_i - \mathbf{r}_H) \times \mathbf{F}_{ij} \ .\end{split}
\end{equation*}
\sphinxAtStartPar
L’equazione di bilancio per la \(i\)\sphinxhyphen{}esima massa diventa quindi
\begin{equation*}
\begin{split}\dot{\mathbf{L}}_{H,i} + \dot{\mathbf{r}}_H \times \mathbf{Q}_i = \mathbf{M}_{H,i}^{ext,i} = \mathbf{M}_{H,i}^{ext} + \sum_{j \ne i} \mathbf{M}_{H,ij} \ .\end{split}
\end{equation*}
\sphinxAtStartPar
Sommando le equazioni di bilancio di tutte le masse, si ottiene
\begin{equation*}
\begin{split}\begin{aligned}
\sum_{i} \left( \dot{\mathbf{L}}_i + \dot{\mathbf{r}}_H \times \mathbf{Q}_i \right) & = \sum_i \mathbf{M}_{H,i}^{ext} + \sum_i \sum_{j \ne i} \mathbf{M}_{H,ij} = \\
                            & = \sum_i \mathbf{M}_{H,i}^{ext} + \sum_{\{i,j\}} \underbrace{\left( \mathbf{M}_{H,ij} + \mathbf{M}_{H,ji} \right)}_{=\mathbf{0}} 
\end{aligned}\end{split}
\end{equation*}
\sphinxAtStartPar
e riconoscendo la quantità di moto del sistema e definendo il momento della quantità di moto di un sistema come la somma del momento della quantità di moto delle sue parti e la risultante dei momenti esterni come somma dei momenti esterni agenti sulle parti del sistema,
\begin{equation*}
\begin{split}\mathbf{L}_H := \sum_i \mathbf{L}_{H,i}\end{split}
\end{equation*}\begin{equation*}
\begin{split}\mathbf{M}_H^e := \sum_i \mathbf{M}_{H,i}^{ext}\end{split}
\end{equation*}
\sphinxAtStartPar
si ritrova la forma generale del bilancio del momento della quantità di moto,
\begin{equation*}
\begin{split}\dot{\mathbf{L}}_{H} + \dot{\mathbf{r}}_H \times \mathbf{Q} = \mathbf{M}_H^e \ .\end{split}
\end{equation*}
\sphinxAtStartPar
\sphinxstylestrong{Bilancio dell’energia cinetica.}
E’ possibile ricavare il bilancio dell’energia cinetica del sistema, moltiplicando scalarmente il bilancio della quantità di moto di ogni punto,
\begin{equation*}
\begin{split}\mathbf{v}_i \cdot m_i \dot{\mathbf{v}}_i = \mathbf{v}_i \cdot \left( \mathbf{F}_i^{e} + \sum_{j \ne i} \mathbf{F}_{ij} \right) \ ,\end{split}
\end{equation*}
\sphinxAtStartPar
riconoscendo nel primo termine la derivata nel tempo dell’energia cinetica dell’\(i\)\sphinxhyphen{}esimo punto,
\begin{equation*}
\begin{split}\dot{K}_i = \dfrac{d}{dt} \left( \frac{1}{2} m_i \mathbf{v}_i \cdot \mathbf{v}_i \right) = m_i \mathbf{v}_i \cdot \dot{\mathbf{v}}_i \ ,\end{split}
\end{equation*}
\sphinxAtStartPar
e sommando queste equazioni di bilancio per ottenere
\begin{equation*}
\begin{split}\begin{aligned}
  \sum_i \dot{K}_i = \sum_i \mathbf{v}_i \cdot  \mathbf{F}_i^{e} + \sum_i \mathbf{v}_i \cdot \sum_{j \ne i} \mathbf{F}_{ij} \ . 
\end{aligned}\end{split}
\end{equation*}
\sphinxAtStartPar
Definendo l’energia cinetica di un sistema come la somma dell’energia cinetica delle sue parti, e definendo la potenza delle forze esterne/interne agenti sul sistema come la somma della potenza di tutte le forze esterne/interni al sistema,
\begin{equation*}
\begin{split}K :=  \sum_i K_i\end{split}
\end{equation*}\begin{equation*}
\begin{split}P^e := \sum_i P^{ext}_i = \sum_i \mathbf{v}_i \cdot  \mathbf{F}_i^{e} \end{split}
\end{equation*}\begin{equation*}
\begin{split}P^i := \sum_i P^{int}_i = \sum_i \mathbf{v}_i \cdot \sum_{j \ne i} \mathbf{F}_{ij}\end{split}
\end{equation*}
\sphinxAtStartPar
si ritrova la forma generale del bilancio dell’energia cinetica,
\begin{equation*}
\begin{split}\dot{K} = P^e + P^i = P^{tot} \ .\end{split}
\end{equation*}
\sphinxstepscope

\begin{sphinxuseclass}{sd-container-fluid}
\begin{sphinxuseclass}{sd-sphinx-override}
\begin{sphinxuseclass}{sd-p-0}
\begin{sphinxuseclass}{sd-mt-2}
\begin{sphinxuseclass}{sd-mb-4}
\begin{sphinxuseclass}{sd-row}
\begin{sphinxuseclass}{sd-row-cols-2}
\begin{sphinxuseclass}{sd-gx-2}
\begin{sphinxuseclass}{sd-gy-1}
\begin{sphinxuseclass}{sd-col}
\begin{sphinxuseclass}{sd-d-flex-row}
\begin{sphinxuseclass}{sd-align-minor-center}
\begin{sphinxuseclass}{sd-container-fluid}
\begin{sphinxuseclass}{sd-sphinx-override}
\begin{sphinxuseclass}{sd-row}
\begin{sphinxuseclass}{sd-row-cols-2}
\begin{sphinxuseclass}{sd-row-cols-xs-2}
\begin{sphinxuseclass}{sd-row-cols-sm-3}
\begin{sphinxuseclass}{sd-row-cols-md-3}
\begin{sphinxuseclass}{sd-row-cols-lg-3}
\begin{sphinxuseclass}{sd-gx-3}
\begin{sphinxuseclass}{sd-gy-1}
\begin{sphinxuseclass}{sd-col}
\begin{sphinxuseclass}{sd-col-auto}
\begin{sphinxuseclass}{sd-d-flex-row}
\begin{sphinxuseclass}{sd-align-minor-center}
\sphinxAtStartPar
basics

\end{sphinxuseclass}
\end{sphinxuseclass}
\end{sphinxuseclass}
\end{sphinxuseclass}
\begin{sphinxuseclass}{sd-col}
\begin{sphinxuseclass}{sd-col-auto}
\begin{sphinxuseclass}{sd-d-flex-row}
\begin{sphinxuseclass}{sd-align-minor-center}
\sphinxAtStartPar
Nov 09, 2024

\end{sphinxuseclass}
\end{sphinxuseclass}
\end{sphinxuseclass}
\end{sphinxuseclass}
\begin{sphinxuseclass}{sd-col}
\begin{sphinxuseclass}{sd-col-auto}
\begin{sphinxuseclass}{sd-d-flex-row}
\begin{sphinxuseclass}{sd-align-minor-center}
\sphinxAtStartPar
0 min read

\end{sphinxuseclass}
\end{sphinxuseclass}
\end{sphinxuseclass}
\end{sphinxuseclass}
\end{sphinxuseclass}
\end{sphinxuseclass}
\end{sphinxuseclass}
\end{sphinxuseclass}
\end{sphinxuseclass}
\end{sphinxuseclass}
\end{sphinxuseclass}
\end{sphinxuseclass}
\end{sphinxuseclass}
\end{sphinxuseclass}
\end{sphinxuseclass}
\end{sphinxuseclass}
\end{sphinxuseclass}
\end{sphinxuseclass}
\end{sphinxuseclass}
\end{sphinxuseclass}
\end{sphinxuseclass}
\end{sphinxuseclass}
\end{sphinxuseclass}
\end{sphinxuseclass}
\end{sphinxuseclass}
\end{sphinxuseclass}

\subsection{Equazioni cardinali della dinamica: corpo rigido}
\label{\detokenize{ch/dynamics-eom-rigid:equazioni-cardinali-della-dinamica-corpo-rigido}}\label{\detokenize{ch/dynamics-eom-rigid:classical-mechanics-dynamics-eom-rigid}}\label{\detokenize{ch/dynamics-eom-rigid::doc}}
\sphinxstepscope

\begin{sphinxuseclass}{sd-container-fluid}
\begin{sphinxuseclass}{sd-sphinx-override}
\begin{sphinxuseclass}{sd-p-0}
\begin{sphinxuseclass}{sd-mt-2}
\begin{sphinxuseclass}{sd-mb-4}
\begin{sphinxuseclass}{sd-row}
\begin{sphinxuseclass}{sd-row-cols-2}
\begin{sphinxuseclass}{sd-gx-2}
\begin{sphinxuseclass}{sd-gy-1}
\begin{sphinxuseclass}{sd-col}
\begin{sphinxuseclass}{sd-d-flex-row}
\begin{sphinxuseclass}{sd-align-minor-center}
\begin{sphinxuseclass}{sd-container-fluid}
\begin{sphinxuseclass}{sd-sphinx-override}
\begin{sphinxuseclass}{sd-row}
\begin{sphinxuseclass}{sd-row-cols-2}
\begin{sphinxuseclass}{sd-row-cols-xs-2}
\begin{sphinxuseclass}{sd-row-cols-sm-3}
\begin{sphinxuseclass}{sd-row-cols-md-3}
\begin{sphinxuseclass}{sd-row-cols-lg-3}
\begin{sphinxuseclass}{sd-gx-3}
\begin{sphinxuseclass}{sd-gy-1}
\begin{sphinxuseclass}{sd-col}
\begin{sphinxuseclass}{sd-col-auto}
\begin{sphinxuseclass}{sd-d-flex-row}
\begin{sphinxuseclass}{sd-align-minor-center}
\sphinxAtStartPar
basics

\end{sphinxuseclass}
\end{sphinxuseclass}
\end{sphinxuseclass}
\end{sphinxuseclass}
\begin{sphinxuseclass}{sd-col}
\begin{sphinxuseclass}{sd-col-auto}
\begin{sphinxuseclass}{sd-d-flex-row}
\begin{sphinxuseclass}{sd-align-minor-center}
\sphinxAtStartPar
Nov 09, 2024

\end{sphinxuseclass}
\end{sphinxuseclass}
\end{sphinxuseclass}
\end{sphinxuseclass}
\begin{sphinxuseclass}{sd-col}
\begin{sphinxuseclass}{sd-col-auto}
\begin{sphinxuseclass}{sd-d-flex-row}
\begin{sphinxuseclass}{sd-align-minor-center}
\sphinxAtStartPar
0 min read

\end{sphinxuseclass}
\end{sphinxuseclass}
\end{sphinxuseclass}
\end{sphinxuseclass}
\end{sphinxuseclass}
\end{sphinxuseclass}
\end{sphinxuseclass}
\end{sphinxuseclass}
\end{sphinxuseclass}
\end{sphinxuseclass}
\end{sphinxuseclass}
\end{sphinxuseclass}
\end{sphinxuseclass}
\end{sphinxuseclass}
\end{sphinxuseclass}
\end{sphinxuseclass}
\end{sphinxuseclass}
\end{sphinxuseclass}
\end{sphinxuseclass}
\end{sphinxuseclass}
\end{sphinxuseclass}
\end{sphinxuseclass}
\end{sphinxuseclass}
\end{sphinxuseclass}
\end{sphinxuseclass}
\end{sphinxuseclass}

\subsection{Equazioni cardinali della dinamica: mezzi continui}
\label{\detokenize{ch/dynamics-eom-continuum:equazioni-cardinali-della-dinamica-mezzi-continui}}\label{\detokenize{ch/dynamics-eom-continuum:classical-mechanics-dynamics-eom-continuum}}\label{\detokenize{ch/dynamics-eom-continuum::doc}}
\sphinxstepscope

\begin{sphinxuseclass}{sd-container-fluid}
\begin{sphinxuseclass}{sd-sphinx-override}
\begin{sphinxuseclass}{sd-p-0}
\begin{sphinxuseclass}{sd-mt-2}
\begin{sphinxuseclass}{sd-mb-4}
\begin{sphinxuseclass}{sd-row}
\begin{sphinxuseclass}{sd-row-cols-2}
\begin{sphinxuseclass}{sd-gx-2}
\begin{sphinxuseclass}{sd-gy-1}
\begin{sphinxuseclass}{sd-col}
\begin{sphinxuseclass}{sd-d-flex-row}
\begin{sphinxuseclass}{sd-align-minor-center}
\begin{sphinxuseclass}{sd-container-fluid}
\begin{sphinxuseclass}{sd-sphinx-override}
\begin{sphinxuseclass}{sd-row}
\begin{sphinxuseclass}{sd-row-cols-2}
\begin{sphinxuseclass}{sd-row-cols-xs-2}
\begin{sphinxuseclass}{sd-row-cols-sm-3}
\begin{sphinxuseclass}{sd-row-cols-md-3}
\begin{sphinxuseclass}{sd-row-cols-lg-3}
\begin{sphinxuseclass}{sd-gx-3}
\begin{sphinxuseclass}{sd-gy-1}
\begin{sphinxuseclass}{sd-col}
\begin{sphinxuseclass}{sd-col-auto}
\begin{sphinxuseclass}{sd-d-flex-row}
\begin{sphinxuseclass}{sd-align-minor-center}
\sphinxAtStartPar
basics

\end{sphinxuseclass}
\end{sphinxuseclass}
\end{sphinxuseclass}
\end{sphinxuseclass}
\begin{sphinxuseclass}{sd-col}
\begin{sphinxuseclass}{sd-col-auto}
\begin{sphinxuseclass}{sd-d-flex-row}
\begin{sphinxuseclass}{sd-align-minor-center}
\sphinxAtStartPar
Nov 09, 2024

\end{sphinxuseclass}
\end{sphinxuseclass}
\end{sphinxuseclass}
\end{sphinxuseclass}
\begin{sphinxuseclass}{sd-col}
\begin{sphinxuseclass}{sd-col-auto}
\begin{sphinxuseclass}{sd-d-flex-row}
\begin{sphinxuseclass}{sd-align-minor-center}
\sphinxAtStartPar
1 min read

\end{sphinxuseclass}
\end{sphinxuseclass}
\end{sphinxuseclass}
\end{sphinxuseclass}
\end{sphinxuseclass}
\end{sphinxuseclass}
\end{sphinxuseclass}
\end{sphinxuseclass}
\end{sphinxuseclass}
\end{sphinxuseclass}
\end{sphinxuseclass}
\end{sphinxuseclass}
\end{sphinxuseclass}
\end{sphinxuseclass}
\end{sphinxuseclass}
\end{sphinxuseclass}
\end{sphinxuseclass}
\end{sphinxuseclass}
\end{sphinxuseclass}
\end{sphinxuseclass}
\end{sphinxuseclass}
\end{sphinxuseclass}
\end{sphinxuseclass}
\end{sphinxuseclass}
\end{sphinxuseclass}
\end{sphinxuseclass}

\section{Moti particolari}
\label{\detokenize{ch/dynamics-motions:moti-particolari}}\label{\detokenize{ch/dynamics-motions:classical-mechanics-dynamics-motions}}\label{\detokenize{ch/dynamics-motions::doc}}
\sphinxAtStartPar
In questa sezione verranno studiati alcuni moti particolari, interessanti e utili da analizzare per motivi didattici, storici, e applicativi.
\begin{itemize}
\item {} 
\sphinxAtStartPar
moto rettilineo uniforme

\item {} 
\sphinxAtStartPar
moto uniformemente accelerato

\item {} 
\sphinxAtStartPar
moto circolare uniforme

\item {} 
\sphinxAtStartPar
moti oscillatori e oscillatori smorzati:
\begin{itemize}
\item {} 
\sphinxAtStartPar
oscillazioni libere:
\begin{itemize}
\item {} 
\sphinxAtStartPar
sistema massa\sphinxhyphen{}molla(\sphinxhyphen{}smorzatore)

\item {} 
\sphinxAtStartPar
pendolo

\end{itemize}

\item {} 
\sphinxAtStartPar
oscillazioni forzate:
\begin{itemize}
\item {} 
\sphinxAtStartPar
primo passo verso l’analisi strutturale e non solo (“ogni sistema fisico è un sistema di tanti oscillatori armonici”)

\item {} 
\sphinxAtStartPar
concetti di risposta in frequenza e risonanza. \sphinxstylestrong{todo} video e/o script su risposta in frequenza di strutture e strutture antisismiche, mass\sphinxhyphen{}damper,…

\end{itemize}

\end{itemize}

\item {} 
\sphinxAtStartPar
\sphinxstylestrong{Gravitazione}: partendo dalla legge di gravitazione universale fornita da Newton, si studia il moto dei corpi celesti in sistemi a due corpi, scoprendo che le loro traiettorie descrivono le sezioni coniche (cerchio, ellisse, parabola, iperbole), e dimostrando le leggi di Keplero

\item {} 
\sphinxAtStartPar
rotazione di un corpo attorno a un punto fisso, moti di Poinsot

\end{itemize}

\sphinxstepscope

\begin{sphinxuseclass}{sd-container-fluid}
\begin{sphinxuseclass}{sd-sphinx-override}
\begin{sphinxuseclass}{sd-p-0}
\begin{sphinxuseclass}{sd-mt-2}
\begin{sphinxuseclass}{sd-mb-4}
\begin{sphinxuseclass}{sd-row}
\begin{sphinxuseclass}{sd-row-cols-2}
\begin{sphinxuseclass}{sd-gx-2}
\begin{sphinxuseclass}{sd-gy-1}
\begin{sphinxuseclass}{sd-col}
\begin{sphinxuseclass}{sd-d-flex-row}
\begin{sphinxuseclass}{sd-align-minor-center}
\begin{sphinxuseclass}{sd-container-fluid}
\begin{sphinxuseclass}{sd-sphinx-override}
\begin{sphinxuseclass}{sd-row}
\begin{sphinxuseclass}{sd-row-cols-2}
\begin{sphinxuseclass}{sd-row-cols-xs-2}
\begin{sphinxuseclass}{sd-row-cols-sm-3}
\begin{sphinxuseclass}{sd-row-cols-md-3}
\begin{sphinxuseclass}{sd-row-cols-lg-3}
\begin{sphinxuseclass}{sd-gx-3}
\begin{sphinxuseclass}{sd-gy-1}
\begin{sphinxuseclass}{sd-col}
\begin{sphinxuseclass}{sd-col-auto}
\begin{sphinxuseclass}{sd-d-flex-row}
\begin{sphinxuseclass}{sd-align-minor-center}
\sphinxAtStartPar
basics

\end{sphinxuseclass}
\end{sphinxuseclass}
\end{sphinxuseclass}
\end{sphinxuseclass}
\begin{sphinxuseclass}{sd-col}
\begin{sphinxuseclass}{sd-col-auto}
\begin{sphinxuseclass}{sd-d-flex-row}
\begin{sphinxuseclass}{sd-align-minor-center}
\sphinxAtStartPar
Nov 09, 2024

\end{sphinxuseclass}
\end{sphinxuseclass}
\end{sphinxuseclass}
\end{sphinxuseclass}
\begin{sphinxuseclass}{sd-col}
\begin{sphinxuseclass}{sd-col-auto}
\begin{sphinxuseclass}{sd-d-flex-row}
\begin{sphinxuseclass}{sd-align-minor-center}
\sphinxAtStartPar
0 min read

\end{sphinxuseclass}
\end{sphinxuseclass}
\end{sphinxuseclass}
\end{sphinxuseclass}
\end{sphinxuseclass}
\end{sphinxuseclass}
\end{sphinxuseclass}
\end{sphinxuseclass}
\end{sphinxuseclass}
\end{sphinxuseclass}
\end{sphinxuseclass}
\end{sphinxuseclass}
\end{sphinxuseclass}
\end{sphinxuseclass}
\end{sphinxuseclass}
\end{sphinxuseclass}
\end{sphinxuseclass}
\end{sphinxuseclass}
\end{sphinxuseclass}
\end{sphinxuseclass}
\end{sphinxuseclass}
\end{sphinxuseclass}
\end{sphinxuseclass}
\end{sphinxuseclass}
\end{sphinxuseclass}
\end{sphinxuseclass}

\subsection{Gravitazione: problema dei due corpi}
\label{\detokenize{ch/dynamics-motions-gravitation-2bodies:gravitazione-problema-dei-due-corpi}}\label{\detokenize{ch/dynamics-motions-gravitation-2bodies:classical-mechanics-dynamics-motions-gravitation-2bodies}}\label{\detokenize{ch/dynamics-motions-gravitation-2bodies::doc}}
\sphinxstepscope


\part{Meccanica analitica}

\sphinxstepscope

\begin{sphinxuseclass}{sd-container-fluid}
\begin{sphinxuseclass}{sd-sphinx-override}
\begin{sphinxuseclass}{sd-p-0}
\begin{sphinxuseclass}{sd-mt-2}
\begin{sphinxuseclass}{sd-mb-4}
\begin{sphinxuseclass}{sd-row}
\begin{sphinxuseclass}{sd-row-cols-2}
\begin{sphinxuseclass}{sd-gx-2}
\begin{sphinxuseclass}{sd-gy-1}
\begin{sphinxuseclass}{sd-col}
\begin{sphinxuseclass}{sd-d-flex-row}
\begin{sphinxuseclass}{sd-align-minor-center}
\begin{sphinxuseclass}{sd-container-fluid}
\begin{sphinxuseclass}{sd-sphinx-override}
\begin{sphinxuseclass}{sd-row}
\begin{sphinxuseclass}{sd-row-cols-2}
\begin{sphinxuseclass}{sd-row-cols-xs-2}
\begin{sphinxuseclass}{sd-row-cols-sm-3}
\begin{sphinxuseclass}{sd-row-cols-md-3}
\begin{sphinxuseclass}{sd-row-cols-lg-3}
\begin{sphinxuseclass}{sd-gx-3}
\begin{sphinxuseclass}{sd-gy-1}
\begin{sphinxuseclass}{sd-col}
\begin{sphinxuseclass}{sd-col-auto}
\begin{sphinxuseclass}{sd-d-flex-row}
\begin{sphinxuseclass}{sd-align-minor-center}
\sphinxAtStartPar
basics

\end{sphinxuseclass}
\end{sphinxuseclass}
\end{sphinxuseclass}
\end{sphinxuseclass}
\begin{sphinxuseclass}{sd-col}
\begin{sphinxuseclass}{sd-col-auto}
\begin{sphinxuseclass}{sd-d-flex-row}
\begin{sphinxuseclass}{sd-align-minor-center}
\sphinxAtStartPar
Nov 09, 2024

\end{sphinxuseclass}
\end{sphinxuseclass}
\end{sphinxuseclass}
\end{sphinxuseclass}
\begin{sphinxuseclass}{sd-col}
\begin{sphinxuseclass}{sd-col-auto}
\begin{sphinxuseclass}{sd-d-flex-row}
\begin{sphinxuseclass}{sd-align-minor-center}
\sphinxAtStartPar
1 min read

\end{sphinxuseclass}
\end{sphinxuseclass}
\end{sphinxuseclass}
\end{sphinxuseclass}
\end{sphinxuseclass}
\end{sphinxuseclass}
\end{sphinxuseclass}
\end{sphinxuseclass}
\end{sphinxuseclass}
\end{sphinxuseclass}
\end{sphinxuseclass}
\end{sphinxuseclass}
\end{sphinxuseclass}
\end{sphinxuseclass}
\end{sphinxuseclass}
\end{sphinxuseclass}
\end{sphinxuseclass}
\end{sphinxuseclass}
\end{sphinxuseclass}
\end{sphinxuseclass}
\end{sphinxuseclass}
\end{sphinxuseclass}
\end{sphinxuseclass}
\end{sphinxuseclass}
\end{sphinxuseclass}
\end{sphinxuseclass}

\chapter{Meccanica analitica: formulazione di Lagrange}
\label{\detokenize{ch/lagrange:meccanica-analitica-formulazione-di-lagrange}}\label{\detokenize{ch/lagrange:classical-mechanics-lagrange}}\label{\detokenize{ch/lagrange::doc}}
\sphinxAtStartPar
Riformulazione della meccanica di Newton:
\begin{itemize}
\item {} 
\sphinxAtStartPar
\sphinxstylestrong{forma debole} delle equazioni: approccio di D’Alembert, lavori virtuali

\item {} 
\sphinxAtStartPar
\sphinxstylestrong{equazioni di Lagrange}

\end{itemize}
\begin{equation*}
\begin{split}\dfrac{d}{dt}\Big( \frac{\partial \mathscr{L}}{\partial \dot{q}} \Big) - \frac{\partial \mathscr{L}}{\partial q} = Q_q\end{split}
\end{equation*}
\sphinxAtStartPar
con \(\mathscr{L}(\dot{q}(t), q(t), t) = K(\dot{q}(t), q(t), t) + U(q(t), t)\).
\begin{itemize}
\item {} 
\sphinxAtStartPar
\sphinxstylestrong{principi di stazionarietà del funzionale azione \(S\)}

\end{itemize}

\sphinxAtStartPar
Nel caso non ci siano azioni non conservative, \(Q_q = 0\), è possibile interpretare le equazioni di Lagrange come risultato di un principio di stazionarietà. Chiamando \(\delta q(t)\) la variazione della funzion \(q(t)\), moltiplicandola per le equazioni di Lagrange e integrando in tempo per \(t \in [t_0, t_1]\),
\begin{equation*}
\begin{split}\begin{aligned}
  0 & = \int_{t_0}^{t_1} \delta q (t) \Big[ \dfrac{d}{dt}\Big( \frac{\partial \mathscr{L}}{\partial \dot{q}} \Big) - \frac{\partial \mathscr{L}}{\partial q} \Big] \, dt = \\
    & = \delta q(t) \Big( \frac{\partial \mathscr{L}}{\partial \dot{q}} \Big) \Big|_{t_0}^{t_1} - \int_{t_0}^{t_1} \Big[ \delta \dot{q}(t) \, \frac{\partial \mathscr{L}}{\partial \dot{q}} + \delta q(t) \, \frac{\partial \mathscr{L}}{\partial q} \Big] \, dt \ . \\
\end{aligned}\end{split}
\end{equation*}
\sphinxAtStartPar
Imponendo che la variazione \(\delta q(t)\) sia nulla per \(t_0\) e \(t_1\), il primo termine si annulla, e si può dimostrare che
\begin{equation*}
\begin{split}\begin{aligned}
    0 & = - \int_{t_0}^{t_1} \Big[ \delta \dot{q}(t) \, \frac{\partial \mathscr{L}}{\partial \dot{q}} + \delta q(t) \, \frac{\partial \mathscr{L}}{\partial q} \Big] \, dt \\
    & = - \delta \int_{t_0}^{t_1} \mathscr{L}(\dot{q}(t), q(t), t) \, dt =: - \delta S \ ,
\end{aligned}\end{split}
\end{equation*}
\sphinxAtStartPar
cioè le equazioni di Lagrange sono equivalenti alla condizione di stazionarietà del funzionale \sphinxstylestrong{azione}
\begin{equation*}
\begin{split}S:= \int_{t_0}^{t_1} \mathscr{L}(\dot{q}(t), q(t), t) \, dt \ .\end{split}
\end{equation*}
\sphinxstepscope

\begin{sphinxuseclass}{sd-container-fluid}
\begin{sphinxuseclass}{sd-sphinx-override}
\begin{sphinxuseclass}{sd-p-0}
\begin{sphinxuseclass}{sd-mt-2}
\begin{sphinxuseclass}{sd-mb-4}
\begin{sphinxuseclass}{sd-row}
\begin{sphinxuseclass}{sd-row-cols-2}
\begin{sphinxuseclass}{sd-gx-2}
\begin{sphinxuseclass}{sd-gy-1}
\begin{sphinxuseclass}{sd-col}
\begin{sphinxuseclass}{sd-d-flex-row}
\begin{sphinxuseclass}{sd-align-minor-center}
\begin{sphinxuseclass}{sd-container-fluid}
\begin{sphinxuseclass}{sd-sphinx-override}
\begin{sphinxuseclass}{sd-row}
\begin{sphinxuseclass}{sd-row-cols-2}
\begin{sphinxuseclass}{sd-row-cols-xs-2}
\begin{sphinxuseclass}{sd-row-cols-sm-3}
\begin{sphinxuseclass}{sd-row-cols-md-3}
\begin{sphinxuseclass}{sd-row-cols-lg-3}
\begin{sphinxuseclass}{sd-gx-3}
\begin{sphinxuseclass}{sd-gy-1}
\begin{sphinxuseclass}{sd-col}
\begin{sphinxuseclass}{sd-col-auto}
\begin{sphinxuseclass}{sd-d-flex-row}
\begin{sphinxuseclass}{sd-align-minor-center}
\sphinxAtStartPar
basics

\end{sphinxuseclass}
\end{sphinxuseclass}
\end{sphinxuseclass}
\end{sphinxuseclass}
\begin{sphinxuseclass}{sd-col}
\begin{sphinxuseclass}{sd-col-auto}
\begin{sphinxuseclass}{sd-d-flex-row}
\begin{sphinxuseclass}{sd-align-minor-center}
\sphinxAtStartPar
Nov 09, 2024

\end{sphinxuseclass}
\end{sphinxuseclass}
\end{sphinxuseclass}
\end{sphinxuseclass}
\begin{sphinxuseclass}{sd-col}
\begin{sphinxuseclass}{sd-col-auto}
\begin{sphinxuseclass}{sd-d-flex-row}
\begin{sphinxuseclass}{sd-align-minor-center}
\sphinxAtStartPar
1 min read

\end{sphinxuseclass}
\end{sphinxuseclass}
\end{sphinxuseclass}
\end{sphinxuseclass}
\end{sphinxuseclass}
\end{sphinxuseclass}
\end{sphinxuseclass}
\end{sphinxuseclass}
\end{sphinxuseclass}
\end{sphinxuseclass}
\end{sphinxuseclass}
\end{sphinxuseclass}
\end{sphinxuseclass}
\end{sphinxuseclass}
\end{sphinxuseclass}
\end{sphinxuseclass}
\end{sphinxuseclass}
\end{sphinxuseclass}
\end{sphinxuseclass}
\end{sphinxuseclass}
\end{sphinxuseclass}
\end{sphinxuseclass}
\end{sphinxuseclass}
\end{sphinxuseclass}
\end{sphinxuseclass}
\end{sphinxuseclass}

\section{Punto materiale}
\label{\detokenize{ch/lagrange-point:punto-materiale}}\label{\detokenize{ch/lagrange-point:classical-mechanics-lagrange-point}}\label{\detokenize{ch/lagrange-point::doc}}
\sphinxAtStartPar
\sphinxstylestrong{Equazione in forma forte.} Dalla meccanica di Newton, l’equazione del moto per un punto materiale \(P\) può essere scritta come
\begin{equation*}
\begin{split}m \dot{\mathbf{v}}_P = \mathbf{R}^e \ ,\end{split}
\end{equation*}
\sphinxAtStartPar
avendo indicato con \(m\) la massa del sistema, \(\mathbf{v}_P\) la velocità del punto, \(\mathbf{a}_P = \dot{\mathbf{v}}_P\) la sua accelerazione e \(\mathbf{R}^{e}\) la risultante delle forze esterne agenti sul sistema.

\sphinxAtStartPar
\sphinxstylestrong{Equazione in forma debole.} La forma debole dell’equazione si ottiene moltiplicando scalarmente l’equazione del moto per una funzione arbitraria \(\mathbf{w}\)
\begin{equation*}
\begin{split}\mathbf{0} = \mathbf{w} \cdot \left( m \dot{\mathbf{v}} - \mathbf{R}^e\right)  \qquad \forall \mathbf{w}\end{split}
\end{equation*}
\sphinxAtStartPar
\sphinxstylestrong{Formulazione della meccanica di Lagrange.} La formulazione di Lagrange della meccanica si ottiene scrivendo la posizione del punto materiale come funzione delle coordinate generalizzate \(q(t)\) e del tempo \(t\),
\begin{equation*}
\begin{split}\mathbf{r}_P(t) = \mathbf{r}(q_k(t),t) \ .\end{split}
\end{equation*}
\sphinxAtStartPar
Di conseguenza, usando la definizione di velocità e la regola di derivazione di funzioni composte, si può ottenere un’espressione della velocità in funzione del tempo, delle coordinate generalizzate e della loro derivata prima rispetto al tempo,
\begin{equation*}
\begin{split}\mathbf{v}_P(t) := \frac{d\mathbf{r}_P}{dt} = \dot{q}_k(t) \underbrace{\frac{\partial \mathbf{r}}{\partial q_k}}_{\frac{\partial \mathbf{v}}{\partial \dot{q}_k}} + \frac{\partial \mathbf{r}}{\partial t} = \mathbf{v}\left(\dot{q}_k(t), q_k(t), t \right) \ ,\end{split}
\end{equation*}
\sphinxAtStartPar
grazie alla quale si può riconoscere la relazione
\begin{equation}\label{equation:ch/lagrange-point:classical-mechanics:lagrange:point:mixed-der}
\begin{split}\dfrac{\partial \mathbf{r}}{\partial q_k} = \dfrac{\partial \mathbf{v}}{\partial \dot{q}_k} \ .\end{split}
\end{equation}
\sphinxAtStartPar
Scegliendo come funzione test \(\mathbf{w} = \dfrac{\partial \mathbf{r}}{\partial q_k} = \dfrac{\partial \mathbf{v}}{\partial \dot{q}_k}\), applicando la regola di derivazione del prodotto e usando il teorema di Schwartz per invertire l’ordine delle derivate e la relazione \eqref{equation:ch/lagrange-point:classical-mechanics:lagrange:point:mixed-der}, si può manipolare l’equazione in forma debole
\begin{equation*}
\begin{split}\begin{aligned}
\mathbf{0} & = \frac{\partial \mathbf{v}}{\partial \dot{q}_k} \cdot \left( m \dot{\mathbf{v}} - \mathbf{R}^e \right) = \\
& = \frac{d}{dt} \left( \frac{\partial \mathbf{v}}{\partial \dot{q}_k} \cdot m \mathbf{v} \right) - \frac{d}{dt} \frac{\partial \mathbf{r}}{\partial q_k} \cdot m \mathbf{v} - \frac{\partial \mathbf{r}}{\partial q_k} \cdot ( \mathbf{R}^{e,c} + \mathbf{R}^{e,nc} ) \\
& = \frac{d}{dt} \left( \frac{\partial \mathbf{v}}{\partial \dot{q}_k} \cdot m \mathbf{v} \right) - \frac{\partial \mathbf{v}}{\partial q_k} \cdot m \mathbf{v} - \frac{\partial \mathbf{r}}{\partial q_k} \cdot ( \nabla U + \mathbf{R}^{e,nc} ) \\
& = \frac{d}{dt} \left( \frac{\partial K}{\partial \dot{q}_k} \right) - \frac{\partial K}{\partial q_k} - \frac{\partial U}{\partial q_k} - \underbrace{\frac{\partial \mathbf{r}}{\partial q_k} \cdot \mathbf{R}^{e,nc}}_{=: Q_k} \ . \\
\end{aligned}\end{split}
\end{equation*}
\sphinxAtStartPar
Introducendo la funzione lagrangiana, \(\mathscr{L}(\dot{q}_k(t), q_k(t), t) := K(\dot{q}_k(t), q_k(t), t) + U(q_k(t),t)\), e ricordando che la funzione potenziale non dipende dalle velocità e quindi dalle derivate delle coordinate generalizzate \(\dot{q}_k\), si possono raccogliere i termini e scrivere le equazioni di Lagrange come
\begin{equation*}
\begin{split}\frac{d}{dt}\left(\frac{\partial \mathscr{L}}{\partial \dot{q}_k} \right) - \frac{\partial \mathscr{L}}{\partial q_k} = Q_k \ ,\end{split}
\end{equation*}
\sphinxAtStartPar
essendo \(Q_k\) la \sphinxstylestrong{forza generalizzata non conservativa}, \(Q_k = \dfrac{\partial \mathbf{r}}{\partial q_k} \cdot \mathbf{R}^{e,nc}\).

\sphinxstepscope

\begin{sphinxuseclass}{sd-container-fluid}
\begin{sphinxuseclass}{sd-sphinx-override}
\begin{sphinxuseclass}{sd-p-0}
\begin{sphinxuseclass}{sd-mt-2}
\begin{sphinxuseclass}{sd-mb-4}
\begin{sphinxuseclass}{sd-row}
\begin{sphinxuseclass}{sd-row-cols-2}
\begin{sphinxuseclass}{sd-gx-2}
\begin{sphinxuseclass}{sd-gy-1}
\begin{sphinxuseclass}{sd-col}
\begin{sphinxuseclass}{sd-d-flex-row}
\begin{sphinxuseclass}{sd-align-minor-center}
\begin{sphinxuseclass}{sd-container-fluid}
\begin{sphinxuseclass}{sd-sphinx-override}
\begin{sphinxuseclass}{sd-row}
\begin{sphinxuseclass}{sd-row-cols-2}
\begin{sphinxuseclass}{sd-row-cols-xs-2}
\begin{sphinxuseclass}{sd-row-cols-sm-3}
\begin{sphinxuseclass}{sd-row-cols-md-3}
\begin{sphinxuseclass}{sd-row-cols-lg-3}
\begin{sphinxuseclass}{sd-gx-3}
\begin{sphinxuseclass}{sd-gy-1}
\begin{sphinxuseclass}{sd-col}
\begin{sphinxuseclass}{sd-col-auto}
\begin{sphinxuseclass}{sd-d-flex-row}
\begin{sphinxuseclass}{sd-align-minor-center}
\sphinxAtStartPar
basics

\end{sphinxuseclass}
\end{sphinxuseclass}
\end{sphinxuseclass}
\end{sphinxuseclass}
\begin{sphinxuseclass}{sd-col}
\begin{sphinxuseclass}{sd-col-auto}
\begin{sphinxuseclass}{sd-d-flex-row}
\begin{sphinxuseclass}{sd-align-minor-center}
\sphinxAtStartPar
Nov 09, 2024

\end{sphinxuseclass}
\end{sphinxuseclass}
\end{sphinxuseclass}
\end{sphinxuseclass}
\begin{sphinxuseclass}{sd-col}
\begin{sphinxuseclass}{sd-col-auto}
\begin{sphinxuseclass}{sd-d-flex-row}
\begin{sphinxuseclass}{sd-align-minor-center}
\sphinxAtStartPar
0 min read

\end{sphinxuseclass}
\end{sphinxuseclass}
\end{sphinxuseclass}
\end{sphinxuseclass}
\end{sphinxuseclass}
\end{sphinxuseclass}
\end{sphinxuseclass}
\end{sphinxuseclass}
\end{sphinxuseclass}
\end{sphinxuseclass}
\end{sphinxuseclass}
\end{sphinxuseclass}
\end{sphinxuseclass}
\end{sphinxuseclass}
\end{sphinxuseclass}
\end{sphinxuseclass}
\end{sphinxuseclass}
\end{sphinxuseclass}
\end{sphinxuseclass}
\end{sphinxuseclass}
\end{sphinxuseclass}
\end{sphinxuseclass}
\end{sphinxuseclass}
\end{sphinxuseclass}
\end{sphinxuseclass}
\end{sphinxuseclass}

\section{Sistemi di punti}
\label{\detokenize{ch/lagrange-points:sistemi-di-punti}}\label{\detokenize{ch/lagrange-points:classical-mechanics-lagrange-points}}\label{\detokenize{ch/lagrange-points::doc}}
\sphinxstepscope

\begin{sphinxuseclass}{sd-container-fluid}
\begin{sphinxuseclass}{sd-sphinx-override}
\begin{sphinxuseclass}{sd-p-0}
\begin{sphinxuseclass}{sd-mt-2}
\begin{sphinxuseclass}{sd-mb-4}
\begin{sphinxuseclass}{sd-row}
\begin{sphinxuseclass}{sd-row-cols-2}
\begin{sphinxuseclass}{sd-gx-2}
\begin{sphinxuseclass}{sd-gy-1}
\begin{sphinxuseclass}{sd-col}
\begin{sphinxuseclass}{sd-d-flex-row}
\begin{sphinxuseclass}{sd-align-minor-center}
\begin{sphinxuseclass}{sd-container-fluid}
\begin{sphinxuseclass}{sd-sphinx-override}
\begin{sphinxuseclass}{sd-row}
\begin{sphinxuseclass}{sd-row-cols-2}
\begin{sphinxuseclass}{sd-row-cols-xs-2}
\begin{sphinxuseclass}{sd-row-cols-sm-3}
\begin{sphinxuseclass}{sd-row-cols-md-3}
\begin{sphinxuseclass}{sd-row-cols-lg-3}
\begin{sphinxuseclass}{sd-gx-3}
\begin{sphinxuseclass}{sd-gy-1}
\begin{sphinxuseclass}{sd-col}
\begin{sphinxuseclass}{sd-col-auto}
\begin{sphinxuseclass}{sd-d-flex-row}
\begin{sphinxuseclass}{sd-align-minor-center}
\sphinxAtStartPar
basics

\end{sphinxuseclass}
\end{sphinxuseclass}
\end{sphinxuseclass}
\end{sphinxuseclass}
\begin{sphinxuseclass}{sd-col}
\begin{sphinxuseclass}{sd-col-auto}
\begin{sphinxuseclass}{sd-d-flex-row}
\begin{sphinxuseclass}{sd-align-minor-center}
\sphinxAtStartPar
Nov 09, 2024

\end{sphinxuseclass}
\end{sphinxuseclass}
\end{sphinxuseclass}
\end{sphinxuseclass}
\begin{sphinxuseclass}{sd-col}
\begin{sphinxuseclass}{sd-col-auto}
\begin{sphinxuseclass}{sd-d-flex-row}
\begin{sphinxuseclass}{sd-align-minor-center}
\sphinxAtStartPar
0 min read

\end{sphinxuseclass}
\end{sphinxuseclass}
\end{sphinxuseclass}
\end{sphinxuseclass}
\end{sphinxuseclass}
\end{sphinxuseclass}
\end{sphinxuseclass}
\end{sphinxuseclass}
\end{sphinxuseclass}
\end{sphinxuseclass}
\end{sphinxuseclass}
\end{sphinxuseclass}
\end{sphinxuseclass}
\end{sphinxuseclass}
\end{sphinxuseclass}
\end{sphinxuseclass}
\end{sphinxuseclass}
\end{sphinxuseclass}
\end{sphinxuseclass}
\end{sphinxuseclass}
\end{sphinxuseclass}
\end{sphinxuseclass}
\end{sphinxuseclass}
\end{sphinxuseclass}
\end{sphinxuseclass}
\end{sphinxuseclass}

\section{Corpo rigido}
\label{\detokenize{ch/lagrange-rigid-body:corpo-rigido}}\label{\detokenize{ch/lagrange-rigid-body:classical-mechanics-lagrange-rigid}}\label{\detokenize{ch/lagrange-rigid-body::doc}}
\sphinxstepscope

\begin{sphinxuseclass}{sd-container-fluid}
\begin{sphinxuseclass}{sd-sphinx-override}
\begin{sphinxuseclass}{sd-p-0}
\begin{sphinxuseclass}{sd-mt-2}
\begin{sphinxuseclass}{sd-mb-4}
\begin{sphinxuseclass}{sd-row}
\begin{sphinxuseclass}{sd-row-cols-2}
\begin{sphinxuseclass}{sd-gx-2}
\begin{sphinxuseclass}{sd-gy-1}
\begin{sphinxuseclass}{sd-col}
\begin{sphinxuseclass}{sd-d-flex-row}
\begin{sphinxuseclass}{sd-align-minor-center}
\begin{sphinxuseclass}{sd-container-fluid}
\begin{sphinxuseclass}{sd-sphinx-override}
\begin{sphinxuseclass}{sd-row}
\begin{sphinxuseclass}{sd-row-cols-2}
\begin{sphinxuseclass}{sd-row-cols-xs-2}
\begin{sphinxuseclass}{sd-row-cols-sm-3}
\begin{sphinxuseclass}{sd-row-cols-md-3}
\begin{sphinxuseclass}{sd-row-cols-lg-3}
\begin{sphinxuseclass}{sd-gx-3}
\begin{sphinxuseclass}{sd-gy-1}
\begin{sphinxuseclass}{sd-col}
\begin{sphinxuseclass}{sd-col-auto}
\begin{sphinxuseclass}{sd-d-flex-row}
\begin{sphinxuseclass}{sd-align-minor-center}
\sphinxAtStartPar
basics

\end{sphinxuseclass}
\end{sphinxuseclass}
\end{sphinxuseclass}
\end{sphinxuseclass}
\begin{sphinxuseclass}{sd-col}
\begin{sphinxuseclass}{sd-col-auto}
\begin{sphinxuseclass}{sd-d-flex-row}
\begin{sphinxuseclass}{sd-align-minor-center}
\sphinxAtStartPar
Nov 09, 2024

\end{sphinxuseclass}
\end{sphinxuseclass}
\end{sphinxuseclass}
\end{sphinxuseclass}
\begin{sphinxuseclass}{sd-col}
\begin{sphinxuseclass}{sd-col-auto}
\begin{sphinxuseclass}{sd-d-flex-row}
\begin{sphinxuseclass}{sd-align-minor-center}
\sphinxAtStartPar
0 min read

\end{sphinxuseclass}
\end{sphinxuseclass}
\end{sphinxuseclass}
\end{sphinxuseclass}
\end{sphinxuseclass}
\end{sphinxuseclass}
\end{sphinxuseclass}
\end{sphinxuseclass}
\end{sphinxuseclass}
\end{sphinxuseclass}
\end{sphinxuseclass}
\end{sphinxuseclass}
\end{sphinxuseclass}
\end{sphinxuseclass}
\end{sphinxuseclass}
\end{sphinxuseclass}
\end{sphinxuseclass}
\end{sphinxuseclass}
\end{sphinxuseclass}
\end{sphinxuseclass}
\end{sphinxuseclass}
\end{sphinxuseclass}
\end{sphinxuseclass}
\end{sphinxuseclass}
\end{sphinxuseclass}
\end{sphinxuseclass}

\chapter{Meccanica analitica: formulazione di Hamilton}
\label{\detokenize{ch/hamilton:meccanica-analitica-formulazione-di-hamilton}}\label{\detokenize{ch/hamilton:classical-mechanics-hamilton}}\label{\detokenize{ch/hamilton::doc}}
\sphinxAtStartPar
Riformulazione ulteriore della meccanica di Newton, a partire dalla meccanica di Lagrange.
Fornisce le basi per un approccio moderno anche in altre teorie della Fisica. \sphinxstylestrong{dots…}

\sphinxAtStartPar
Partendo dalle equazioni di Lagrange,
\begin{equation*}
\begin{split}\dfrac{d}{dt}\Big( \frac{\partial \mathscr{L}}{\partial \dot{q}} \Big) - \frac{\partial \mathscr{L}}{\partial q} = Q_q\end{split}
\end{equation*}
\sphinxAtStartPar
si definisce il \sphinxstylestrong{momento generalizzato} (dall’inglese, momentum = quantità di moto)
\begin{equation*}
\begin{split}p := \frac{\partial \mathscr{L}}{\partial \dot{q}} \ ,\end{split}
\end{equation*}
\sphinxAtStartPar
e la funzione \sphinxstylestrong{hamiltoniana}
\begin{equation*}
\begin{split}H(q(t), p(t), t) := p_k \dot{q}^k - \mathscr{L} \ .\end{split}
\end{equation*}\begin{equation*}
\begin{split}\begin{aligned}
dH & = dq^k \, \frac{\partial H}{\partial q^k} + dp_k \, \frac{\partial H}{\partial p_k} + dt \,  \frac{\partial H}{\partial t} = \\
& = d p_k \, \dot{q}^k + \underbrace{ p_k \, d \dot{q}^k - d \dot{q}^k \, \frac{\partial \mathscr{L}}{\partial \dot{q}^k}}_{=0} - d q^k \, \frac{\partial \mathscr{L}}{\partial q^k} - dt \, \frac{\partial \mathscr{L}}{\partial t}
\end{aligned}\end{split}
\end{equation*}
\sphinxAtStartPar
e quindi segue
\begin{equation*}
\begin{split}\begin{cases}
 \dot{q}^k & = \dfrac{\partial H}{\partial p_k} \\
 \dfrac{\partial \mathscr{H}}{\partial q^k} & = - \dfrac{\partial \mathscr{L}}{\partial q^k} \\
 \dfrac{\partial H}{\partial t} & = - \dfrac{\partial\mathscr{L}}{\partial t} \ .
\end{cases}\end{split}
\end{equation*}
\sphinxAtStartPar
Riscrivendo le equazioni di Lagrange come
\begin{equation*}
\begin{split}\frac{\partial \mathscr{L}}{\partial q^k} = - Q_{q^k} + \dfrac{d}{dt}\Big( \frac{\partial \mathscr{L}}{\partial \dot{q}^k} \Big) = -Q_{q^k} + \dot{p}_k\end{split}
\end{equation*}
\sphinxAtStartPar
si ottengono le \sphinxstylestrong{equazioni di Hamilton}
\begin{equation*}
\begin{split}\begin{cases}
 \dot{q}^k & = \dfrac{\partial H}{\partial p_k} \\
 \dot{p}_k & =-\dfrac{\partial H}{\partial q^k} + Q_{q^k} \\
\end{cases}\end{split}
\end{equation*}






\renewcommand{\indexname}{Index}
\printindex
\end{document}