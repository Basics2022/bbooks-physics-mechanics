%% Generated by Sphinx.
\def\sphinxdocclass{jupyterBook}
\documentclass[letterpaper,10pt,english]{jupyterBook}
\ifdefined\pdfpxdimen
   \let\sphinxpxdimen\pdfpxdimen\else\newdimen\sphinxpxdimen
\fi \sphinxpxdimen=.75bp\relax
\ifdefined\pdfimageresolution
    \pdfimageresolution= \numexpr \dimexpr1in\relax/\sphinxpxdimen\relax
\fi
%% let collapsible pdf bookmarks panel have high depth per default
\PassOptionsToPackage{bookmarksdepth=5}{hyperref}
%% turn off hyperref patch of \index as sphinx.xdy xindy module takes care of
%% suitable \hyperpage mark-up, working around hyperref-xindy incompatibility
\PassOptionsToPackage{hyperindex=false}{hyperref}
%% memoir class requires extra handling
\makeatletter\@ifclassloaded{memoir}
{\ifdefined\memhyperindexfalse\memhyperindexfalse\fi}{}\makeatother

\PassOptionsToPackage{warn}{textcomp}

\catcode`^^^^00a0\active\protected\def^^^^00a0{\leavevmode\nobreak\ }
\usepackage{cmap}
\usepackage{fontspec}
\defaultfontfeatures[\rmfamily,\sffamily,\ttfamily]{}
\usepackage{amsmath,amssymb,amstext}
\usepackage{polyglossia}
\setmainlanguage{english}



\setmainfont{FreeSerif}[
  Extension      = .otf,
  UprightFont    = *,
  ItalicFont     = *Italic,
  BoldFont       = *Bold,
  BoldItalicFont = *BoldItalic
]
\setsansfont{FreeSans}[
  Extension      = .otf,
  UprightFont    = *,
  ItalicFont     = *Oblique,
  BoldFont       = *Bold,
  BoldItalicFont = *BoldOblique,
]
\setmonofont{FreeMono}[
  Extension      = .otf,
  UprightFont    = *,
  ItalicFont     = *Oblique,
  BoldFont       = *Bold,
  BoldItalicFont = *BoldOblique,
]



\usepackage[Bjarne]{fncychap}
\usepackage[,numfigreset=1,mathnumfig]{sphinx}

\fvset{fontsize=\small}
\usepackage{geometry}


% Include hyperref last.
\usepackage{hyperref}
% Fix anchor placement for figures with captions.
\usepackage{hypcap}% it must be loaded after hyperref.
% Set up styles of URL: it should be placed after hyperref.
\urlstyle{same}

\addto\captionsenglish{\renewcommand{\contentsname}{Newton Mechanics}}

\usepackage{sphinxmessages}



        % Start of preamble defined in sphinx-jupyterbook-latex %
         \usepackage[Latin,Greek]{ucharclasses}
        \usepackage{unicode-math}
        % fixing title of the toc
        \addto\captionsenglish{\renewcommand{\contentsname}{Contents}}
        \hypersetup{
            pdfencoding=auto,
            psdextra
        }
        % End of preamble defined in sphinx-jupyterbook-latex %
        

\title{Classical Mechanics}
\date{May 08, 2025}
\release{}
\author{basics}
\newcommand{\sphinxlogo}{\vbox{}}
\renewcommand{\releasename}{}
\makeindex
\begin{document}

\pagestyle{empty}
\sphinxmaketitle
\pagestyle{plain}
\sphinxtableofcontents
\pagestyle{normal}
\phantomsection\label{\detokenize{intro::doc}}


\sphinxAtStartPar
This material is part of the \sphinxhref{https://basics2022.github.io/bbooks}{\sphinxstylestrong{basics\sphinxhyphen{}books project}}. It is also available as a \DUrole{xref,myst}{.pdf document}.

\sphinxAtStartPar
Classical mechanics deal with the motion of systems and its causes.

\sphinxAtStartPar
Different formulations of mechanics are available. Newton formulation was developed at the end of XVII century and starts from mass conservation and Newton’s three principles of dynamics, summarised in invariance under Galileian transformations, the relation between force and the change of momentum of a system, and action/reaction principle. Analytical mechanics was developed in the following centuries by D’Alembert and Lagrange and starts from variational principles, leading to Lagrange or Hamiltonian equations of motion.

\sphinxAtStartPar
\sphinxstylestrong{Newton Mechanics.}
\subsubsection*{Kinematics}
\subsubsection*{Actions}
\subsubsection*{Inertia}
\subsubsection*{Dynamics}

\sphinxAtStartPar
\sphinxstylestrong{Analytical Mechanics.}
\subsubsection*{Lagrangian Mechanics}
\subsubsection*{Hamiltonian Mechanics}

\sphinxAtStartPar
Classical mechanics provides a reliable and useful theory for systems:
\begin{itemize}
\item {} 
\sphinxAtStartPar
much larger than atomic scales; at atomic scales, \sphinxhref{https://basics2022.github.io/bbooks-physics-modern/ch/quantum-mechanics/intro.html}{quantum mechanics} is needed

\item {} 
\sphinxAtStartPar
with velocity much slower than the speed of light or in domains where the finite value of finite speed of interactions can be neglected, as classical mechanics relies on istantaneous action at distance; if these assumptions fail, Einstein theory is needed either \sphinxhref{https://basics2022.github.io/bbooks-physics-modern/ch/relativity-special/intro.html}{special relativity} \sphinxhyphen{} as a consistent theory of mechanics and \sphinxhref{https://basics2022.github.io/bbooks-physics-electromagnetism/intro.html}{electromagnetism} \sphinxhyphen{} or \sphinxhref{https://basics2022.github.io/bbooks-physics-modern/ch/relativity-general/intro.html}{general relativity} \sphinxhyphen{} as a theory of gravitation.

\item {} 
\sphinxAtStartPar
with a small number of components, so that the integration of the governing equations of motion is feasible; continuous model of the systems are object of classical continuum mechanics, relying on the equations of classical mechanics and thermodynamics; systems with large number of components can be approached with the techniques developed in \sphinxhref{https://basics2022.github.io/bbooks-physics-modern/ch/statistical-mechanics/intro.html}{statistical mechanics}.

\end{itemize}

\sphinxAtStartPar
Under these assumptions, mass conservation (Lavoisier principle) holds, inertial reference frames are related by Galileian relativity and the equations of motions are deterministic and can be solved with a reasonable effort \sphinxhyphen{} compared to the information and detail contained in the results \sphinxhyphen{} either analytically or numerically.
Classical mechanics treats time and space as individually absolute physical quantities: this can be a good model whenever Einstein relativity effects are negligible.





\sphinxstepscope


\part{Newton Mechanics}

\sphinxstepscope


\chapter{Kinematics}
\label{\detokenize{ch/kinematics:kinematics}}\label{\detokenize{ch/kinematics:classical-mechanics-kinematics}}\label{\detokenize{ch/kinematics::doc}}
\sphinxAtStartPar
Kinematics deals with the motion of mechanical systems, without taking into account the causes of motion.



\sphinxAtStartPar
Classical mechanics relies on the concepts of \sphinxstylestrong{absolute 3\sphinxhyphen{}dimensional Euclidean space}, \(E^3\), and \sphinxstylestrong{absolute time}.

\sphinxAtStartPar
\sphinxstylestrong{Space and time.} Briefly, what is space? It’s something you can measure with ruler (for distances) and square (for angles), or other space\sphinxhyphen{}measurment devices. Newton mechanics relies on space modeled as \sphinxstylestrong{Euclidean space}, a physical entity where the Euclidean geometry holds.
What is time? It’s something you can measure with a clock or other timekeeping devices, that can be related to order of events, and causality (cause comes before consequences).



\sphinxAtStartPar
\sphinxstylestrong{Models.} Different models of physical systems can be derived with an abstraction and modelling process, depending on the characteristics of the system under investigation and on the level of detail required by the analysis.

\sphinxAtStartPar
These models can be classified by:
\begin{itemize}
\item {} 
\sphinxAtStartPar
dimensions: 0: point; 1: line; 2: surfaces; 3: volume solid

\item {} 
\sphinxAtStartPar
deformation: deformable or rigid components

\end{itemize}

\sphinxAtStartPar
A system can be composed of several components, either free or connected with constraints.

\sphinxAtStartPar
Here, the focus goes to the kinematics of {\hyperref[\detokenize{ch/kinematics-point:classical-mechanics-kinematics-point}]{\sphinxcrossref{\DUrole{std,std-ref}{points}}}} and {\hyperref[\detokenize{ch/kinematics-rigid:classical-mechanics-kinematics-rigid-body}]{\sphinxcrossref{\DUrole{std,std-ref}{rigid bodies}}}}, while deformable bodies are described in \sphinxhref{https://basics2022.github.io/bbooks-physics-continuum-mechanics/intro.html}{continuous mechanics} \sphinxhyphen{} \sphinxhref{https://basics2022.github.io/bbooks-physics-continuum-mechanics/ch/continuum/kinematics.html}{kinematics}.

\sphinxAtStartPar
While space and time are absolute, the motion of a system is usually the \sphinxstylestrong{motion relative to an observer} or to a reference frame. After treating the kinematics of points and rigid bodies w.r.t. a given reference frame, {\hyperref[\detokenize{ch/kinematics-relative:classical-mechanics-kinematics-relative}]{\sphinxcrossref{\DUrole{std,std-ref}{relative kinematics}}}} provides the description of the motion of the same system w.r.t. 2 different observers/reference frames in relative motion.
\label{ch/kinematics:definition-0}
\begin{sphinxadmonition}{note}{Definition 1.1 (Configuration)}


\end{sphinxadmonition}
\label{ch/kinematics:definition-1}
\begin{sphinxadmonition}{note}{Definition 1.2 (State)}


\end{sphinxadmonition}

\sphinxstepscope


\section{Point}
\label{\detokenize{ch/kinematics-point:point}}\label{\detokenize{ch/kinematics-point:classical-mechanics-kinematics-point}}\label{\detokenize{ch/kinematics-point::doc}}
\sphinxAtStartPar
The configuration of a point system is determined by its position in space, its state by its position and its velocity. Acceleration is usually required in mechanics, since equations of motions may be recast as a system of second\sphinxhyphen{}orde ordinary differential equations in the configuration of the system. These physical quantites are defined here w.r.t. a reference frame \(O_o \hat{x}^0 \hat{y}^0 \hat{z}^0, t^0\), keeping constant the vectors of the base w.r.t. time \(t^0\).

\sphinxAtStartPar
\sphinxstylestrong{Position.}
\begin{equation*}
\begin{split}\vec{r}_P(t) = P - O_0 = x^0_{P,i}(t) \, \hat{e}^0_i\end{split}
\end{equation*}
\sphinxAtStartPar
\sphinxstylestrong{Velocity.}
\begin{equation*}
\begin{split}\vec{v}_P(t) = \dfrac{d \vec{r}_P}{dt} = \dot{x}^0_{P,i} \, \hat{e}^0_i\end{split}
\end{equation*}
\sphinxAtStartPar
\sphinxstylestrong{Acceleration.}
\begin{equation*}
\begin{split}\begin{aligned}
  \vec{a}_P(t)
  & = \dfrac{d \vec{v}_P}{dt}     = \dot{v}^0_{P,i} \, \hat{e}^0_i = \\
  & = \dfrac{d^2 \vec{r}_P}{dt^2} =\ddot{x}^0_{P,i} \, \hat{e}^0_i 
\end{aligned}\end{split}
\end{equation*}


\sphinxstepscope


\section{Rigid Body}
\label{\detokenize{ch/kinematics-rigid:rigid-body}}\label{\detokenize{ch/kinematics-rigid:classical-mechanics-kinematics-rigid-body}}\label{\detokenize{ch/kinematics-rigid::doc}}

\subsection{Rigid motion}
\label{\detokenize{ch/kinematics-rigid:rigid-motion}}
\sphinxAtStartPar
Rigid motion preserves distance between any pair of points, and thus angles. The motion of two material points \(P\), \(Q\) performing a rigid motion obeys
\begin{equation}\label{equation:ch/kinematics-rigid:eq:kin:rigid:vel}
\begin{split}\vec{v}_P - \vec{v}_Q = \vec{\omega} \times (P - Q) \ ,\end{split}
\end{equation}
\sphinxAtStartPar
being \(\vec{v}_P\), \(\vec{v}_Q\) the velocity of the points and \(\vec{\omega}\) the \sphinxstylestrong{angular velocity} of the rigid motion. Taking a point \(Q\) as the reference point of the motion, the velocity of all other points can be found
\begin{equation*}
\begin{split}\vec{v}_P = \vec{v}_Q + \vec{\omega} \times (P-Q) \ ,\end{split}
\end{equation*}
\sphinxAtStartPar
as a function of the velocity of \(Q\), the angular velocity of the rigid motion, and the relative position \(P-Q\).
\subsubsection*{Proof.}

\sphinxAtStartPar
Given 3 points \(P(t)\), \(Q(t)\), \(R(t)\), the distance bewteen each pair of points is constant and thus its time derivative zero,
\begin{equation*}
\begin{split}0 = \dfrac{d}{dt} |P(t)-Q(t)|^2 = 2 \left(P - Q\right) \cdot \left( \vec{v}_P - \vec{v}_Q \right) \quad \rightarrow \quad \Delta \vec{v}_{QP} = \vec{\omega}_{QP} \times \Delta \vec{r}_{QP}\end{split}
\end{equation*}\begin{equation*}
\begin{split}0 = \dfrac{d}{dt} |P(t)-R(t)|^2 = 2 \left(P - R\right) \cdot \left( \vec{v}_P - \vec{v}_R \right) \quad \rightarrow \quad \Delta \vec{v}_{RP} = \vec{\omega}_{RP} \times \Delta \vec{r}_{RP}\end{split}
\end{equation*}\begin{equation*}
\begin{split}\begin{aligned}
  0 = \dfrac{d}{dt} \left[ (P-Q) \cdot (P-R) \right] 
  & = \Delta \vec{v}_{QP} \cdot \Delta \vec{r}_{RP} + \Delta \vec{r}_{QP} \cdot \Delta \vec{v}_{RP} = \\
  & = \vec{\omega}_{QP} \times \Delta \vec{r}_{QP} \cdot \Delta \vec{r}_{RP} + \Delta \vec{r}_{QP} \cdot \Delta \vec{\omega}_{RP} \times \Delta \vec{r}_{RP} = \\
  & = \Delta \vec{r}_{QP} \times \Delta \vec{r}_{RP} \cdot ( \vec{\omega}_{QP} - \vec{\omega}_{RP} ) \ ,
\end{aligned}\end{split}
\end{equation*}
\sphinxAtStartPar
and since \(\Delta \vec{r}_{QP}\), \(\Delta \vec{r}_{RP}\) are arbitrary it follows that the vector \(\vec{\omega} = \vec{\omega}_{QP} = \vec{\omega}_{RP}\) is unique for all the points performing a rigid motion.

\sphinxAtStartPar
The configuration of a material vector \(\vec{a}\) undergoing a rotation is described by the product of the rotation tensor \(\mathbb{R}\) by the reference configuration \(\vec{a}^0\),
\begin{equation*}
\begin{split}
\vec{a} = \mathbb{R} \cdot \vec{a}^0 
\qquad , \qquad
\vec{b} = \mathbb{R} \cdot \vec{b}^0 \ .
\end{split}
\end{equation*}
\sphinxAtStartPar
In order to preserve distance, and angles
\begin{equation*}
\begin{split}\begin{cases}
 |\vec{a}|^2 = & \vec{a} \cdot \vec{a} = \vec{a}^0 \cdot \mathbb{R}^T \cdot \mathbb{R} \cdot \vec{a}^0 = \vec{a}^0 \cdot \vec{a}^0 = |\vec{a}|^2 \\
               & \vec{a} \cdot \vec{b} = \vec{a}^0 \cdot \mathbb{R}^T \cdot \mathbb{R} \cdot \vec{b}^0 = \vec{a}^0 \cdot \vec{a}^0
\end{cases}
\qquad \rightarrow \qquad \mathbb{R}^T \cdot \mathbb{R} = \mathbb{I}
\end{split}
\end{equation*}
\sphinxAtStartPar
the \sphinxstylestrong{rotation tensor} is \sphinxstylestrong{unitary}
\begin{equation}\label{equation:ch/kinematics-rigid:eq:rot:unitary}
\begin{split}\mathbb{I} = \mathbb{R}^T \cdot \mathbb{R} = \mathbb{R} \cdot \mathbb{R}^T\end{split}
\end{equation}
\begin{sphinxadmonition}{note}{Note:}
\sphinxAtStartPar
From relation \eqref{equation:ch/kinematics-rigid:eq:rot:unitary}, it follows that
\begin{equation*}
\begin{split}1 = |\mathbb{I}| = |\mathbb{R}^T| |\mathbb{R}| = |\mathbb{R}|^2 \ ,\end{split}
\end{equation*}
\sphinxAtStartPar
and thus \(|\mathbb{R}| = \mp 1\). If \(|\mathbb{R}| = 1\), \(\mathbb{R}\) represents a rotation, and implies conservation of orientation of space; if \(|\mathbb{R}| = -1\) represents a reflection w.r.t. a plane, and implies inversion of orientation of space.

\sphinxAtStartPar
Orientation of space is determined by the transformation of a RHS triad of vectors: if the transform triad is RHS, then orientation of space is preserved; if it becomes LHS, then orientation of space is inverted.
\end{sphinxadmonition}

\begin{sphinxadmonition}{note}{Note:}
\sphinxAtStartPar
Rotation tensor \(\mathbb{R}\) is not singular and its determinant equals \(|\mathbb{R}| = 1\). Thus, \(\mathbb{R}^T \cdot \mathbb{R} = \mathbb{I}\) implies \(\mathbb{R} \cdot \mathbb{R}^T = \mathbb{I}\). Multiplying \eqref{equation:ch/kinematics-rigid:eq:rot:unitary} by \(\mathbb{R}\) on the left
\begin{equation*}
\begin{split}\mathbb{0} = \mathbb{R} \cdot \mathbb{R}^T \cdot \mathbb{R} - \mathbb{R} \cdot \mathbb{I} = (\mathbb{R} \cdot \mathbb{R}^T - \mathbb{I}) \cdot  \mathbb{R} \ ,\end{split}
\end{equation*}
\sphinxAtStartPar
and since \(\mathbb{R}\) is non\sphinxhyphen{}singular, it follows that \(\mathbb{R} \cdot \mathbb{R}^T = \mathbb{I}\).
\end{sphinxadmonition}

\sphinxAtStartPar
Time derivative of the relation \eqref{equation:ch/kinematics-rigid:eq:rot:unitary} reads
\begin{equation*}
\begin{split}\mathbb{0} = \dfrac{d}{dt} \left( \mathbb{R} \cdot \mathbb{R}^T \right) = \dot{\mathbb{R}} \cdot \mathbb{R}^T + \mathbb{R} \cdot \dot{\mathbb{R}}^T\end{split}
\end{equation*}
\sphinxAtStartPar
It follows that the 2\sphinxhyphen{}nd order tensor \(\dot{\mathbb{R}} \cdot \mathbb{R}^T = - \mathbb{R} \cdot \dot{\mathbb{R}}^T\) is anti\sphinxhyphen{}symmetric, and thus it can be written as
\begin{equation}\label{equation:ch/kinematics-rigid:eq:omegax:def}
\begin{split}\dot{\mathbb{R}} \cdot \mathbb{R}^T =: \vec{\omega}_{\times} \ ,\end{split}
\end{equation}
\sphinxAtStartPar
being the vector \(\vec{\omega}\) the angular velocity. Since \(\mathbb{R}\) is unitary by \eqref{equation:ch/kinematics-rigid:eq:rot:unitary}, multiplying \eqref{equation:ch/kinematics-rigid:eq:omegax:def} with the dot\sphinxhyphen{}product on the right by \(\mathbb{R}\), it follows
\begin{equation*}
\begin{split}\dot{\mathbb{R}} = \vec{\omega}_{\times} \cdot \mathbb{R} \ ,\end{split}
\end{equation*}
\sphinxAtStartPar
and the expression of the time derivative of a material vector \(\vec{a}\),
\begin{equation}\label{equation:ch/kinematics-rigid:eq:material-v:time-derivative}
\begin{split}\dfrac{d \vec{a}}{d t} = \dot{\mathbb{R}} \cdot \vec{a}^0 = \vec{\omega}_{\times} \cdot \mathbb{R} \cdot \vec{a}^0 = \vec{\omega}_{\times} \cdot \vec{a} = \vec{\omega} \times \vec{a} \  .\end{split}
\end{equation}


\sphinxAtStartPar
\sphinxstylestrong{Position and Orientation.}
The most general rigid motion is the combination of the translation of a reference point \(Q\) and the rotation w.r.t. this point of other material points,
\begin{equation}\label{equation:ch/kinematics-rigid:eq:rigid:pos}
\begin{split}\begin{aligned}
  \vec{r}_P
  & = \vec{r}_Q + (P - Q) = \\
  & = \vec{r}_Q + \mathbb{R} \cdot (P-Q)^0 
\end{aligned}\end{split}
\end{equation}
\sphinxAtStartPar
\sphinxstylestrong{Velocity and Angular velocity.}
Time derivative of the relation \eqref{equation:ch/kinematics-rigid:eq:rigid:pos} between positions of material points gives again \eqref{equation:ch/kinematics-rigid:eq:kin:rigid:vel}
\begin{equation}\label{equation:ch/kinematics-rigid:eq:rigid:vel}
\begin{split}\begin{aligned}
  \vec{v}_P
  & = \vec{v}_Q + \dot{\mathbb{R}} \cdot (P-Q)^0 = \\
  & = \vec{v}_Q + \vec{\omega}_{\times} \mathbb{R} \cdot (P-Q)^0 = \\
  & = \vec{v}_Q + \vec{\omega} \times (P-Q)
\end{aligned}\end{split}
\end{equation}
\sphinxAtStartPar
\sphinxstylestrong{Acceleration and Angular acceleration.}
Time derivatives of the relation \eqref{equation:ch/kinematics-rigid:eq:rigid:vel} gives
\begin{equation*}
\begin{split}\begin{aligned}
\vec{a}_P 
 & = \vec{a}_Q + \vec{\alpha} \times (P-Q) + \vec{\omega} \times \left( \vec{v}_P - \vec{Q} \right) = \\
 & = \vec{a}_Q + \vec{\alpha} \times (P-Q) + \vec{\omega} \times \left[ \, \vec{\omega} \times (P - Q) \, \right] \ .
\end{aligned}\end{split}
\end{equation*}


\sphinxstepscope


\section{Continuous Medium}
\label{\detokenize{ch/kinematics-continuum:continuous-medium}}\label{\detokenize{ch/kinematics-continuum:classical-mechanics-kinematics-contiuum}}\label{\detokenize{ch/kinematics-continuum::doc}}
\sphinxstepscope


\section{Relative Kinematics}
\label{\detokenize{ch/kinematics-relative:relative-kinematics}}\label{\detokenize{ch/kinematics-relative:classical-mechanics-kinematics-relative}}\label{\detokenize{ch/kinematics-relative::doc}}
\sphinxAtStartPar
Relative kinematics is discussed here using two Cartesian reference frames.
\begin{equation*}
\begin{split}P   - O_0 = x^{0i}_{  P/O_0} \hat{e}^0_i\end{split}
\end{equation*}\begin{equation*}
\begin{split}O_1 - O_0 = x^{0i}_{O_1/O_0} \hat{e}^0_i\end{split}
\end{equation*}\begin{equation*}
\begin{split}P   - O_1 = x^{1i}_{  P/O_1} \hat{e}^1_i\end{split}
\end{equation*}\begin{equation*}
\begin{split}\begin{aligned}
  \hat{e}^1_i 
  = \hat{e}^{1}_i \cdot \hat{e}^0_k \, \hat{e}^0_k
  & = \hat{e}^{1}_j \cdot \hat{e}^0_k \, \hat{e}^0_k \otimes \hat{e}^0_j \cdot \hat{e}^0_i = \\
  & = R^{0\rightarrow 1}_{kj} \hat{e}^0_k \otimes \hat{e}^0_j \cdot \hat{e}^0_i 
    = \mathbb{R}^{0\rightarrow 1} \cdot \hat{e}^0_i  \ .
\end{aligned}\end{split}
\end{equation*}

\subsection{Points}
\label{\detokenize{ch/kinematics-relative:points}}\label{\detokenize{ch/kinematics-relative:classical-mechanics-kinematics-relative-points}}
\sphinxAtStartPar
\sphinxstylestrong{Position.}
Given two reference frames \(Ox^i\), \(O' x^{i'}\), for the position of a point \(P\) reads
\begin{equation}\label{equation:ch/kinematics-relative:eq:relative:point:pos}
\begin{split}(P - O_0) = ( O_1 - O_0 ) + ( P - O_1) \ ,\end{split}
\end{equation}\begin{equation*}
\begin{split}x^0_{P/O_0,i} \hat{e}^0_i = x^0_{O_1/O_0,i} \hat{e}^0_i + x^1_{P/O_1,k} \hat{e}^1_k \ ,\end{split}
\end{equation*}
\sphinxAtStartPar
i.e. the position vector \(P-O\) of the point \(P\) w.r.t. point \(O\) \sphinxhyphen{} origin of the reference frame \(O x^i\) \sphinxhyphen{} is the sum of the position vector \(P-O'\) of the point \(P\) w.r.t. to the point \(O'\) \sphinxhyphen{} origin of the reference frame \(O' x^{'i}\) \sphinxhyphen{}  and the position vector \(O' - O\), of the origin \(O'\) w.r.t. to \(O\).

\sphinxAtStartPar
\sphinxstylestrong{Velocity.} Time derivative of relative position relation \eqref{equation:ch/kinematics-relative:eq:relative:point:pos} w.r.t. to reference frame \(0\) is performed keeping \(\hat{e}^0_i\) constant.
\begin{equation*}
\begin{split}\begin{aligned}
  \dfrac{{}^0 d}{dt} (P-O_0)
  & = \dfrac{{}^0 d}{dt} \left[ (O_1 - O_0) + (P - O_1) \right] = \\
  & = \dfrac{{}^0 d}{dt} \left( x^0_{O_1/O_0,i} \hat{e}^0_i  \right) + \dfrac{{}^0 d}{dt} \left( x^1_{P/O_1,k} \hat{e}^1_k \right) = \\
  & = \dfrac{{}^0 d}{dt} x^0_{O_1/O_0,i} \, \hat{e}^0_i + \dfrac{{}^0 d}{dt} x^1_{P/O_1,k} \, \hat{e}^1_k + x^1_{P/O_1,k} \, \dfrac{{}^0 d}{dt}  \hat{e}^1_k = \\
  & = \vec{v}^0_{O_1/O_0} + \vec{v}^1_{P/O_1} + \vec{\omega}_{1/0} \times  \hat{e}^1_k x^1_{P/O_1,k} = \\
  & = \vec{v}^0_{O_1/O_0} + \vec{v}^1_{P/O_1} + \vec{\omega}_{1/0} \times ( P - O_1 )  \ ,
\end{aligned}\end{split}
\end{equation*}
\sphinxAtStartPar
so that
\begin{equation}\label{equation:ch/kinematics-relative:eq:relative:point:vel}
\begin{split} \vec{v}^0_{P/O} = \vec{v}^0_{O_1/O_0} + \vec{v}^1_{P/O_1} + \vec{\omega}_{1/0} \times ( P - O_1 )  \ .\end{split}
\end{equation}
\sphinxAtStartPar
\sphinxstylestrong{Acceleration.} Time derivative of relative velocity relation \eqref{equation:ch/kinematics-relative:eq:relative:point:vel} w.r.t. reference frame \(0\) reads
\begin{equation*}
\begin{split}\begin{aligned}
  \dfrac{{}^0 d}{dt} \vec{v}^0_{P/O_0}
  & = \dfrac{{}^0 d}{dt} \left[ \vec{v}^0_{O_1/O_0} + \vec{v}^1_{P/O_1} + \vec{\omega}_{1/0} \times ( P - O_1 ) \right] = \\
  & = \dots \\
  & = \vec{a}^0_{O_1/O_0} + \vec{a}^{1}_{P/O_1} + 2 \vec{\omega}_{1/0} \times \vec{v}^1_{P/O_1} + \vec{\alpha}_{1/0} \times (P - O_1) + \vec{\omega}_{1/0} \times [ \, \vec{\omega}_{1/0} \times (P - O_1) \, ]
\end{aligned}\end{split}
\end{equation*}
\sphinxAtStartPar
so that
\begin{equation}\label{equation:ch/kinematics-relative:eq:relative:point:acc}
\begin{split}
\vec{a}^0_{P/O_0} = \vec{a}^0_{O_1/O_0} + \vec{a}^{1}_{P/O_1} + \underbrace{\vec{\alpha}_{1/0} \times (P - O_1)}_{\text{tangential}} + \underbrace{2 \vec{\omega}_{1/0} \times \vec{v}^1_{P/O_1}}_{\text{Coriolis}} + \underbrace{\vec{\omega}_{1/0} \times [ \, \vec{\omega}_{1/0} \times (P - O_1) \, ]}_{\text{centripetal}} \ .
\end{split}
\end{equation}
\sphinxAtStartPar
where:
\begin{itemize}
\item {} 
\sphinxAtStartPar
the “tangential component” is orthogonal to the instantaneous angular acceleration and radius,

\item {} 
\sphinxAtStartPar
the “centripetal component” is orthogonal w.r.t. the instantaneous angular velocity

\end{itemize}

\sphinxAtStartPar
\sphinxstylestrong{todo} \sphinxstyleemphasis{tangent to what, centripetal w.r.t. what?} \sphinxstylestrong{state it clearly, otherwise delete this}


\subsection{Rigid bodies}
\label{\detokenize{ch/kinematics-relative:rigid-bodies}}
\sphinxAtStartPar
\sphinxstylestrong{Orientation.}

\sphinxAtStartPar
\sphinxstylestrong{Angular velocity.}

\sphinxAtStartPar
\sphinxstylestrong{Angular acceleration.}

\sphinxstepscope


\section{Rotations}
\label{\detokenize{ch/kinematics-rotations:rotations}}\label{\detokenize{ch/kinematics-rotations:classical-mechanics-kinematics-rotations}}\label{\detokenize{ch/kinematics-rotations::doc}}

\subsection{Rotation tensor}
\label{\detokenize{ch/kinematics-rotations:rotation-tensor}}
\sphinxAtStartPar
Given 2 Cartesian bases \(\{ \hat{e}^0_i \}_{i=1:3}\), \(\{ \hat{e}^1_j \}_{j=1:3}\), the rotation tensor providing the transformation
\begin{equation*}
\begin{split}\hat{e}^1_i = \mathbb{R}^{0 \rightarrow 1} \cdot \hat{e}^0_i \ ,\end{split}
\end{equation*}
\sphinxAtStartPar
is
\begin{equation*}
\begin{split}\mathbb{R}^{0 \rightarrow 1}
 = R_{ij}^{0 \rightarrow 1} \hat{e}^0_i \otimes \hat{e}^0_j  
 = R_{ij}^{0 \rightarrow 1} \hat{e}^1_i \otimes \hat{e}^1_j 
\end{split}
\end{equation*}
\sphinxAtStartPar
with \(R^{0 \rightarrow 1}_{ij} = \hat{e}^0_i \cdot \hat{e}^1_j\).

\sphinxAtStartPar
\sphinxstylestrong{Angular velocity.}
\begin{equation*}
\begin{split}\vec{\omega}^{01}_{\times} = \mathbb{\Omega}^{01} = \dot{\mathbb{R}}^{01} \cdot \mathbb{R}^{01,T}\end{split}
\end{equation*}
\sphinxAtStartPar
Using index notation
\begin{equation*}
\begin{split}\varepsilon_{ijk} \omega_j = \dot{R}_{ij} R_{kj}\end{split}
\end{equation*}
\sphinxAtStartPar
and the identities
\begin{equation*}
\begin{split}\varepsilon_{ijk} \varepsilon_{lmk} = \delta_{il} \delta_{jm} - \delta_{jl} \delta_{im}\end{split}
\end{equation*}\begin{equation*}
\begin{split}\varepsilon_{ijk} \varepsilon_{ljk} = \delta_{il} \delta_{jj} - \delta_{ij} \delta_{jl} = 3 \delta_{il} - \delta_{il} = 2 \delta_{il}\end{split}
\end{equation*}
\sphinxAtStartPar
it follows
\begin{equation*}
\begin{split}\varepsilon_{ilk} \varepsilon_{ijk} \omega_j = \varepsilon_{ilk} \dot{R}_{ij} R_{kj}\end{split}
\end{equation*}\begin{equation*}
\begin{split}2 \delta_{lj} \omega_j = \varepsilon_{ilk} \dot{R}_{ij} R_{kj}\end{split}
\end{equation*}\begin{equation*}
\begin{split}\omega_l = \frac{1}{2} \varepsilon_{ilk} \dot{R}_{ij} R_{kj} = - \frac{1}{2} \varepsilon_{lik} \dot{R}_{ij} R_{kj} = -\frac{1}{2} \varepsilon_{lij} \Omega_{ij}\end{split}
\end{equation*}
\sphinxAtStartPar
\sphinxstylestrong{Angular acceleration.} Angular acceleration, \(\vec{\alpha}\), is the time derivative of angular velocity, \(\vec{\omega}\),
\begin{equation*}
\begin{split}\vec{\alpha} = \dot{\vec{\omega}} \ .\end{split}
\end{equation*}

\subsection{Successive rotations}
\label{\detokenize{ch/kinematics-rotations:successive-rotations}}
\sphinxAtStartPar
\sphinxstylestrong{Orientation.} Given 3 Cartesian bases \(\{ \hat{e}^0_i \}_{i=1:3}\), \(\{ \hat{e}^1_j \}_{j=1:3}\), \(\{ \hat{e}^2_k \}_{k=1:3}\),
\begin{equation*}
\begin{split}\begin{aligned}
 \hat{e}^2_i 
  & = \mathbb{R}^{1 \rightarrow 2} \cdot \hat{e}^1_i = \\ 
  & = \mathbb{R}^{1 \rightarrow 2} \cdot \mathbb{R}^{0 \rightarrow 1} \cdot\hat{e}^0_i 
\end{aligned} \ , \end{split}
\end{equation*}
\sphinxAtStartPar
i.e composition of rotations holds
\begin{equation*}
\begin{split}\mathbb{R}^{0 \rightarrow 2} = \mathbb{R}^{1 \rightarrow 2} \cdot \mathbb{R}^{0 \rightarrow 1} \ .\end{split}
\end{equation*}
\sphinxAtStartPar
\sphinxstylestrong{Angular velocity.} Time derivative w.r.t. reference frame 0 is indicated as the standard time derivative
\begin{equation*}
\begin{split}\dot{a} = \dfrac{d a}{d t} = \dfrac{{}^0 d a}{d t} = \dfrac{{}^1 d}{dt} + \vec{\omega}_{1/0} \times \ ,\end{split}
\end{equation*}\begin{equation*}
\begin{split}\begin{aligned}
\dfrac{d}{dt} \mathbb{R}^{21} 
  & = \frac{d}{dt} \left[ R^{21}_{ik} \hat{e}^1_i \otimes \hat{e}^1_k \right] = \\
  & = \dot{R}^{21}_{ik} \hat{e}^1_i \otimes \hat{e}^1_k + \mathbb{\Omega}^{10} \cdot  \mathbb{R}^{21} - \mathbb{R}^{21} \cdot \mathbb{\Omega}^{10} = \\
  & = \dfrac{{}^1 d}{dt} \mathbb{R}^{21} + \mathbb{\Omega}^{10} \cdot  \mathbb{R}^{21} - \mathbb{R}^{21} \cdot \mathbb{\Omega}^{10} = \\
\end{aligned}\end{split}
\end{equation*}\begin{equation*}
\begin{split}\begin{aligned}
 \mathbb{\Omega}^{20}
 & = \dot{\mathbb{R}}^{20} \cdot \mathbb{R}^{20, T} = \\
 & = \frac{d}{dt} \left( \mathbb{R}^{21} \cdot \mathbb{R}^{10} \right) \cdot \mathbb{R}^{20, T} = \\
 & = \left\{ \left[ \dfrac{{}^1 d}{dt} \mathbb{R}^{21} + \mathbb{\Omega}^{10} \cdot  \mathbb{R}^{21} - \mathbb{R}^{21} \cdot \mathbb{\Omega}^{10} \right] \cdot \mathbb{R}^{10} + \mathbb{R}^{21} \cdot \dot{\mathbb{R}}^{10}  \right\} \cdot \mathbb{R}^{01} \cdot \mathbb{R}^{12} = \\
 & =  \dfrac{{}^1 d}{dt} \mathbb{R}^{21} \cdot \mathbb{R}^{12} + \mathbb{\Omega}^{10} = \\
 & = \mathbb{\Omega}^{21} + \mathbb{\Omega}^{10} \ .
\end{aligned}\end{split}
\end{equation*}
\sphinxAtStartPar
so that addition of relative angular velocity holds
\begin{equation*}
\begin{split}\mathbb{\Omega}^{20} = \mathbb{\Omega}^{21} + \mathbb{\Omega}^{10} \qquad , \qquad \vec{\omega}_{2/0} = \vec{\omega}_{2/1} + \vec{\omega}_{1/0} \ .\end{split}
\end{equation*}
\sphinxAtStartPar
\sphinxstylestrong{Angular acceleration.} Time derivative of angular velocity composition provides the addition of relative angular accelerations
\begin{equation*}
\begin{split}\dfrac{{}^0 d}{dt} \vec{\omega}_{2/0} = \dfrac{{}^0 d}{dt} \vec{\omega}_{2/1} + \dfrac{{}^0 d}{dt} \vec{\omega}_{1/0} \ ,\end{split}
\end{equation*}
\sphinxAtStartPar
or
\begin{equation*}
\begin{split}\vec{\alpha}_{2/0} = \vec{\alpha}_{2/1} + \vec{\alpha}_{1/0} \ .\end{split}
\end{equation*}

\subsection{Linearization of rotations}
\label{\detokenize{ch/kinematics-rotations:linearization-of-rotations}}\begin{equation*}
\begin{split}\mathbb{I} = \mathbb{R} \cdot \mathbb{R}^T\end{split}
\end{equation*}
\sphinxAtStartPar
Increment
\begin{equation*}
\begin{split}\mathbb{0} = \delta \mathbb{R} \cdot \mathbb{R}^T + \mathbb{R} \cdot \delta \mathbb{R}^T\end{split}
\end{equation*}
\sphinxAtStartPar
and thus the antisymmetric tensor can be written as
\begin{equation*}
\begin{split}\delta \theta_{\times} := \delta \mathbb{R} \cdot \mathbb{R}^T = \delta \mathbb{\Theta} \ ,\end{split}
\end{equation*}
\sphinxAtStartPar
so that
\begin{equation*}
\begin{split}\delta \theta_l = -\frac{1}{2} \varepsilon_{lij} \delta R_{ik} \, R_{jk} = - \frac{1}{2} \varepsilon_{lij} \delta \Theta_{ij}\end{split}
\end{equation*}

\subsection{Parametrizations}
\label{\detokenize{ch/kinematics-rotations:parametrizations}}\label{\detokenize{ch/kinematics-rotations:classical-mechanics-kinematics-rotations-param}}
\sphinxAtStartPar
Minimal sets of parameters to represent rotations have 3 parameters. However these sets of parameters are not regular over all the possible rotations, and the transformation becomes singular somewhere. {\hyperref[\detokenize{ch/kinematics-rotations:classical-mechanics-kinematics-rotations-param-quaternions}]{\sphinxcrossref{\DUrole{std,std-ref}{Quaternions}}}} provide a set of 4 parameters for a regular parametrization of rotations


\subsubsection{Euler angles}
\label{\detokenize{ch/kinematics-rotations:euler-angles}}\label{\detokenize{ch/kinematics-rotations:classical-mechanics-kinematics-rotations-param-euler}}

\subsubsection{Axis and rotation angle}
\label{\detokenize{ch/kinematics-rotations:axis-and-rotation-angle}}\label{\detokenize{ch/kinematics-rotations:classical-mechanics-kinematics-rotations-param-axis}}

\subsubsection{Quaternions}
\label{\detokenize{ch/kinematics-rotations:quaternions}}\label{\detokenize{ch/kinematics-rotations:classical-mechanics-kinematics-rotations-param-quaternions}}
\sphinxstepscope




\chapter{Actions}
\label{\detokenize{ch/actions:actions}}\label{\detokenize{ch/actions:classical-mechanics-actions}}\label{\detokenize{ch/actions::doc}}


\sphinxAtStartPar
\sphinxstylestrong{What is an action?}%
\begin{footnote}[1]\sphinxAtStartFootnote
The answer to the question “what is an action?” might imply a “true” knowledge—whatever that means—of the object\sphinxhyphen{}concept “action.” Here too, as in other cases, the question “what is…?” can be replaced with “what do we mean by…?”, and an “operational answer” can be considered satisfactory, as it reflects the mode of knowledge and formation of understanding in the scientific field: without delving into more abstract philosophical domains, in physics, we are content to define something through its interactions and effects on other systems, its properties, and a reliable process for its measurement.
%
\end{footnote} Following the introduction of fundamental mechanics concepts in his \sphinxstyleemphasis{Principia Naturalis}, Newton conceived the concept of action—including both \sphinxstylestrong{forces} and \sphinxstylestrong{moments}—as the possible \sphinxstylestrong{causes of variation in the “true motion”} of a mechanical system or, equivalently, the causes of the \sphinxstylestrong{difference between true motion and a} {\hyperref[\detokenize{ch/kinematics-relative:classical-mechanics-kinematics-relative}]{\sphinxcrossref{\DUrole{std,std-ref}{\sphinxstylestrong{generic relative motion}}}}}.
\label{ch/actions:def:true-motion}
\begin{sphinxadmonition}{note}{Definition 2.1 (“True motion”)}



\sphinxAtStartPar
Newton’s concept of \sphinxstyleemphasis{true motion} is meant as the motion w.r.t. an inertial reference frame. So what is an {\hyperref[\detokenize{ch/dynamics-principles:classical-mehcanics-dynamics-principles-inertial-ref-frame}]{\sphinxcrossref{\DUrole{std,std-ref}{\sphinxstylestrong{inertial reference frame}}}}}? From an operational point of view, dynamometers measure no force and moment associated with uniform motion w.r.t. a inertial reference frame.
\end{sphinxadmonition}

\sphinxAtStartPar
A force is a vectorial physical quantity that, from an operational point of view, can be measured with a \sphinxstylestrong{3\sphinxhyphen{}axis force sensor} (which measures its three components in a 3\sphinxhyphen{}dimensional space) or with a \sphinxstylestrong{dynamometer} (which measures its intensity), provided it is free to orient itself along the force direction or if the force direction is known and the instrument is aligned with it.


\label{ch/actions:def:true-forces}
\begin{sphinxadmonition}{note}{Definition 2.2 (True forces in classical mechanics)}



\sphinxAtStartPar
Referring to the \DUrole{xref,myst}{four fundamental interactions}, the significant interactions in the realm of classical mechanics are only those of {\hyperref[\detokenize{ch/actions-examples:classical-mechanics-actions-gravitation}]{\sphinxcrossref{\DUrole{std,std-ref}{\sphinxstylestrong{gravitational}}}}} and \sphinxstylestrong{electromagnetic} nature. Electromagnetic interactions can manifest with bodies having a net charge or, more frequently in classical mechanics, between bodies with no net charge. Among the latter cases, in classical mechanics, it is common to observe the macroscopic manifestation of the microscopic electromagnetic interaction between the elementary components of matter in the form of:
\begin{itemize}
\item {} 
\sphinxAtStartPar
{\hyperref[\detokenize{ch/actions-reactions:classical-mechanics-actions-reactions-contact}]{\sphinxcrossref{\DUrole{std,std-ref}{\sphinxstylestrong{contact}}}}} interactions, where it is possible to distinguish:
\begin{itemize}
\item {} 
\sphinxAtStartPar
a \sphinxstylestrong{normal} component to the contact surfaces responsible for the impenetrability of bodies,

\item {} 
\sphinxAtStartPar
and a tangential component to the surfaces that manifests as {\hyperref[\detokenize{ch/actions-reactions:classical-mechanics-actions-reactions-contact-friction}]{\sphinxcrossref{\DUrole{std,std-ref}{\sphinxstylestrong{friction}}}}}

\end{itemize}

\item {} 
\sphinxAtStartPar
material response to stresses, such as in the elastic constitutive law for {\hyperref[\detokenize{ch/actions-examples:classical-mechanics-actions-spring}]{\sphinxcrossref{\DUrole{std,std-ref}{springs}}}}

\end{itemize}
\end{sphinxadmonition}

\sphinxAtStartPar
A measured action that is not a result of the fundamental interactions is due to non\sphinxhyphen{}inertial motion of the dynamometer \sphinxhyphen{} or, very unlikely, to a new interaction you’ve just discover. If you experience this situation, please remember to send me an invitation for the Nobel cerimony.


\bigskip\hrule\bigskip


\sphinxstepscope


\section{Force, Moment of a Force, Distributed Actions}
\label{\detokenize{ch/actions-types:force-moment-of-a-force-distributed-actions}}\label{\detokenize{ch/actions-types:physics-hs-mechanics-actions-def}}\label{\detokenize{ch/actions-types::doc}}

\subsection{Concentrated Force}
\label{\detokenize{ch/actions-types:concentrated-force}}
\sphinxAtStartPar
A (concentrated) force is a vector quantity with physical dimensions,
\begin{equation*}
\begin{split}[\text{force}] = \frac{\text{[mass]}\text{[length]}}{\text{[time]}^2}\end{split}
\end{equation*}
\sphinxAtStartPar
which can be measured using a dynamometer, and whose effect can alter the equilibrium or motion conditions of a physical system.

\sphinxAtStartPar
In addition to the typical information of a vector quantity \sphinxhyphen{} magnitude, direction, and sense \sphinxhyphen{} contained in the force vector \(\vec{F}\), it is often necessary to know the \sphinxstylestrong{point of application} or the line of application of the force.


\subsection{Moment of a Concentrated Force}
\label{\detokenize{ch/actions-types:moment-of-a-concentrated-force}}
\sphinxAtStartPar
The moment of a force \(\vec{F}\) applied at point \(P\), or with a line of application passing through \(P\), relative to point \(H\) is defined as the vector product,
\begin{equation*}
\begin{split}\vec{M}_H = (P - H) \times \vec{F}\end{split}
\end{equation*}

\subsection{System of Forces, Resultant of Actions, and Equivalent Loads}
\label{\detokenize{ch/actions-types:system-of-forces-resultant-of-actions-and-equivalent-loads}}
\sphinxAtStartPar
Given a system of \(N\) forces \(\left\{ \vec{F}_n \right\}_{n=1:N}\), applied at points \(P_n\), we define:
\begin{itemize}
\item {} 
\sphinxAtStartPar
\sphinxstylestrong{resultant} of the system of forces: the sum of the forces,
\begin{equation*}
\begin{split}\vec{R} = \sum_{n=1}^{N} \vec{F}_n \ ,\end{split}
\end{equation*}
\item {} 
\sphinxAtStartPar
resultant of the moments with respect to a point \(H\): the sum of the moments
\begin{equation*}
\begin{split}\vec{M}_H = \sum_{n=1}^{N} (P_n - H) \times \vec{F}_n \ ,\end{split}
\end{equation*}
\item {} 
\sphinxAtStartPar
an \sphinxstylestrong{equivalent load}: a system of forces that has the same resultant of forces and moments; for a system of forces, an equivalent load can be defined as a single force, the resultant of the forces \(\vec{R}\) applied at point \(Q\) derived from the equivalence of moments
\begin{equation*}
\begin{split}\begin{aligned}
    \vec{R} & = \sum_{n=1}^{N} \vec{F}_n \\
    (Q - H) \times \vec{R} & = \sum_{n=1}^{N} (P_n - H) \times \vec{F}_n \\
  \end{aligned}\end{split}
\end{equation*}
\end{itemize}


\subsection{Couple of Forces}
\label{\detokenize{ch/actions-types:couple-of-forces}}
\sphinxAtStartPar
A couple of forces is an equivalent load to two forces of equal magnitude and opposite sense, \(\vec{F}_2 = - \vec{F}_1\), applied at two points \(P_1\), \(P_2\) not aligned along the line of application of the forces to have non\sphinxhyphen{}zero effects.

\sphinxAtStartPar
\sphinxstylestrong{todo} \sphinxstyleemphasis{image}

\sphinxAtStartPar
The resultant of the forces is zero,
\begin{equation*}
\begin{split}\vec{R} = \vec{F}_1 + \vec{F}_2 = \vec{F}_1 - \vec{F}_1 = \vec{0} \ ,\end{split}
\end{equation*}
\sphinxAtStartPar
while the resultant of the moments does not depend on the moment pole,
\begin{equation*}
\begin{split}\begin{aligned}
  \vec{M}_H & = (P_1 - H) \times \vec{F}_1 + (P_2 - H) \times \vec{F}_2 = \\
  & = (P_1 - H) \times \vec{F}_1 - (P_2 - H) \times \vec{F}_1 = \\
  & = (P_1 - P_2) \times \vec{F}_1 =: \vec{C} \ .
\end{aligned}\end{split}
\end{equation*}

\subsection{Force Fields}
\label{\detokenize{ch/actions-types:force-fields}}
\sphinxAtStartPar
\sphinxstylestrong{todo}


\subsection{Distributed Actions}
\label{\detokenize{ch/actions-types:distributed-actions}}
\sphinxAtStartPar
\sphinxstylestrong{todo}

\sphinxstepscope


\section{Work and Power}
\label{\detokenize{ch/actions-work:work-and-power}}\label{\detokenize{ch/actions-work:classical-mechanics-actions-work}}\label{\detokenize{ch/actions-work::doc}}
\sphinxAtStartPar
In mechanics, as will become clearer later (\sphinxstylestrong{todo} add reference), the concept of work is linked to the concept of energy. \sphinxstylestrong{todo}


\subsection{Work and Power of a Force}
\label{\detokenize{ch/actions-work:work-and-power-of-a-force}}
\sphinxAtStartPar
\sphinxstylestrong{Work.} The elementary work of a force \(\vec{F}\) applied at point \(P\) that undergoes an elementary displacement \(d \vec{r}_P\) is defined as the dot product between the force and the displacement,
\begin{equation}\label{equation:ch/actions-work:eq:work:diff}
\begin{split}\delta W := \vec{F} \cdot d \vec{r}_P \ .\end{split}
\end{equation}
\sphinxAtStartPar
The work done by the force \(\vec{F}\) applied at point \(P\) moving from point \(A\) to point \(B\) along the path \(\ell_{AB}\) is the sum of all elementary contributions \sphinxhyphen{} and hence, in the limit for elementary displacements \(\rightarrow 0\) for continuous variations, the line integral,
\begin{equation}\label{equation:ch/actions-work:eq:work:int}
\begin{split}W_{\ell_{AB}} = \int_{\ell_{AB}} \delta W = \int_{\ell_{AB}} \vec{F} \cdot d \vec{r}_{P} \ .\end{split}
\end{equation}
\sphinxAtStartPar
In general, the work of a force or a field of forces depends on the path \({\ell}_{AB}\). In cases where the work is independent of the path but depends only on the endpoints, we talk about {\hyperref[\detokenize{ch/actions-conservative:classical-mechanics-actions-conservative}]{\sphinxcrossref{\DUrole{std,std-ref}{conservative actions}}}}.

\sphinxAtStartPar
\sphinxstylestrong{Power.} The power of the force is defined as the time derivative of the work,
\begin{equation*}
\begin{split}P := \frac{\delta W}{dt} = \vec{F} \cdot \frac{d \vec{r}_P}{d t} = \vec{F} \cdot \vec{v}_P \ , \end{split}
\end{equation*}
\sphinxAtStartPar
and coincides with the dot product between the force and the velocity of the point of application. Be cautious if a force is applied to geometric points rather than material points, such as in the case of a disk rolling without slipping on a surface: at every instant, the (new) material contact point has zero velocity, while the geometric contact point is the projection of the center of the disk and moves with the same velocity, \(v = R \theta\)


\subsection{Work and Power of a System of Forces}
\label{\detokenize{ch/actions-work:work-and-power-of-a-system-of-forces}}
\sphinxAtStartPar
\sphinxstylestrong{Work.} The work of a system of forces is the sum of the works of the individual forces,
\begin{equation*}
\begin{split}\delta W = \sum_{n=1}^{N} \delta W_n = \sum_{n=1}^{N} \vec{F}_n \cdot d \vec{r}_n\end{split}
\end{equation*}
\sphinxAtStartPar
\sphinxstylestrong{Power.} The power of a system of forces is the sum of the powers of the individual forces,
\begin{equation*}
\begin{split}P = \sum_{n=1}^{N} P_n = \sum_{n=1}^{N} \vec{F}_n \cdot \vec{v}_n \ .\end{split}
\end{equation*}

\subsection{Work and Power of a Couple of Forces}
\label{\detokenize{ch/actions-work:work-and-power-of-a-couple-of-forces}}
\sphinxAtStartPar
\sphinxstylestrong{Work.} The elementary work of a couple of forces is the sum of the elementary works,
\begin{equation*}
\begin{split}\begin{aligned}
  \delta W & = \vec{F}_1 \cdot d \vec{r}_1 + \vec{F}_2 \cdot d \vec{r}_2 = \\
           & = \vec{F}_1 \cdot ( d \vec{r}_1 - d \vec{r}_2 ) = 
\end{aligned}\end{split}
\end{equation*}
\sphinxAtStartPar
\sphinxstylestrong{Power.} The power of a couple of forces,
\begin{equation*}
\begin{split}P = \vec{F}_1 \cdot (\vec{v}_1 - \vec{v}_2)\end{split}
\end{equation*}
\sphinxAtStartPar
can be rewritten if the points of application perform a rigid motion act (\sphinxstylestrong{todo} verify the definition of motion act and if it should be introduced),
\begin{equation*}
\begin{split}\vec{v}_1 - \vec{v}_2 = \vec{\omega} \times (P_1 - P_2) \ ,\end{split}
\end{equation*}
\sphinxAtStartPar
as
\begin{equation*}
\begin{split}\begin{aligned}
  P & =  \vec{F}_1 \cdot (\vec{v}_1 - \vec{v}_2) = \\
    & =  \vec{F}_1 \cdot \left[ \vec{\omega} \times (P_1 - P_2) \right] = \\
    & =  \vec{\omega} \cdot \left[ (P_1 - P_2) \times \vec{F}_1\right] = \\
    & =  \vec{\omega} \cdot \vec{C} \ . 
\end{aligned}\end{split}
\end{equation*}
\sphinxstepscope


\section{Conservative Actions}
\label{\detokenize{ch/actions-conservative:conservative-actions}}\label{\detokenize{ch/actions-conservative:classical-mechanics-actions-conservative}}\label{\detokenize{ch/actions-conservative::doc}}
\sphinxAtStartPar
In general, the work of a force field acting on a point \(P\) moving in space from point \(A\) to point \(B\) along a path \(\ell_{AB}\) represented by integral \eqref{equation:ch/actions-work:eq:work:int} depends on the path, and this dependence on the path is usually highlighted with the use of the symbol \(\delta\) in the elementary work \eqref{equation:ch/actions-work:eq:work:diff}.

\sphinxAtStartPar
If the work of a force field does not depend on the path \(\ell_{AB}\) but only on the endpoints \(A\), \(B\), for all pairs of points within a region of space \(\Omega\), the \sphinxstylestrong{force field} is said to be \sphinxstylestrong{conservative} in the region \(\Omega\) of space. In this case, the work integral can be written as the difference of a scalar field, \(U(P)\) or its opposite \(V(P) := - U(P)\),
\begin{equation*}
\begin{split}\begin{aligned}
  W_{AB} &  = \int_{\ell_{AB}} \vec{F} \cdot d \vec{r} = \\
         &  = \int_{\ell_{AB}} \delta W = \\
         &  = U(B) - U(A) = \Delta_{AB} U  \\
         &  = V(A) - V(B) = -\Delta_{AB} V  \\
\end{aligned}\end{split}
\end{equation*}
\sphinxAtStartPar
The functions \(U\), \(V\) are respectively defined as the \sphinxstylestrong{potential} and \sphinxstylestrong{potential energy} of the force field. From the definition of a conservative force field it readily follows that
\begin{equation*}
\begin{split}\oint_{\ell} \vec{F} \cdot d \vec{r} = 0 \ .\end{split}
\end{equation*}
\sphinxAtStartPar
The elementary work can thus be expressed in terms of the differential of these functions,
\begin{equation*}
\begin{split}\begin{aligned}
  \delta W & = \ \ \ d U =\ \ \ d \vec{r} \cdot \nabla U  = \\
           & =     - d V =    - d \vec{r} \cdot \nabla V \\
\end{aligned}\end{split}
\end{equation*}
\sphinxAtStartPar
Comparing this relation with the definition of work \(\delta W = d \vec{r} \cdot \vec{F}\), it is possible to identify the force field with the gradient of the potential function, and the opposite of the gradient of the potential energy,
\begin{equation*}
\begin{split}\vec{F} = \nabla U = - \nabla V \ .\end{split}
\end{equation*}
\sphinxAtStartPar
Since the force field can be written as the gradient of a scalar field, and the curl of a gradient is identically zero, the curl of a potential force field is identically zero,
\begin{equation*}
\begin{split}\nabla \times \vec{F} = \vec{0} \ .\end{split}
\end{equation*}
\begin{sphinxadmonition}{note}{Note:}
\sphinxAtStartPar
The reverse logical process \sphinxhyphen{} \(\nabla \times \vec{F} = \vec{0}\) implies \(\vec{F} = \nabla U\) implies \(\vec{F}\) conservative, i.e. independence of the work from the path \sphinxhyphen{} requires the domain continaing the integration path \(\ell\) to be simply connected.
\end{sphinxadmonition}
\label{ch/actions-conservative:example-0}
\begin{sphinxadmonition}{note}{Example 2.3.1 (Force fields in non\sphinxhyphen{}simply connected domains)}



\sphinxAtStartPar
In a region of \(E^2\), described with Cartesian coordinates, containing the origin \(O \equiv (x_0, y_0) \equiv (0,0)\) the vector field
\begin{equation*}
\begin{split}\vec{F}_1(\vec{r}) = \frac{x}{x^2 + y^2} \hat{x} + \frac{y}{x^2+y^2} \hat{y}\end{split}
\end{equation*}
\sphinxAtStartPar
is conservative, while the vector field
\begin{equation*}
\begin{split}\vec{F}_2(\vec{r}) = -\frac{y}{x^2 + y^2} \hat{x} + \frac{x}{x^2+y^2} \hat{y}\end{split}
\end{equation*}
\sphinxAtStartPar
is not conservative, even though their curl is zero in all the points of the domain where the field is defined \sphinxhyphen{} they’re not defined in the origin.
\end{sphinxadmonition}

\sphinxstepscope


\section{Examples of Forces}
\label{\detokenize{ch/actions-examples:examples-of-forces}}\label{\detokenize{ch/actions-examples:classical-mechanics-actions-examples}}\label{\detokenize{ch/actions-examples::doc}}

\subsection{Gravitation}
\label{\detokenize{ch/actions-examples:gravitation}}\label{\detokenize{ch/actions-examples:classical-mechanics-actions-gravitation}}

\subsubsection{Universal Law of Gravitation}
\label{\detokenize{ch/actions-examples:universal-law-of-gravitation}}\label{\detokenize{ch/actions-examples:classical-mechanics-actions-gravitation-newton}}
\sphinxAtStartPar
The force \(\vec{F}_{12}\) exerted by a mass \(m_2\) at \(P_2\) on a mass \(m_1\) at \(P_1\) is described by \sphinxstylestrong{Newton’s Universal Law of Gravitation},
\begin{equation*}
\begin{split}\vec{F}_{12} = G \, m_1 \, m_2 \dfrac{P_2 - P_1}{|P_2 - P_1|^3} \ ,\end{split}
\end{equation*}
\sphinxAtStartPar
or
\begin{equation*}
\begin{split}\vec{F}_{12} = G \, m_1 \, m_2 \dfrac{\hat{r}_{12}}{|\vec{r}_{12}|^2} \ ,\end{split}
\end{equation*}
\sphinxAtStartPar
where \(\vec{r}_{12} = (P_2 - P_1)\) is the vector pointing from point \(P_1\) to point \(P_2\), \(r_{12} = |\vec{r}_{12}|\) is its magnitude, and \(\hat{r}_{12} = \frac{\vec{r}_{12}}{|\vec{r}_{12}|}\) is the unit vector in the same direction. The \sphinxstylestrong{universal gravitational constant} \(G\) is
\begin{equation*}
\begin{split}G = 6.67 \cdot 10^{-11} \frac{N \, m^2}{kg^2}\end{split}
\end{equation*}
\sphinxAtStartPar
and is considered a constant of nature.

\sphinxAtStartPar
\sphinxstylestrong{Principle of Superposition of Causes and Effects (PSCE).} Principle of superposition holds, i.e. the force acting on a mass \(m\) placed in \(P\) due to a set of \(N\) masses \(\{ m_k \}_{k=1:N}\) placed in \(P_k\) is the sum of individual forces \(\vec{F}_{k}\),
\begin{equation}\label{equation:ch/actions-examples:eq:gravitation:force:psce}
\begin{split}\vec{F} = \sum_{k=1}^N \vec{F}_{k} = G \, m \, \sum_{k=1}^N m_k \dfrac{P_k - P}{|P_k - P|^3} \ .\end{split}
\end{equation}

\subsubsection{Gravitational Field}
\label{\detokenize{ch/actions-examples:gravitational-field}}\label{\detokenize{ch/actions-examples:classical-mechanics-actions-gravitation-field}}
\sphinxAtStartPar
The gravitational field generated by a set of masses \(\{ m_k \}_{k=1:N}\) located at \(P_k\) is a vector field associating a vector with physical dimensions \(\frac{\text{[force]}}{\text{[mass]}}\) to each point in space \(P\), that can be thought as the force per unit\sphinxhyphen{}mass acting on a \sphinxstylestrong{test mass} \(m\) placed in \(P\), whose expression directly follows from \eqref{equation:ch/actions-examples:eq:gravitation:force:psce}
\begin{equation*}
\begin{split}\vec{g}(P) = \dfrac{\vec{F}}{m} = G \, \sum_{k=1}^N m_k \dfrac{P_k - P}{|P_k - P|^3} \ .\end{split}
\end{equation*}
\sphinxAtStartPar
Given the gravitational field \(\vec{g}(P)\), the gravitational force experienced by a system of mass \(m\) at \(P\) can be written as
\begin{equation*}
\begin{split}\vec{F}_g = m \vec{g}(P)\end{split}
\end{equation*}
\sphinxAtStartPar
\sphinxstylestrong{Gravitational Potential Energy.} Gravitational potential of a system of 2 masses reads
\begin{equation*}
\begin{split}V(P) = - G \, m \, m_1 \frac{1}{|P - P_1|} \ ,\end{split}
\end{equation*}
\sphinxAtStartPar
as it can be easily shown evaluating its gradient,
\begin{equation*}
\begin{split}\begin{aligned}
  \nabla V(P)
  & = - G \, m \, m_1 \, \hat{x}_k \, \dfrac{\partial}{\partial x_k} \dfrac{1}{|P - P_1|} = \\
  & = - G \, m \, m_1 \, \hat{x}_k \, \left( - \dfrac{1}{|P - P_1|^2} \right) \dfrac{\partial}{\partial x_k} |P - P_1| = \\
  & =   G \, m \, m_1 \, \hat{x}_k \, \left(   \dfrac{1}{|P - P_1|^2} \right) \dfrac{x_k - x_{1,k}}{|P - P_1|} = \\
  & =   G \, m \, m_1 \, \dfrac{x_k - x_{1,k}}{|P - P_1|^3}  \, \hat{x}_k= \\
  & =   G \, m \, m_1 \, \dfrac{P-P_1}{|P - P_1|^3} \ .
\end{aligned}\end{split}
\end{equation*}
\sphinxAtStartPar
Potential energy stored in a system of \(N\) point masses \(\{ m_k \}_{k=1:N}\) coincides with the work needed to build the system \sphinxhyphen{} a common choice to set the arbitrary additional constant of the energy is setting it equal to zero when masses are at infinite distances \sphinxhyphen{}, namely
\begin{equation*}
\begin{split}V(P_k) = \sum_{\{i,k\}, i \ne k} G \, m_i \, m_k \frac{1}{|P_i - P_k|} \ ,\end{split}
\end{equation*}
\sphinxAtStartPar
summing over different unordered pairs, i.e. \(\{ 1, 2 \}\) and \(\{2,1\}\) are the same pair and thus considered only once, or
\begin{equation*}
\begin{split}V(P_k) = \frac{1}{2} \sum_{(i,k),  i\ne k} G \, m_i \, m_k \frac{1}{|P_i - P_k|} \ ,\end{split}
\end{equation*}
\sphinxAtStartPar
summing over different ordered pairs, i.e. \((1,2)\) and \((2,1)\) are different pairs.


\subsubsection{Gravitational Field Near Earth’s Surface}
\label{\detokenize{ch/actions-examples:gravitational-field-near-earth-s-surface}}
\sphinxAtStartPar
Within a limited domain near Earth’s surface, it is common to approximate Earth’s gravitational field as a uniform field, directed along the local vertical toward the center of the Earth, with intensity \(g = G \frac{M_E}{R_E^2}\).

\sphinxAtStartPar
This model can be derived by approximating the position vector relative to the Earth’s center \(P - P_E \sim R_E \hat{r}\) and the unit vector identifying the direction from a point in the domain to the Earth’s center with the local vertical \(\hat{r}_{12} \sim - \hat{z}\),
\begin{equation*}
\begin{split}\vec{g}(\vec{r}) = - G \dfrac{M_E}{R_E^2} \hat{z} = - g \hat{z} \ .\end{split}
\end{equation*}
\sphinxAtStartPar
The gravitational force experienced by a body of mass \(m\) near Earth’s surface is thus
\begin{equation*}
\begin{split}\vec{F}_g = - m g \hat{z} \ ,\end{split}
\end{equation*}
\sphinxAtStartPar
commonly referred to as \sphinxstylestrong{weight}.

\sphinxAtStartPar
\sphinxstylestrong{Gravitational Potential Energy.} It can be shown that the gravitational potential near Earth’s surface becomes
\begin{equation*}
\begin{split}V(P) = m \, g \, z_P \ .\end{split}
\end{equation*}\subsubsection*{Proof.}

\sphinxAtStartPar
With the series expansion, with \(P - P_E = R_E \hat{r} + \vec{d}\), and \(|\vec{d}| \ll R_E\),
\begin{equation*}
\begin{split}\begin{aligned}
  V(P) & = - G \, m \, M_E \frac{1}{|P - P_E|} = \\
       & \approx G M_E \, m \left[ - \frac{1}{R_E} + \frac{R_E \hat{r} \cdot \vec{d}}{R_E^3}  \right]   = \\
       & = \underbrace{- m \, \frac{ G M_E}{R_E}}_{\text{const}} + m \, \underbrace{\frac{G \, M_E}{R_E^2}}_{= g} \underbrace{\hat{r} \cdot \vec{d}}_{= z}
\end{aligned}\end{split}
\end{equation*}

\subsubsection{Gravitational field of a continuous mass distribution}
\label{\detokenize{ch/actions-examples:gravitational-field-of-a-continuous-mass-distribution}}
\sphinxAtStartPar
Mass density field \(\rho(\vec{r}_0)\) for \(\vec{r}_0 \in V_0\) produces the \sphinxstylestrong{gravitational field} in \(\vec{r}\),
\begin{equation*}
\begin{split}\vec{g}(\vec{r}) = \int_{\vec{r}_0 \in V_0} d \vec{g}(\vec{r}, \vec{r}_0) = - \int_{\vec{r}_0 \in V_0} G \rho(\vec{r}_0) \dfrac{\vec{r} - \vec{r}_0}{\left|\vec{r} - \vec{r}_0\right|^3} \ .\end{split}
\end{equation*}
\sphinxAtStartPar
By direct computation, the \sphinxstylestrong{gravitational potential} \(\phi(\vec{r})\), s.t. \(\vec{g} = \nabla \phi\), reads
\begin{equation*}
\begin{split}\phi(\vec{r}) = \int_{\vec{r}_0 \in V_0} G \rho(\vec{r}_0) \dfrac{1}{|\vec{r}-\vec{r}_0|}\end{split}
\end{equation*}

\paragraph{Gauss’ law for the gravitational field}
\label{\detokenize{ch/actions-examples:gauss-law-for-the-gravitational-field}}
\sphinxAtStartPar
The flux of the gravitational field produced by mass distribution \(\rho(\vec{r}_0)\) in volume \(V_0\) thorugh a closed surface \(\partial V\) reads
\begin{equation*}
\begin{split}\begin{aligned}
  \oint_{\vec{r} \in \partial V} \vec{g}(\vec{r}) \cdot \hat{n}(\vec{r}) 
  & = - G \oint_{\vec{r} \in \partial V} \int_{\vec{r}_0 \in V_0} \rho(\vec{r}_0) \dfrac{\vec{r} - \vec{r}_0}{\left|\vec{r} - \vec{r}_0\right|^3}  \cdot \hat{n}(\vec{r}) = \\
  & = - G \int_{\vec{r}_0 \in V_0} \rho(\vec{r}_0) \oint_{\vec{r} \in \partial V} \dfrac{\vec{r} - \vec{r}_0}{\left|\vec{r} - \vec{r}_0\right|^3}  \cdot \hat{n}(\vec{r}) 
\end{aligned}\end{split}
\end{equation*}
\sphinxAtStartPar
The inner integral can be written as the solid angle of the surface \(\partial V\) as seen by the point \(\vec{r}_0\), whose value is
\begin{equation*}
\begin{split}
  \oint_{\vec{r} \in \partial V} \dfrac{\vec{r} - \vec{r}_0}{\left|\vec{r} - \vec{r}_0\right|^3}  \cdot \hat{n}(\vec{r}) = 4 \pi
  \begin{cases}
     1 & \quad \text{if $ \vec{r}_0 \in V$} \\
     \theta(\vec{r}_0, \partial V) & \quad \text{if $ \vec{r}_0 \in \partial V$} \\
     0 & \quad \text{if $ \vec{r}_0 \notin V \cup \partial V$} \\
  \end{cases}
\end{split}
\end{equation*}
\sphinxAtStartPar
Thus, net contributions of the flux of the gravitational field \(\vec{g}(\vec{r})\) through \(\partial V\) come only from points \(\vec{r}_0\) inside \(V\), \(\vec{r}_0 \in V\).%
\begin{footnote}[1]\sphinxAtStartFootnote
Contributions from points outside \(\partial V\) are identically zero; contributions from surface \(\partial V\) are zero if volume mass density \(\rho(\vec{r}_0)\) is regular enough, i.e. it contains Dirac’s \(\delta\) representing surface distribution that would have non\sphinxhyphen{}negligible contributions in integration over \(V\).
%
\end{footnote} Thus the flux becomes
\begin{equation*}
\begin{split}
  \oint_{\vec{r} \in \partial V} \vec{g}(\vec{r}) \cdot \hat{n}(\vec{r}) = - G \int_{\vec{r}_0 \in V_0 \cap V} 4 \pi \rho(\vec{r}_0) 
\end{split}
\end{equation*}
\sphinxAtStartPar
or, setting \(\rho(\vec{r}_0) = \) in all the points \(\vec{r} \in V\), \(\vec{r} \notin V_0\), and changing the name of the dummy integration variable \(\vec{r}_0 \rightarrow \vec{r}\),
\begin{equation*}
\begin{split}
  \oint_{\vec{r} \in \partial V} \vec{g}(\vec{r}) \cdot \hat{n}(\vec{r}) = - G \int_{\vec{r} \in V} 4 \pi \rho(\vec{r}) \ .
\end{split}
\end{equation*}
\sphinxAtStartPar
If the gravitational field \(\vec{g}(\vec{r})\) is regular enough for the divergence theorem to hold, it follows
\begin{equation}\label{equation:ch/actions-examples:eq:g:flux:V}
\begin{split}
  \oint_{\vec{r} \in \partial V} \vec{g}(\vec{r}) \cdot \hat{n}(\vec{r}) = \int_{\vec{r} \in V} \nabla \cdot \vec{g}(\vec{r})  = - G \int_{\vec{r} \in V} 4 \pi \rho(\vec{r}) \ ,
\end{split}
\end{equation}
\sphinxAtStartPar
or, for the arbitrariety of the volume \(V\),
\begin{equation*}
\begin{split}- \nabla \cdot \vec{g} = 4 \pi G \rho \ .\end{split}
\end{equation*}
\sphinxAtStartPar
Introducing the gravitational potential \(\phi(\vec{r})\), whose gradient equals the gravitational field \(\nabla \phi = \vec{g}\) by definition, a \sphinxstylestrong{Poisson equation for the gravitational potential} follows
\begin{equation}\label{equation:ch/actions-examples:eq:g:poisson}
\begin{split}- \nabla^2 \phi = 4 \pi G \rho \ .\end{split}
\end{equation}

\subsection{Elastic Actions: Linear Springs}
\label{\detokenize{ch/actions-examples:elastic-actions-linear-springs}}\label{\detokenize{ch/actions-examples:classical-mechanics-actions-spring}}
\sphinxAtStartPar
\sphinxstylestrong{todo}


\bigskip\hrule\bigskip


\sphinxstepscope


\section{Constraint Reactions}
\label{\detokenize{ch/actions-reactions:constraint-reactions}}\label{\detokenize{ch/actions-reactions:classical-mechanics-actions-reactions}}\label{\detokenize{ch/actions-reactions::doc}}
\sphinxAtStartPar
Kinematic constraints act on a system by limiting its possible movements, exerting forces and moments, which are defined as constraint reactions.

\sphinxAtStartPar
In general, at an \sphinxstylestrong{ideal} constraint (\sphinxstylestrong{todo} provide definition of ideal constraint and discuss/mention/refer to friction), a constraint reaction corresponds to each constrained degree of freedom: for example, the constraint of translation of a point in a direction has a corresponding reaction force in that direction; the constraint of rotation around an axis has a corresponding moment aligned with that axis.

\sphinxAtStartPar
These conditions can be derived from the equations of dynamics for massless systems, as often considered in the ideal constraint model.


\subsection{Contact Actions}
\label{\detokenize{ch/actions-reactions:contact-actions}}\label{\detokenize{ch/actions-reactions:classical-mechanics-actions-reactions-contact}}

\subsubsection{Constraint Reactions of Ideal Constraints}
\label{\detokenize{ch/actions-reactions:constraint-reactions-of-ideal-constraints}}\label{\detokenize{ch/actions-reactions:classical-mechanics-actions-reactions-contact-ideal}}
\sphinxAtStartPar
Ideal constraints are models that \sphinxstylestrong{do not perform net work}, and are thus \sphinxstylestrong{conservative elements}. As should become evident in the subsequent sections from the expressions of relative velocities and exchanged actions,
\begin{equation*}
\begin{split}\begin{aligned}
P & = \vec{v}_1     \cdot \vec{F}_{21} + \vec{v}_2     \cdot \vec{F}_{12} 
    + \vec{\omega}_1 \cdot \vec{M}_{21} + \vec{\omega}_2 \cdot \vec{M}_{12} = \\ 
  & = ( \vec{v}_1 - \vec{v}_2 ) \cdot \vec{F}_{21}
    + ( \vec{\omega}_1 - \vec{\omega}_2 ) \cdot \vec{M}_{21} = \\ 
  & = \vec{v}^{rel}_{21} \cdot \vec{F}_{21}
    + \vec{\omega}^{rel}_{21} \cdot \vec{M}_{21} \ ,
\end{aligned}\end{split}
\end{equation*}
\sphinxAtStartPar
both terms are zero either because the relative motion is zero, or the actions act orthogonally to the relative motions.


\paragraph{Fixed Joint}
\label{\detokenize{ch/actions-reactions:fixed-joint}}
\sphinxAtStartPar
The fixed joint constraint prevents both relative motion and relative rotation,
\begin{equation*}
\begin{split}
\begin{cases}
  \vec{0} = \vec{v}^{rel}_{21}     = \vec{v}_{2}     - \vec{v}_{1} \\
  \vec{0} = \vec{\omega}^{rel}_{21} = \vec{\omega}_{2} - \vec{\omega}_{1} \\
\end{cases}
\qquad , \qquad
\begin{cases}
  \qquad \vec{F}_{12} = - \vec{F}_{21} \\
  \qquad \vec{M}_{12} = - \vec{M}_{21} \\
\end{cases}
\end{split}
\end{equation*}

\paragraph{Slider}
\label{\detokenize{ch/actions-reactions:slider}}
\sphinxAtStartPar
The slider constraint prevents relative motion in one direction and relative rotation.
\begin{equation*}
\begin{split}
\begin{cases}
  \quad \forall \ \vec{v}^{rel}_{\hat{t},21}     = \vec{v}_{\hat{t},2}     - \vec{v}_{\hat{t},1} \\
          0  = v^{rel}_{\hat{n},21}     = v_{\hat{n},2}     - v_{\hat{n},1} \\
  \vec{0} = \vec{\omega}^{rel}_{21} = \vec{\omega}_{2} - \vec{\omega}_{1} \\
\end{cases}
\qquad , \qquad
\begin{cases}
  \vec{0} = \vec{F}_{\hat{t},12} = \vec{F}_{\hat{t},21} \\
  \qquad F_{\hat{n},12} = - F_{\hat{n},21} \\
  \qquad \vec{M}_{12} = - \vec{M}_{21} \\
\end{cases}
\end{split}
\end{equation*}

\paragraph{Cylindrical Joint}
\label{\detokenize{ch/actions-reactions:cylindrical-joint}}
\sphinxAtStartPar
The cylindrical joint constraint prevents relative motion and allows rotation around one axis.
\begin{equation*}
\begin{split}
\begin{cases}
  \vec{0} = \vec{v}^{rel}_{21}     = \vec{v}_{2}     - \vec{v}_{1} \\
  \quad \forall \ \omega^{rel}_{\hat{t},21} = \omega_{\hat{t},2} - \omega_{\hat{t},1} \\
  \vec{0} = \vec{\omega}^{rel}_{\hat{n},21} = \vec{\omega}_{\hat{n},2} - \vec{\omega}_{\hat{n},1} \\
\end{cases}
\qquad , \qquad
\begin{cases}
  \qquad \vec{F}_{12} = - \vec{F}_{21} \\
  0 =  M_{\hat{t},12} = M_{\hat{t},21} \\
  \qquad \vec{M}_{\hat{n},12} = - \vec{M}_{\hat{n},21} \\
\end{cases}
\end{split}
\end{equation*}

\paragraph{Spherical Joint}
\label{\detokenize{ch/actions-reactions:spherical-joint}}
\sphinxAtStartPar
The spherical joint constraint prevents relative motion but allows general rotation.
\begin{equation*}
\begin{split}
\begin{cases}
  \vec{0} = \vec{v}^{rel}_{21}     = \vec{v}_{2}     - \vec{v}_{1} \\
  \quad \forall \vec{\omega}^{rel}_{21} = \vec{\omega}_{2} - \vec{\omega}_{1} \\
\end{cases}
\qquad , \qquad
\begin{cases}
  \qquad \vec{F}_{12} = - \vec{F}_{21} \\
  \vec{0} =  \vec{M}_{12} = \vec{M}_{21} \\
\end{cases}
\end{split}
\end{equation*}

\paragraph{Roller}
\label{\detokenize{ch/actions-reactions:roller}}
\sphinxAtStartPar
The roller constraint can be thought of as a combination of a slider and a cylindrical joint.
\begin{equation*}
\begin{split}
\begin{cases}
  \quad \forall \ \vec{v}^{rel}_{\hat{t},21}     = \vec{v}_{\hat{t},2}     - \vec{v}_{\hat{t},1} \\
          0  = v^{rel}_{\hat{n},21}     = v_{\hat{n},2}     - v_{\hat{n},1} \\
  \quad \forall \ \omega^{rel}_{\hat{t},21} = \omega_{\hat{t},2} - \omega_{\hat{t},1} \\
  \vec{0} = \vec{\omega}^{rel}_{\hat{n},21} = \vec{\omega}_{\hat{n},2} - \vec{\omega}_{\hat{n},1} \\
\end{cases}
\qquad , \qquad
\begin{cases}
  \vec{0} = \vec{F}_{\hat{t},12} = \vec{F}_{\hat{t},21} \\
  \qquad F_{\hat{n},12} = - F_{\hat{n},21} \\
  0 =  M_{\hat{t},12} = M_{\hat{t},21} \\
  \qquad \vec{M}_{\hat{n},12} = - \vec{M}_{\hat{n},21} \\
\end{cases}
\end{split}
\end{equation*}

\paragraph{Support}
\label{\detokenize{ch/actions-reactions:support}}
\sphinxAtStartPar
The support constraint is a unilateral constraint \sphinxstylestrong{todo} \sphinxstyleemphasis{add description}


\subsubsection{Friction}
\label{\detokenize{ch/actions-reactions:friction}}\label{\detokenize{ch/actions-reactions:classical-mechanics-actions-reactions-contact-friction}}

\paragraph{Static Friction}
\label{\detokenize{ch/actions-reactions:static-friction}}\label{\detokenize{ch/actions-reactions:classical-mechanics-actions-reactions-contact-friction-static}}
\sphinxAtStartPar
Static friction is the type of friction that occurs between two bodies when there is no relative motion between them, acting as a tangential force to the contact surface. The simplest model of static friction assumes that the maximum static friction force \(F^s_{max}\) that can be exerted between two bodies is proportional to the normal reaction between them, \(N\),
\begin{equation*}
\begin{split}F^s_{max} = \mu^s \, N \ .\end{split}
\end{equation*}
\sphinxAtStartPar
The proportionality constant \(\mu^s\) is defined as the \sphinxstylestrong{coefficient of static friction}. Generally, static friction forces are determined by the equilibrium conditions of the body, if these conditions can be met, and the relation
\begin{equation*}
\begin{split}|F^s| \ge F^s_{max} \ .\end{split}
\end{equation*}

\paragraph{Dynamic Friction}
\label{\detokenize{ch/actions-reactions:dynamic-friction}}\label{\detokenize{ch/actions-reactions:classical-mechanics-actions-reactions-contact-friction-dynamic}}
\sphinxAtStartPar
Dynamic friction occurs between two bodies in contact and in relative motion, acting as a tangential force to the contact surface. The simplest model of dynamic friction assumes that the dynamic friction force is proportional to the normal reaction between the two bodies and directed opposite to the relative velocity,
\begin{equation*}
\begin{split}\vec{F}_{12} = - \mu^d N \frac{\vec{v}_{12}}{|\vec{v}_{12}|} \ ,\end{split}
\end{equation*}
\sphinxAtStartPar
where \(\vec{F}_{12}\) is the force acting on body 1 due to body 2, and \(\vec{v}_{12} = \vec{v}_1 - \vec{v}_2\) is the velocity of body 1 relative to body 2.


\paragraph{Pure Rolling}
\label{\detokenize{ch/actions-reactions:pure-rolling}}\label{\detokenize{ch/actions-reactions:classical-mechanics-actions-reactions-contact-friction-pure-rolling}}
\sphinxAtStartPar
\sphinxstylestrong{todo} \sphinxstyleemphasis{add description}

\sphinxstepscope


\chapter{Inertia}
\label{\detokenize{ch/inertia:inertia}}\label{\detokenize{ch/inertia:classical-mechanics-inertia}}\label{\detokenize{ch/inertia::doc}}
\sphinxAtStartPar
Inertia deals with mass and mass distribution of systems.

\sphinxAtStartPar
\sphinxstylestrong{But what is mass?} Mass is a physical quantity, a property of the system, that manifests itself:
\begin{itemize}
\item {} 
\sphinxAtStartPar
in {\hyperref[\detokenize{ch/actions-examples:classical-mechanics-actions-gravitation}]{\sphinxcrossref{\DUrole{std,std-ref}{gravitational attraction}}}} (being both the origin of gravitational force and the property that makes a system sensible to gravitational attraction),

\item {} 
\sphinxAtStartPar
in resistance to change of motion of a system under external {\hyperref[\detokenize{ch/actions:classical-mechanics-actions}]{\sphinxcrossref{\DUrole{std,std-ref}{actions}}}}, as it will be clear from principles and equations of motions in {\hyperref[\detokenize{ch/dynamics:classical-mechanics-dynamics}]{\sphinxcrossref{\DUrole{std,std-ref}{dynamics}}}}

\end{itemize}

\sphinxAtStartPar
In the range of application of classical mechanics \sphinxstylestrong{mass conservation} holds, as stated by \sphinxstylestrong{Lavoisier principle}: the mass of a closed system is constant.

\sphinxAtStartPar
Beside mass, three main \sphinxstylestrong{additive dynamical quantities} are introduced: \sphinxstylestrong{momentum}, \sphinxstylestrong{angular momentum}, and \sphinxstylestrong{kinetic energy}. Even though their meaning could not be clear in this chapter, it would be clear in the following chapters, in the derivation of {\hyperref[\detokenize{ch/dynamics-eom:classical-mechanics-dynamics-eom-eom}]{\sphinxcrossref{\DUrole{std,std-ref}{equations of motion}}}} of mechanical systems, like {\hyperref[\detokenize{ch/dynamics-eom-point:classical-mechanics-dynamics-eom-point}]{\sphinxcrossref{\DUrole{std,std-ref}{point mass}}}}, {\hyperref[\detokenize{ch/dynamics-eom-points:classical-mechanics-dynamics-eom-points}]{\sphinxcrossref{\DUrole{std,std-ref}{system of point masses}}}} and {\hyperref[\detokenize{ch/dynamics-eom-rigid:classical-mechanics-dynamics-eom-rigid}]{\sphinxcrossref{\DUrole{std,std-ref}{rigid bodies}}}},…


\section{Point mass}
\label{\detokenize{ch/inertia:point-mass}}

\section{Discrete masses}
\label{\detokenize{ch/inertia:discrete-masses}}
\sphinxAtStartPar
\sphinxstylestrong{Momentum.}
\begin{equation*}
\begin{split}\vec{Q} := \sum_k m_k \vec{v}_k \end{split}
\end{equation*}
\sphinxAtStartPar
\sphinxstylestrong{Angular momentum.}
\begin{equation*}
\begin{split}\vec{L}_H := \int_{V_t} (P_k - H) \times m_k \vec{v}_k \end{split}
\end{equation*}
\sphinxAtStartPar
\sphinxstylestrong{Kinetic energy.}
\begin{equation*}
\begin{split}K := \sum_k \dfrac{1}{2} m_k |\vec{v}_k|^2\end{split}
\end{equation*}

\section{Continuous systems}
\label{\detokenize{ch/inertia:continuous-systems}}
\sphinxAtStartPar
\sphinxstylestrong{Momentum.}
\begin{equation*}
\begin{split}\vec{Q} := \int_{V_t} \rho \vec{v} \end{split}
\end{equation*}
\sphinxAtStartPar
\sphinxstylestrong{Angular momentum.}
\begin{equation*}
\begin{split}\vec{L}_H := \int_{V_t} (P - H) \times \rho \vec{v} \end{split}
\end{equation*}
\sphinxAtStartPar
\sphinxstylestrong{Kinetic energy.}
\begin{equation*}
\begin{split}K := \int_{V_t} \dfrac{1}{2} \rho |\vec{v}|^2\end{split}
\end{equation*}

\section{Rigid systems}
\label{\detokenize{ch/inertia:rigid-systems}}
\sphinxAtStartPar
The expression of dynamical quantities for rigid bodies can be written in terms of the velocity \(\vec{v}_Q\) of a point \(Q\) of the rigid body and its angular velocity \(\vec{\omega}\), exploiting the law of rigid motion \eqref{equation:ch/kinematics-rigid:eq:kin:rigid:vel} to write the velocity of each points of the rigid system as functions of \(\vec{v}_Q\) and \(\vec{\omega}\),
\begin{equation*}
\begin{split}\vec{v}_P = \vec{v}_Q + \vec{\omega} \times (P - Q) \ .\end{split}
\end{equation*}

\subsection{Discrete systems}
\label{\detokenize{ch/inertia:discrete-systems}}
\sphinxAtStartPar
\sphinxstylestrong{Momentum.}
\begin{equation*}
\begin{split}\begin{aligned}
  \vec{Q} = \sum_k m_k \vec{v}_k 
  & = \sum_k m_k \left( \vec{v}_Q + \vec{\omega} \times (P_k - Q) \right) = \\
  & = m \vec{v}_Q - \sum_k m_k (P_k - Q) \times \vec{\omega} = \\
  & = m \vec{v}_Q + \mathbb{S}_Q \cdot \vec{\omega} \ ,
\end{aligned}\end{split}
\end{equation*}
\sphinxAtStartPar
having defined the static moment of inertia (a \(2^{nd}\)\sphinxhyphen{}order antisymmetric tensor)
\begin{equation*}
\begin{split}\mathbb{S}_Q := \vec{s}_{P \times} := - \sum_k m_k (P_k - Q)_{\times} \ .\end{split}
\end{equation*}
\sphinxAtStartPar
\sphinxstylestrong{Angular momentum.}
\begin{equation*}
\begin{split}\begin{aligned}
  \vec{L}_H = \sum_k (P_k-H) \times  m_k \vec{v}_k 
  & = \underbrace{\sum_k (P_k-Q) \times m_k \vec{v}_k}_{\vec{L}_Q} + (Q - H) \times \vec{Q}
\end{aligned}\end{split}
\end{equation*}
\sphinxAtStartPar
and
\begin{equation*}
\begin{split}\begin{aligned}
\vec{L}_Q = \sum_k (P_k - Q) \times m_k \vec{v}_k 
 & = \sum_k (P_k - Q) \times m_k \left( \vec{v}_Q - (P_k - Q) \times \vec{\omega} \right) = \\
 & = \mathbb{S}_Q^T \cdot \vec{v}_Q + \mathbb{I}_Q \cdot \vec{\omega}  \ ,
\end{aligned}\end{split}
\end{equation*}
\sphinxAtStartPar
having recognized the transpose of the static moment of inertia, and introduced the tensor of inertia w.r.t. reference point \(Q\)
\begin{equation*}
\begin{split}\mathbb{I}_Q := - \sum_k m_k \left( P_k - Q \right)_{\times} \left( P_k - Q \right)_{\times} \ .\end{split}
\end{equation*}
\sphinxAtStartPar
\sphinxstylestrong{Kinetic energy.}
\begin{equation*}
\begin{split}\begin{aligned}
 K = \sum_k \dfrac{1}{2} m_k |\vec{v}_k|^2 
 & = \sum_k \dfrac{1}{2} m_k \left( \vec{v}_Q + \vec{\omega} \times (P_k - Q)  \right) \cdot \left( \vec{v}_Q + \vec{\omega} \times (P_k - Q) \right) = \\
 & = \sum_k \dfrac{1}{2} m_k | \vec{v}_Q |^2 + \dfrac{1}{2} \sum_k 2 m_k \vec{v}_Q \cdot \left( - (P_k - Q) \times \vec{\omega} \right) + \dfrac{1}{2} \sum_k \vec{\omega} \cdot (P_k - Q)_{\times} (P_k - Q)_\times \cdot \vec{\omega} = \\
 & = \dfrac{1}{2} \left[\sum_k m_k \right] | \vec{v}_Q |^2 
   + \dfrac{1}{2} \vec{v}_Q \cdot \left[ - \sum_k m_k (P_k - Q)_\times \right] \cdot \vec{\omega} + \\
 & + \dfrac{1}{2} \vec{\omega} \cdot \left[ \sum_k m_k (P_k - Q)_\times \right] \cdot \vec{v}_Q
   + \dfrac{1}{2} \vec{\omega} \cdot \left[ \sum_k m_k (P_k - Q)_{\times} (P_k - Q)_\times \right] \cdot \vec{\omega} = \\
 & = \dfrac{1}{2} m |\vec{v}_Q|^2 + \dfrac{1}{2} \vec{v}_Q \cdot \mathbb{S}_Q \cdot \vec{\omega} + \dfrac{1}{2} \vec{\omega} \cdot \mathbb{S}^T \cdot \vec{v}_Q + \dfrac{1}{2} \vec{\omega} \cdot \mathbb{I}_Q \cdot \vec{\omega} = \\
 & = \dfrac{1}{2} \vec{v}_Q \cdot \left[ m \vec{v}_Q + \mathbb{S}_Q \cdot \vec{\omega} \right] + \dfrac{1}{2} \vec{\omega} \cdot \left[  \mathbb{S}_Q^T \cdot \vec{v}_Q + \mathbb{I}_Q \cdot \vec{\omega} \right] = \\
 & = \dfrac{1}{2} \vec{v}_Q \cdot \vec{Q} + \dfrac{1}{2} \vec{\omega} \cdot \vec{L}_Q \ .
\end{aligned}\end{split}
\end{equation*}
\sphinxAtStartPar
having used the vector identities
\begin{equation*}
\begin{split}\vec{a} \cdot \vec{b} \times \vec{c} = \vec{b} \cdot \vec{c} \times \vec{a}\end{split}
\end{equation*}\begin{equation*}
\begin{split}\vec{a} \times \vec{b} = - \vec{b} \times \vec{a}\end{split}
\end{equation*}
\sphinxAtStartPar
and having introduced the expression of momentum and angular momentum in the last step.

\begin{sphinxadmonition}{note}{Note:}
\sphinxAtStartPar
The components of static moment of inertia and tensor of inertia in a material basis \sphinxhyphen{} following the motion of the system \sphinxhyphen{} are constant.
\end{sphinxadmonition}


\subsection{Continuous systems}
\label{\detokenize{ch/inertia:id1}}
\sphinxAtStartPar
\sphinxstylestrong{Momentum.}
\begin{equation*}
\begin{split}\begin{aligned}
  \vec{Q} = \int_{V_t} \rho \vec{v}
  & = \int_{V_t} \rho \left( \vec{v}_Q + \vec{\omega} \times (P - Q) \right) = \\
  & = m \vec{v}_Q - \int_{V_t} \rho (P_k - Q) \times \vec{\omega} = \\
  & = m \vec{v}_Q + \mathbb{S}_Q \cdot \vec{\omega} \ ,
\end{aligned}\end{split}
\end{equation*}
\sphinxAtStartPar
having defined the static moment of inertia (a \(2^{nd}\)\sphinxhyphen{}order antisymmetric tensor) for continuous systems,
\begin{equation*}
\begin{split}\mathbb{S}_Q := \vec{s}_{P \times} := - \int_{V_t} \rho (P - Q)_{\times} \ .\end{split}
\end{equation*}
\begin{sphinxadmonition}{note}{Note:}
\sphinxAtStartPar
Using the definition of the center of mass \(G\),
\begin{equation*}
\begin{split}G = \dfrac{1}{m} \int_{V_t} \rho P \ ,\end{split}
\end{equation*}
\sphinxAtStartPar
the static moment of inertia can be written as
\begin{equation*}
\begin{split}\mathbb{S}_Q = - \int_{V_t} \rho (P-Q)_{\times} = - m (G - Q)_{\times} \ .\end{split}
\end{equation*}\end{sphinxadmonition}

\begin{sphinxadmonition}{note}{Note:}
\sphinxAtStartPar
The components of static moment of inertia w.r.t. a material reference frame are constant. Using a material Carteisan reference frame the tensor reads
\begin{equation*}
\begin{split}\begin{aligned}
\mathbb{S}_{Q} 
 & = - \int_{V_t} \rho (P - Q)_{\times} = \\
 & = - \int_{V_t} \rho \left[ ( x^0 - x_Q^0 ) \hat{x}^0 +  ( y^0 - y_Q^0 ) \hat{y}^0 +  ( z^0 - z_Q^0 ) \hat{z}^0 \right]_\times
 = S_{ij} \hat{e}^0_i \hat{e}^0_j \ ,
\end{aligned}\end{split}
\end{equation*}
\sphinxAtStartPar
whose components can be collected in the \sphinxstylestrong{antisymmetric matrix}
\begin{equation*}
\begin{split}\underline{\underline{S}}_Q = \left[ S_{Q,ij} \right] = - \int_{V_0} \rho \begin{bmatrix} 0 & -(z^0-z^0_Q) & (y^0-y^0_Q) \\ (z^0 - z_Q^0) & 0 & -(x^0-x_Q^0) \\ -(y^0-y^0_Q) & (x^0-x^0_Q) & 0 \end{bmatrix} \ ,\end{split}
\end{equation*}
\sphinxAtStartPar
so that the vector product between \(-\int_V \rho (P-Q)\) and a vector \(\vec{a}\) reads
\begin{equation*}
\begin{split}\begin{aligned}
  - \int_{V} \rho (P-Q) \times a
  & = - \int_V \rho \left[ \hat{x}^0 \left( \Delta y^0 a_z - \Delta z^0 a_y \right) + \hat{y}^0 \left( \Delta z^0 a_x - \Delta x^0 a_z \right) + \hat{z}^0 \left( \Delta x^0 a_y - \Delta y^0 a_x  \right) \right] \\
  & = \mathbb{S} \cdot \vec{a} \ .
\end{aligned}\end{split}
\end{equation*}\end{sphinxadmonition}

\sphinxAtStartPar
\sphinxstylestrong{Angular momentum.}
\begin{equation*}
\begin{split}\begin{aligned}
  \vec{L}_H = \int_{V_t} (P-H) \times  \rho \vec{v}
  & = \underbrace{\int_{V_t}(P-Q) \times \rho \vec{v}_k}_{\vec{L}_Q} + (Q - H) \times \vec{Q}
\end{aligned}\end{split}
\end{equation*}
\sphinxAtStartPar
and
\begin{equation*}
\begin{split}\begin{aligned}
\vec{L}_Q = \int_{V_t} (P - Q) \times \rho \vec{v}
 & = \int_{V_t} (P - Q) \times \rho \left( \vec{v}_Q - (P - Q) \times \vec{\omega} \right) = \\
 & = \mathbb{S}_Q^T \cdot \vec{v}_Q + \mathbb{I}_Q \cdot \vec{\omega}  \ ,
\end{aligned}\end{split}
\end{equation*}
\sphinxAtStartPar
having recognized the transpose of the static moment of inertia, and introduced the tensor of inertia w.r.t. reference point \(Q\)
\begin{equation*}
\begin{split}\begin{aligned}
  \mathbb{I}_Q
  & := - \int_{V_t} \rho \left( P - Q \right)_{\times} \left( P - Q \right)_{\times}  = \\
  & := \int_{V_t} \rho \left[ |P-Q|^2 \mathbb{I} - (P-Q) \otimes (P-Q) \right] \ ,
\end{aligned}\end{split}
\end{equation*}
\sphinxAtStartPar
having used the tensor identity
\begin{equation*}
\begin{split}- \vec{a}_{\times} \cdot \vec{a}_{\times} = |\vec{a}|^2 \mathbb{I} - \vec{a} \otimes \vec{a}\end{split}
\end{equation*}
\begin{sphinxadmonition}{note}{Note:}
\sphinxAtStartPar
The components of tensor of inertia w.r.t. a material reference frame are constant. Using a material Carteisan reference frame the tensor reads
\begin{equation*}
\begin{split}\begin{aligned}
\mathbb{I}_{Q} 
 & = - \int_{V_t} \rho (P - Q)_{\times} (P-Q)_{\times} = \\
 & = - \int_{V_t} \rho \left[ |P-Q|^2 \mathbb{I} - (P-Q) \otimes (P-Q) \right] = \\
 & = I^0_{Q,ij} \hat{e}^0_i \hat{e}^0_j \ ,
\end{aligned}\end{split}
\end{equation*}
\sphinxAtStartPar
whose components can be collected in the \sphinxstylestrong{symmetric matrix}
\begin{equation*}
\begin{split}\underline{\underline{I}}^0_Q = \left[ I^0_{Q,ij} \right] = \int_{V_0} \rho \begin{bmatrix} \Delta y_0^2 + \Delta z_0^2 & -\Delta x_0 \Delta y_0 & -\Delta x_0 \Delta z_0 \\ -\Delta y_0 \Delta x_0 & \Delta x_0^2 + \Delta y_0^2 & -\Delta y_0 \Delta z_0 \\ -\Delta z_0 \Delta x_0 & - \Delta z_0 \Delta y_0 & \Delta x_0^2 + \Delta z_0^2 \end{bmatrix} \ ,\end{split}
\end{equation*}
\sphinxAtStartPar
being \(\Delta x^0 := x^0_P - x_Q^0\).
\end{sphinxadmonition}


\subsection{Properties of inertia tensors of rigid bodies}
\label{\detokenize{ch/inertia:properties-of-inertia-tensors-of-rigid-bodies}}

\subsubsection{Static inertia}
\label{\detokenize{ch/inertia:static-inertia}}
\sphinxAtStartPar
\sphinxstylestrong{Center of mass, \(G\).} Center of mass of of a rigid body is defined as the point \(G\) for which \(\mathbb{S}_G \equiv \mathbb{0}\), whose coordinates are given by
\begin{equation*}
\begin{split}G = \dfrac{1}{m} \int_{V_t} \rho \, P\end{split}
\end{equation*}
\sphinxAtStartPar
\sphinxstylestrong{Anti\sphinxhyphen{}symmetric.} From the definition of the static inertia tensor
\begin{equation*}
\begin{split}\mathbb{S}_Q \cdot \vec{a} = \int_{V_t} \rho (P-Q) \times \vec{a} = - \vec{a} \times \int_{V_t} \rho (P-Q) = - \vec{a} \cdot \mathbb{S}_Q = - \mathbb{S}^T \cdot \vec{a} \ .\end{split}
\end{equation*}
\sphinxAtStartPar
\sphinxstylestrong{Transport.}
\begin{equation*}
\begin{split}\begin{aligned}
  \mathbb{S}_Q 
  & = - \int_{V_t} \rho (P - Q)_{\times} = \\
  & = - \int_{V_t} \rho (P - R)_{\times} - \int_{V_t} \rho (R - Q)_{\times} = \\
  & = \mathbb{S}_R - m (R-Q)_{\times} \ ,
\end{aligned}\end{split}
\end{equation*}
\sphinxAtStartPar
or w.r.t. the center of mass \(G\),
\begin{equation*}
\begin{split}\mathbb{S}_Q = \mathbb{S}_G - m (G-Q)_{\times} \ .\end{split}
\end{equation*}

\subsubsection{Tensor of inertia}
\label{\detokenize{ch/inertia:tensor-of-inertia}}
\sphinxAtStartPar
\sphinxstylestrong{Symmetric (semi)\sphinxhyphen{}definite positive.} Inertia tensor is symmetric
\begin{equation*}
\begin{split}\vec{v} \cdot \mathbb{I}_Q \cdot \vec{w} = \vec{v} \cdot \int_{V_t} \rho \left[ |\Delta \vec{r}|^2 \mathbb{I} - \Delta \vec{r} \otimes \Delta \vec{r} \right] \cdot \vec{w} = \vec{w} \cdot \mathbb{I}_Q \cdot \vec{v} \ .\end{split}
\end{equation*}
\sphinxAtStartPar
for all \(\forall \vec{v}, \vec{w}\), and semi\sphinxhyphen{}definite positive
\begin{equation*}
\begin{split}\begin{aligned}
  \vec{v} \cdot \mathbb{I}_Q \cdot \vec{v} 
  & = - \vec{v} \cdot \int_{V_t} \rho \Delta \vec{r}_{\times} \Delta \vec{r}_{\times} \cdot \vec{v} = \\
  & = - \int_{V_t} \rho \vec{v} \cdot \Delta \vec{r}_{\times} \Delta \vec{r}_{\times} \cdot \vec{v} = \\
  & = - \int_{V_t} \rho \vec{v} \cdot \left[ \Delta \vec{r} \times \left( \Delta \vec{r} \times \cdot \vec{v} \right) \right] = \\
  & = - \int_{V_t} \rho \left( \Delta \vec{r} \times \vec{v} \right) \cdot \left( \vec{v} \times  \Delta \vec{r} \right) = \\
  & =   \int_{V_t} \rho \left( \Delta \vec{r} \times \vec{v} \right) \cdot \left( \Delta \vec{r} \times \vec{v} \right) = \\
  & =   \int_{V_t} \rho | \Delta \vec{r} \times \vec{v} |^2 \ge 0 \\
\end{aligned}\end{split}
\end{equation*}
\sphinxAtStartPar
\sphinxstylestrong{Principal axes of inertia.} As the tensor of inertia is symmetric and definite positive, a set of orthogonal vectors \(\hat{E}^0_i\) so that it can be written in diagonal form,
\begin{equation*}
\begin{split}\mathbb{I}_Q = I^0_{XX} \, \hat{E}^0_X \otimes \hat{E}^0_X +  I^0_{YY} \, \hat{E}^0_Y \otimes \hat{E}^0_Y + I^0_{ZZ} \, \hat{E}^0_Z \otimes \hat{E}^0_Z \ ,\end{split}
\end{equation*}
\sphinxAtStartPar
with \(I_{ii}^0 \ge 0\) (no sum).
\label{ch/inertia:thm:huygens}
\begin{sphinxadmonition}{note}{Theorem 3.4.1 (Transport \sphinxhyphen{} Huygens’ theorem.)}


\begin{equation*}
\begin{split}\begin{aligned}
  \mathbb{I}_Q 
  & = - \int_{V_t} \rho (P - Q)_{\times} (P-Q)_{\times} = \\
  & =    - \int_{V_t} \rho (P - R)_{\times} (P - R)_{\times}     
         - \int_{V_t} \rho (P - R)_{\times} (R - Q)_{\times} \\  
& \quad  - \int_{V_t} \rho (R - Q)_{\times} (P - R)_{\times}     
         - \int_{V_t} \rho (R - Q)_{\times} (R - Q)_{\times} = \\ 
  & = \mathbb{I}_R + \mathbb{S}_R \cdot (R-Q)_{\times} + (R-Q)_{\times} \cdot \mathbb{S}_R - m (R - Q)_{\times} (R - Q)_{\times}
\end{aligned}\end{split}
\end{equation*}
\sphinxAtStartPar
or w.r.t. the center of mass \(G\),
\begin{equation*}
\begin{split} \mathbb{I}_Q = \mathbb{I}_G - m (Q-G)_\times (Q-G)_\times \ .\end{split}
\end{equation*}\end{sphinxadmonition}


\subsection{Time derivatives of dynamical quantities}
\label{\detokenize{ch/inertia:time-derivatives-of-dynamical-quantities}}
\sphinxAtStartPar
Time derivatives of dynamical quantities are easily evaluated using a Cartesian material reference frame.

\sphinxAtStartPar
\sphinxstylestrong{Momentum.}
\begin{equation*}
\begin{split}\begin{aligned}
\dfrac{d}{dt} \vec{Q} 
  & = \dfrac{d}{dt} \left( m \vec{v}_Q + \mathbb{S}_Q \cdot \vec{\omega} \right) =  \\
  & =  m \dot{\vec{v}}_Q + \dfrac{d}{dt} \left( \vec{E}^0_i S^0_{ij} \omega^0_j \right) = \\
  & =  m \dot{\vec{v}}_Q + \dfrac{d\vec{E}^0_i }{dt} S^0_{ij} \omega^0_j + \vec{E}^0_i S^0_{ij} \dfrac{d} \omega^0_j{dt} = \\
  & =  m \dot{\vec{v}}_Q + \vec{\omega} \times \vec{E}^0_i S^0_{ij} \omega^0_j + \vec{E}^0_i S^0_{ij} \dfrac{d}{dt}\omega^0_j = \\
  & =  m \dot{\vec{v}}_Q + \vec{\omega} \times \left( \mathbb{S}_Q \cdot \vec{\omega} \right) + \mathbb{S}_Q \cdot \dot{\vec{\omega}} \ .
\end{aligned}\end{split}
\end{equation*}\begin{equation*}
\begin{split}\begin{aligned}
 \dfrac{d}{dt} \vec{\omega} 
 & = \dfrac{d}{dt} \left( \hat{E}^0_i \omega_i^0 \right) = \\
 & = \vec{\omega} \times \hat{E}^0_i \omega_i^0 + \hat{E}^0_i \dfrac{d \omega^0_i}{dt} = \\
 & = \underbrace{\vec{\omega} \times \vec{\omega}}_{=\vec{0}} + \hat{E}^0_i \dfrac{d \omega^0_i}{dt} \ .
\end{aligned}\end{split}
\end{equation*}\begin{equation*}
\begin{split}\begin{aligned}
 \dfrac{d}{dt} \vec{v} 
 & = \dfrac{d}{dt} \left( \hat{E}^0_i v_i^0 \right) = \\
 & = \vec{\omega} \times \hat{E}^0_i v_i^0 + \hat{E}^0_i \dfrac{d v^0_i}{dt} = \\
 & = \vec{\omega} \times \vec{v}  + \dfrac{{}^0 d}{dt} \vec{v} \\
\end{aligned}\end{split}
\end{equation*}
\sphinxAtStartPar
\sphinxstylestrong{Angular momentum.}
\begin{equation*}
\begin{split}\begin{aligned}
\dfrac{d}{dt} \vec{L}_H
& =  \dfrac{d}{dt} \left( (Q-H) \times \vec{Q} + \vec{L}_Q  \right) \ ,
\end{aligned}\end{split}
\end{equation*}
\sphinxAtStartPar
and
\begin{equation*}
\begin{split}\dfrac{d}{dt} \left( (Q-H) \times \vec{Q} \right) = ( \vec{v}_Q - \dot{\vec{x}}_H ) \times \vec{Q} + ( Q - H ) \times \dot{\vec{Q}}\end{split}
\end{equation*}
\sphinxAtStartPar
and
\begin{equation*}
\begin{split}\begin{aligned}
 \dfrac{d \vec{L}_Q}{dt} 
 & = \dfrac{d}{dt} \left( \mathbb{S}^T_Q \cdot \vec{v}_Q + \mathbb{I}_Q \cdot \vec{\omega} \right) = \\
 & = \dfrac{d}{dt} \left[ \hat{E}_i^0 \left( S^0_{ji} v^0_{Q,j} + I^0_{ij} \omega^0_j \right) \right] = \\
 & = \vec{\omega} \times \hat{E}_i^0 \left( S^0_{ji} v^0_{Q,j} + I^0_{ij} \omega^0_j \right)
   + \hat{E}^0_i \left( S_{ji}^0 \dot{v}^0_{Q,j} + I^0_{ij} \dot{\omega}^0_j \right) = \\
 & = \vec{\omega} \times ( \mathbb{S}_Q^T \cdot \vec{v}_Q + \mathbb{I}_Q \cdot \vec{\omega} )
   + \left( \mathbb{S}_Q^T \cdot \dfrac{{}^0 d}{dt} \vec{v}_Q + \mathbb{I} \cdot \dot{\vec{\omega}} \right) = \\
 & = \vec{\omega} \times ( \mathbb{S}_Q^T \cdot \vec{v}_Q + \mathbb{I}_Q \cdot \vec{\omega} )
   + \left( \mathbb{S}_Q^T \cdot \left( \dot{v}_Q - \vec{\omega} \times \vec{v}_Q \right) + \mathbb{I} \cdot \dot{\vec{\omega}} \right) \\
 & = \left( \mathbb{S}_Q^T \cdot \left( \dot{v}_Q - \vec{\omega} \times \vec{v}_Q \right) + \mathbb{I} \cdot \dot{\vec{\omega}} \right) 
    + \vec{\omega} \times ( \mathbb{S}_Q^T \cdot \vec{v}_Q + \mathbb{I}_Q \cdot \vec{\omega} )
\end{aligned}\end{split}
\end{equation*}

\subsubsection{Dynamical quantities and time derivatives with \protect\(G\protect\) as reference point}
\label{\detokenize{ch/inertia:dynamical-quantities-and-time-derivatives-with-g-as-reference-point}}\begin{equation*}
\begin{split}
\begin{cases}
  \vec{Q} = m \vec{v}_G \\
  \vec{L}_G = \mathbb{I}_G \cdot \vec{\omega}
\end{cases}
\qquad , \qquad
\begin{cases}
  \dot{\vec{Q}} = m \dot{\vec{v}}_G \\
  \dot{\vec{L}}_G = \mathbb{I}_G \cdot \dot{\vec{\omega}} + \vec{\omega} \times \mathbb{I}_G \cdot \vec{\omega}
\end{cases}\end{split}
\end{equation*}
\sphinxstepscope




\chapter{Dynamics}
\label{\detokenize{ch/dynamics:dynamics}}\label{\detokenize{ch/dynamics:classical-mechanics-dynamics}}\label{\detokenize{ch/dynamics::doc}}
\sphinxAtStartPar
Dynamics provides the link between the motion of a body, described by {\hyperref[\detokenize{ch/kinematics:classical-mechanics-kinematics}]{\sphinxcrossref{\DUrole{std,std-ref}{kinematics}}}}, and the \DUrole{xref,myst}{actions} causing that motion.

\sphinxAtStartPar
Newton’s principles of dynamics and the cardinal equations of dynamics are the physical laws that govern the motion of mechanical systems: {\hyperref[\detokenize{ch/dynamics-principles:classical-mechanics-dynamics-principles}]{\sphinxcrossref{\DUrole{std,std-ref}{Newton’s principles}}}} agree with the experimental observations (for systems with negligible quantum and Einstein relativity effects) and are the starting point \sphinxhyphen{} the principles \sphinxhyphen{} of Newton’s formulation of mechanics; from these principles, {\hyperref[\detokenize{ch/dynamics-eom:classical-mechanics-dynamics-eom}]{\sphinxcrossref{\DUrole{std,std-ref}{equations of motion}}}} of mechanical systems are derived.
These physical laws are formulated in terms of certain physical quantities, such as momentum, angular momentum, or the kinetic energy of the system \sphinxhyphen{} already discussed in the section about {\hyperref[\detokenize{ch/inertia:classical-mechanics-inertia}]{\sphinxcrossref{\DUrole{std,std-ref}{inertia}}}}. These dynamic quantities have the property of being additive (by definition), and making it particularly easy to write and interpret a general form of the equations governing motion. In general, these equations relate the time derivatives of these dynamic quantities to the causes of their variation. In the absence of net causes, conservation principles hold.



\sphinxstepscope


\section{Principles of Newtonian Mechanics}
\label{\detokenize{ch/dynamics-principles:principles-of-newtonian-mechanics}}\label{\detokenize{ch/dynamics-principles:classical-mechanics-dynamics-principles}}\label{\detokenize{ch/dynamics-principles::doc}}
\sphinxAtStartPar
Newton’s principle of dynamics are first established for \sphinxstylestrong{closed systems}, i.e. systems that don’t exchange mass with the external environment and thus have constant mass.
Balance equations for mass, momentum, angular momentum and other quantities for arbitary systems (closed or open) are derived in \sphinxhref{https://basics2022.github.io/bbooks-physics-continuum-mechanics/intro.html}{Continuum Mechanics}:\sphinxhref{https://basics2022.github.io/bbooks-physics-continuum-mechanics/ch/continuum/governing-equations.html}{Governing Equations}, exploiting \sphinxhref{https://basics2022.github.io/bbooks-math-miscellanea/ch/tensor-algebra-calculus/time-derivative-of-integrals.html\#volume-density}{Reynolds’ transport theorem}.%
\begin{footnote}[1]\sphinxAtStartFootnote
Reynolds’s transport theorem provides the rules for time derivatives of integrals on arbitrary domains in Euclidean space, that allows to change representation to and from closed systems (material volume, a geometric volume moving with the physical system) and open systems (usually control volumes of some regions of the system that can exchage mass with the exterrnal environment).
%
\end{footnote}

\sphinxAtStartPar
\sphinxstylestrong{Mass conservation.} In the regime of classical mechanics, Lavoisier principle states that the mass of a closed system is constant. Roughly speaking “nothing is created nothing is destroyed”.

\sphinxAtStartPar
\sphinxstylestrong{First Principle of Dynamics (Newton’s First Law): inertia and Galileian invariance.} A body (more precisely, the center of mass of a body) on which no net force acts remains in its state of rest or uniform rectilinear motion relative to an {\hyperref[\detokenize{ch/dynamics-principles:classical-mehcanics-dynamics-principles-inertial-ref-frame}]{\sphinxcrossref{\DUrole{std,std-ref}{inertial reference frame}}}}.



\sphinxAtStartPar
\sphinxstylestrong{Second Principle of Dynamics (Newton’s Second Law): momentum balance.} Relative to an {\hyperref[\detokenize{ch/dynamics-principles:classical-mehcanics-dynamics-principles-inertial-ref-frame}]{\sphinxcrossref{\DUrole{std,std-ref}{inertial reference frame}}}}, the change in momentum of a system is equal to the impulse of \sphinxstyleemphasis{true external forces} (see {\hyperref[\detokenize{ch/actions:def:true-forces}]{\sphinxcrossref{Definition 2.2}}}) acting on it,
\begin{equation*}
\begin{split}\Delta \vec{Q} = \vec{I}^e \ .\end{split}
\end{equation*}
\sphinxAtStartPar
In the case of smooth motion, where the momentum can be represented as a continuous and differentiable function of time, the second principle of dynamics can be expressed in differential form,
\begin{equation*}
\begin{split}\dot{\vec{Q}} = \vec{R}^e \ ,\end{split}
\end{equation*}
\sphinxAtStartPar
where the resultant of the external forces, \(\vec{R}^e = \frac{d \vec{I}^e}{dt}\), is the time derivative of the impulse.

\sphinxAtStartPar
\sphinxstylestrong{Third Principle of Dynamics (Newton’s Third Law): action\sphinxhyphen{}reaction.} If a system \(i\) exerts a force \(\vec{F}_{ji}\) on a system \(j\), then system \(j\) exerts an “equal and opposite” force \(\vec{F}_{ij}\) on system \(i\), with equal magnitude and opposite direction,
\begin{equation*}
\begin{split}\vec{F}_{ij} = - \vec{F}_{ji} \ .\end{split}
\end{equation*}

\subsection{Inertial reference frame}
\label{\detokenize{ch/dynamics-principles:inertial-reference-frame}}\label{\detokenize{ch/dynamics-principles:classical-mehcanics-dynamics-principles-inertial-ref-frame}}
\sphinxAtStartPar
Force sensors in an inertial reference frame measure only \sphinxstyleemphasis{true forces}, as defined in {\hyperref[\detokenize{ch/actions:def:true-forces}]{\sphinxcrossref{Definition 2.2}}}.



\sphinxAtStartPar
Here the effect of non\sphinxhyphen{}inertial reference frame in the description of the motion is shown on the \sphinxstylestrong{dynamics of a point system}, using \(2^{nd}\) principle for the description of the dynamics for an inertial observer, and the rules for the {\hyperref[\detokenize{ch/kinematics-relative:classical-mechanics-kinematics-relative-points}]{\sphinxcrossref{\DUrole{std,std-ref}{relative kinematics of a point}}}}.

\sphinxAtStartPar
Here, an inertial reference frame is referred as 0, while the generic reference frame is referred as 1. The equation of motion of a point system \(P\) w.r.t. the inertial reference frame is governed by Newton’s second principle,
\begin{equation*}
\begin{split}m \vec{a}^{0}_{P/O_0} = \vec{F} \ ,\end{split}
\end{equation*}
\sphinxAtStartPar
being \(\vec{a}^0_{P/O_0}\) the acceleration of point \(P\) w.r.t. the origin \(O_0\) of the inertial reference frame \(0\), as seen by the same reference frame \(0\) (the meaning of the apex).
From relative kinematics, \eqref{equation:ch/kinematics-relative:eq:relative:point:acc}
\begin{equation*}
\begin{split}\begin{aligned}
  m \vec{a}^{1}_{P/O_1} & =  \vec{F}  + \\
   & \quad - m \left[ \vec{a}^0_{O_1/O_0} + \vec{\alpha}_{1/0} \times (P - O_1) + 2 \vec{\omega}_{1/0} \times \vec{v}^1_{P/O_1} + \vec{\omega}_{1/0} \times [ \, \vec{\omega}_{1/0} \times (P - O_1) \right]
\end{aligned}\end{split}
\end{equation*}
\sphinxAtStartPar
In order for the reference frame \(1\) to agree with the acceleration of any point \(P\) with the inertial reference frame \(0\), the content of the square brackets must vanish, and thus:
\begin{itemize}
\item {} 
\sphinxAtStartPar
\(\vec{a}^0_{O_1/O_0} = \vec{0}\), i.e. the acceleration of the origin of the reference frame \(O_1\) w.r.t. to the reference frame \(0\) must be zero

\item {} 
\sphinxAtStartPar
\(\vec{\omega}_{1/0} = \vec{0}\), and thus implying \(\vec{\alpha}_{1/0}= \vec{0}\), i.e. the angular velocity of the reference frame \(1\) w.r.t. the reference frame \(0\) are zero.

\end{itemize}

\sphinxAtStartPar
Thus, reference frame \(1\) agrees with the measurements of accelerations and forces of an inertial reference frame if:
\begin{itemize}
\item {} 
\sphinxAtStartPar
its origin \(O_1\) is in uniform motion w.r.t. \(O_0\)

\item {} 
\sphinxAtStartPar
its orientation \(\mathbb{R}^{1/0}\) is constant in time, so that angular velocity and acceleration are identically zero.

\end{itemize}
\label{ch/dynamics-principles:definition-0}
\begin{sphinxadmonition}{note}{Definition 4.1.1 (Class of inertial reference frames)}



\sphinxAtStartPar
If these conditions hold, it’s not possible to find which reference frame is “more absolute” than the other, but all the reference frames with constant relative orientation and origin in uniform relative motion are inertial reference frames, w.r.t. which the governing equations have the same expression.
\end{sphinxadmonition}


\bigskip\hrule\bigskip


\sphinxstepscope


\section{Equations of Motion and Conservation Principles}
\label{\detokenize{ch/dynamics-eom:equations-of-motion-and-conservation-principles}}\label{\detokenize{ch/dynamics-eom:classical-mechanics-dynamics-eom}}\label{\detokenize{ch/dynamics-eom::doc}}
\sphinxAtStartPar
Starting from the {\hyperref[\detokenize{ch/dynamics-principles:classical-mechanics-dynamics-principles}]{\sphinxcrossref{\DUrole{std,std-ref}{principles of Newtonian mechanics}}}}, it is possible to derive the dynamical equations governing the motion of mechanical systems. These equations governs the change of dynamical quantities, \sphinxstylestrong{momentum}, \sphinxstylestrong{angular momentum}, \sphinxstylestrong{kinetic energy}, linking them to (external) \sphinxstylestrong{forces}, (external) \sphinxstylestrong{moments} and (total) \sphinxstylestrong{power}. Under certain conditions, and only in these cases, the {\hyperref[\detokenize{ch/dynamics-eom:classical-mechanics-dynamics-eom-eom}]{\sphinxcrossref{\DUrole{std,std-ref}{cardinal equations}}}} of dynamics become{\hyperref[\detokenize{ch/dynamics-eom:classical-mechanics-dynamics-eom-conservation}]{\sphinxcrossref{\DUrole{std,std-ref}{principles of conservation of dynamic quantities}}}}: by observing the expressions of the cardinal equations, it is easy to infer that the condition to obtain a conservation principle is the vanishing of all terms except for the time derivative of the conserved quantity.




\subsection{Equations of Motion}
\label{\detokenize{ch/dynamics-eom:equations-of-motion}}\label{\detokenize{ch/dynamics-eom:classical-mechanics-dynamics-eom-eom}}
\sphinxAtStartPar
The general form of these equations is easily expressed in terms of the dynamical quantities discussed in the section about \DUrole{xref,myst}{inertia}. Cardinal equations, or equations of motion, are collected here in their most general form for closed systems, and derived in the following sectinos for different systems: {\hyperref[\detokenize{ch/dynamics-eom-point:classical-mechanics-dynamics-eom-point}]{\sphinxcrossref{\DUrole{std,std-ref}{point mass}}}}, {\hyperref[\detokenize{ch/dynamics-eom-points:classical-mechanics-dynamics-eom-points}]{\sphinxcrossref{\DUrole{std,std-ref}{systems of point masses}}}}, {\hyperref[\detokenize{ch/dynamics-eom-rigid:classical-mechanics-dynamics-eom-rigid}]{\sphinxcrossref{\DUrole{std,std-ref}{rigid body}}}},…

\sphinxAtStartPar
\sphinxstylestrong{Momentum Balance.} The time derivative of the momentum is equal to the resultant of the external forces,
\begin{equation}\label{equation:ch/dynamics-eom:principle:q}
\begin{split}\dot{\vec{Q}} = \vec{R}^e \ .\end{split}
\end{equation}
\sphinxAtStartPar
\sphinxstylestrong{Angular Momentum Balance with respect to a point \(H\).} The time derivative of the angular momentum with respect to a point \(H\), minus the “transport term,” is equal to the resultant of the external moments with respect to the point \(H\),
\begin{equation}\label{equation:ch/dynamics-eom:principle:l}
\begin{split}\dot{\vec{L}}_H + \dot{\vec{x}}_H \times \vec{Q} = \vec{M}_H^e \ .\end{split}
\end{equation}
\sphinxAtStartPar
\sphinxstylestrong{Kinetic Energy Balance.} The time derivative of the kinetic energy is equal to the total power acting on the system, which is the sum of the power of the external actions and the power of the internal actions within the system,
\begin{equation}\label{equation:ch/dynamics-eom:principle:k}
\begin{split}\dot{K} = P^{tot} = P^e + P^i\end{split}
\end{equation}

\subsection{Conservation Principles}
\label{\detokenize{ch/dynamics-eom:conservation-principles}}\label{\detokenize{ch/dynamics-eom:classical-mechanics-dynamics-eom-conservation}}
\sphinxAtStartPar
Under certain condiitons, equations of motion become principles of conservation of dynamical quantities. These conditions are easily derived by inspection of the equations of motion, nullyfing the causes of change of the dynamical quantities. Beside the conservation of momentum, angular momentum and kinetic energy, a \sphinxstylestrong{principle of conservation of mechanical energy} arises when actions acting on the system are {\hyperref[\detokenize{ch/actions-conservative:classical-mechanics-actions-conservative}]{\sphinxcrossref{\DUrole{std,std-ref}{conservative}}}}, so that its power can be written as a time derivative of a potential energy.

\sphinxAtStartPar
\sphinxstylestrong{Conservation of Momentum in the presence of zero net external forces.} If the resultant of the external forces is zero, \(\vec{R}^e = \vec{0}\), from the momentum balance, we immediately obtain
\begin{equation*}
\begin{split}\dot{\vec{Q}} = \vec{0} \qquad \rightarrow \qquad \vec{Q} = \bar{\vec{Q}} = \text{const.}\end{split}
\end{equation*}
\sphinxAtStartPar
\sphinxstylestrong{Conservation of Angular Momentum in the presence of zero net external moments.} If the choice of the point \(H\) nullifies the transport term, \(\dot{\vec{r}}_H \times \vec{Q} = \vec{0}\), and if the resultant of the external moments is zero, \(\vec{M}^e_H = \vec{0}\), from the angular momentum balance, we immediately obtain
\begin{equation*}
\begin{split}\dot{\vec{L}}_H = \vec{0} \qquad \rightarrow \qquad \vec{L}_H = \bar{\vec{L}}_H = \text{const.}\end{split}
\end{equation*}
\sphinxAtStartPar
\sphinxstylestrong{Conservation of Kinetic Energy in the presence of zero total power.} If the total power of the actions on the system is zero, \(P^{tot} = 0\), from the kinetic energy balance, we immediately obtain
\begin{equation*}
\begin{split}\dot{K} = 0 \qquad \rightarrow \qquad K = \bar{K} = \text{const.}\end{split}
\end{equation*}
\sphinxAtStartPar
\sphinxstylestrong{Conservation of Mechanical Energy in the absence of non\sphinxhyphen{}conservative forces.} In addition to the three conservation principles directly derived from the cardinal equations, we add the principle of the conservation of mechanical energy, which is the sum of the system’s kinetic and potential energy,
\begin{equation*}
\begin{split}E^{mech} = K + V \ ,\end{split}
\end{equation*}
\sphinxAtStartPar
in the absence of non\sphinxhyphen{}conservative actions. If there are no non\sphinxhyphen{}conservative forces, the power of the actions on the system can be written as the negative of the time derivative of the system’s potential energy,
\begin{equation*}
\begin{split}P^{tot} = -\dot{V}\end{split}
\end{equation*}
\sphinxAtStartPar
From the kinetic energy balance, we get
\begin{equation*}
\begin{split}\dot{K} = - \dot{V} \qquad \rightarrow \qquad \dfrac{d}{dt}(K+V) = 0 \qquad \rightarrow \qquad \dot{E}^{mech} = 0 \qquad \rightarrow \qquad E^{mech} = \bar{E}^{mech} = \text{const.}\end{split}
\end{equation*}
\sphinxstepscope


\section{Equations of motion of a point mass}
\label{\detokenize{ch/dynamics-eom-point:equations-of-motion-of-a-point-mass}}\label{\detokenize{ch/dynamics-eom-point:classical-mechanics-dynamics-eom-point}}\label{\detokenize{ch/dynamics-eom-point::doc}}
\sphinxAtStartPar
\sphinxstylestrong{Dynamic quantities.}
\begin{equation*}
\begin{split}\begin{aligned}
  \vec{Q}_P & := m_P \vec{v}_P \\
  \vec{L}_{P,H} & := (\vec{r}_P - \vec{r}_H) \times \vec{Q} = m_P (\vec{r}_P - \vec{r}_H) \times \vec{v}_P \\
  K & := \frac{1}{2} m_P \vec{v}_P \cdot \vec{v}_P = \frac{1}{2} m_P |\vec{v}_P|^2
\end{aligned}\end{split}
\end{equation*}
\sphinxAtStartPar
\sphinxstylestrong{Momentum balance equation.} The balance equation of momentum of a point \(P\) with mass \(m\), \(\vec{Q}_P = m \vec{v}_P\) readily follows the second principle of dynamics,
\begin{equation*}
\begin{split}\dot{\vec{Q}}_P = \vec{R}^e_P\end{split}
\end{equation*}
\sphinxAtStartPar
\sphinxstylestrong{Angular momentum balance equation.} Time derivative of the angular momentum is evaluated with the rule of derivative of product,
\begin{equation*}
\begin{split}\begin{aligned}
\dot{\vec{L}}_{P,H} & = \dfrac{d}{dt} \left[ m_P (\vec{r}_P - \vec{r}_H) \times \vec{v}_P \right] = \\
& = m \left[ ( \dot{\vec{r}}_P - \dot{\vec{r}}_H ) \times \vec{v}_P + m_P (\vec{r}_P - \vec{r}_H) \times \dot{\vec{v}}_P \right] = \\
& = - m_P \dot{\vec{r}}_H \times \vec{v}_P + m_P (\vec{r}_P - \vec{r}_H) \times \dot{\vec{v}}_P = \\
& = - \dot{\vec{r}}_H \times \vec{Q} + \vec{M}_H^{ext} \ .
\end{aligned}\end{split}
\end{equation*}
\sphinxAtStartPar
\sphinxstylestrong{Kinetic energy blanace equation.}
\begin{equation*}
\begin{split}\begin{aligned}
\dot{K}_{P} & = \dfrac{d}{dt} \left( \frac{1}{2} m_P \vec{v}_P \cdot \vec{v}_P \right) = \\
            & = m_P \dot{\vec{v}}_P \cdot \vec{v}_P = \\
            & = \vec{R}^e \cdot \vec{v}_P = P^e = P^{tot} \\
\end{aligned}\end{split}
\end{equation*}
\sphinxAtStartPar
being the power of external actions \(P^e\) equal to the total power acting on the system, assuming there is no internal action in the point system, or at least they have zero net power.

\sphinxstepscope


\section{Equations of motion of a discrete system of point masses}
\label{\detokenize{ch/dynamics-eom-points:equations-of-motion-of-a-discrete-system-of-point-masses}}\label{\detokenize{ch/dynamics-eom-points:classical-mechanics-dynamics-eom-points}}\label{\detokenize{ch/dynamics-eom-points::doc}}
\sphinxAtStartPar
Starting from the dynamic equations for a single point, the dynamic equations for a system of particles can be derived using the third principle of dynamics, action/reaction. The development of these equations helps us understand that the additive nature of dynamical quantities (momentum, angular momentum, kinetic energy) directly follows from their definition.

\sphinxAtStartPar
\sphinxstylestrong{Momentum Balance.}
The momentum balance for each point \(i\) in the system can be written by expressing the resultant of the external forces acting on the point as the sum of the external forces acting on the entire system and the internal forces exchanged with the other points of the system,
\begin{equation*}
\begin{split}\vec{R}_i^{ext,i} = \vec{F}_i^{ext} + \sum_{j \ne i} \vec{F}_{ij} \ .\end{split}
\end{equation*}
\sphinxAtStartPar
The momentum balance equation for the \(i\)\sphinxhyphen{}th mass thus becomes
\begin{equation*}
\begin{split}\dot{\vec{Q}}_i = \vec{R}_i^{ext,i} = \vec{F}_i^{ext} + \sum_{j \ne i} \vec{F}_{ij} \ .\end{split}
\end{equation*}
\sphinxAtStartPar
By summing the momentum balance equations for all masses, we obtain
\begin{equation*}
\begin{split}\begin{aligned}
\sum_{i} \dot{\vec{Q}}_i & = \sum_i \vec{F}_{i}^{ext} + \sum_i \sum_{j \ne i} \vec{F}_{ij} = \\
                            & = \sum_i \vec{F}_{i}^{ext} + \sum_{\{i,j\}} \underbrace{\left( \vec{F}_{ij} + \vec{F}_{ji} \right)}_{=\vec{0}} 
\end{aligned}\end{split}
\end{equation*}
\sphinxAtStartPar
\sphinxstyleemphasis{Defining the momentum of the system as the sum of the momenta of its parts}, and the resultant of the external forces as the sum of the external forces acting on the parts of the system,
\begin{equation*}
\begin{split}\vec{Q} := \sum_i \vec{Q}_i\end{split}
\end{equation*}\begin{equation*}
\begin{split}\vec{R}^e := \sum_i \vec{F}_i^{ext}\end{split}
\end{equation*}
\sphinxAtStartPar
we recover the general form of the momentum balance equation,
\begin{equation*}
\begin{split}\dot{\vec{Q}} = \vec{R}^e \ .\end{split}
\end{equation*}
\sphinxAtStartPar
\sphinxstylestrong{Angular Momentum Balance.}
The angular momentum balance for each point \(i\) in the system can be written by expressing the resultant of the external moments acting on the point as the sum of the external moments acting on the entire system and the internal moments exchanged with the other points of the system,
\begin{equation*}
\begin{split}\vec{M}_{H,i}^{ext,i} = \vec{M}_{H,i}^{ext} + \sum_{j \ne i} \vec{M}_{H,ij} \ .\end{split}
\end{equation*}
\sphinxAtStartPar
In the case where parts of the system interact via forces, the moment with respect to a point \(H\) generated by mass \(j\) on mass \(i\) is given by
\begin{equation*}
\begin{split}\vec{M}_{H,ij} = (\vec{r}_i - \vec{r}_H) \times \vec{F}_{ij} \ .\end{split}
\end{equation*}
\sphinxAtStartPar
The angular momentum balance equation for the \(i\)\sphinxhyphen{}th mass thus becomes
\begin{equation*}
\begin{split}\dot{\vec{L}}_{H,i} + \dot{\vec{r}}_H \times \vec{Q}_i = \vec{M}_{H,i}^{ext,i} = \vec{M}_{H,i}^{ext} + \sum_{j \ne i} \vec{M}_{H,ij} \ .\end{split}
\end{equation*}
\sphinxAtStartPar
By summing the angular momentum balance equations for all masses, we obtain
\begin{equation*}
\begin{split}\begin{aligned}
\sum_{i} \left( \dot{\vec{L}}_i + \dot{\vec{r}}_H \times \vec{Q}_i \right) & = \sum_i \vec{M}_{H,i}^{ext} + \sum_i \sum_{j \ne i} \vec{M}_{H,ij} = \\
                            & = \sum_i \vec{M}_{H,i}^{ext} + \sum_{\{i,j\}} \underbrace{\left( \vec{M}_{H,ij} + \vec{M}_{H,ji} \right)}_{=\vec{0}} 
\end{aligned}\end{split}
\end{equation*}
\sphinxAtStartPar
Recognizing the total momentum of the system, and \sphinxstyleemphasis{defining the angular momentum of the system as the sum of the angular momentum of its parts}, and the resultant of the external moments as the sum of the external moments acting on the parts of the system,
\begin{equation*}
\begin{split}\vec{L}_H := \sum_i \vec{L}_{H,i}\end{split}
\end{equation*}\begin{equation*}
\begin{split}\vec{M}_H^e := \sum_i \vec{M}_{H,i}^{ext}\end{split}
\end{equation*}
\sphinxAtStartPar
we recover the general form of the angular momentum balance equation,
\begin{equation*}
\begin{split}\dot{\vec{L}}_{H} + \dot{\vec{r}}_H \times \vec{Q} = \vec{M}_H^e \ .\end{split}
\end{equation*}
\sphinxAtStartPar
\sphinxstylestrong{Kinetic Energy Balance.}
The kinetic energy balance of the system can be derived by taking the scalar product of the momentum balance equation for each point,
\begin{equation*}
\begin{split}\vec{v}_i \cdot m_i \dot{\vec{v}}_i = \vec{v}_i \cdot \left( \vec{F}_i^{e} + \sum_{j \ne i} \vec{F}_{ij} \right) \ ,\end{split}
\end{equation*}
\sphinxAtStartPar
recognizing in the first term the time derivative of the kinetic energy of the \(i\)\sphinxhyphen{}th point,
\begin{equation*}
\begin{split}\dot{K}_i = \dfrac{d}{dt} \left( \frac{1}{2} m_i \vec{v}_i \cdot \vec{v}_i \right) = m_i \vec{v}_i \cdot \dot{\vec{v}}_i \ ,\end{split}
\end{equation*}
\sphinxAtStartPar
and summing these equations to obtain
\begin{equation*}
\begin{split}\begin{aligned}
  \sum_i \dot{K}_i = \sum_i \vec{v}_i \cdot  \vec{F}_i^{e} + \sum_i \vec{v}_i \cdot \sum_{j \ne i} \vec{F}_{ij} \ . 
\end{aligned}\end{split}
\end{equation*}
\sphinxAtStartPar
\sphinxstyleemphasis{Defining the kinetic energy of the system as the sum of the kinetic energies of its parts}, and defining the power of the external/internal forces acting on the system as the sum of the power of all external/internal forces in the system,
\begin{equation*}
\begin{split}K :=  \sum_i K_i\end{split}
\end{equation*}\begin{equation*}
\begin{split}P^e := \sum_i P^{ext}_i = \sum_i \vec{v}_i \cdot  \vec{F}_i^{e} \end{split}
\end{equation*}\begin{equation*}
\begin{split}P^i := \sum_i P^{int}_i = \sum_i \vec{v}_i \cdot \sum_{j \ne i} \vec{F}_{ij}\end{split}
\end{equation*}
\sphinxAtStartPar
we recover the general form of the kinetic energy balance equation,
\begin{equation*}
\begin{split}\dot{K} = P^e + P^i = P^{tot} \ .\end{split}
\end{equation*}
\begin{sphinxadmonition}{note}{Note:}
\sphinxAtStartPar
While internal forces and moments have zero net resultant in momentum and angular momentum balance, this is not true for power of internal actions in kinetic energy equation.
\end{sphinxadmonition}

\sphinxstepscope




\section{Equations of motion of a rigid body}
\label{\detokenize{ch/dynamics-eom-rigid:equations-of-motion-of-a-rigid-body}}\label{\detokenize{ch/dynamics-eom-rigid:classical-mechanics-dynamics-eom-rigid}}\label{\detokenize{ch/dynamics-eom-rigid::doc}}
\sphinxAtStartPar
With different choices of the reference point \(H\), the general expression of dynamical equations may have different, but equivalent, forms.


\subsection{General equations}
\label{\detokenize{ch/dynamics-eom-rigid:general-equations}}
\sphinxAtStartPar
\sphinxstylestrong{Momentum balance equation.}
\begin{equation*}
\begin{split}\dfrac{d}{dt} \vec{Q} = \vec{R}^e\end{split}
\end{equation*}
\sphinxAtStartPar
\sphinxstylestrong{Angular momentum balance equation.}
\begin{equation*}
\begin{split}\dfrac{d}{dt} \vec{L}_H + \dot{\vec{x}}_H \times \vec{Q} = \vec{M}_H^e\end{split}
\end{equation*}
\sphinxAtStartPar
\sphinxstylestrong{Kinetic energy balance equation.}
\begin{equation*}
\begin{split}\dfrac{d}{dt} K = P^{tot}\end{split}
\end{equation*}

\subsection{Dynamical equations w.r.t. the center of mass \protect\(G\protect\)}
\label{\detokenize{ch/dynamics-eom-rigid:dynamical-equations-w-r-t-the-center-of-mass-g}}
\sphinxAtStartPar
\sphinxstylestrong{Momentum, angular momentum and kinetic energy}
\begin{equation*}
\begin{split}
\begin{cases}
 \vec{Q} = m \vec{v}_G \\
 \vec{L}_G = \mathbb{I}_G \cdot \vec{\omega} \\
 K = \frac{1}{2} m |\vec{v}_G|^2 + \frac{1}{2} \vec{\omega} \cdot \mathbb{I}_G \cdot \vec{\omega} \\
\end{cases}
\qquad , \qquad
\begin{cases}
 \dot{\vec{Q}} = m \dot{\vec{v}}_G \\
 \dot{\vec{L}}_G = \mathbb{I}_G \cdot \dot{\vec{\omega}} + \vec{\omega} \times \mathbb{I}_G \cdot \vec{\omega} \\
 \dot{K} = m \vec{v}_G \cdot \dot{\vec{v}}_G + \vec{\omega} \cdot \mathbb{I}_G \cdot \dot{\vec{\omega}}
\end{cases}
\end{split}
\end{equation*}
\sphinxAtStartPar
\sphinxstylestrong{Equations of motion.}
\begin{equation*}
\begin{split}
\begin{cases}
 \dot{\vec{Q}} = \vec{R}^e \\
 \dot{\vec{L}}_G = \vec{M}_G^e \\
\end{cases}
\qquad , \qquad
\begin{cases}
 m \dot{\vec{v}}_G = \vec{R}^e \\
 \mathbb{I}_G \cdot \dot{\vec{\omega}} + \vec{\omega} \times \mathbb{I}_G \cdot \vec{\omega} = \vec{M}_G^e \\
\end{cases}
\end{split}
\end{equation*}\subsubsection*{Time derivative of kinetic energy}
\begin{equation*}
\begin{split}\begin{aligned}
  \dfrac{d K}{dt}
  & = \dfrac{1}{2} \dfrac{d}{dt} \left( \vec{v}_G \cdot \vec{Q} + \vec{\omega}_G \cdot \vec{L}_G \right) = \\
  & = \dfrac{1}{2} \left( \dot{\vec{v}}_G \cdot m \vec{v}_G + \vec{v}_G \cdot m \dot{\vec{v}}_G + \dot{\vec{\omega}} \cdot \mathbb{I}_G \cdot \vec{\omega} + \vec{\omega} \cdot \left(  \mathbb{I} \cdot \dot{\vec{\omega}} + \vec{\omega} \times \mathbb{I}_G \cdot \vec{\omega} \right) \right) \\
  & = \vec{v}_G \cdot m \dot{\vec{v}}_G + \vec{\omega} \cdot \mathbb{I}_G \cdot \dot{\vec{\omega}} \\
  & = \vec{v}_G \cdot \dot{\vec{Q}} + \vec{\omega} \cdot \dot{\vec{L}}_G \\
  & = \dot{\vec{v}}_G \cdot \vec{Q} + \dot{\vec{\omega}} \cdot \vec{L}_G \ .
\end{aligned}\end{split}
\end{equation*}

\subsection{Dynamical equations w.r.t. a material point \protect\(Q\protect\)}
\label{\detokenize{ch/dynamics-eom-rigid:dynamical-equations-w-r-t-a-material-point-q}}
\sphinxAtStartPar
\sphinxstylestrong{Momentum, angular momentum and kinetic energy}
\begin{equation*}
\begin{split}
\begin{cases}
 \vec{Q} = m \vec{v}_Q + \mathbb{S}_Q \cdot \vec{\omega} \\
 \vec{L}_Q = \mathbb{S}_Q^T \cdot \vec{v}_Q + \mathbb{I}_Q \cdot \vec{\omega} \\
 K = \dfrac{1}{2} \vec{v}_Q \cdot \vec{Q} + \dfrac{1}{2} \vec{\omega} \cdot \vec{L}_Q
\end{cases}
\qquad , \qquad
\begin{cases}
 \dot{\vec{Q}} = m \dot{\vec{v}}_Q + \mathbb{S}_Q \cdot \dot{\vec{\omega}} + \vec{\omega} \times \mathbb{S}_Q \cdot \vec{\omega} \\
 \dot{\vec{L}}_Q =
   \left[ \mathbb{S}_Q^T \cdot \left( \dot{v}_Q - \vec{\omega} \times \vec{v}_Q \right) + \mathbb{I} \cdot \dot{\vec{\omega}} \right] 
 + \vec{\omega} \times ( \mathbb{S}_Q^T \cdot \vec{v}_Q + \mathbb{I}_Q \cdot \vec{\omega} ) \\
 \dot{K} = \dots
\end{cases}
\end{split}
\end{equation*}
\sphinxAtStartPar
\sphinxstylestrong{Equations of motion.}
\begin{equation*}
\begin{split}
\begin{cases}
 \dot{\vec{Q}} = \vec{R}^e \\
 \dot{\vec{L}}_Q + \vec{v}_Q \times \vec{Q} = \vec{M}_Q^e \\
\end{cases}
\end{split}
\end{equation*}\begin{equation*}
\begin{split}
\begin{cases}
 m \dot{\vec{v}}_Q + \mathbb{S}_Q \cdot \dot{\vec{\omega}} + \vec{\omega} \times \mathbb{S}_Q \cdot \vec{\omega} = \vec{R}^e \\
 \left[ \mathbb{S}_Q^T \cdot \left( \dot{v}_Q - \vec{\omega} \times \vec{v}_Q \right) + \mathbb{I} \cdot \dot{\vec{\omega}} \right] 
 + \vec{\omega} \times ( \mathbb{S}_Q^T \cdot \vec{v}_Q + \mathbb{I}_Q \cdot \vec{\omega} ) + \vec{v}_Q \times \left[  m \vec{v}_Q + \mathbb{S}_Q \cdot \vec{\omega}   \right] = \vec{M}_Q^e \\
\end{cases}
\end{split}
\end{equation*}
\sphinxAtStartPar
or using the “material time derivative”
\begin{equation*}
\begin{split}\dfrac{{}^0 d}{dt} \underline{\hspace{10pt}} = \dfrac{d}{dt} \underline{\hspace{10pt}} - \vec{\omega} \times \underline{\hspace{10pt}} \ ,\end{split}
\end{equation*}
\sphinxAtStartPar
and matrix formalism to write these two vector equations
\begin{equation*}
\begin{split}
\begin{bmatrix} m \mathbb{I} & \mathbb{S}_Q \\ \mathbb{S}_Q^T & \mathbb{I}_Q \end{bmatrix} \, \dfrac{{}^0 d}{dt} \begin{bmatrix} \vec{v}_Q \\ \vec{\omega} \end{bmatrix} + \begin{bmatrix} \vec{\omega}_\times & \vec{0} \\ \vec{v}_{Q \times} & \vec{\omega}_{\times} \end{bmatrix} \begin{bmatrix} m \mathbb{I} & \mathbb{S}_Q \\ \mathbb{S}_Q^T & \mathbb{I}_Q \end{bmatrix} \begin{bmatrix} \vec{v}_Q \\ \vec{\omega} \end{bmatrix} = \begin{bmatrix} \vec{R}^e \\ \vec{M}_Q^e \end{bmatrix} \ .
\end{split}
\end{equation*}\subsubsection*{Starting from differential equations}

\sphinxAtStartPar
\sphinxstylestrong{todo}

\sphinxstepscope




\section{Equations of motion for continuous media}
\label{\detokenize{ch/dynamics-eom-continuum:equations-of-motion-for-continuous-media}}\label{\detokenize{ch/dynamics-eom-continuum:classical-mechanics-dynamics-eom-continuum}}\label{\detokenize{ch/dynamics-eom-continuum::doc}}
\sphinxstepscope




\section{Particular Motions}
\label{\detokenize{ch/dynamics-motions:particular-motions}}\label{\detokenize{ch/dynamics-motions:classical-mechanics-dynamics-motions}}\label{\detokenize{ch/dynamics-motions::doc}}
\sphinxAtStartPar
In this section, we will study certain particular motions that are interesting and useful to analyze for educational, historical, and practical reasons.
\begin{itemize}
\item {} 
\sphinxAtStartPar
Uniform rectilinear motion

\item {} 
\sphinxAtStartPar
Uniformly accelerated motion

\item {} 
\sphinxAtStartPar
Uniform circular motion

\item {} 
\sphinxAtStartPar
Oscillatory and damped oscillatory motions:
\begin{itemize}
\item {} 
\sphinxAtStartPar
Free oscillations:
\begin{itemize}
\item {} 
\sphinxAtStartPar
Mass\sphinxhyphen{}spring(\sphinxhyphen{}damper) system

\item {} 
\sphinxAtStartPar
Pendulum

\end{itemize}

\item {} 
\sphinxAtStartPar
Forced oscillations:
\begin{itemize}
\item {} 
\sphinxAtStartPar
A first step towards structural analysis and beyond (“every physical system is a system of many harmonic oscillators”)

\item {} 
\sphinxAtStartPar
Concepts of frequency response and resonance. \sphinxstylestrong{todo} video and/or script on frequency response of structures and seismic structures, mass\sphinxhyphen{}damper,…

\end{itemize}

\end{itemize}

\item {} 
\sphinxAtStartPar
\sphinxstylestrong{Gravitation}: Starting from Newton’s universal law of gravitation, we study the motion of celestial bodies in two\sphinxhyphen{}body systems, discovering that their trajectories describe conic sections (circle, ellipse, parabola, hyperbola), and demonstrating Kepler’s laws.

\item {} 
\sphinxAtStartPar
Rotation of a body around a fixed point, Poinsot’s motions

\end{itemize}

\sphinxstepscope


\section{Equilibrium and Stability}
\label{\detokenize{ch/dynamics-equilibrium-stability:equilibrium-and-stability}}\label{\detokenize{ch/dynamics-equilibrium-stability:classical-mechanics-dynamics-equilibrium-and-stability}}\label{\detokenize{ch/dynamics-equilibrium-stability::doc}}
\sphinxAtStartPar
A system is in equilibrium is all its components are in equilibrium, and thus there exists a reference frame w.r.t. which the momentum and the angular momentum of all its components are equal to zero.


\subsection{Eigenvalue stability}
\label{\detokenize{ch/dynamics-equilibrium-stability:eigenvalue-stability}}
\sphinxAtStartPar
…
\begin{equation*}
\begin{split}\delta \vec{Q} = m \delta \vec{v}_G\end{split}
\end{equation*}

\subsection{}
\label{\detokenize{ch/dynamics-equilibrium-stability:id1}}
\sphinxstepscope


\part{Analytical Mechanics}

\sphinxstepscope




\chapter{Lagrangian Mechanics}
\label{\detokenize{ch/lagrange:lagrangian-mechanics}}\label{\detokenize{ch/lagrange:classical-mechanics-lagrange}}\label{\detokenize{ch/lagrange::doc}}
\sphinxAtStartPar
Classical mechanics can be re\sphinxhyphen{}formulated starting principles of \sphinxhref{https://basics2022.github.io/bbooks-math-miscellanea/ch/calculus-variations/intro.html}{calculus of variations}, usually referred as \sphinxstylestrong{analytical mechanics}. Under some assumptions, that will be discussed during the derivation, analytical mechanics is equivalent to Newton mechanics.

\sphinxAtStartPar
{\hyperref[\detokenize{ch/lagrange-ii-type:classical-mechanics-lagrange-ii-type}]{\sphinxcrossref{\DUrole{std,std-ref}{\sphinxstylestrong{Lagrange equations \sphinxhyphen{} II kind}}}}}. Lagrange equations of the II kind provides \sphinxstylestrong{pure equations of motion}, in which constraint forces do not appear. Given a system with \(N\) degrees of freedom, that can be described with \(N\) independent generalized coordinates \(\{ q^k \}_{k=1:N}\), Lagrange equations of the II kind are a set of \(N\) \sphinxstylestrong{\(2^{nd}\) order ODEs} in the generalized coordinates.

\sphinxAtStartPar
The equivalence with Newton’s approach to classical mechanics is discussed in detail for different kind of systems (points, rigid bodies,…): starting from Newton’s dynamical equations of motion (strong form), D’Alembert approach (weak form) is derived, and Lagrange equations are derived from that with a proper choice of test functions. Lagrange equations are then recast as the stationariety condition of a functional, providing a variational approach to classical mechanics.

\begin{sphinxadmonition}{tip}{Tip:}
\sphinxAtStartPar
Lagrange equations of the II kind could be the best approach for small\sphinxhyphen{}dimensional problems, when there’s no interest in evaluating constraint reactions.
\end{sphinxadmonition}

\sphinxAtStartPar
{\hyperref[\detokenize{ch/lagrange-i-type:classical-mechanics-lagrange-i-type}]{\sphinxcrossref{\DUrole{std,std-ref}{\sphinxstylestrong{Lagrange equations \sphinxhyphen{} I kind}}}}}. Lagrange equations of the I kind provides a set of \sphinxstylestrong{DAEs}, explicitly including constraints as independent equations and adding their effects \sphinxhyphen{} their constraint reactions \sphinxhyphen{} in the dynamical equations of the degrees of freedom as \sphinxstylestrong{Lagrange multipliers}.

\begin{sphinxadmonition}{tip}{Tip:}
\sphinxAtStartPar
Lagrange equations of the I kind could be the best approach for numerical approach to generic mechanical systems, as it is a less problem\sphinxhyphen{}dependent approach, without any case\sphinxhyphen{}dependent manipulation.
\end{sphinxadmonition}

\sphinxAtStartPar
{\hyperref[\detokenize{ch/lagrange-t-explicit:classical-mechanics-lagrange-time-dependent}]{\sphinxcrossref{\DUrole{std,std-ref}{\sphinxstylestrong{Lagrangian mechanics, with expliicitly time\sphinxhyphen{}dependent Lagrangian function}}}}}. Energy conservation is related to explicit time independent of its Lagrangian function, and can be related \sphinxhyphen{} beside the absence of non\sphinxhyphen{}conservative actions \sphinxhyphen{} to the absence of any time\sphinxhyphen{}dependent external input, or the choice of an inertial reference frame. Consequence of explicit time dependence in the Lagrangian functions are discussed, with examples.

\sphinxAtStartPar
{\hyperref[\detokenize{ch/lagrange-constraints:classical-mechanics-lagrange-constraints}]{\sphinxcrossref{\DUrole{std,std-ref}{\sphinxstylestrong{Constraints in Lagrangian mechanics}}}}}. Constraints in Lagrangian mechanics are discussed: holonomic and non\sphinxhyphen{}holonomic are discussed, and an approach to ideal non\sphinxhyphen{}holonomic (semi\sphinxhyphen{}linear?) constraints in Lagrange mechanics is outlined.

\sphinxAtStartPar
{\hyperref[\detokenize{ch/lagrange-properties:classical-mechanics-lagrange-properties}]{\sphinxcrossref{\DUrole{std,std-ref}{\sphinxstylestrong{Properties}}}}}. Some properties of Lagrangian approach to mechanics are discussed. As an example, Lagrange mechanics provides a \sphinxstylestrong{symmetric form} of the (linearised?) governing equations, without any additional effort. This could be quite useful, especially for exploiting numerical methods for symmetric (and definite positive, sometimes) matrices.



\sphinxstepscope


\section{Lagrange Equations of the Second Kind}
\label{\detokenize{ch/lagrange-ii-type:lagrange-equations-of-the-second-kind}}\label{\detokenize{ch/lagrange-ii-type:classical-mechanics-lagrange-ii-type}}\label{\detokenize{ch/lagrange-ii-type::doc}}
\sphinxAtStartPar
Here, the equivalence of analytical mechanics and Newton mechanics is stressed, by means of the derivation of the principle of analytical mechanics starting from the equations of motions derived in Newtonian mechanics, relying on the conservation of mass and the three principles of Newton mechanics. The process is shown in the following sections for {\hyperref[\detokenize{ch/lagrange-point:classical-mechanics-lagrange-point}]{\sphinxcrossref{\DUrole{std,std-ref}{point systems}}}}, {\hyperref[\detokenize{ch/lagrange-points:classical-mechanics-lagrange-points}]{\sphinxcrossref{\DUrole{std,std-ref}{systems of points}}}}, {\hyperref[\detokenize{ch/lagrange-rigid-body:classical-mechanics-lagrange-rigid}]{\sphinxcrossref{\DUrole{std,std-ref}{extended rigid bodies}}}} and follows these steps:
\begin{itemize}
\item {} 
\sphinxAtStartPar
\sphinxstylestrong{strong form of equations.} Starting point is the dynamical equations of Newton mechanics, here also referred as the strong form of equations

\item {} 
\sphinxAtStartPar
\sphinxstylestrong{weak form of equations.} Strong form are recast in weak form, also referred as \sphinxstylestrong{D’Alembert approach} or \sphinxstylestrong{virtual work formulation}, multiplying strong form of equations for arbitrary test functions

\item {} 
\sphinxAtStartPar
\sphinxstylestrong{Lagrange equations.} A proper choice of test functions as a function of generalized coordinates, and some manipulation, leads to Lagrange equations. While the choice of test functions depends on the nature of the system, their expression always reads
\begin{equation}\label{equation:ch/lagrange-ii-type:eq:lagrange-eq}
\begin{split}\dfrac{d}{dt}\left( \frac{\partial \mathscr{L}}{\partial \dot{q}^k} \right) - \frac{\partial \mathscr{L}}{\partial q^k} = Q_{q^k} \ ,\end{split}
\end{equation}
\sphinxAtStartPar
being \(q^k(t)\) the generalized coordinates, \(\mathscr{L}\left(\dot{q}^k(t), q?k(t), t \right) = K\left(\dot{q}^k(t), q^k(t), t\right) + U(q^k(t), t)\) the Lagrangian function of the system, defined as the sum of the kinetic energy \(K\) and the potential function \(U = - V\), being \(V\) the potential energy \sphinxhyphen{} s.t. the conservative vector field reads \(\vec{F} = - \nabla V\), and \(Q_q\) the generalized force.

\item {} 
\sphinxAtStartPar
Lagrange equations can be interpreted as a result of a stationary principle of a functional, \(S\), defined \sphinxstylestrong{action functional}, as it can be shown with the tools of \sphinxhref{https://basics2022.github.io/bbooks-math-miscellanea/ch/calculus-variations/intro.html}{calculus of variations}.
\begin{itemize}
\item {} 
\sphinxAtStartPar
\sphinxstylestrong{If \(Q_{q^{k}} = 0\)}, multiplying by \(w^k(t)\), integrating over time from \(t_0\), \(t_1\), and assuming that \(w(t_0) = w(t_1) = 0\),
\begin{equation*}
\begin{split}\begin{aligned}
         0 & = \int_{t_0}^{t_1} w^k (t) \left[ \dfrac{d}{dt}\left( \frac{\partial \mathscr{L}}{\partial \dot{q}^k} \right) - \frac{\partial \mathscr{L}}{\partial q} \right] \, dt = \\
           & = w^k(t) \left.\left( \frac{\partial \mathscr{L}}{\partial \dot{q}^k} \right)\right|_{t_0}^{t_1} - \int_{t_0}^{t_1} \left[ \dot{w}^k(t) \, \frac{\partial \mathscr{L}}{\partial \dot{q}^k} + w^k(t) \, \frac{\partial \mathscr{L}}{\partial q^k} \right] \, dt \ . \\
     \end{aligned}\end{split}
\end{equation*}
\sphinxAtStartPar
If \(w^k(t)\) is equal to zero for \(t\) equal to \(t_0\) and \(t_1\), first term vanishes
\begin{equation*}
\begin{split}\begin{aligned}
           0 & = - \int_{t_0}^{t_1} \left[ \dot{w}^k(t) \, \frac{\partial \mathscr{L}}{\partial \dot{q}^k} + w^k(t) \, \frac{\partial \mathscr{L}}{\partial q^k} \right] \, dt \\
           & = - \frac{1}{\varepsilon} \int_{t_0}^{t_1} \varepsilon \left[ \dot{w}^k(t) \, \frac{\partial \mathscr{L}}{\partial \dot{q}^k}\left(\dot{q}^l(t), q^l(t), t \right) + w^k(t) \, \frac{\partial \mathscr{L}}{\partial q^k}\left(\dot{q}^l(t), q^l(t), t \right) \right] \, dt = \\
           & = - \lim_{\varepsilon \rightarrow 0} \left\{ \frac{1}{\varepsilon} \int_{t_0}^{t_1} \varepsilon \left[ \dot{w}^k(t) \, \frac{\partial \mathscr{L}}{\partial \dot{q}^k}\left(\dot{q}^l(t), q^l(t), t \right) + w^k(t) \, \frac{\partial \mathscr{L}}{\partial q^k}\left(\dot{q}^l(t), q^l(t), t \right) \right] \, dt \right\}= \\
           & = - \lim_{\varepsilon \rightarrow 0} \left\{ \frac{1}{\varepsilon} \int_{t_0}^{t_1} \left[ \mathscr{L}\left(\dot{q}^l(t)+\varepsilon \dot{w}^l(t), q^l(t) + \varepsilon w^l(t), t \right) - \mathscr{L}\left(\dot{q}^l(t), q^l(t), t \right) \right] \, dt + o(\varepsilon) \right\}= \\
           & = - \delta \int_{t_0}^{t_1} \mathscr{L}(\dot{q}^l(t), q^l(t), t) \, dt =: - \delta S[q^k(t)] \ ,
       \end{aligned}\end{split}
\end{equation*}
\sphinxAtStartPar
i.e. Lagrange equations are equivalent to the stationary condition of the action functional
\begin{equation*}
\begin{split}S[q^k(t)]:= \int_{t_0}^{t_1} \mathscr{L}\left(\dot{q}^l(t), q^l(t), t\right) \, dt \ .\end{split}
\end{equation*}
\item {} 
\sphinxAtStartPar
\sphinxstylestrong{If \(Q_k \ne 0\)}, the variational principle becomes
\begin{equation*}
\begin{split}\begin{aligned}
           0 & = \int_{t_0}^{t_1} w^k (t) \left[ \dfrac{d}{dt}\left( \frac{\partial \mathscr{L}}{\partial \dot{q}^k} \right) - \frac{\partial \mathscr{L}}{\partial q^k} - Q_k \right] \, dt = \\
             & = w^k(t) \left.\left( \frac{\partial \mathscr{L}}{\partial \dot{q}^k} \right)\right|_{t_0}^{t_1} - \int_{t_0}^{t_1} \left[ \dot{w}^k(t) \, \frac{\partial \mathscr{L}}{\partial \dot{q}^k} + w^k(t) \, \frac{\partial \mathscr{L}}{\partial q^k} \right] \, dt 
             + \int_{t_0}^{t_1} w^k(t) Q_k \, dt \ . \\
             & = \dots \\
             & = - \delta \int_{t_0}^{t_1} \mathscr{L}\left(\dot{q}^l(t), q^l(t), t\right) \, dt + \int_{t_0}^{t_1} \delta q^k(t) Q_k \, dt \ ,
       \end{aligned}\end{split}
\end{equation*}
\sphinxAtStartPar
having written the arbitrary test function as \(w^k(t) =: \delta q^k(t)\) to keep in mind that they’re used as variations of functions \(q^k(t)\).

\sphinxAtStartPar
The second contribution is usually defined \sphinxstylestrong{virtual work} of generalized forces \(Q_k\) \sphinxhyphen{} that is equal to the virtual work of actions not included in the potential \(U\left(q^k(t),t\right)\). For the very nature of variation, it can be thought as the \sphinxstyleemphasis{infinitesimal work done by forces for small displacements compatible with constraints, keeping time constant}.

\end{itemize}

\end{itemize}

\sphinxstepscope




\subsection{Point}
\label{\detokenize{ch/lagrange-point:point}}\label{\detokenize{ch/lagrange-point:classical-mechanics-lagrange-point}}\label{\detokenize{ch/lagrange-point::doc}}
\sphinxAtStartPar
\sphinxstylestrong{Newton dynamical equations \sphinxhyphen{} strong form.} Dynamical equation governing the motion of a point \(P\) reads
\begin{equation*}
\begin{split}m \dot{\vec{v}}_P = \vec{R}^e \ ,\end{split}
\end{equation*}
\sphinxAtStartPar
being \(m\) the mass of the system, \(\vec{v}_P\) the velocity of point \(P\), \(\vec{a}_P = \dot{\vec{v}}_P\) its acceleration and \(\vec{R}^{e}\) the net external force acting on the system..

\sphinxAtStartPar
\sphinxstylestrong{Weak form.} Weak form of dynamical equations is derived with scalar multiplication of the strong form by an arbitrary test function \(\vec{w}\),
\begin{equation}\label{equation:ch/lagrange-point:eq:lagrange:point:weak}
\begin{split}\vec{0} = \vec{w} \cdot \left( m \dot{\vec{v}} - \vec{R}^e\right)  \qquad \forall \vec{w}\end{split}
\end{equation}
\sphinxAtStartPar
\sphinxstylestrong{Lagrange equations.} Lagrange equations are derived from a proper choice of the test function. The position of the point \(P\) is written as a function of the generalized coordinates \(q^k(t)\) and time \(t\)
\begin{equation*}
\begin{split}\vec{r}_P(t) = \vec{r}(q^k(t),t) \ ,\end{split}
\end{equation*}
\sphinxAtStartPar
so that its velocity can be written as
\begin{equation*}
\begin{split}\vec{v}_P(t) := \frac{d\vec{r}_P}{dt} = \dot{q}^k(t) \underbrace{\frac{\partial \vec{r}}{\partial q^k}}_{\frac{\partial \vec{v}}{\partial \dot{q}^k}}(q^l(t), t) + \frac{\partial \vec{r}}{\partial t}(q^l(t), t) = \vec{v}\left(\dot{q}^k(t), q^k(t), t \right) \ ,\end{split}
\end{equation*}
\sphinxAtStartPar
from which the relation between partial derivatives
\begin{equation}\label{equation:ch/lagrange-point:classical-mechanics:lagrange:point:mixed-der}
\begin{split}\dfrac{\partial \vec{r}}{\partial q^k} = \dfrac{\partial \vec{v}}{\partial \dot{q}^k} \ .\end{split}
\end{equation}
\sphinxAtStartPar
follows. Choosing the test function \(\vec{w}\) as
\begin{equation*}
\begin{split}\vec{w} = \dfrac{\partial \vec{r}}{\partial q^k} = \dfrac{\partial \vec{v}}{\partial \dot{q}^k} \ ,\end{split}
\end{equation*}
\sphinxAtStartPar
applying the rule of derivative of product, using Schwartz theorem to switch order of derivation, and exploiting relation \eqref{equation:ch/lagrange-point:classical-mechanics:lagrange:point:mixed-der} it’s possible to recast weak form \eqref{equation:ch/lagrange-point:eq:lagrange:point:weak} as
\begin{equation*}
\begin{split}\begin{aligned}
\vec{0} & = \frac{\partial \vec{v}}{\partial \dot{q}^k} \cdot \left( m \dot{\vec{v}} - \vec{R}^e \right) = \\
& = \frac{d}{dt} \left( \frac{\partial \vec{v}}{\partial \dot{q}^k} \cdot m \vec{v} \right) - \frac{d}{dt} \frac{\partial \vec{r}}{\partial q^k} \cdot m \vec{v} - \frac{\partial \vec{r}}{\partial q^k} \cdot ( \vec{R}^{e,c} + \vec{R}^{e,nc} ) \\
& = \frac{d}{dt} \left( \frac{\partial \vec{v}}{\partial \dot{q}^k} \cdot m \vec{v} \right) - \frac{\partial \vec{v}}{\partial q^k} \cdot m \vec{v} - \frac{\partial \vec{r}}{\partial q^k} \cdot ( \nabla U + \vec{R}^{e,nc} ) \\
& = \frac{d}{dt} \left( \frac{\partial K}{\partial \dot{q}^k} \right) - \frac{\partial K}{\partial q^k} - \frac{\partial U}{\partial q^k} - \underbrace{\frac{\partial \vec{r}}{\partial q^k} \cdot \vec{R}^{e,nc}}_{=: Q^k} \ . \\
\end{aligned}\end{split}
\end{equation*}
\sphinxAtStartPar
Introducing the \sphinxstylestrong{Lagrangian function}
\begin{equation*}
\begin{split}\mathscr{L}(\dot{q}^k(t), q^k(t), t) := K(\dot{q}^k(t), q^k(t), t) + U(q^k(t),t) \ ,\end{split}
\end{equation*}
\sphinxAtStartPar
and recalling that potential function \(U\) is not a function of velocity and thus of time derivatives of the generalized coordinates \(\dot{q}^k\), it’s possible to recast the dynamical equation as the \sphinxstylestrong{Lagrange equations}
\begin{equation*}
\begin{split}\frac{d}{dt}\left(\frac{\partial \mathscr{L}}{\partial \dot{q}^k} \right) - \frac{\partial \mathscr{L}}{\partial q^k} = Q^k \ ,\end{split}
\end{equation*}
\sphinxAtStartPar
being \(Q^k\) the \sphinxstylestrong{generalized force} not included in the gradient of the potential \(\nabla U\) \sphinxhyphen{} usually a non conservative contribution \sphinxhyphen{}, \(Q^k = \dfrac{\partial \vec{r}}{\partial q^k} \cdot \vec{R}^{e,nc}\).

\sphinxstepscope




\subsection{System of points}
\label{\detokenize{ch/lagrange-points:system-of-points}}\label{\detokenize{ch/lagrange-points:classical-mechanics-lagrange-points}}\label{\detokenize{ch/lagrange-points::doc}}
\sphinxAtStartPar
\sphinxstylestrong{Newton dynamical equations \sphinxhyphen{} strong form.}

\sphinxAtStartPar
\sphinxstylestrong{Weak form.}

\sphinxAtStartPar
\sphinxstylestrong{Lagrange equations.}

\sphinxstepscope


\subsection{Rigid Body}
\label{\detokenize{ch/lagrange-rigid-body:rigid-body}}\label{\detokenize{ch/lagrange-rigid-body:classical-mechanics-lagrange-rigid}}\label{\detokenize{ch/lagrange-rigid-body::doc}}
\sphinxAtStartPar
\sphinxstylestrong{Newton dynamical equations \sphinxhyphen{} strong form.} Dynamical equations governing the motion of a rigid body, referred to its center of mass \(G\) read
\begin{equation*}
\begin{split}\begin{cases}
  \dot{\vec{Q}} = \vec{R}^{e} \\
  \dot{\vec{\Gamma}}_G = \vec{M}^e_G \ ,
\end{cases}\end{split}
\end{equation*}
\sphinxAtStartPar
with momentum \(\vec{Q} = m \vec{v}_G\) and angular momentum \(\vec{\Gamma}_G = \mathbb{I}_G \cdot \vec{\omega}\).

\sphinxAtStartPar
\sphinxstylestrong{Weak form.} Weak form of dynamical equations is derived with scalar multiplication of the strong form by an arbitrary test functions \(\vec{w}_t\), \(\vec{w}_r\)
\begin{equation}\label{equation:ch/lagrange-rigid-body:eq:lagrange:rigid:weak}
\begin{split}\vec{0} = \vec{w}_t \cdot \left( m \dot{\vec{v}}_G - \vec{R}^e\right) + \vec{w}_r \cdot \left( \dot{\vec{\Gamma}}_G - \vec{M}^e_G \right) \qquad \forall \vec{w}_t, \, \vec{w}_r\end{split}
\end{equation}
\sphinxAtStartPar
\sphinxstylestrong{Lagrange equations.} Lagrange equations are derived from the weak form, with a proper choice of the weak test functions. The “translational part” is recasted after choosing
\begin{equation*}
\begin{split}\vec{w}_t = \frac{\partial \vec{r}}{\partial q^k} = \frac{\partial \vec{v}}{\partial \dot{q}^k} \ .\end{split}
\end{equation*}
\sphinxAtStartPar
Following the same steps show to derive {\hyperref[\detokenize{ch/lagrange-point:classical-mechanics-lagrange-point}]{\sphinxcrossref{\DUrole{std,std-ref}{Lagrange equations for a point system}}}}, the translational part becomes
\begin{equation*}
\begin{split}\dfrac{d}{dt}\frac{\partial K^{tr}}{\partial \dot{q}^k} - \frac{\partial K^{tr}}{\partial q^k} - \frac{\partial U^{tr}}{\partial q^{k}} = Q^{tr}_{k} \ ,\end{split}
\end{equation*}
\sphinxAtStartPar
being \(K^{tr} = \frac{1}{2} m |\vec{v}_G|^2\) the contribution to kinetic energy of the velocity of the center of mass \(G\) deriving from the momentum equation, \(U^{tr}\) the contribution to the potential energy \(U\) from the momentum equation, and \(Q^{tr}_{k}\) the contribution to the generalized force from the momentum equation.

\sphinxAtStartPar
The “rotational part” is recasted after choosing
\begin{equation*}
\begin{split}\vec{w}_r = \frac{\partial \vec{\theta}}{\partial q^k} = \frac{\partial \vec{\omega}}{\partial \dot{q}^k} \end{split}
\end{equation*}
\sphinxAtStartPar
Angular velocity \(\vec{\omega}\) can be written w.r.t the inertial \(\{ \hat{e}_i \}\) or the material reference frame \(\{ \hat{E}_i \}\),
\begin{equation*}
\begin{split}\vec{\omega} = \omega_i \hat{e}_i = \sigma_j \hat{E}_j \ ,\end{split}
\end{equation*}
\sphinxAtStartPar
and the inertia tensor as
\begin{equation*}
\begin{split}\mathbb{I}_G = I_{ij} \, \hat{E}_i \otimes \hat{E}_j \ ,\end{split}
\end{equation*}
\sphinxAtStartPar
being the components \(I_{ij}\) constant.
\begin{equation*}
\begin{split}0 = \frac{\partial \vec{\omega}}{\partial \dot{q}^k} \cdot \dfrac{d}{d t} \left( \mathbb{I}_G \cdot \vec{\omega} \right) - \frac{\partial \vec{\omega}}{\partial \dot{q}^k } \cdot \vec{M}^e_G = \dfrac{d}{dt}\left( \frac{\partial \vec{\omega}}{\partial \dot{q}^k} \cdot \mathbb{I}_G \cdot \vec{\omega} \right) - \dfrac{d}{dt} \frac{\partial \vec{\omega}}{\partial \dot{q}^k} \cdot \mathbb{I}_G \cdot \vec{\omega} - \frac{\partial \vec{\theta}}{\partial q^k} \cdot \vec{M}^e_G\end{split}
\end{equation*}
\sphinxAtStartPar
The first term becomes
\begin{equation*}
\begin{split}\dfrac{d}{dt}\left( \frac{\partial \vec{\omega}}{\partial \dot{q}^k} \cdot \mathbb{I}_G \cdot \vec{\omega} \right) = \dfrac{d}{dt} \left( \frac{\partial \sigma_a}{\partial \dot{q}^k} I_{ab} \sigma_b \right) = \dfrac{d}{dt} \dfrac{\partial}{\partial \dot{q}^k}\left( \dfrac{1}{2} \vec{\omega} \cdot \mathbb{I}_G \cdot \vec{\omega} \right) = \dfrac{d}{dt}\dfrac{\partial K^{rot}}{\partial \dot{q}^k}\end{split}
\end{equation*}
\sphinxAtStartPar
The second term becomes
\begin{equation*}
\begin{split}\begin{aligned}
 \dfrac{d}{dt} \frac{\partial \vec{\theta}}{\partial q^k} \cdot \mathbb{I}_G \cdot \vec{\omega} 
  & = \dfrac{\partial }{\partial q^k} \underbrace{\dfrac{d \vec{\theta}}{d t}}_{\vec{\omega}} \cdot \mathbb{I}_G \cdot \vec{\omega} = \\
  & = \dfrac{\partial \vec{\omega}}{\partial q^k} \cdot \mathbb{I}_G \cdot \vec{\omega} = \\
  & = \dfrac{\partial}{\partial q^k} \left( \sigma_a \hat{E}_a \right) \cdot \hat{E}_b \, I_{bc} \sigma_c = \\
  & = \dfrac{\partial \sigma_a}{\partial q^k} \underbrace{\hat{E}_a \cdot \hat{E}_b}_{= \delta_{ab}} \, I_{bc} \sigma_c 
    + \sigma_a \underbrace{\dfrac{\partial \hat{E}_a}{\partial q^k}  \cdot \hat{E}_b}_{= 0} \, I_{bc} \sigma_c = \\
  & = \dfrac{\partial}{\partial q^k} \left( \frac{1}{2} \sigma_a \, I_{ab} \sigma_b \right) = \\
  & = \dfrac{\partial }{\partial q^k} \left( \frac{1}{2} \vec{\omega} \cdot \mathbb{I}_G \cdot \vec{\omega} \right) 
    = \dfrac{\partial K^{rot}}{\partial q^k} \ .
\end{aligned}\end{split}
\end{equation*}
\sphinxAtStartPar
The third term can be written as the sum of the derivative of a potential function and a generalized force,
\begin{equation*}
\begin{split}\frac{\partial \vec{\theta}}{\partial q^k} \cdot \vec{M}_G^e = \frac{\partial U^{rot}}{\partial q^k} + Q^{rot}_{q^k}\end{split}
\end{equation*}
\sphinxAtStartPar
The rotational part of the wak form becomes
\begin{equation*}
\begin{split}\dfrac{d}{dt}\frac{\partial K^{rot}}{\partial \dot{q}^k} - \frac{\partial K^{rot}}{\partial q^k} - \frac{\partial U^{rot}}{\partial q^{k}} = Q^{rot}_{q^k} \ ,\end{split}
\end{equation*}
\sphinxAtStartPar
being \(K^{rot} = \frac{1}{2} \vec{\omega} \cdot \mathbb{I}_G \cdot \vec{\omega}\) the contribution to kinetic energy of the rotation aroung the center of mass \(G\) deriving from the angular momentum equation, \(U^{rot}\) the contribution to the potential energy \(U\) from the angular momentum equation, and \(Q^{rot}_{k}\) the contribution to the generalized force from the angular momentum equation.

\sphinxAtStartPar
Adding together the contributions of the momentum and the angular momentum equations, the Lagrange equation can be formally written with the same expression found for the system of points,
\begin{equation*}
\begin{split}\dfrac{d}{dt}\left(\frac{\partial \mathscr{L}}{\partial \dot{q}^k}\right) - \frac{\partial \mathscr{L}}{\partial q^k} = Q_{q^k} \ ,\end{split}
\end{equation*}
\sphinxAtStartPar
being \(\mathscr{L} = K + U\) the Lagrangian function of the system, and \(K = K^{tr} + K^{rot}\), \(U = U^{tr} + U^{rot}\), \(Q_{q^k} = Q_{q^k}^{tr} + Q_{q^k}^{rot}\) the kinetic energy the potential function and the generalized force of the system, defined as the sum of the contributions coming from the momentum and the angular momentum equations.



\sphinxstepscope


\section{Lagrange Equations of the First Kind}
\label{\detokenize{ch/lagrange-i-type:lagrange-equations-of-the-first-kind}}\label{\detokenize{ch/lagrange-i-type:classical-mechanics-lagrange-i-type}}\label{\detokenize{ch/lagrange-i-type::doc}}
\sphinxAtStartPar
Explicitly making appear constraint forces, due to constraints
\begin{equation*}
\begin{split}\begin{aligned}
 & \dfrac{d}{dt}\frac{\partial L}{\partial \dot{q}^k} - \frac{\partial L}{\partial q^k} = Q^e_k + Q^c_k \\
 & g^j \left(q^k(t), t\right) = 0
\end{aligned}\end{split}
\end{equation*}\label{ch/lagrange-i-type:example-0}
\begin{sphinxadmonition}{note}{Example 5.2.1}



\sphinxAtStartPar
Pendulum with point mass \(m\) and length \(\ell\), with hinge position \(x_H(t)\) w.r.t. an inertial reference frame, in a gravitational field \(\vec{g} = g \hat{y}\)

\sphinxAtStartPar
Position, and velocity of the point mass in \(P\)
\begin{equation*}
\begin{split}\begin{aligned}
  & \vec{r}_P(t) = x_P(t) \, \hat{x} + y_P(t) \, \hat{y} = \left( x_H(t) + \ell \sin \theta(t) \right) \, \hat{x} + \ell \cos \theta(t) \, \hat{y} \\
  & \vec{v}_P(t) = \dot{x}_P(t) \, \hat{x} + \dot{y}_P(t) \, \hat{y} = (\dot{x}_H + \ell \dot{\theta}(t) \cos \theta(t)) \, \hat{x} - \ell \dot{\theta}(t) \sin \theta(t) \, \hat{y} \\
\end{aligned}\end{split}
\end{equation*}
\sphinxAtStartPar
\sphinxstylestrong{Approach 1. LE of the II Kind.} LE of the II Kind provides free equations of motion. The system has one degree of freedom. Here the angle \(\theta(t)\) is chosen as the generalized dof. Kinetic energy \(K\) and potential function \(U\),
\begin{equation*}
\begin{split}\begin{aligned}
  K & = \frac{1}{2} m \left( \dot{x}_H^2 + 2 \ell \dot{x}_H \dot \theta \cos \theta + \ell^2 \dot{\theta}^2  \right) \\
  U & = m g \ell \cos \theta \\
\end{aligned}\end{split}
\end{equation*}
\sphinxAtStartPar
and Lagrange equation of the II\sphinxhyphen{}kind provides a free equation of motion, that immediately follows from direct evaluation of the required derivatives
\begin{equation*}
\begin{split}\begin{aligned}
  \dfrac{d}{dt}\frac{\partial L}{\partial \dot{\theta}}
  & = \dfrac{d}{dt} \left[ m (\ell \dot{x}_H \cos \theta + \ell^2 \dot{\theta} ) \right]
    = m \ell \ddot{x}_H \cos \theta - m \ell \dot{x}_H \dot{\theta} \sin \theta + m \ell^2 \ddot{\theta} \\
  \frac{\partial L}{\partial \theta}
  & = - m \ell \dot{x}_H \dot{\theta} \sin \theta - m g \ell \sin \theta
\end{aligned}\end{split}
\end{equation*}
\sphinxAtStartPar
Thus, Lagrange equation reads
\begin{equation*}
\begin{split}
  0 = \dfrac{d}{d t} \frac{\partial L}{\partial \dot{\theta}} - \frac{\partial L}{\partial \theta}
    = m \ell^2 \ddot{\theta} + m g \ell \sin \theta - m \ell \ddot{x}_H (t) \cos \theta
\end{split}
\end{equation*}
\sphinxAtStartPar
\sphinxstylestrong{Approach 2. LE of the I Kind.}
\end{sphinxadmonition}

\sphinxstepscope


\section{Lagrangian functions and time dependence}
\label{\detokenize{ch/lagrange-t:lagrangian-functions-and-time-dependence}}\label{\detokenize{ch/lagrange-t:classical-mechanics-lagrange-time}}\label{\detokenize{ch/lagrange-t::doc}}
\sphinxAtStartPar
Some problems may have a Lagrangian function with an explicit dependence on time,
\begin{equation*}
\begin{split}\mathscr{L}(\dot{q}^k(t),q^k(t),t) \ .\end{split}
\end{equation*}
\sphinxAtStartPar
Using the general form \eqref{equation:ch/lagrange-ii-type:eq:lagrange-eq} of Lagrange equations, the time derivative of the Lagrange function reads
\begin{equation*}
\begin{split}\begin{aligned}
\dfrac{d \mathscr{L}}{dt}
 & =
  \ddot{q}^k \dfrac{\partial \mathscr{L}}{\partial \dot{q}^k}
 + \dot{q}^k \dfrac{\partial \mathscr{L}}{\partial       q^k}
 + \dfrac{\partial \mathscr{L}}{\partial  t} = && \qquad \text{(IxP)} \\
 & = \dfrac{d}{dt} \left( \dot{q}^k \dfrac{\partial \mathscr{L}}{\partial \dot{q}^k} \right) - \dot{q}^k \dfrac{d}{dt} \dfrac{\partial \mathscr{L}}{\partial \dot{q}^k}
 + \dot{q}^k \dfrac{\partial \mathscr{L}}{\partial       q^k}
 + \dfrac{\partial \mathscr{L}}{\partial  t} = && \qquad \text{(Lagrange eq.)} \\
 & = \dfrac{d}{dt} \left( \dot{q}^k \dfrac{\partial \mathscr{L}}{\partial \dot{q}^k} \right) - \dot{q}^k Q_k
 + \dfrac{\partial \mathscr{L}}{\partial  t} \ .
\end{aligned}\end{split}
\end{equation*}
\sphinxAtStartPar
This latter relation can be recast as
\begin{equation}\label{equation:ch/lagrange-t:eq:euler-beltrami}
\begin{split}\dfrac{d}{dt} \left[ \dot{q}^k \dfrac{\partial \mathscr{L}}{\partial \dot{q}^k} - \mathscr{L} \right] = \dot{q}^k Q_k - \dfrac{\partial \mathscr{L}}{\partial t} \ ,\end{split}
\end{equation}
\sphinxAtStartPar
i.e. time derivative of a physical quantity equals the power of actions not included in the potential, \(\dot{q}^k Q_k\) and a contribution of partial derivative of the Lagrangian function, \(\partial_t \mathscr{L}\).

\sphinxAtStartPar
As it’s discusseed in the {\hyperref[\detokenize{ch/lagrange-t-no:classical-mechanics-lagrange-time-independent}]{\sphinxcrossref{\DUrole{std,std-ref}{section for systems with Lagrangian function with no explicit dependence on time}}}}  \sphinxstylestrong{when the Lagrangian function of the system is not an explicit function of time} \sphinxstylestrong{todo} *discuss the cases when \(\partial_t \mathscr{L} \ne 0\), the equation \eqref{equation:ch/lagrange-t:eq:euler-beltrami} is nothing but the \sphinxstylestrong{balance equation of mechanical energy}.

\sphinxAtStartPar
When there is no generalized force, that can’t be included in the potential, \(Q_k  = 0\), and no explicit dependence of Lagrangian function on time, \(\partial_t \mathscr{L} = 0\), the equation \eqref{equation:ch/lagrange-t:eq:euler-beltrami} can be recast as an \sphinxhref{https://basics2022.github.io/bbooks-math-miscellanea/ch/calculus-variations/intro.html\#euler-beltrami-equation}{Euler\sphinxhyphen{}Beltrami equation},
\begin{equation*}
\begin{split}\dot{q}^k \dfrac{\partial \mathscr{L}}{\partial \dot{q}^k} - \mathscr{L} = \overline{E} \quad \text{const.}\end{split}
\end{equation*}
\sphinxAtStartPar
describing the conservation of mechanical energy w.r.t. an inertial reference frame, in absence of non\sphinxhyphen{}conservative forces, as discussed in the .

\sphinxstepscope


\section{Constraints classification}
\label{\detokenize{ch/lagrange-constraints:constraints-classification}}\label{\detokenize{ch/lagrange-constraints:classical-mechanics-lagrange-constraints}}\label{\detokenize{ch/lagrange-constraints::doc}}

\subsection{Holonomic vs. non\sphinxhyphen{}holonomic constraints}
\label{\detokenize{ch/lagrange-constraints:holonomic-vs-non-holonomic-constraints}}\label{ch/lagrange-constraints:holonomic-constraint}
\begin{sphinxadmonition}{note}{Definition 5.4.1 (Holonomic constraint)}



\sphinxAtStartPar
A holonomic constraints can be written in a form
\begin{equation*}
\begin{split}g(q^k(t), t) = 0 \ .\end{split}
\end{equation*}\end{sphinxadmonition}
\label{ch/lagrange-constraints:non-holonomic-constraint}
\begin{sphinxadmonition}{note}{Definition 5.4.2 (Non\sphinxhyphen{}holonomic constraint)}



\sphinxAtStartPar
Every constraint that is not holonomic, is non\sphinxhyphen{}holonomic. (\sphinxstylestrong{wow!})
\end{sphinxadmonition}

\sphinxAtStartPar
\sphinxstylestrong{todo} \sphinxstyleemphasis{Add some examples…; even some constraints that looks like a non\sphinxhyphen{}holonomic constraint that are holonomic: “integrable constraints”, Pfaffian method?…}


\subsection{Ideal constraints}
\label{\detokenize{ch/lagrange-constraints:classical-mechanics-lagrange-constraints-ideal}}\label{\detokenize{ch/lagrange-constraints:ideal-constraints}}\label{ch/lagrange-constraints:definition-2}
\begin{sphinxadmonition}{note}{Definition 5.4.3 (Ideal constraint)}



\sphinxAtStartPar
An ideal constraint produces no net power.
\end{sphinxadmonition}

\sphinxAtStartPar
Given the generalized actions \(\mathbf{f}_c\) introduced in the dynamical systems by constraints,
\begin{equation*}
\begin{split}\mathbf{M} \ddot{\mathbf{q}} = \mathbf{f} + \mathbf{f}_c \ ,\end{split}
\end{equation*}
\sphinxAtStartPar
power of ideal constraints reads
\begin{equation*}
\begin{split}\dot{\mathbf{q}}^T \mathbf{f}_c = 0 \ .\end{split}
\end{equation*}

\subsubsection{Ideal holonomic constraints}
\label{\detokenize{ch/lagrange-constraints:classical-mechanics-lagrange-constraints-ideal-holonomic}}\label{\detokenize{ch/lagrange-constraints:ideal-holonomic-constraints}}
\sphinxAtStartPar
If constraints don’t have explicit dependence (what happens if there’s explicit {\hyperref[\detokenize{ch/lagrange-t-explicit:classical-mechanics-lagrange-time-dependent}]{\sphinxcrossref{\DUrole{std,std-ref}{time dependence}}}}? Treat in the proper section…) from time \(t\)
\begin{equation*}
\begin{split}\begin{aligned}
  & \mathbf{M} \ddot{\mathbf{q}} = \mathbf{f} + \mathbf{f}_c \\
  & \mathbf{g}(\mathbf{q}(t)) = \mathbf{0}
\end{aligned}\end{split}
\end{equation*}
\sphinxAtStartPar
with \(\mathbf{q} \in \mathbb{R}^N\), \(\mathbf{g} \in \mathbb{R}^C\).

\sphinxAtStartPar
\sphinxstylestrong{Power of ideal constraint.} Power of ideal constraints is zero, and this condition provides the most general form of constraint reactions \(\mathbf{f}_c\) as a linear combination of the gradient of the constraints, and a set of well\sphinxhyphen{}defined (\sphinxstyleemphasis{extra conditions?}) and determined DAEs, with equal number of equations and unknowns. Power of contraint reactions of ideal constrains is zero,
\begin{equation*}
\begin{split}0 = \dot{\mathbf{q}}^T \mathbf{f}_c\end{split}
\end{equation*}
\sphinxAtStartPar
Time derivative of the constraint equation reads \(0 = g_i (q^k) = \dot{q}^k \frac{\partial g_i}{\partial q^k}\), and thus for every \(\mathbf{s} \in \mathbb{R}^C\),
\begin{equation}\label{equation:ch/lagrange-constraints:eq:constraint:reaction:holonomic}
\begin{split}0 = \dot{\mathbf{q}}^T \nabla_{\mathbf{q}} \mathbf{g} \, \mathbf{s} \ ,\end{split}
\end{equation}
\sphinxAtStartPar
and thus constraint reactions can be written as a linear combination of the (columns of the) gradient of the constraint equation w.r.t. the generalized coordiantes,
\begin{equation*}
\begin{split}\mathbf{f}_c = \nabla_\mathbf{q} \mathbf{g} \, \mathbf{s}\end{split}
\end{equation*}
\sphinxAtStartPar
or
\begin{equation*}
\begin{split}\begin{aligned}
  \mathbf{f}_c = \nabla_{\mathbf{q}} g_i \, s_i \qquad , \qquad
  f_{c,k}      = \dfrac{\partial\mathbf{g}^T}{\partial q^k} \mathbf{s} = \dfrac{\partial g^i}{\partial q^k} \, s_i \ .
\end{aligned}\end{split}
\end{equation*}
\sphinxAtStartPar
\sphinxstylestrong{Determined set of DAEs.} Introducing the expression \eqref{equation:ch/lagrange-constraints:eq:constraint:reaction:holonomic} of the constraint reactions in the original set of DAEs, the set of equations governing the constrained system reads
\begin{equation*}
\begin{split}\begin{aligned}
  & \mathbf{M} \ddot{\mathbf{q}} = \mathbf{f} + \nabla_{\mathbf{q}} \mathbf{g} \, \mathbf{s} \\
  & \mathbf{g}(\mathbf{q}) = \mathbf{0}
\end{aligned}\end{split}
\end{equation*}

\subsubsection{Ideal non\sphinxhyphen{}holonomic constraints}
\label{\detokenize{ch/lagrange-constraints:classical-mechanics-lagrange-constraints-ideal-non-holonomic}}\label{\detokenize{ch/lagrange-constraints:ideal-non-holonomic-constraints}}
\sphinxAtStartPar
The equations of a constrained system with non\sphinxhyphen{}holonomic constraints in semi\sphinxhyphen{}linear form and no explicit time dependence reads
\begin{equation*}
\begin{split}\begin{aligned}
  & \mathbf{M} \ddot{\mathbf{q}} = \mathbf{f} + \mathbf{f}_c \\
  & \mathbf{a}(\mathbf{q}(t)) \, \dot{\mathbf{q}}(t) = \mathbf{0}
\end{aligned}\end{split}
\end{equation*}
\sphinxAtStartPar
\sphinxstylestrong{Power of ideal constraints.} As done in the {\hyperref[\detokenize{ch/lagrange-constraints:classical-mechanics-lagrange-constraints-ideal-holonomic}]{\sphinxcrossref{\DUrole{std,std-ref}{section about holonomic constraints}}}}, the condition of zero power of ideal constraints provides the most general form of constraint reactions \(\mathbf{f}_c\) as a linear combination of the gradient of the constraints, and thus a determined set of DAEs.
\begin{equation*}
\begin{split}0 = \dot{\mathbf{q}}^T \, \mathbf{f}_c\end{split}
\end{equation*}
\sphinxAtStartPar
compared with the (transpose of the) non\sphinxhyphen{}holonomic constraint,
\begin{equation*}
\begin{split}0 = \dot{\mathbf{q}}^T \mathbf{a}^T(\mathbf{q}) \mathbf{s}  \ , \qquad \forall \mathbf{s} \in \mathbb{R}^C\end{split}
\end{equation*}
\sphinxAtStartPar
and thus constraint reactions can be written as the linear combination of the columns of the transpose of matrix \(\mathbf{a}(\mathbf{q})\),
\begin{equation}\label{equation:ch/lagrange-constraints:eq:constraint:reaction:non-holonomic}
\begin{split}\mathbf{f} = \mathbf{a}^T(\mathbf{q}) \mathbf{s} \ .\end{split}
\end{equation}
\sphinxAtStartPar
\sphinxstylestrong{Determined set of DAEs.} Introducing the expression \eqref{equation:ch/lagrange-constraints:eq:constraint:reaction:non-holonomic} of the constraint reactions in the original set of DAEs, the determined set of equations governing the constrained system reads
\begin{equation*}
\begin{split}\begin{aligned}
  & \mathbf{M} \ddot{\mathbf{q}} = \mathbf{f} + \mathbf{a}^T(\mathbf{q}) \, \mathbf{s} \\
  & \mathbf{a}(\mathbf{q}) \, \dot{\mathbf{q}}(t) = \mathbf{0}
\end{aligned}\end{split}
\end{equation*}
\sphinxstepscope


\section{Properties of the Lagrangian approach to classical mechanics}
\label{\detokenize{ch/lagrange-properties:properties-of-the-lagrangian-approach-to-classical-mechanics}}\label{\detokenize{ch/lagrange-properties:classical-mechanics-lagrange-properties}}\label{\detokenize{ch/lagrange-properties::doc}}
\sphinxstepscope


\subsection{Lagrangian function with no explicit dependence on time}
\label{\detokenize{ch/lagrange-t-no:lagrangian-function-with-no-explicit-dependence-on-time}}\label{\detokenize{ch/lagrange-t-no:classical-mechanics-lagrange-time-independent}}\label{\detokenize{ch/lagrange-t-no::doc}}
\sphinxAtStartPar
Let’s analyse first some properties of systems, whose Lagrangian function are not an explicit function of time,
\begin{equation*}
\begin{split}\mathscr{L}(\dot{q}^k(t), q^k(t)) = K(\dot{q}^k(t), q^k(t)) + U(q^k(t)) \ ,\end{split}
\end{equation*}
\sphinxAtStartPar
and then go back to the most general case.
As the Lagrange equation is not an explicit function of time, relation \eqref{equation:ch/lagrange-t:eq:euler-beltrami} reads
\begin{equation*}
\begin{split}\dfrac{d}{dt} \left[ \dot{q}^k \dfrac{\partial \mathscr{L}}{\partial \dot{q}^k} - \mathscr{L} \right] = \dot{q}^k Q_k \ .\end{split}
\end{equation*}
\sphinxAtStartPar
Since the Lagrangian doesn’t expliclty depend on time, and potential is not a function of time, relation \eqref{equation:ch/lagrange-t-no:eq:lagrange:time:ind:dqdK_dq} gives \(\dot{q}^k \frac{\partial \mathscr{L}}{\partial \dot{q}^k} = 2 K\), and thus the content of the braces is the mechanical energy of the system,
\begin{equation*}
\begin{split}\dot{q}^k \dfrac{\partial \mathscr{L}}{\partial \dot{q}^k} - \mathscr{L} = 2 K - \mathscr{L} = 2 K - K - U = K - U = E^{mec} \ ,\end{split}
\end{equation*}
\sphinxAtStartPar
and it becomes clear that the relation is nothhing but the balance equation of mechanical energy
\begin{equation*}
\begin{split}\dfrac{d E^{mec}}{dt} = \dot{q}^k Q_k \ ,\end{split}
\end{equation*}
\sphinxAtStartPar
that becomes conservation of mechanical energy, in absence of non\sphinxhyphen{}conservative actions, \(Q_k = 0\),
\begin{equation*}
\begin{split}E^{mec} = \overline{E}^{mec} \quad \text{const.}\end{split}
\end{equation*}

\subsubsection{Properties of kinetic energy and potential}
\label{\detokenize{ch/lagrange-t-no:properties-of-kinetic-energy-and-potential}}
\sphinxAtStartPar
This section collects some properties of the kinetic energy and potential of systems, where physical coordinates of the system are written as a function of generalized coordinates only, \(q^k(t)\). As an example, coordinates of point masses, material points of rigid bodies and the rotation tensor representing their orientation in space can be written as
\begin{equation*}
\begin{split}\vec{r}_P\left(q^k(t)\right) \quad , \quad \mathbb{R} \left( q^k(t) \right) \ ,\end{split}
\end{equation*}
\sphinxAtStartPar
so that velocities and angular velocities become
\begin{equation*}
\begin{split}\begin{aligned}
  \vec{v}_P\left(\dot{q}^k(t), q^k(t)\right) & = \dfrac{d \vec{r}_P}{dt} = \dot{q}^k \dfrac{\partial \vec{r}}{\partial q^k}\left(q^k(t)\right) \\ 
  \vec{\omega}_{\times}\left(\dot{q}^k(t), q^k(t)\right) & = \dfrac{d \mathbb{R}}{d t} \cdot \mathbb{R}^T = \dot{q}^k(t) \, \dfrac{\partial \mathbb{R}}{\partial q^k} \cdot \mathbb{R}^T = \dot{q}^k(t) \, \dfrac{\partial \vec{\theta}_{\times}}{\partial q^k}\left(q^k(t)\right) \ ,
\end{aligned}\end{split}
\end{equation*}
\sphinxAtStartPar
As the kinetic energy of a mechanical system is a quadratic function of velocity and angular velocity of its sub\sphinxhyphen{}systems, the kinetice energy can be written as
\begin{equation*}
\begin{split}K\left( \dot{q}^k(t), q^k(t) \right) = \frac{1}{2} A_{ij}\left(q^k(t)\right) \, \dot{q}^i(t) \, \dot{q}^j(t) \ .\end{split}
\end{equation*}
\sphinxAtStartPar
Since \(A_{ij}\) is symmetric w.r.t. the swap of indices (or it can be written in a symmetric form), partial derivative of the kinetic energy w.r.t. \(\dot{q}^l\) reads
\begin{equation*}
\begin{split}\dfrac{\partial K}{\partial \dot{q}^l} = A_{lj} \dot{q}^j \ ,\end{split}
\end{equation*}
\sphinxAtStartPar
and
\begin{equation}\label{equation:ch/lagrange-t-no:eq:lagrange:time:ind:dqdK_dq}
\begin{split}\dot{q}^l \dfrac{\partial K}{\partial \dot{q}^l} = \dot{q}^l A_{lj} \dot{q}^j = 2 K \ .\end{split}
\end{equation}\subsubsection*{Proofs}
\begin{equation*}
\begin{split}\begin{aligned}
  \dfrac{\partial K}{\partial \dot{q}^l} 
    = \dfrac{\partial }{\partial \dot{q}^l} \left[ \frac{1}{2} A_{ij} \dot{q}^i \dot{q}^j \right] 
    = \frac{1}{2} A_{ij} \bigg[ \underbrace{ \dfrac{\partial \dot{q}^i}{\partial \dot{q}^l} }_{\delta^i_l} \dot{q}^j + \dot{q}^i \underbrace{ \dfrac{\partial \dot{q}^j}{\partial \dot{q}^l}}_{\delta^j_l} \bigg]  
    = \frac{1}{2} \left[ A_{lj} \dot{q}^j + A_{il} \dot{q}^i \right]
    = A_{lj} \dot{q}^j
\end{aligned}\end{split}
\end{equation*}
\sphinxstepscope


\subsection{Lagrangian function with explicit dependence on time}
\label{\detokenize{ch/lagrange-t-explicit:lagrangian-function-with-explicit-dependence-on-time}}\label{\detokenize{ch/lagrange-t-explicit:classical-mechanics-lagrange-time-dependent}}\label{\detokenize{ch/lagrange-t-explicit::doc}}
\sphinxAtStartPar
Some problems may have a Lagrangian function with an explicit dependence on time,
\begin{equation*}
\begin{split}\mathscr{L}\left(\dot{q}^k(t),q^k(t),t\right) \ ,\end{split}
\end{equation*}
\sphinxAtStartPar
in some cases like:
\begin{enumerate}
\sphinxsetlistlabels{\arabic}{enumi}{enumii}{}{.}%
\item {} 
\sphinxAtStartPar
time\sphinxhyphen{}dependent contraints, whose motion is prescribed

\item {} 
\sphinxAtStartPar
time\sphinxhyphen{}dependent forces that can be included in the potential energy

\item {} 
\sphinxAtStartPar
choice of coordinates \(q^k(t)\) that gives an explicit dependence on time of the physical coordinates, like positions and orientations of rigid bodies
\begin{equation*}
\begin{split}\vec{r}_P\left(q^k(t),t \right) \quad , \quad \mathbb{R}_P\left( q^{k}(t), t \right)\end{split}
\end{equation*}
\sphinxAtStartPar
and leading to an explicit dependence on time of the velocities and angular velocities
\begin{equation*}
\begin{split}\begin{aligned}
      \vec{v}_P\left(\dot{q}^k(t), q^k(t),t \right)  = \dot{q}^i(t) \dfrac{\partial \vec{r}_P}{\partial q^i} \left(q^k(t),t \right) + \dfrac{\partial \vec{r}_P}{\partial t} \left(q^k(t),t \right)  \\
      \vec{\omega}_P\left(\dot{q}^k(t), q^{k}(t), t \right) = \dot{q}^i(t) \dfrac{\partial \vec{\theta}_P}{\partial q^i} \left(q^k(t),t \right) + \dfrac{\partial \vec{\theta}_P}{\partial t}\left(q^k(t),t \right) 
    \end{aligned}\end{split}
\end{equation*}
\sphinxAtStartPar
and thus of the kinetic energy.

\end{enumerate}

\sphinxAtStartPar
In general, external actions with net power are required for conditions 1., 2., i.e. for moving constraints or for changing potential fields acting on the system: even if variable constraints or external forcing acting on a system are prescribed and thus add no degree of freedom to the system, they’re not free in general but requires external actions and power, as it can be realized looking at the balance equation of mechanica energy.

\sphinxAtStartPar
Condition 3. is usually a result of a choice of coordinates in a non\sphinxhyphen{}inertial reference frame, whose motion is not described by the coordinates themselves.


\subsubsection{Energy balance}
\label{\detokenize{ch/lagrange-t-explicit:energy-balance}}\begin{equation*}
\begin{split}\begin{aligned}
  \dfrac{d E}{d t}
  & = \dfrac{d K}{d t} - \dfrac{d U}{d t} = \\
  & = 2 \dfrac{d K}{d t} - \dfrac{d \mathscr{L}}{dt} = \\
  & = 2 \dfrac{d K}{d t} - \ddot{q}^k \dfrac{\partial \mathscr{L}}{\partial \dot{q}^k} - \dot{q}^k \dfrac{\partial \mathscr{L}}{\partial q^k} - \dfrac{\partial \mathscr{L}}{\partial t} = \\
  & = 2 \dfrac{d K}{d t} - \dfrac{d}{dt} \left( \dot{q}^k \dfrac{\partial \mathscr{L}}{\partial \dot{q}^k} \right) + \dot{q}^k \dfrac{d}{dt}\dfrac{\partial \mathscr{L}}{\partial \dot{q}^k} - \dot{q}^k \dfrac{\partial \mathscr{L}}{\partial q^k} - \dfrac{\partial \mathscr{L}}{\partial t} = \\
  & = \dfrac{d }{d t} \left[ 2 K - \dot{q}^k \dfrac{\partial \mathscr{L}}{\partial \dot{q}^k} \right] + \dot{q}^k \underbrace{ \left[ \dfrac{d}{dt}\dfrac{\partial \mathscr{L}}{\partial \dot{q}^k} - \dfrac{\partial \mathscr{L}}{\partial q^k} \right]}_{= Q_k} - \dfrac{\partial \mathscr{L}}{\partial t} = \\
  & = \dot{q}^k Q_k - \dfrac{\partial \mathscr{L}}{\partial t} + \dfrac{d}{dt} \left[ \dot{q}^k B_k + 2C \right]
\end{aligned}\end{split}
\end{equation*}

\subsubsection{Properties of kinetic energy and potential}
\label{\detokenize{ch/lagrange-t-explicit:properties-of-kinetic-energy-and-potential}}
\sphinxAtStartPar
The kinetic energy has a general expression
\begin{equation*}
\begin{split}K = \frac{1}{2} \dot{q}^i \dot{q}^j A_{ij}\left(q^k(t),t\right) + \dot{q}^i B_{i}\left(q^k(t),t\right) + C\left(q^k(t),t\right)\end{split}
\end{equation*}\begin{equation*}
\begin{split}\dot{q}^k \dfrac{\partial K}{\partial \dot{q}^k} = \dot{q}^k \left( A_{kj} \dot{q}^j + B_k \right)\end{split}
\end{equation*}\begin{equation*}
\begin{split}\begin{aligned}
  2K - \dot{q}^k \dfrac{\partial K}{\partial \dot{q}^k}
  & = \dot{q}^k A_{kj} \dot{q}^j + 2 \dot{q}^k B_k + 2 C - \left[ \dot{q}^k \left( A_{kj} \dot{q}^j + B_k \right) \right] = \\
  & = \dot{q}^k B_k + 2 C
\end{aligned}\end{split}
\end{equation*}\label{ch/lagrange-t-explicit:example-0}
\begin{sphinxadmonition}{note}{Example 5.5.1}


\end{sphinxadmonition}

\sphinxstepscope




\chapter{Hamiltonian Mechanics}
\label{\detokenize{ch/hamilton:hamiltonian-mechanics}}\label{\detokenize{ch/hamilton:classical-mechanics-hamilton}}\label{\detokenize{ch/hamilton::doc}}
\sphinxAtStartPar
Riformulazione ulteriore della meccanica di Newton, a partire dalla meccanica di Lagrange.
Fornisce le basi per un approccio moderno anche in altre teorie della Fisica. \sphinxstylestrong{dots…}

\sphinxAtStartPar
Starting from Lagrange equations derived in {\hyperref[\detokenize{ch/lagrange:classical-mechanics-lagrange}]{\sphinxcrossref{\DUrole{std,std-ref}{Lagrangian mechanics}}}},
\begin{equation*}
\begin{split}\dfrac{d}{dt}\Big( \frac{\partial \mathscr{L}}{\partial \dot{q}} \Big) - \frac{\partial \mathscr{L}}{\partial q} = Q_q\end{split}
\end{equation*}
\sphinxAtStartPar
the \sphinxstylestrong{generalized moment} is defined as
\begin{equation*}
\begin{split}p_k := \frac{\partial \mathscr{L}}{\partial \dot{q}^k} \ ,\end{split}
\end{equation*}
\sphinxAtStartPar
and the \sphinxstylestrong{Hamiltonian function} as
\begin{equation*}
\begin{split}\mathscr{H}(q^k(t), p_k(t), t) := p_k \dot{q}^k - \mathscr{L}(\dot{q}^l(q^k, p_k, t), q^l(t), t) \ ,\end{split}
\end{equation*}
\sphinxAtStartPar
its differential reads
\begin{equation*}
\begin{split}\begin{aligned}
d\mathscr{H} & = dq^k \, \frac{\partial \mathscr{H}}{\partial q^k} + dp_k \, \frac{\partial \mathscr{H}}{\partial p_k} + dt \,  \frac{\partial \mathscr{H}}{\partial t} = \\
& = d p_k \, \dot{q}^k + \underbrace{ p_k \, d \dot{q}^k - d \dot{q}^k \, \frac{\partial \mathscr{L}}{\partial \dot{q}^k}}_{=0} - d q^k \, \frac{\partial \mathscr{L}}{\partial q^k} - dt \, \frac{\partial \mathscr{L}}{\partial t}
\end{aligned}\end{split}
\end{equation*}
\sphinxAtStartPar
and thus it follows
\begin{equation*}
\begin{split}\begin{cases}
 \dot{q}^k & = \dfrac{\partial \mathscr{H}}{\partial p_k} \\
 \dfrac{\partial \mathscr{H}}{\partial q^k} & = - \dfrac{\partial \mathscr{L}}{\partial q^k} \\
 \dfrac{\partial \mathscr{H}}{\partial t} & = - \dfrac{\partial\mathscr{L}}{\partial t} \ .
\end{cases}\end{split}
\end{equation*}
\sphinxAtStartPar
Recasting Lagrange equations as
\begin{equation*}
\begin{split}\frac{\partial \mathscr{L}}{\partial q^k} = - Q_{q^k} + \dfrac{d}{dt}\Big( \frac{\partial \mathscr{L}}{\partial \dot{q}^k} \Big) = -Q_{q^k} + \dot{p}_k\end{split}
\end{equation*}
\sphinxAtStartPar
\sphinxstylestrong{Hamilton equations} follow
\begin{equation*}
\begin{split}\begin{cases}
 \dot{q}^k & = \dfrac{\partial H}{\partial p_k} \\
 \dot{p}_k & =-\dfrac{\partial H}{\partial q^k} + Q_{q^k} \ .
\end{cases}\end{split}
\end{equation*}
\sphinxstepscope


\part{Exercises}

\sphinxstepscope


\chapter{Dynamics}
\label{\detokenize{ch/exercises-dynamics:dynamics}}\label{\detokenize{ch/exercises-dynamics:classical-mechanics-exercise-dynamics}}\label{\detokenize{ch/exercises-dynamics::doc}}

\phantomsection \label{exercise:ch/exercises-dynamics-exercise-0}

\begin{sphinxadmonition}{note}{Exercise 7.1}



\sphinxAtStartPar
\sphinxstylestrong{todo} Add description of the problem and image

\sphinxAtStartPar
Find:
\begin{itemize}
\item {} 
\sphinxAtStartPar
the pure equations of motion (solution methods: dynamical equilibria, Lagrange II,…) \sphinxstylestrong{done}

\item {} 
\sphinxAtStartPar
constraint reactions (solution methods: dynamical equilibria, Lagrange I,…) \sphinxstylestrong{todo}

\item {} 
\sphinxAtStartPar
equilibria and their stability \sphinxstylestrong{todo}

\item {} 
\sphinxAtStartPar
evolution of the linear(ized) dynamics of the system around stable equilibria \sphinxstylestrong{todo}

\end{itemize}
\end{sphinxadmonition}
\subsubsection*{Solution.}

\sphinxAtStartPar
\sphinxstylestrong{Kinematics.} Using \(x\), \(\theta\) as generalized coordinates,
\begin{equation*}
\begin{split}\begin{cases}
  \vec{r}_A = x \hat{x} \\
  \vec{r}_B = ( x + \ell \sin \theta ) \hat{x} + \ell \cos \theta \hat{y} \\
\end{cases}\end{split}
\end{equation*}\begin{equation*}
\begin{split}\begin{cases}
  \vec{v}_A = \dot{x} \hat{x} \\
  \vec{v}_B = ( \dot{x} + \ell \dot{\theta} \cos \theta ) \hat{x} - \ell \dot{\theta} \sin \theta \hat{y} \\
\end{cases}\end{split}
\end{equation*}
\sphinxAtStartPar
\sphinxstylestrong{Lagrangian function.}
\begin{equation*}
\begin{split}\mathscr{L} = K + U\end{split}
\end{equation*}\begin{equation*}
\begin{split}\begin{aligned}
  K & = \frac{1}{2} m_A |\vec{v}_A|^2 + \frac{1}{2} m_B |\vec{v}_B|^2 = \\
    & = \frac{1}{2} m_A \dot{x}^2 + \frac{1}{2} m_B \left( \dot{x}^2 + \ell^2 \dot{\theta}^2 + 2 \ell \dot{x} \dot{\theta} \cos \theta  \right) 
\end{aligned}\end{split}
\end{equation*}\begin{equation*}
\begin{split}\begin{aligned}
  U & = - \frac{1}{2} k x_A^2 + m_B g y_B = \\
    & = - \frac{1}{2} k x^2 + m_B g \ell \cos \theta  
\end{aligned}\end{split}
\end{equation*}
\sphinxAtStartPar
\sphinxstylestrong{Lagrange equations (II type)} RHS of Lagrange equations read
\begin{equation*}
\begin{split}\begin{aligned}
  \dfrac{d}{dt}\left( \frac{\partial \mathscr{L}}{\partial \dot{x}} \right) & = \dfrac{d}{dt} \left( m_A \dot{x} + m_B \dot{x} + m_B \ell \dot{\theta} \cos \theta \right) = \left( m_A + m_B \right) \ddot{x} + m_B \ell \ddot{\theta} \cos \theta - m_B \ell \dot{\theta}^2 \sin \theta \\
                      \frac{\partial \mathscr{L}}{\partial      x }         & = -k x \\
\end{aligned}\end{split}
\end{equation*}\begin{equation*}
\begin{split}\begin{aligned}
  \dfrac{d}{dt}\left( \frac{\partial \mathscr{L}}{\partial \dot{\theta}} \right) & = \dfrac{d}{dt} \left( m_B \ell^2 \dot{\theta} + m_B \ell \dot{x} \cos \theta \right) = m_B \ell^2 \ddot{\theta} + m_B \ell \ddot{x} \cos \theta - m_B \ell \dot{x} \dot{\theta} \sin \theta  \\
                      \frac{\partial \mathscr{L}}{\partial      \theta }         & = - m_B \ell \dot{x} \dot{\theta} \sin \theta - m_B g \ell \sin \theta  \\
\end{aligned}\end{split}
\end{equation*}
\sphinxAtStartPar
Generalized forces read
\begin{equation*}
\begin{split}Q_x = F\end{split}
\end{equation*}\begin{equation*}
\begin{split}Q_\theta = C\end{split}
\end{equation*}
\sphinxAtStartPar
so that the pure equations of motion follows from Lagrange equations \(\frac{d}{dt} \left( \frac{\partial \mathscr{L}}{\partial \dot{q}} \right) - \frac{\partial \mathscr{L}}{\partial q} = Q_q\)
\begin{equation*}
\begin{split}\begin{cases}
  \left( m_A + m_B \right) \ddot{x} + m_B \ell \ddot{\theta} \cos \theta - m_B \ell \dot{\theta}^2 \sin \theta + k x = F \\ 
  m_B \ell^2 \ddot{\theta} + m_B \ell \ddot{x} \cos \theta - m_B \ell \dot{x} \dot{\theta} \sin \theta + m_B \ell \dot{x} \dot{\theta} \sin \theta + m_B g \ell \sin \theta = C \\ 
\end{cases}\end{split}
\end{equation*}
\sphinxAtStartPar
and after the simplifications
\begin{equation*}
\begin{split}\begin{cases}
  \left( m_A + m_B \right) \ddot{x} + m_B \ell \ddot{\theta} \cos \theta - m_B \ell \dot{\theta}^2 \sin \theta + k x = F \\ 
  m_B \ell \ddot{x} \cos \theta + m_B \ell^2 \ddot{\theta} + m_B g \ell \sin \theta = C \\ 
\end{cases}\end{split}
\end{equation*}
\sphinxAtStartPar
\sphinxstylestrong{Obs.} The first equation is the \(x\)\sphinxhyphen{}component of the momentum equation of the whole system. The second equation is the angular momentum equation of the rod around the hinge in \(A\).
\subsubsection*{Generalized forces on rigid bodies}

\sphinxAtStartPar
Following the derivation of the {\hyperref[\detokenize{ch/lagrange-rigid-body:classical-mechanics-lagrange-rigid}]{\sphinxcrossref{\DUrole{std,std-ref}{Lagrange equations for rigid bodies}}}}, generalized forces are
\begin{equation*}
\begin{split}Q_q = \frac{\partial \vec{r}_G}{\partial q} \cdot \vec{R}^e + \frac{\partial \vec{\theta}}{\partial q} \cdot \vec{M}_G^e\end{split}
\end{equation*}
\sphinxAtStartPar
With the definition of \(\theta_{\delta}\) in the increment of the rigid body motion
\begin{equation*}
\begin{split}d \vec{r}_A = d \vec{r}_B + \theta_{\delta} \times (A - B) \ ,\end{split}
\end{equation*}
\sphinxAtStartPar
and writing the resultant of forces and moments as
\begin{equation*}
\begin{split}\begin{aligned}
  \vec{R}^e   & = \sum_i \vec{F}_i \\
  \vec{M}^e_G & = \sum_i \left( \vec{r}_G - \vec{r}_i \right) \times \vec{F}_i + \sum_i \vec{C}_i \\
\end{aligned}\end{split}
\end{equation*}
\sphinxAtStartPar
the generalized force can be recast as
\begin{equation*}
\begin{split}\begin{aligned}
  Q_q 
  & = \frac{\partial \vec{r}_G}{\partial q} \cdot \vec{R}^e + \frac{\partial \vec{\theta}}{\partial q} \cdot \vec{M}_G^e = \\
  & = \frac{\partial \vec{r}_G}{\partial q} \cdot \sum_i \vec{F}_i + \frac{\partial \vec{\theta}}{\partial q} \cdot \left[ \sum_i \left( \vec{r}_G - \vec{r}_i \right) \times \vec{F}_i + \sum_i \vec{C}_i  \right] = \\
  & = \sum_i \left[ \frac{\partial \vec{r}_G}{\partial q} + \frac{\partial \vec{\theta}}{\partial q} \times \left( \vec{r}_G - \vec{r}_i \right) \right] \cdot \vec{F}_i + \sum_i \frac{\partial \vec{\theta}}{\partial q} \cdot \vec{C}_i = \\
  & = \sum_i \frac{\partial \vec{r}_i}{\partial q} \cdot \vec{F}_i + \sum_i \frac{\partial \vec{\theta}}{\partial q} \cdot \vec{C}_i \ ,
\end{aligned}\end{split}
\end{equation*}
\sphinxAtStartPar
i.e. as the contribution of the single forces \(\vec{F}_i\) acting on different points \(\vec{r}_i\) the rigid body and the overall contribution of the couples of forces \(\vec{C}_i\).

\sphinxAtStartPar
With \(\vec{r}_C\left(q(t),t \right)\) and \(\mathbb{R}\left(q(t), t \right)\),
\begin{equation*}
\begin{split}\vec{r}_i(q(t), t) - \vec{r}_C(q(t), t) = \mathbb{R}(q(t), t) \cdot \left( \vec{r}_i^0 - \vec{r}_C^0 \right)\end{split}
\end{equation*}
\sphinxAtStartPar
Time derivative becomes
\begin{equation*}
\begin{split}\begin{aligned}
\vec{v}_i - \vec{v}_C
 & = \left[ \dot{q}(t) \, \frac{\partial \mathbb{R}}{\partial q} + \frac{\partial \mathbb{R}}{\partial t} \right] \cdot \left( \vec{r}_i^0 - \vec{r}_C^0 \right) = \\
 & =  \left[ \dot{q}(t) \, \frac{\partial \mathbb{R}}{\partial q} + \frac{\partial \mathbb{R}}{\partial t} \right] \cdot \mathbb{R}^T \cdot \underbrace{\mathbb{R} \cdot \left( \vec{r}_i^0 - \vec{r}_C^0 \right)}_{\vec{r}_i - \vec{r}_C}
\end{aligned}\end{split}
\end{equation*}\begin{equation*}
\begin{split}\begin{aligned}
  \vec{\omega}_\times & = \dot{\mathbb{R}} \cdot \mathbb{R}^T \\
  \delta \vec{\theta}_{\times} & = \delta \mathbb{R} \cdot \mathbb{R}^T \\
  \frac{\partial \vec{\theta}_{\times}}{\partial q} & = \dfrac{\partial \mathbb{R}}{\partial q} \cdot \mathbb{R}^T
\end{aligned}\end{split}
\end{equation*}

\phantomsection \label{exercise:ch/exercises-dynamics-exercise-1}

\begin{sphinxadmonition}{note}{Exercise 7.2}



\sphinxAtStartPar
\sphinxstylestrong{todo} Add description of the problem and image

\sphinxAtStartPar
Find:
\begin{itemize}
\item {} 
\sphinxAtStartPar
the pure equations of motion (solution methods: dynamical equilibria, Lagrange II, Kinetic energy theorem \sphinxhyphen{} energy conservation \sphinxhyphen{} since the problem has 1 dof,…) \sphinxstylestrong{done}

\item {} 
\sphinxAtStartPar
constraint reactions (solution methods: dynamical equilibria, Lagrange I,…) \sphinxstylestrong{todo}

\item {} 
\sphinxAtStartPar
equilibria and their stability \sphinxstylestrong{todo}

\item {} 
\sphinxAtStartPar
evolution of the linear(ized) dynamics of the system around stable equilibria \sphinxstylestrong{todo}

\end{itemize}
\end{sphinxadmonition}
\subsubsection*{Solution}

\sphinxAtStartPar
\sphinxstylestrong{Geometry.}
\begin{equation*}
\begin{split}\begin{aligned}
  R & = d \sin \alpha \\
  b & = d \cos^2 \alpha \\
\end{aligned}\end{split}
\end{equation*}
\sphinxAtStartPar
so that \(\frac{b}{R} = \frac{\cos^2 \alpha}{\sin \alpha}\).

\sphinxAtStartPar
\sphinxstylestrong{Kinematics.}

\sphinxAtStartPar
Position of the center of mass, \(C\)
\begin{equation*}
\begin{split}\begin{aligned}
  \vec{r}_C & = b \cos \theta \hat{x} + b \sin \theta \hat{y} - h \hat{z} \\
  \vec{v}_C & = -b \dot{\theta} \sin \theta \hat{x} + b \dot{\theta} \cos \theta \hat{y} = b \dot{\theta} \hat{y}_1 \\
\end{aligned}\end{split}
\end{equation*}
\sphinxAtStartPar
Angular velocity \(\vec{\omega}\) of the rigid body
\begin{equation*}
\begin{split}\vec{\omega} = \dot{\theta} \hat{z} + \dot{\varphi} \hat{x}_1\end{split}
\end{equation*}
\sphinxAtStartPar
Velocity of point contact point \(A\) is zero, \(\vec{v}_A = \vec{0}\) for \sphinxstylestrong{pure rolling} constraint. Being \((A-C) = R \hat{z}_1\), the general expression of \(\vec{v}_A\) as function of \(\theta\) and \(\varphi\) reads
\begin{equation*}
\begin{split}\begin{aligned}
 \vec{v}_A
 & = \vec{v}_C + \vec{\omega} \times \left( A - C \right) = \\
 & = \hat{y}_1 b \dot{\theta} + \left( \dot{\theta} \hat{z} + \dot{\varphi} \hat{x}_1 \right) \times R \hat{z}_1 = \\
 & = \hat{y}_1 \left( b \dot{\theta} + R \dot{\theta} \sin \alpha - R \dot{\varphi} \right) = \\
\end{aligned}\end{split}
\end{equation*}
\sphinxAtStartPar
so that the kinetatic constraint (integrable, with arbitrary initial condition) between \(\theta\) and \(\varphi\) is
\begin{equation*}
\begin{split}R \varphi = \left( R \sin \alpha + b \right) \, \theta \ .\end{split}
\end{equation*}
\sphinxAtStartPar
\sphinxstylestrong{Lagrangian function.}

\sphinxAtStartPar
With
\begin{equation*}
\begin{split}\mathbb{I}_C = I_{x} \hat{x}_1 \hat{x}_1 + I_{y} \hat{y}_1 \hat{y}_1 + I_{z} \hat{z}_1 \hat{z}_1 \ ,\end{split}
\end{equation*}\begin{equation*}
\begin{split}\vec{\omega} = \dot{\theta} \hat{z} + \dot{\varphi} \hat{x}_1 = \left( \dot{\varphi} - \dot{\theta} \sin \alpha  \right) \hat{x}_1 + \dot{\theta} \cos \alpha \hat{z}_1 \end{split}
\end{equation*}
\sphinxAtStartPar
and using
\begin{equation*}
\begin{split}\dot{\varphi} - \dot{\theta} \sin \alpha = \frac{b}{R} \dot{\theta} \ ,\end{split}
\end{equation*}
\sphinxAtStartPar
the Lagrangian function becomes
\begin{equation*}
\begin{split}\begin{aligned}
  \mathscr{L} & = K + U = \\
  & =\frac{1}{2} m |\vec{v}_C|^2 + \frac{1}{2} \vec{\omega} \cdot \mathbb{I}_C \cdot \vec{\omega} + m g x_C = \\
  & =\frac{1}{2} m b^2 \dot{\theta}^2 + \frac{1}{2} \left[ I_x \left( \dot{\varphi} - \dot{\theta} \sin \alpha \right)^2 + I_z \dot{\theta}^2 \cos^2 \alpha \right] + m g b \cos \theta = \\
  & =\frac{1}{2} m b^2 \dot{\theta}^2 + \frac{1}{2} \left[ I_x \left( \frac{b}{R} \right)^2 + I_z \cos^2 \alpha \right] \dot{\theta}^2 + m g b \cos \theta = \\
  & =\frac{1}{2} \widetilde{I} \dot{\theta}^2 + m g b \cos \theta  \ ,
\end{aligned}\end{split}
\end{equation*}
\sphinxAtStartPar
with the equivalent inertia
\begin{equation*}
\begin{split}\widetilde{I} = m b^2 + I_x \left( \frac{b}{R} \right)^2 + I_z \cos^2 \alpha \ .\end{split}
\end{equation*}
\sphinxAtStartPar
\sphinxstylestrong{Method 1. Lagrange equation (II).} Legrange equation gives a pure equation of motion
\begin{equation*}
\begin{split}\begin{aligned}
  0 & = \dfrac{d}{dt}\left( \frac{\partial \mathscr{L}}{\partial \dot{\theta}} \right) - \frac{\partial \mathscr{L}}{\partial \theta} = \widetilde{I} \ddot{\theta} + m g b \sin \theta \ . \\
\end{aligned}\end{split}
\end{equation*}
\sphinxAtStartPar
\sphinxstylestrong{Method 2. Kinetic energy theorem \sphinxhyphen{} or energy conservation}


\phantomsection \label{exercise:ch/exercises-dynamics-exercise-2}

\begin{sphinxadmonition}{note}{Exercise 7.3}


\end{sphinxadmonition}
\subsubsection*{Solution}

\sphinxAtStartPar
\sphinxstylestrong{Kinematics.} \sphinxstylestrong{todo} \sphinxstyleemphasis{check kinematic constraints. No influence of \(\theta\)?}
\begin{equation*}
\begin{split}\varphi_2 = \frac{R_1}{R_2} \varphi_1 = r \varphi_1\end{split}
\end{equation*}
\sphinxAtStartPar
\sphinxstylestrong{Lagrangian function.}
\begin{equation*}
\begin{split}\begin{aligned}
K 
& = \frac{1}{2} I_1 \omega_1^2 + \frac{1}{2} I_2 \omega_2^2 + \frac{1}{2} m_1 |\vec{v}_1|^2 = \\
& = \frac{1}{2} I_1 \dot{\varphi}_1^2 + \frac{1}{2} I_2 \dot{\varphi}_2^2 + \frac{1}{2} m_1 \ell^2 \dot{\theta}^2 = \\
& = \frac{1}{2} \left( I_1 + I_2 r^2 \right) \dot{\varphi}_1^2 + \frac{1}{2} m_1 \ell^2 \dot{\theta}^2 = \\
\end{aligned}\end{split}
\end{equation*}\begin{equation*}
\begin{split}\begin{aligned}
  U = - \frac{1}{2} k \theta^2 - m g \ell \sin \theta
\end{aligned}\end{split}
\end{equation*}
\sphinxAtStartPar
Generalized forces read
\begin{equation*}
\begin{split}\begin{aligned}
  Q_{\varphi_1} & = C - R_1 F \\
  Q_{\theta} & = C
\end{aligned}\end{split}
\end{equation*}
\sphinxAtStartPar
\sphinxstylestrong{Lagrange functions (II).}
\begin{equation*}
\begin{split}\begin{cases}
  \left( I_1 + I_2 r^2 \right) \ddot{\varphi} = C - R_1 F \\
  m_1 \ell^2 \ddot{\theta} + k \theta + m g \ell \cos \theta = C \\
\end{cases}\end{split}
\end{equation*}

\phantomsection \label{exercise:ch/exercises-dynamics-exercise-3}

\begin{sphinxadmonition}{note}{Exercise 7.4 (Inverted pendulum)}


\end{sphinxadmonition}
\subsubsection*{Solution}

\sphinxstepscope


\chapter{Gravitation}
\label{\detokenize{ch/exercises-gravitation:gravitation}}\label{\detokenize{ch/exercises-gravitation:classical-mechanics-exercise-gravitation}}\label{\detokenize{ch/exercises-gravitation::doc}}\phantomsection \label{exercise:ch/exercises-gravitation-exercise-0}

\begin{sphinxadmonition}{note}{Exercise 8.1 (Ball falling in a tunnel through a planet)}



\sphinxAtStartPar
A ball of mass \(m\) moves through a tunnel drilled through a planet of radius \(R\), mass \(M\) and uniform mass distribution. Neglecting the “mas defect” due to the tunnel in the planet,
\begin{itemize}
\item {} 
\sphinxAtStartPar
provide the expression of the force acting on the ball in the tunnel

\item {} 
\sphinxAtStartPar
assuming no rotation of the planet, and zero initial velocity of the ball, provide the dynamical equation governing the motion of the ball and integrate it to find the lwa of motion

\end{itemize}
\end{sphinxadmonition}

\sphinxAtStartPar
Uniform mass density reads \(\rho = \frac{M}{V} = \frac{M}{\frac{4}{3}\pi R^3}\).
Exploiting symmetry, the gravitational field can be a function of the distance \(r\) of the center of the planet only, and have radial direction,
\begin{equation}\label{equation:ch/exercises-gravitation:eq:g:radial}
\begin{split}\vec{g} = - g(r) \hat{r} \ .\end{split}
\end{equation}
\sphinxAtStartPar
Formula \eqref{equation:ch/actions-examples:eq:g:flux:V} of the flux of the gravitational field,
\begin{equation*}
\begin{split}  \oint_{\vec{r} \in \partial V} \vec{g}(\vec{r}) \cdot \hat{n}(\vec{r}) = - G \int_{\vec{r} \in V} 4 \pi \rho(\vec{r}) \end{split}
\end{equation*}
\sphinxAtStartPar
across the surface of a sphere of radius \(r\) \sphinxhyphen{} that has outward pointing unit normal vector \(\hat{n}(\vec{r}) = \hat{r}(\vec{r})\) \sphinxhyphen{}, exploting expression \eqref{equation:ch/exercises-gravitation:eq:g:radial} from symmetry, becomes for \(r < R\)
\begin{equation*}
\begin{split}- g(r) 4 \pi r^2 = - G \, 4  \pi \rho \frac{4}{3} \pi r^3 = - 4 \pi G M \frac{r^3}{R^3}\end{split}
\end{equation*}
\sphinxAtStartPar
and thus
\begin{equation*}
\begin{split}g(r) = \frac{GM}{R^3} r \ \qquad \rightarrow \qquad \vec{g}(\vec{r}) = - m \frac{G M}{R^3} r \hat{r} = - m \frac{GM}{R^3} \vec{r} \ .\end{split}
\end{equation*}
\sphinxAtStartPar
Force acting on the ball of mass \(m\) thus reads \(\vec{F}(\vec{r}) = m \vec{g}(\vec{r})\). The equation of motion becomes
\begin{equation*}
\begin{split}m \ddot{\vec{r}} = \vec{F}(\vec{r}) = - m \frac{G M}{R^3} \vec{r} \ ,\end{split}
\end{equation*}
\sphinxAtStartPar
a linear second\sphinxhyphen{}order ODE with constant coefficients, whose solution provides an harmonic motion with pulsation \(\Omega = \sqrt{\frac{GM}{R^3}}\).
The solution of this equation, with initial conditions at rest on the surface of the planet,
\begin{equation*}
\begin{split}\begin{cases}
  \vec{r}(0) = \vec{r}_0 = R \hat{r} \\
  \dot{\vec{r}}(0) = \vec{0}
\end{cases}\end{split}
\end{equation*}
\sphinxAtStartPar
reads
\begin{equation*}
\begin{split}\vec{r}(t) = \vec{r}_0 \cos \left( \sqrt{\dfrac{GM}{R^3}} \, t \right) = \hat{r} \, R \cos \left( \sqrt{\dfrac{GM}{R^3}} \, t \right) \ .\end{split}
\end{equation*}\phantomsection \label{exercise:ch/exercises-gravitation-exercise-1}

\begin{sphinxadmonition}{note}{Exercise 8.2}



\sphinxAtStartPar
Investigate the dynamics of the ball in the previous problem, if the rotational motion of the planet around its axis is considered and if the ball is thrown in the tunnel with non\sphinxhyphen{}zero velocity.
\begin{itemize}
\item {} 
\sphinxAtStartPar
Normal actions of the wall of the tunnel on the ball

\item {} 
\sphinxAtStartPar
At the end of the tunnel, the ball moves above the planet surface while it’s attracted “downwards”. When the ball comes back to the planet surface, does it target the tunnel?

\end{itemize}
\end{sphinxadmonition}






\renewcommand{\indexname}{Proof Index}
\begin{sphinxtheindex}
\let\bigletter\sphinxstyleindexlettergroup
\bigletter{def:true\sphinxhyphen{}forces}
\item\relax\sphinxstyleindexentry{def:true\sphinxhyphen{}forces}\sphinxstyleindexextra{ch/actions}\sphinxstyleindexpageref{ch/actions:\detokenize{def:true-forces}}
\indexspace
\bigletter{def:true\sphinxhyphen{}motion}
\item\relax\sphinxstyleindexentry{def:true\sphinxhyphen{}motion}\sphinxstyleindexextra{ch/actions}\sphinxstyleindexpageref{ch/actions:\detokenize{def:true-motion}}
\indexspace
\bigletter{definition\sphinxhyphen{}0}
\item\relax\sphinxstyleindexentry{definition\sphinxhyphen{}0}\sphinxstyleindexextra{ch/kinematics}\sphinxstyleindexpageref{ch/kinematics:\detokenize{definition-0}}
\indexspace
\bigletter{definition\sphinxhyphen{}1}
\item\relax\sphinxstyleindexentry{definition\sphinxhyphen{}1}\sphinxstyleindexextra{ch/kinematics}\sphinxstyleindexpageref{ch/kinematics:\detokenize{definition-1}}
\indexspace
\bigletter{definition\sphinxhyphen{}2}
\item\relax\sphinxstyleindexentry{definition\sphinxhyphen{}2}\sphinxstyleindexextra{ch/lagrange\sphinxhyphen{}constraints}\sphinxstyleindexpageref{ch/lagrange-constraints:\detokenize{definition-2}}
\indexspace
\bigletter{example\sphinxhyphen{}0}
\item\relax\sphinxstyleindexentry{example\sphinxhyphen{}0}\sphinxstyleindexextra{ch/lagrange\sphinxhyphen{}t\sphinxhyphen{}explicit}\sphinxstyleindexpageref{ch/lagrange-t-explicit:\detokenize{example-0}}
\indexspace
\bigletter{holonomic\sphinxhyphen{}constraint}
\item\relax\sphinxstyleindexentry{holonomic\sphinxhyphen{}constraint}\sphinxstyleindexextra{ch/lagrange\sphinxhyphen{}constraints}\sphinxstyleindexpageref{ch/lagrange-constraints:\detokenize{holonomic-constraint}}
\indexspace
\bigletter{non\sphinxhyphen{}holonomic\sphinxhyphen{}constraint}
\item\relax\sphinxstyleindexentry{non\sphinxhyphen{}holonomic\sphinxhyphen{}constraint}\sphinxstyleindexextra{ch/lagrange\sphinxhyphen{}constraints}\sphinxstyleindexpageref{ch/lagrange-constraints:\detokenize{non-holonomic-constraint}}
\indexspace
\bigletter{thm:huygens}
\item\relax\sphinxstyleindexentry{thm:huygens}\sphinxstyleindexextra{ch/inertia}\sphinxstyleindexpageref{ch/inertia:\detokenize{thm:huygens}}
\end{sphinxtheindex}

\renewcommand{\indexname}{Index}
\printindex
\end{document}