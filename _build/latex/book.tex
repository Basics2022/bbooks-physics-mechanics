%% Generated by Sphinx.
\def\sphinxdocclass{jupyterBook}
\documentclass[letterpaper,10pt,italian]{jupyterBook}
\ifdefined\pdfpxdimen
   \let\sphinxpxdimen\pdfpxdimen\else\newdimen\sphinxpxdimen
\fi \sphinxpxdimen=.75bp\relax
\ifdefined\pdfimageresolution
    \pdfimageresolution= \numexpr \dimexpr1in\relax/\sphinxpxdimen\relax
\fi
%% let collapsible pdf bookmarks panel have high depth per default
\PassOptionsToPackage{bookmarksdepth=5}{hyperref}
%% turn off hyperref patch of \index as sphinx.xdy xindy module takes care of
%% suitable \hyperpage mark-up, working around hyperref-xindy incompatibility
\PassOptionsToPackage{hyperindex=false}{hyperref}
%% memoir class requires extra handling
\makeatletter\@ifclassloaded{memoir}
{\ifdefined\memhyperindexfalse\memhyperindexfalse\fi}{}\makeatother

\PassOptionsToPackage{warn}{textcomp}

\catcode`^^^^00a0\active\protected\def^^^^00a0{\leavevmode\nobreak\ }
\usepackage{cmap}
\usepackage{fontspec}
\defaultfontfeatures[\rmfamily,\sffamily,\ttfamily]{}
\usepackage{amsmath,amssymb,amstext}
\usepackage{polyglossia}
\setmainlanguage{italian}



\setmainfont{FreeSerif}[
  Extension      = .otf,
  UprightFont    = *,
  ItalicFont     = *Italic,
  BoldFont       = *Bold,
  BoldItalicFont = *BoldItalic
]
\setsansfont{FreeSans}[
  Extension      = .otf,
  UprightFont    = *,
  ItalicFont     = *Oblique,
  BoldFont       = *Bold,
  BoldItalicFont = *BoldOblique,
]
\setmonofont{FreeMono}[
  Extension      = .otf,
  UprightFont    = *,
  ItalicFont     = *Oblique,
  BoldFont       = *Bold,
  BoldItalicFont = *BoldOblique,
]



\usepackage[Sonny]{fncychap}
\ChNameVar{\Large\normalfont\sffamily}
\ChTitleVar{\Large\normalfont\sffamily}
\usepackage[,numfigreset=1,mathnumfig]{sphinx}

\fvset{fontsize=\small}
\usepackage{geometry}


% Include hyperref last.
\usepackage{hyperref}
% Fix anchor placement for figures with captions.
\usepackage{hypcap}% it must be loaded after hyperref.
% Set up styles of URL: it should be placed after hyperref.
\urlstyle{same}

\addto\captionsitalian{\renewcommand{\contentsname}{Newton Mechanics}}

\usepackage{sphinxmessages}



        % Start of preamble defined in sphinx-jupyterbook-latex %
         \usepackage[Latin,Greek]{ucharclasses}
        \usepackage{unicode-math}
        % fixing title of the toc
        \addto\captionsenglish{\renewcommand{\contentsname}{Contents}}
        \hypersetup{
            pdfencoding=auto,
            psdextra
        }
        % End of preamble defined in sphinx-jupyterbook-latex %
        

\title{Meccanica classica}
\date{16 gen 2025}
\release{}
\author{basics}
\newcommand{\sphinxlogo}{\vbox{}}
\renewcommand{\releasename}{}
\makeindex
\begin{document}

\pagestyle{empty}
\sphinxmaketitle
\pagestyle{plain}
\sphinxtableofcontents
\pagestyle{normal}
\phantomsection\label{\detokenize{intro::doc}}


\sphinxAtStartPar
Classical mechanics deal with the motion of systems and its causes.

\sphinxAtStartPar
Different formulations of mechanics are available. Newton formulation was developed at the end of XVII century and starts from mass conservation and Newton’s three principles of dynamics, summarised in invariance under Galileian transformations, the relation between force and the change of momentum of a system, and action/reaction principle. Analytical mechanics was developed in the following centuries by D’Alembert and Lagrange and starts from variational principles, leading to Lagrange or Hamiltonian equations of motion.

\sphinxAtStartPar
\sphinxstylestrong{Newton Mechanics.}
\subsubsection*{Kinematics}
\subsubsection*{Actions}
\subsubsection*{Inertia}
\subsubsection*{Dynamics}

\sphinxAtStartPar
\sphinxstylestrong{Analytical Mechanics.}
\subsubsection*{Lagrangian Mechanics}
\subsubsection*{Hamiltonian Mechanics}

\sphinxAtStartPar
Classical mechanics provides a reliable and useful theory for systems:
\begin{itemize}
\item {} 
\sphinxAtStartPar
much larger than atomic scales; at atomic scales, \sphinxhref{https://basics2022.github.io/bbooks-physics-modern/ch/quantum-mechanics/intro.html}{quantum mechanics} is needed

\item {} 
\sphinxAtStartPar
with velocity much slower than the speed of light or in domains where the finite value of finite speed of interactions can be neglected, as classical mechanics relies on istantaneous action at distance; if these assumptions fail, Einstein theory is needed either \sphinxhref{https://basics2022.github.io/bbooks-physics-modern/ch/relativity-special/intro.html}{special relativity} \sphinxhyphen{} as a consistent theory of mechanics and \sphinxhref{https://basics2022.github.io/bbooks-physics-electromagnetism/intro.html}{electromagnetism} \sphinxhyphen{} or \sphinxhref{https://basics2022.github.io/bbooks-physics-modern/ch/relativity-general/intro.html}{general relativity} \sphinxhyphen{} as a theory of gravitation.

\item {} 
\sphinxAtStartPar
with a small number of components, so that the integration of the governing equations of motion is feasible; continuous model of the systems are object of classical continuum mechanics, relying on the equations of classical mechanics and thermodynamics; systems with large number of components can be approached with the techniques developed in \sphinxhref{https://basics2022.github.io/bbooks-physics-modern/ch/statistical-mechanics/intro.html}{statistical mechanics}.

\end{itemize}

\sphinxAtStartPar
Under these assumptions, mass conservation (Lavoisier principle) holds, inertial reference frames are related by Galileian relativity and the equations of motions are deterministic and can be solved with a reasonable effort \sphinxhyphen{} compared to the information and detail contained in the results \sphinxhyphen{} either analytically or numerically.
Classical mechanics treats time and space as individually absolute physical quantities: this can be a good model whenever Einstein relativity effects are negligible.





\sphinxstepscope


\part{Newton Mechanics}

\sphinxstepscope


\chapter{Kinematics}
\label{\detokenize{ch/kinematics:kinematics}}\label{\detokenize{ch/kinematics:classical-mechanics-kinematics}}\label{\detokenize{ch/kinematics::doc}}
\sphinxAtStartPar
Kinematics deals with the motion of mechanical systems, without taking into account the causes of motion.

\sphinxstepscope


\section{Space and time}
\label{\detokenize{ch/kinematics-space-time:space-and-time}}\label{\detokenize{ch/kinematics-space-time:classical-mechanics-kinematics-space-time}}\label{\detokenize{ch/kinematics-space-time::doc}}
\sphinxAtStartPar
Classical mechanics relies on the concepts of \sphinxstylestrong{absolute 3\sphinxhyphen{}dimensional Euclidean space}, \(E^3\), and \sphinxstylestrong{absolute time}.

\sphinxAtStartPar
Briefly, what is space? It’s something you can measure with ruler (for distances) and square (for angles), or other space\sphinxhyphen{}measurment devices.
What is time? It’s something you can measure with a clock or other timekeeping devices.

\sphinxstepscope


\section{Models}
\label{\detokenize{ch/kinematics-models:models}}\label{\detokenize{ch/kinematics-models:classical-mechanics-kinematics-models}}\label{\detokenize{ch/kinematics-models::doc}}
\sphinxAtStartPar
Different models of physical systems can be derived with an abstraction and modelling process, depending on the characteristics of the system under investigation and on the level of detail required by the analysis.

\sphinxAtStartPar
These models can be classified by:
\begin{itemize}
\item {} 
\sphinxAtStartPar
dimensions: 0: point; 1: line; 2: surfaces; 3: volume solid

\item {} 
\sphinxAtStartPar
deformation: deformable or rigid components

\end{itemize}

\sphinxAtStartPar
A system can be composed of several components, either free or connected with constraints.

\sphinxAtStartPar
Here, the focus goes to the kinematics of {\hyperref[\detokenize{ch/kinematics-point:classical-mechanics-kinematics-point}]{\sphinxcrossref{\DUrole{std,std-ref}{points}}}} and {\hyperref[\detokenize{ch/kinematics-rigid:classical-mechanics-kinematics-rigid-body}]{\sphinxcrossref{\DUrole{std,std-ref}{rigid bodies}}}}, while deformable bodies are described in \DUrole{xref,myst}{continuous mechanics} \sphinxhyphen{} \DUrole{xref,myst}{kinematics}.

\sphinxAtStartPar
While space and time are absolute, the motion of a system is usually the \sphinxstylestrong{motion relative to an observer} or to a reference frame. After treating the kinematics of points and rigid bodies w.r.t. a given reference frame, {\hyperref[\detokenize{ch/kinematics-relative:classical-mechanics-kinematics-relative}]{\sphinxcrossref{\DUrole{std,std-ref}{relative kinematics}}}} provides the description of the motion of the same system w.r.t. 2 different observers/reference frames in relative motion.

\sphinxstepscope


\section{Point}
\label{\detokenize{ch/kinematics-point:point}}\label{\detokenize{ch/kinematics-point:classical-mechanics-kinematics-point}}\label{\detokenize{ch/kinematics-point::doc}}


\sphinxstepscope


\section{Rigid Body}
\label{\detokenize{ch/kinematics-rigid:rigid-body}}\label{\detokenize{ch/kinematics-rigid:classical-mechanics-kinematics-rigid-body}}\label{\detokenize{ch/kinematics-rigid::doc}}

\subsection{Rigid motion}
\label{\detokenize{ch/kinematics-rigid:rigid-motion}}
\sphinxAtStartPar
Rigid motion preserves distance between any pair of points, and thus angles. Rotations are described by rotation matrices.

\sphinxAtStartPar
The configuration of a material vector \(\vec{a}\) undergoing a rotation is described by the product of the rotation tensor \(\mathbb{R}\) by the reference configuration \(\vec{a}^0\),
\begin{equation*}
\begin{split}\vec{a} = \mathbb{R} \cdot \vec{a}^0 \ .\end{split}
\end{equation*}
\sphinxAtStartPar
A rotation tensor is unitary to preserve distance,
\begin{equation}\label{equation:ch/kinematics-rigid:eq:rot:unitary}
\begin{split}\mathbb{I} = \mathbb{R} \cdot \mathbb{R}^T = \mathbb{R}^T \cdot \mathbb{R}\end{split}
\end{equation}
\sphinxAtStartPar
so that its time derivative reads,
\begin{equation*}
\begin{split}\mathbb{0} = \dfrac{d}{dt} \left( \mathbb{R} \cdot \mathbb{R}^T \right) = \dot{\mathbb{R}} \cdot \mathbb{R}^T + \mathbb{R} \cdot \dot{\mathbb{R}}^T\end{split}
\end{equation*}
\sphinxAtStartPar
It follows that the 2\sphinxhyphen{}nd order tensor \(\dot{\mathbb{R}} \cdot \mathbb{R}^T = - \mathbb{R} \cdot \dot{\mathbb{R}}^T\) is anti\sphinxhyphen{}symmetric, and thus it can be written as
\begin{equation}\label{equation:ch/kinematics-rigid:eq:omegax:def}
\begin{split}\dot{\mathbb{R}} \cdot \mathbb{R}^T =: \vec{\omega}_{\times} \ ,\end{split}
\end{equation}
\sphinxAtStartPar
being the vector \(\vec{\omega}\) the angular velocity. Since \(\mathbb{R}\) is unitary by \eqref{equation:ch/kinematics-rigid:eq:rot:unitary}, multiplying \eqref{equation:ch/kinematics-rigid:eq:omegax:def} with the dot\sphinxhyphen{}product on the right by \(\mathbb{R}\), it follows
\begin{equation*}
\begin{split}\dot{\mathbb{R}} = \vec{\omega}_{\times} \cdot \mathbb{R} \ ,\end{split}
\end{equation*}
\sphinxAtStartPar
and the expression of the time derivative of a material vector \(\vec{a}\),
\begin{equation*}
\begin{split}\dfrac{d \vec{a}}{d t} = \dot{\mathbb{R}} \cdot \vec{a}^0 = \vec{\omega}_{\times} \cdot \mathbb{R} \cdot \vec{a}^0 = \vec{\omega}_{\times} \cdot \vec{a} = \vec{\omega} \times \vec{a} \  .\end{split}
\end{equation*}

\subsection{Kinematics of rigid body}
\label{\detokenize{ch/kinematics-rigid:kinematics-of-rigid-body}}
\sphinxAtStartPar
Rigid motion allows to describe the kinematics of a rigid body \sphinxhyphen{} determining position, velocity and acceleration of all of its points \sphinxhyphen{} given the position of a material point \(P\) and its orientation with respect to the reference frame, as an example using a rotation tensor \(\mathbb{R}\).

\sphinxAtStartPar
\sphinxstylestrong{Orientation.}

\sphinxAtStartPar
\sphinxstylestrong{Angular velocity.}

\sphinxAtStartPar
\sphinxstylestrong{Angular acceleration.}



\sphinxstepscope


\section{Rotations}
\label{\detokenize{ch/kinematics-rotations:rotations}}\label{\detokenize{ch/kinematics-rotations:classical-mechanics-kinematics-rotations}}\label{\detokenize{ch/kinematics-rotations::doc}}

\subsection{Rotation tensor}
\label{\detokenize{ch/kinematics-rotations:rotation-tensor}}
\sphinxAtStartPar
Given 2 Cartesian bases \(\{ \hat{e}^0_i \}_{i=1:3}\), \(\{ \hat{e}^1_j \}_{j=1:3}\), the rotation tensor providing the transformation
\begin{equation*}
\begin{split}\hat{e}^1_i = \mathbb{R}^{0 \rightarrow 1} \cdot \hat{e}^0_i \ ,\end{split}
\end{equation*}
\sphinxAtStartPar
is
\begin{equation*}
\begin{split}\mathbb{R}^{0 \rightarrow 1}
 = R_{ij}^{0 \rightarrow 1} \hat{e}^0_i \otimes \hat{e}^0_j  
 = R_{ij}^{0 \rightarrow 1} \hat{e}^1_i \otimes \hat{e}^1_j 
\end{split}
\end{equation*}
\sphinxAtStartPar
with \(R^{0 \rightarrow 1}_{ij} = \hat{e}^0_i \cdot \hat{e}^1_j\).

\sphinxAtStartPar
\sphinxstylestrong{Angular velocity.}
\begin{equation*}
\begin{split}\vec{\omega}^{01}_{\times} = \mathbb{\Omega}^{01} = \dot{\mathbb{R}}^{01} \cdot \mathbb{R}^{01,T}\end{split}
\end{equation*}
\sphinxAtStartPar
Using index notation
\begin{equation*}
\begin{split}\varepsilon_{ijk} \omega_j = \dot{R}_{ij} R_{kj}\end{split}
\end{equation*}
\sphinxAtStartPar
and the identities
\begin{equation*}
\begin{split}\varepsilon_{ijk} \varepsilon_{lmk} = \delta_{il} \delta_{jm} - \delta_{jl} \delta_{im}\end{split}
\end{equation*}\begin{equation*}
\begin{split}\varepsilon_{ijk} \varepsilon_{ljk} = \delta_{il} \delta_{jj} - \delta_{ij} \delta_{jl} = 3 \delta_{il} - \delta_{il} = 2 \delta_{il}\end{split}
\end{equation*}
\sphinxAtStartPar
it follows
\begin{equation*}
\begin{split}\varepsilon_{ilk} \varepsilon_{ijk} \omega_j = \varepsilon_{ilk} \dot{R}_{ij} R_{kj}\end{split}
\end{equation*}\begin{equation*}
\begin{split}2 \delta_{lj} \omega_j = \varepsilon_{ilk} \dot{R}_{ij} R_{kj}\end{split}
\end{equation*}\begin{equation*}
\begin{split}\omega_l = \frac{1}{2} \varepsilon_{ilk} \dot{R}_{ij} R_{kj} = - \frac{1}{2} \varepsilon_{lik} \dot{R}_{ij} R_{kj} = -\frac{1}{2} \varepsilon_{lij} \Omega_{ij}\end{split}
\end{equation*}

\subsubsection{Successive rotations}
\label{\detokenize{ch/kinematics-rotations:successive-rotations}}
\sphinxAtStartPar
Given 3 Cartesian bases \(\{ \hat{e}^0_i \}_{i=1:3}\), \(\{ \hat{e}^1_j \}_{j=1:3}\), \(\{ \hat{e}^2_k \}_{k=1:3}\),
\begin{equation*}
\begin{split}\begin{aligned}
 \hat{e}^2_i 
  & = \mathbb{R}^{1 \rightarrow 2} \cdot \hat{e}^1_i = \\ 
  & = \mathbb{R}^{1 \rightarrow 2} \cdot \mathbb{R}^{0 \rightarrow 1} \cdot\hat{e}^0_i 
\end{aligned}\end{split}
\end{equation*}
\sphinxAtStartPar
Time derivative w.r.t. reference frame 0 is indicated as the standard time derivative
\begin{equation*}
\begin{split}\dot{a} = \dfrac{d a}{d t} = \dfrac{{}^0 d a}{d t} \ ,\end{split}
\end{equation*}\begin{equation*}
\begin{split}\begin{aligned}
\dfrac{d}{dt} \mathbb{R}^{21} 
  & = \frac{d}{dt} \left[ R^{21}_{ik} \hat{e}^1_i \otimes \hat{e}^1_k \right] = \\
  & = \dot{R}^{21}_{ik} \hat{e}^1_i \otimes \hat{e}^1_k + \mathbb{\Omega}^{10} \cdot  \mathbb{R}^{21} - \mathbb{R}^{21} \cdot \mathbb{\Omega}^{10} = \\
  & = \dfrac{{}^1 d}{dt} \mathbb{R}^{21} + \mathbb{\Omega}^{10} \cdot  \mathbb{R}^{21} - \mathbb{R}^{21} \cdot \mathbb{\Omega}^{10} = \\
\end{aligned}\end{split}
\end{equation*}\begin{equation*}
\begin{split}\begin{aligned}
 \mathbb{\Omega}^{20}
 & = \dot{\mathbb{R}}^{20} \cdot \mathbb{R}^{20, T} = \\
 & = \frac{d}{dt} \left( \mathbb{R}^{21} \cdot \mathbb{R}^{10} \right) \cdot \mathbb{R}^{20, T} = \\
 & = \left\{ \left[ \dfrac{{}^1 d}{dt} \mathbb{R}^{21} + \mathbb{\Omega}^{10} \cdot  \mathbb{R}^{21} - \mathbb{R}^{21} \cdot \mathbb{\Omega}^{10} \right] \cdot \mathbb{R}^{10} + \mathbb{R}^{21} \cdot \dot{\mathbb{R}}^{10}  \right\} \cdot \mathbb{R}^{01} \cdot \mathbb{R}^{12} = \\
 & =  \dfrac{{}^1 d}{dt} \mathbb{R}^{21} \cdot \mathbb{R}^{12} + \mathbb{\Omega}^{10} = \\
 & = \mathbb{\Omega}^{21} + \mathbb{\Omega}^{10} \ .
\end{aligned}\end{split}
\end{equation*}

\subsubsection{Linearization of rotations}
\label{\detokenize{ch/kinematics-rotations:linearization-of-rotations}}\begin{equation*}
\begin{split}\mathbb{I} = \mathbb{R} \cdot \mathbb{R}^T\end{split}
\end{equation*}
\sphinxAtStartPar
Increment
\begin{equation*}
\begin{split}\mathbb{0} = \delta \mathbb{R} \cdot \mathbb{R}^T + \mathbb{R} \cdot \delta \mathbb{R}^T\end{split}
\end{equation*}
\sphinxAtStartPar
and thus the antisymmetric tensor can be written as
\begin{equation*}
\begin{split}\delta \theta_{\times} := \delta \mathbb{R} \cdot \mathbb{R}^T\end{split}
\end{equation*}
\sphinxAtStartPar
so that
\begin{equation*}
\begin{split}\delta \theta_l = -\frac{1}{2} \varepsilon_{lij} \delta R_{ik} \, R_{jk}\end{split}
\end{equation*}

\subsection{Quaternions}
\label{\detokenize{ch/kinematics-rotations:quaternions}}
\sphinxstepscope


\section{Continuous Medium}
\label{\detokenize{ch/kinematics-continuum:continuous-medium}}\label{\detokenize{ch/kinematics-continuum:classical-mechanics-kinematics-contiuum}}\label{\detokenize{ch/kinematics-continuum::doc}}
\sphinxstepscope


\section{Relative Kinematics}
\label{\detokenize{ch/kinematics-relative:relative-kinematics}}\label{\detokenize{ch/kinematics-relative:classical-mechanics-kinematics-relative}}\label{\detokenize{ch/kinematics-relative::doc}}
\sphinxAtStartPar
Relative kinematics is discussed here using two Cartesian reference frames.
\begin{equation*}
\begin{split}P   - O_0 = x^{0i}_{  P/O_0} \hat{e}^0_i\end{split}
\end{equation*}\begin{equation*}
\begin{split}O_1 - O_0 = x^{0i}_{O_1/O_0} \hat{e}^0_i\end{split}
\end{equation*}\begin{equation*}
\begin{split}P   - O_1 = x^{1i}_{  P/O_1} \hat{e}^1_i\end{split}
\end{equation*}\begin{equation*}
\begin{split}\begin{aligned}
  \hat{e}^1_i 
  = \hat{e}^{1}_i \cdot \hat{e}^0_k \, \hat{e}^0_k
  & = \hat{e}^{1}_j \cdot \hat{e}^0_k \, \hat{e}^0_k \otimes \hat{e}^0_j \cdot \hat{e}^0_i = \\
  & = R^{0\rightarrow 1}_{kj} \hat{e}^0_k \otimes \hat{e}^0_j \cdot \hat{e}^0_i 
    = \mathbb{R}^{0\rightarrow 1} \cdot \hat{e}^0_i  \ .
\end{aligned}\end{split}
\end{equation*}

\subsection{Points}
\label{\detokenize{ch/kinematics-relative:points}}
\sphinxAtStartPar
\sphinxstylestrong{Position.}
Given two reference frames \(Ox^i\), \(O' x^{i'}\), for the position of a point \(P\) reads
\begin{equation*}
\begin{split}(P - O) = (P - O') + ( O' - O) \ ,\end{split}
\end{equation*}
\sphinxAtStartPar
i.e. the position vector \(P-O\) of the point \(P\) w.r.t. point \(O\) \sphinxhyphen{} origin of the reference frame \(O x^i\) \sphinxhyphen{} is the sum of the position vector \(P-O'\) of the point \(P\) w.r.t. to the point \(O'\) \sphinxhyphen{} origin of the reference frame \(O' x^{'i}\) \sphinxhyphen{}  and the position vector \(O' - O\), of the origin \(O'\) w.r.t. to \(O\).

\sphinxAtStartPar
\sphinxstylestrong{Velocity.}

\sphinxAtStartPar
\sphinxstylestrong{Acceleration.}


\subsection{Rigid bodies}
\label{\detokenize{ch/kinematics-relative:rigid-bodies}}
\sphinxAtStartPar
\sphinxstylestrong{Orientation.}

\sphinxAtStartPar
\sphinxstylestrong{Angular velocity.}

\sphinxAtStartPar
\sphinxstylestrong{Angular acceleration.}

\sphinxstepscope




\chapter{Actions}
\label{\detokenize{ch/actions:actions}}\label{\detokenize{ch/actions:classical-mechanics-actions}}\label{\detokenize{ch/actions::doc}}
\sphinxAtStartPar
\sphinxstylestrong{Cos’è una forza?}
Newton lega il concetto di azione, che comprende i concetti di forza e momento, alle cause che distinguono il \sphinxstyleemphasis{moto vero} dal \sphinxstyleemphasis{moto relativo}: per Newton, queste cause sono le forze.

\sphinxAtStartPar
\sphinxstylestrong{Argomenti.}
\begin{itemize}
\item {} 
\sphinxAtStartPar
Forza

\item {} 
\sphinxAtStartPar
Momento di una forza e coppia di forze

\item {} 
\sphinxAtStartPar
Azioni distribuite: cenni su sforzo e pressione

\item {} 
\sphinxAtStartPar
Lavoro e potenza

\item {} 
\sphinxAtStartPar
Azioni conservative

\item {} 
\sphinxAtStartPar
\sphinxstylestrong{Reazioni vincolari}

\item {} 
\sphinxAtStartPar
Esempi di forze:
\begin{itemize}
\item {} 
\sphinxAtStartPar
legge di attrazione di gravità generale

\item {} 
\sphinxAtStartPar
forza di gravità nei pressi della superficie terrestre

\item {} 
\sphinxAtStartPar
legge costitutiva elastica della molla e molle senza massa

\item {} 
\sphinxAtStartPar
attrito

\item {} 
\sphinxAtStartPar
legge costitutiva di smorzatori lineari

\item {} 
\sphinxAtStartPar
reazioni vincolari

\end{itemize}

\end{itemize}

\sphinxstepscope

\begin{sphinxuseclass}{sd-container-fluid}
\begin{sphinxuseclass}{sd-sphinx-override}
\begin{sphinxuseclass}{sd-p-0}
\begin{sphinxuseclass}{sd-mt-2}
\begin{sphinxuseclass}{sd-mb-4}
\begin{sphinxuseclass}{sd-row}
\begin{sphinxuseclass}{sd-row-cols-2}
\begin{sphinxuseclass}{sd-gx-2}
\begin{sphinxuseclass}{sd-gy-1}
\begin{sphinxuseclass}{sd-col}
\begin{sphinxuseclass}{sd-d-flex-row}
\begin{sphinxuseclass}{sd-align-minor-center}
\begin{sphinxuseclass}{sd-container-fluid}
\begin{sphinxuseclass}{sd-sphinx-override}
\begin{sphinxuseclass}{sd-row}
\begin{sphinxuseclass}{sd-row-cols-2}
\begin{sphinxuseclass}{sd-row-cols-xs-2}
\begin{sphinxuseclass}{sd-row-cols-sm-3}
\begin{sphinxuseclass}{sd-row-cols-md-3}
\begin{sphinxuseclass}{sd-row-cols-lg-3}
\begin{sphinxuseclass}{sd-gx-3}
\begin{sphinxuseclass}{sd-gy-1}
\begin{sphinxuseclass}{sd-col}
\begin{sphinxuseclass}{sd-col-auto}
\begin{sphinxuseclass}{sd-d-flex-row}
\begin{sphinxuseclass}{sd-align-minor-center}
\sphinxAtStartPar
basics

\end{sphinxuseclass}
\end{sphinxuseclass}
\end{sphinxuseclass}
\end{sphinxuseclass}
\begin{sphinxuseclass}{sd-col}
\begin{sphinxuseclass}{sd-col-auto}
\begin{sphinxuseclass}{sd-d-flex-row}
\begin{sphinxuseclass}{sd-align-minor-center}
\sphinxAtStartPar
16 gen 2025

\end{sphinxuseclass}
\end{sphinxuseclass}
\end{sphinxuseclass}
\end{sphinxuseclass}
\begin{sphinxuseclass}{sd-col}
\begin{sphinxuseclass}{sd-col-auto}
\begin{sphinxuseclass}{sd-d-flex-row}
\begin{sphinxuseclass}{sd-align-minor-center}
\sphinxAtStartPar
0 min read

\end{sphinxuseclass}
\end{sphinxuseclass}
\end{sphinxuseclass}
\end{sphinxuseclass}
\end{sphinxuseclass}
\end{sphinxuseclass}
\end{sphinxuseclass}
\end{sphinxuseclass}
\end{sphinxuseclass}
\end{sphinxuseclass}
\end{sphinxuseclass}
\end{sphinxuseclass}
\end{sphinxuseclass}
\end{sphinxuseclass}
\end{sphinxuseclass}
\end{sphinxuseclass}
\end{sphinxuseclass}
\end{sphinxuseclass}
\end{sphinxuseclass}
\end{sphinxuseclass}
\end{sphinxuseclass}
\end{sphinxuseclass}
\end{sphinxuseclass}
\end{sphinxuseclass}
\end{sphinxuseclass}
\end{sphinxuseclass}

\section{Forze, momenti, forze distribuite}
\label{\detokenize{ch/actions-types:forze-momenti-forze-distribuite}}\label{\detokenize{ch/actions-types:classical-mechanics-actions-density}}\label{\detokenize{ch/actions-types::doc}}
\sphinxAtStartPar
\sphinxstylestrong{Forza.} Grandezza vettoriale, caratterizzata da intensità, direzione, verso, e punto di applicazione.

\sphinxAtStartPar
\sphinxstylestrong{Momento.} Il momento di una forza rispetto a un polo \(H\)
\begin{equation*}
\begin{split}\mathbf{M}_H = (\mathbf{r}_P - \mathbf{r}_H) \times \mathbf{F} \ .\end{split}
\end{equation*}
\sphinxAtStartPar
\sphinxstylestrong{Forze distribuite.} \sphinxstylestrong{todo}

\sphinxstepscope

\begin{sphinxuseclass}{sd-container-fluid}
\begin{sphinxuseclass}{sd-sphinx-override}
\begin{sphinxuseclass}{sd-p-0}
\begin{sphinxuseclass}{sd-mt-2}
\begin{sphinxuseclass}{sd-mb-4}
\begin{sphinxuseclass}{sd-row}
\begin{sphinxuseclass}{sd-row-cols-2}
\begin{sphinxuseclass}{sd-gx-2}
\begin{sphinxuseclass}{sd-gy-1}
\begin{sphinxuseclass}{sd-col}
\begin{sphinxuseclass}{sd-d-flex-row}
\begin{sphinxuseclass}{sd-align-minor-center}
\begin{sphinxuseclass}{sd-container-fluid}
\begin{sphinxuseclass}{sd-sphinx-override}
\begin{sphinxuseclass}{sd-row}
\begin{sphinxuseclass}{sd-row-cols-2}
\begin{sphinxuseclass}{sd-row-cols-xs-2}
\begin{sphinxuseclass}{sd-row-cols-sm-3}
\begin{sphinxuseclass}{sd-row-cols-md-3}
\begin{sphinxuseclass}{sd-row-cols-lg-3}
\begin{sphinxuseclass}{sd-gx-3}
\begin{sphinxuseclass}{sd-gy-1}
\begin{sphinxuseclass}{sd-col}
\begin{sphinxuseclass}{sd-col-auto}
\begin{sphinxuseclass}{sd-d-flex-row}
\begin{sphinxuseclass}{sd-align-minor-center}
\sphinxAtStartPar
basics

\end{sphinxuseclass}
\end{sphinxuseclass}
\end{sphinxuseclass}
\end{sphinxuseclass}
\begin{sphinxuseclass}{sd-col}
\begin{sphinxuseclass}{sd-col-auto}
\begin{sphinxuseclass}{sd-d-flex-row}
\begin{sphinxuseclass}{sd-align-minor-center}
\sphinxAtStartPar
16 gen 2025

\end{sphinxuseclass}
\end{sphinxuseclass}
\end{sphinxuseclass}
\end{sphinxuseclass}
\begin{sphinxuseclass}{sd-col}
\begin{sphinxuseclass}{sd-col-auto}
\begin{sphinxuseclass}{sd-d-flex-row}
\begin{sphinxuseclass}{sd-align-minor-center}
\sphinxAtStartPar
1 min read

\end{sphinxuseclass}
\end{sphinxuseclass}
\end{sphinxuseclass}
\end{sphinxuseclass}
\end{sphinxuseclass}
\end{sphinxuseclass}
\end{sphinxuseclass}
\end{sphinxuseclass}
\end{sphinxuseclass}
\end{sphinxuseclass}
\end{sphinxuseclass}
\end{sphinxuseclass}
\end{sphinxuseclass}
\end{sphinxuseclass}
\end{sphinxuseclass}
\end{sphinxuseclass}
\end{sphinxuseclass}
\end{sphinxuseclass}
\end{sphinxuseclass}
\end{sphinxuseclass}
\end{sphinxuseclass}
\end{sphinxuseclass}
\end{sphinxuseclass}
\end{sphinxuseclass}
\end{sphinxuseclass}
\end{sphinxuseclass}

\section{Lavoro e potenza}
\label{\detokenize{ch/actions-work-power:lavoro-e-potenza}}\label{\detokenize{ch/actions-work-power:classical-mechanics-actions-work-power}}\label{\detokenize{ch/actions-work-power::doc}}

\subsection{Lavoro e potenza di una forza}
\label{\detokenize{ch/actions-work-power:lavoro-e-potenza-di-una-forza}}
\sphinxAtStartPar
\sphinxstylestrong{Lavoro.} Il lavoro elementare di una forza \(\mathbf{F}\) applicata in un punto di applicazione \(\mathbf{r}_P\) che subisce uno spostamento \(d\mathbf{r}\) viene definito come
\begin{equation*}
\begin{split}\delta L = \mathbf{F} \cdot d \mathbf{r} \ .\end{split}
\end{equation*}
\sphinxAtStartPar
Per uno spostamento finito del punto \(P\) dal punto \(\mathbf{r}_A\) al punto \(\mathbf{r}_B\) lungo la linea \(\ell_{AB}\), è necessario sommare i contributi elementari (e quindi integrare) lungo il percorso \(\ell_{AB}\)
\begin{equation*}
\begin{split}L = \int_{\ell_{AB}} \delta L = \int_{\ell_{AB}} \mathbf{F} \cdot d \mathbf{r}_P \ .\end{split}
\end{equation*}
\sphinxAtStartPar
\sphinxstylestrong{todo} In generale il lavoro di una forza o di un campo di forze \sphinxstylestrong{dipende dal percorso di integrazione} \(\ell_{AB}\). Sarebbe meglio usare \(\delta L\) per ricordare questa proprietà del lavoro, che quindi non è un \sphinxstylestrong{differenziale esatto}.

\sphinxAtStartPar
\sphinxstylestrong{Potenza.} La potenza di una forza \(\mathbf{F}\) applicata in un punto di applicazione \(\mathbf{r}_P\) che ha velocità \(\mathbf{v}_P\) viene definita come
\begin{equation*}
\begin{split}P = \mathbf{F} \cdot \mathbf{v}_P \ ,\end{split}
\end{equation*}
\sphinxAtStartPar
cioè come la derivata nel tempo del lavoro,
\begin{equation*}
\begin{split}\dfrac{dL}{dt} = \mathbf{F} \cdot \dfrac{d \mathbf{r}_P}{d t} = \mathbf{F} \cdot \mathbf{v}_P = P \ .\end{split}
\end{equation*}

\subsection{Lavoro e potenza di una coppia di forze}
\label{\detokenize{ch/actions-work-power:lavoro-e-potenza-di-una-coppia-di-forze}}
\sphinxAtStartPar
Una coppia di forze è definita come il momento generato da due forze con uguale intensità e verso opposto, \(\mathbf{F}_1 = - \mathbf{F}_2\),
\begin{equation*}
\begin{split}\mathbf{C} = (\mathbf{r}_1 - \mathbf{r}_2) \times \mathbf{F}_1 \ .\end{split}
\end{equation*}
\sphinxAtStartPar
\sphinxstylestrong{Potenza.} La potenza di una coppia di forze è la somma della potenza generata da entrambe le forze,
\begin{equation*}
\begin{split}P = P_1 + P_2 = \mathbf{F}_1 \cdot \mathbf{v}_1 + \mathbf{F}_2 \cdot \mathbf{v}_2 = \mathbf{F}_1 \cdot (\mathbf{v}_1 - \mathbf{v}_2) \ .\end{split}
\end{equation*}
\sphinxAtStartPar
Se la coppia di forze è applicata a una coppia di punti che descrivono un atto di moto rigido,
\begin{equation*}
\begin{split}\mathbf{v}_1 - \mathbf{v}_2 = \symbf{\omega} \times (\mathbf{r}_1 - \mathbf{r}_2) \ ,\end{split}
\end{equation*}
\sphinxAtStartPar
si può riscrivere la potenza della coppia di forze come
\begin{equation*}
\begin{split}\begin{aligned}
P & = \mathbf{F}_1 \cdot \symbf{\omega} \times (\mathbf{r}_1 - \mathbf{r}_2) = \\
  & = \symbf{\omega} \cdot ( \mathbf{r}_1 - \mathbf{r}_2 ) \times \mathbf{F}_1 = \\
  & = \symbf{\omega} \cdot \mathbf{C}
\end {aligned}\end{split}
\end{equation*}
\sphinxstepscope

\begin{sphinxuseclass}{sd-container-fluid}
\begin{sphinxuseclass}{sd-sphinx-override}
\begin{sphinxuseclass}{sd-p-0}
\begin{sphinxuseclass}{sd-mt-2}
\begin{sphinxuseclass}{sd-mb-4}
\begin{sphinxuseclass}{sd-row}
\begin{sphinxuseclass}{sd-row-cols-2}
\begin{sphinxuseclass}{sd-gx-2}
\begin{sphinxuseclass}{sd-gy-1}
\begin{sphinxuseclass}{sd-col}
\begin{sphinxuseclass}{sd-d-flex-row}
\begin{sphinxuseclass}{sd-align-minor-center}
\begin{sphinxuseclass}{sd-container-fluid}
\begin{sphinxuseclass}{sd-sphinx-override}
\begin{sphinxuseclass}{sd-row}
\begin{sphinxuseclass}{sd-row-cols-2}
\begin{sphinxuseclass}{sd-row-cols-xs-2}
\begin{sphinxuseclass}{sd-row-cols-sm-3}
\begin{sphinxuseclass}{sd-row-cols-md-3}
\begin{sphinxuseclass}{sd-row-cols-lg-3}
\begin{sphinxuseclass}{sd-gx-3}
\begin{sphinxuseclass}{sd-gy-1}
\begin{sphinxuseclass}{sd-col}
\begin{sphinxuseclass}{sd-col-auto}
\begin{sphinxuseclass}{sd-d-flex-row}
\begin{sphinxuseclass}{sd-align-minor-center}
\sphinxAtStartPar
basics

\end{sphinxuseclass}
\end{sphinxuseclass}
\end{sphinxuseclass}
\end{sphinxuseclass}
\begin{sphinxuseclass}{sd-col}
\begin{sphinxuseclass}{sd-col-auto}
\begin{sphinxuseclass}{sd-d-flex-row}
\begin{sphinxuseclass}{sd-align-minor-center}
\sphinxAtStartPar
16 gen 2025

\end{sphinxuseclass}
\end{sphinxuseclass}
\end{sphinxuseclass}
\end{sphinxuseclass}
\begin{sphinxuseclass}{sd-col}
\begin{sphinxuseclass}{sd-col-auto}
\begin{sphinxuseclass}{sd-d-flex-row}
\begin{sphinxuseclass}{sd-align-minor-center}
\sphinxAtStartPar
1 min read

\end{sphinxuseclass}
\end{sphinxuseclass}
\end{sphinxuseclass}
\end{sphinxuseclass}
\end{sphinxuseclass}
\end{sphinxuseclass}
\end{sphinxuseclass}
\end{sphinxuseclass}
\end{sphinxuseclass}
\end{sphinxuseclass}
\end{sphinxuseclass}
\end{sphinxuseclass}
\end{sphinxuseclass}
\end{sphinxuseclass}
\end{sphinxuseclass}
\end{sphinxuseclass}
\end{sphinxuseclass}
\end{sphinxuseclass}
\end{sphinxuseclass}
\end{sphinxuseclass}
\end{sphinxuseclass}
\end{sphinxuseclass}
\end{sphinxuseclass}
\end{sphinxuseclass}
\end{sphinxuseclass}
\end{sphinxuseclass}

\section{Azioni conservative}
\label{\detokenize{ch/actions-conservative:azioni-conservative}}\label{\detokenize{ch/actions-conservative:classical-mechanics-actions-conservative}}\label{\detokenize{ch/actions-conservative::doc}}
\sphinxAtStartPar
In generale, il lavoro di una forza o di un campo di forze dipende dalla traiettoria \(\ell_{AB}\) del punto di applicazione tra i punti iniziale e finale \(\mathbf{r}_A\), \(\mathbf{r}_B\).

\sphinxAtStartPar
In alcuni casi, il lavoro è indipendente dal percorso \(\ell_{AB}\), ma dipende solo dai suoi punti estremi. Se questo è vero per tutti i percorsi, il lavoro è svolto da una \sphinxstylestrong{forza conservativa}.

\sphinxAtStartPar
In questo caso, l’integrale di linea può essere riscritto senza indicare esplicitamente la dipendenza dal percorso \(\ell_{AB}\), ma indicandone solo gli estremi
\begin{equation*}
\begin{split}L = \int_{\mathbf{r}_A}^{\mathbf{r}_B} \mathbf{F} \cdot d\mathbf{r} \ ,\end{split}
\end{equation*}
\sphinxAtStartPar
e il suo valore può essere calcolato come differenza di una funzione scalare della variabile spaziale \(U(\mathbf{r}) = -V(\mathbf{r})\) valutata nei due estremi del percorso,
\begin{equation*}
\begin{split}L_{AB} = U(\mathbf{r}_B) - U(\mathbf{r}_A) = - V(\mathbf{r}_B) + V(\mathbf{r}_A) .\end{split}
\end{equation*}
\sphinxAtStartPar
o in forma differenziale,
\begin{equation*}
\begin{split}\delta L = dU = - dV \ .\end{split}
\end{equation*}
\sphinxAtStartPar
Sotto l’ipotesi di sufficiente regolarità (\sphinxstylestrong{todo}), e confrontando le espressioni del lavoro infinitesimale di un campo di forze \(\mathbf{F}(\mathbf{r})\),
\begin{equation*}
\begin{split}\delta L = dU = \mathbf{F} \cdot \mathbf{r} \ ,\end{split}
\end{equation*}
\sphinxAtStartPar
si può scrivere il campo di forze come il gradiente della funzione \(U\),
\begin{equation*}
\begin{split}\mathbf{F}(\mathbf{r}) = \nabla U(\mathbf{r}) = - \nabla V(\mathbf{r} )\ .\end{split}
\end{equation*}
\sphinxAtStartPar
Per le proprietà dei campi e degli operatori vettoriali, se un campo vettoriale può essere scritto come gradiente di una funzione scalare, il suo rotore è nullo,
\begin{equation*}
\begin{split}\nabla \times \mathbf{F}(\mathbf{r}) = \mathbf{0} \ .\end{split}
\end{equation*}
\sphinxstepscope

\begin{sphinxuseclass}{sd-container-fluid}
\begin{sphinxuseclass}{sd-sphinx-override}
\begin{sphinxuseclass}{sd-p-0}
\begin{sphinxuseclass}{sd-mt-2}
\begin{sphinxuseclass}{sd-mb-4}
\begin{sphinxuseclass}{sd-row}
\begin{sphinxuseclass}{sd-row-cols-2}
\begin{sphinxuseclass}{sd-gx-2}
\begin{sphinxuseclass}{sd-gy-1}
\begin{sphinxuseclass}{sd-col}
\begin{sphinxuseclass}{sd-d-flex-row}
\begin{sphinxuseclass}{sd-align-minor-center}
\begin{sphinxuseclass}{sd-container-fluid}
\begin{sphinxuseclass}{sd-sphinx-override}
\begin{sphinxuseclass}{sd-row}
\begin{sphinxuseclass}{sd-row-cols-2}
\begin{sphinxuseclass}{sd-row-cols-xs-2}
\begin{sphinxuseclass}{sd-row-cols-sm-3}
\begin{sphinxuseclass}{sd-row-cols-md-3}
\begin{sphinxuseclass}{sd-row-cols-lg-3}
\begin{sphinxuseclass}{sd-gx-3}
\begin{sphinxuseclass}{sd-gy-1}
\begin{sphinxuseclass}{sd-col}
\begin{sphinxuseclass}{sd-col-auto}
\begin{sphinxuseclass}{sd-d-flex-row}
\begin{sphinxuseclass}{sd-align-minor-center}
\sphinxAtStartPar
basics

\end{sphinxuseclass}
\end{sphinxuseclass}
\end{sphinxuseclass}
\end{sphinxuseclass}
\begin{sphinxuseclass}{sd-col}
\begin{sphinxuseclass}{sd-col-auto}
\begin{sphinxuseclass}{sd-d-flex-row}
\begin{sphinxuseclass}{sd-align-minor-center}
\sphinxAtStartPar
16 gen 2025

\end{sphinxuseclass}
\end{sphinxuseclass}
\end{sphinxuseclass}
\end{sphinxuseclass}
\begin{sphinxuseclass}{sd-col}
\begin{sphinxuseclass}{sd-col-auto}
\begin{sphinxuseclass}{sd-d-flex-row}
\begin{sphinxuseclass}{sd-align-minor-center}
\sphinxAtStartPar
1 min read

\end{sphinxuseclass}
\end{sphinxuseclass}
\end{sphinxuseclass}
\end{sphinxuseclass}
\end{sphinxuseclass}
\end{sphinxuseclass}
\end{sphinxuseclass}
\end{sphinxuseclass}
\end{sphinxuseclass}
\end{sphinxuseclass}
\end{sphinxuseclass}
\end{sphinxuseclass}
\end{sphinxuseclass}
\end{sphinxuseclass}
\end{sphinxuseclass}
\end{sphinxuseclass}
\end{sphinxuseclass}
\end{sphinxuseclass}
\end{sphinxuseclass}
\end{sphinxuseclass}
\end{sphinxuseclass}
\end{sphinxuseclass}
\end{sphinxuseclass}
\end{sphinxuseclass}
\end{sphinxuseclass}
\end{sphinxuseclass}

\section{Reazioni vincolari}
\label{\detokenize{ch/actions-reactions:reazioni-vincolari}}\label{\detokenize{ch/actions-reactions:classical-mechanics-actions-reactions}}\label{\detokenize{ch/actions-reactions::doc}}
\sphinxAtStartPar
I vincoli cinematici agiscono su un sistema limitandone la possibilità di movimento, esercitando forze e momenti su di esso, definibili come reazioni vincolari.

\sphinxAtStartPar
In generale, in corrispondenza di un vincolo \sphinxstylestrong{ideale} (\sphinxstylestrong{todo} dare definizione di vincolo ideale, e trattare/accennare/rimandare all’attrito) nasce un’azione vincolare corrispondente a ogni grado di libertà vincolato: così ad esempio il vincolo della traslazione di un punto in una direzione ha come reazione corrispondente una forza in quella direzione; il vincolo della rotazione attorno a un asse ha come azione corrispondente un momento allineato con quell’asse.

\sphinxAtStartPar
Queste condizioni possono essere ricavate dalle equazioni della dinamica per sistemi senza massa, come spesso considerato nel modello ideale dei vincoli.


\subsection{Esempi 2D}
\label{\detokenize{ch/actions-reactions:esempi-2d}}\begin{itemize}
\item {} 
\sphinxAtStartPar
Incastro

\item {} 
\sphinxAtStartPar
Cerniera

\item {} 
\sphinxAtStartPar
Pattino

\item {} 
\sphinxAtStartPar
Carrello

\item {} 
\sphinxAtStartPar
Manicotto

\end{itemize}


\subsection{Esempi 3D}
\label{\detokenize{ch/actions-reactions:esempi-3d}}\begin{itemize}
\item {} 
\sphinxAtStartPar
Incastro

\item {} 
\sphinxAtStartPar
Cerniera sferica

\item {} 
\sphinxAtStartPar
Cerniera cilindrica

\item {} 
\sphinxAtStartPar
Pattino

\item {} 
\sphinxAtStartPar
Carrello

\item {} 
\sphinxAtStartPar
Manicotto

\end{itemize}

\sphinxstepscope

\begin{sphinxuseclass}{sd-container-fluid}
\begin{sphinxuseclass}{sd-sphinx-override}
\begin{sphinxuseclass}{sd-p-0}
\begin{sphinxuseclass}{sd-mt-2}
\begin{sphinxuseclass}{sd-mb-4}
\begin{sphinxuseclass}{sd-row}
\begin{sphinxuseclass}{sd-row-cols-2}
\begin{sphinxuseclass}{sd-gx-2}
\begin{sphinxuseclass}{sd-gy-1}
\begin{sphinxuseclass}{sd-col}
\begin{sphinxuseclass}{sd-d-flex-row}
\begin{sphinxuseclass}{sd-align-minor-center}
\begin{sphinxuseclass}{sd-container-fluid}
\begin{sphinxuseclass}{sd-sphinx-override}
\begin{sphinxuseclass}{sd-row}
\begin{sphinxuseclass}{sd-row-cols-2}
\begin{sphinxuseclass}{sd-row-cols-xs-2}
\begin{sphinxuseclass}{sd-row-cols-sm-3}
\begin{sphinxuseclass}{sd-row-cols-md-3}
\begin{sphinxuseclass}{sd-row-cols-lg-3}
\begin{sphinxuseclass}{sd-gx-3}
\begin{sphinxuseclass}{sd-gy-1}
\begin{sphinxuseclass}{sd-col}
\begin{sphinxuseclass}{sd-col-auto}
\begin{sphinxuseclass}{sd-d-flex-row}
\begin{sphinxuseclass}{sd-align-minor-center}
\sphinxAtStartPar
basics

\end{sphinxuseclass}
\end{sphinxuseclass}
\end{sphinxuseclass}
\end{sphinxuseclass}
\begin{sphinxuseclass}{sd-col}
\begin{sphinxuseclass}{sd-col-auto}
\begin{sphinxuseclass}{sd-d-flex-row}
\begin{sphinxuseclass}{sd-align-minor-center}
\sphinxAtStartPar
16 gen 2025

\end{sphinxuseclass}
\end{sphinxuseclass}
\end{sphinxuseclass}
\end{sphinxuseclass}
\begin{sphinxuseclass}{sd-col}
\begin{sphinxuseclass}{sd-col-auto}
\begin{sphinxuseclass}{sd-d-flex-row}
\begin{sphinxuseclass}{sd-align-minor-center}
\sphinxAtStartPar
0 min read

\end{sphinxuseclass}
\end{sphinxuseclass}
\end{sphinxuseclass}
\end{sphinxuseclass}
\end{sphinxuseclass}
\end{sphinxuseclass}
\end{sphinxuseclass}
\end{sphinxuseclass}
\end{sphinxuseclass}
\end{sphinxuseclass}
\end{sphinxuseclass}
\end{sphinxuseclass}
\end{sphinxuseclass}
\end{sphinxuseclass}
\end{sphinxuseclass}
\end{sphinxuseclass}
\end{sphinxuseclass}
\end{sphinxuseclass}
\end{sphinxuseclass}
\end{sphinxuseclass}
\end{sphinxuseclass}
\end{sphinxuseclass}
\end{sphinxuseclass}
\end{sphinxuseclass}
\end{sphinxuseclass}
\end{sphinxuseclass}

\section{Esempi}
\label{\detokenize{ch/actions-examples:esempi}}\label{\detokenize{ch/actions-examples:classical-mechanics-actions-examples}}\label{\detokenize{ch/actions-examples::doc}}
\sphinxstepscope


\chapter{Inertia}
\label{\detokenize{ch/inertia:inertia}}\label{\detokenize{ch/inertia:classical-mechanics-inertia}}\label{\detokenize{ch/inertia::doc}}
\sphinxstepscope




\chapter{Dynamics}
\label{\detokenize{ch/dynamics:dynamics}}\label{\detokenize{ch/dynamics:classical-mechanics-dynamics}}\label{\detokenize{ch/dynamics::doc}}
\sphinxAtStartPar
La dinamica fornisce il legame tra il moto di un corpo e le azioni causa del moto stesso.

\sphinxAtStartPar
I principi della dinamica di Newton e le equazioni cardinali della dinamica sono le leggi fisiche che governano il moto dei sistemi meccanici. Queste leggi fisiche vengono formulate nei termini di alcune grandezze fisiche, come la quantità di moto, il momento della quantità di moto o l’energia cinetica del sistema. Queste grandezze dinamiche hanno la proprietà di essere additive (per definizione) e rendono particolarmente facile la scrittura di una forma generale delle equazioni che governano il moto, e che si riducono a una relazione tra la derivate nel tempo di queste grandezze dinamiche e le cause della loro variazione. In assenza di cause nette, si ottengono i princìpi di conservazione.

\sphinxstepscope

\begin{sphinxuseclass}{sd-container-fluid}
\begin{sphinxuseclass}{sd-sphinx-override}
\begin{sphinxuseclass}{sd-p-0}
\begin{sphinxuseclass}{sd-mt-2}
\begin{sphinxuseclass}{sd-mb-4}
\begin{sphinxuseclass}{sd-row}
\begin{sphinxuseclass}{sd-row-cols-2}
\begin{sphinxuseclass}{sd-gx-2}
\begin{sphinxuseclass}{sd-gy-1}
\begin{sphinxuseclass}{sd-col}
\begin{sphinxuseclass}{sd-d-flex-row}
\begin{sphinxuseclass}{sd-align-minor-center}
\begin{sphinxuseclass}{sd-container-fluid}
\begin{sphinxuseclass}{sd-sphinx-override}
\begin{sphinxuseclass}{sd-row}
\begin{sphinxuseclass}{sd-row-cols-2}
\begin{sphinxuseclass}{sd-row-cols-xs-2}
\begin{sphinxuseclass}{sd-row-cols-sm-3}
\begin{sphinxuseclass}{sd-row-cols-md-3}
\begin{sphinxuseclass}{sd-row-cols-lg-3}
\begin{sphinxuseclass}{sd-gx-3}
\begin{sphinxuseclass}{sd-gy-1}
\begin{sphinxuseclass}{sd-col}
\begin{sphinxuseclass}{sd-col-auto}
\begin{sphinxuseclass}{sd-d-flex-row}
\begin{sphinxuseclass}{sd-align-minor-center}
\sphinxAtStartPar
basics

\end{sphinxuseclass}
\end{sphinxuseclass}
\end{sphinxuseclass}
\end{sphinxuseclass}
\begin{sphinxuseclass}{sd-col}
\begin{sphinxuseclass}{sd-col-auto}
\begin{sphinxuseclass}{sd-d-flex-row}
\begin{sphinxuseclass}{sd-align-minor-center}
\sphinxAtStartPar
16 gen 2025

\end{sphinxuseclass}
\end{sphinxuseclass}
\end{sphinxuseclass}
\end{sphinxuseclass}
\begin{sphinxuseclass}{sd-col}
\begin{sphinxuseclass}{sd-col-auto}
\begin{sphinxuseclass}{sd-d-flex-row}
\begin{sphinxuseclass}{sd-align-minor-center}
\sphinxAtStartPar
1 min read

\end{sphinxuseclass}
\end{sphinxuseclass}
\end{sphinxuseclass}
\end{sphinxuseclass}
\end{sphinxuseclass}
\end{sphinxuseclass}
\end{sphinxuseclass}
\end{sphinxuseclass}
\end{sphinxuseclass}
\end{sphinxuseclass}
\end{sphinxuseclass}
\end{sphinxuseclass}
\end{sphinxuseclass}
\end{sphinxuseclass}
\end{sphinxuseclass}
\end{sphinxuseclass}
\end{sphinxuseclass}
\end{sphinxuseclass}
\end{sphinxuseclass}
\end{sphinxuseclass}
\end{sphinxuseclass}
\end{sphinxuseclass}
\end{sphinxuseclass}
\end{sphinxuseclass}
\end{sphinxuseclass}
\end{sphinxuseclass}

\section{Principi della dinamica}
\label{\detokenize{ch/dynamics-principles:principi-della-dinamica}}\label{\detokenize{ch/dynamics-principles:classical-mechanics-dynamics-principles}}\label{\detokenize{ch/dynamics-principles::doc}}
\sphinxAtStartPar
\sphinxstylestrong{Primo principio della dinamica.}
Un corpo (o meglio, il baricentro di un corpo) sul quale agisce una forza netta nulla, persevera nel suo stato di quiete o di moto rettilineo uniforme rispetto a un sistema di riferimento inerziale.

\sphinxAtStartPar
\sphinxstylestrong{Secondo principio della dinamica.} Rispetto a un sistema di riferimento inerziale, la variazione della quantità di moto di un sistema è uguale all’impulso delle forze esterne agenti su di esso,
\begin{equation*}
\begin{split}\Delta \mathbf{Q} = \mathbf{I}^e \ .\end{split}
\end{equation*}
\sphinxAtStartPar
Nel caso di moto regolare, in cui la quantità di moto può essere rappresentata da una funzione continua e differenziabile in funzione del tempo, si può scrivere il secondo principio della dinamica in forma differenziale,
\begin{equation*}
\begin{split}\dot{\mathbf{Q}} = \mathbf{R}^e \ ,\end{split}
\end{equation*}
\sphinxAtStartPar
essendo la risultante delle forze esterne, \(\mathbf{R}^e = \frac{d \mathbf{I}^e}{dt}\), la derivata nel tempo dell’impulso.

\sphinxAtStartPar
\sphinxstylestrong{Terzo principio della dinamica.} Se un sistema \(i\) esercita su un sistema \(j\) una forza \(\mathbf{F}_{ji}\), allora il sistema \(j\) esercita sul sistema \(i\) una forza \(\mathbf{F}_{ij}\) «uguale e contraria», con modulo uguale e verso opposto,
\begin{equation*}
\begin{split}\mathbf{F}_{ij} = - \mathbf{F}_{ji} \ .\end{split}
\end{equation*}
\sphinxstepscope

\begin{sphinxuseclass}{sd-container-fluid}
\begin{sphinxuseclass}{sd-sphinx-override}
\begin{sphinxuseclass}{sd-p-0}
\begin{sphinxuseclass}{sd-mt-2}
\begin{sphinxuseclass}{sd-mb-4}
\begin{sphinxuseclass}{sd-row}
\begin{sphinxuseclass}{sd-row-cols-2}
\begin{sphinxuseclass}{sd-gx-2}
\begin{sphinxuseclass}{sd-gy-1}
\begin{sphinxuseclass}{sd-col}
\begin{sphinxuseclass}{sd-d-flex-row}
\begin{sphinxuseclass}{sd-align-minor-center}
\begin{sphinxuseclass}{sd-container-fluid}
\begin{sphinxuseclass}{sd-sphinx-override}
\begin{sphinxuseclass}{sd-row}
\begin{sphinxuseclass}{sd-row-cols-2}
\begin{sphinxuseclass}{sd-row-cols-xs-2}
\begin{sphinxuseclass}{sd-row-cols-sm-3}
\begin{sphinxuseclass}{sd-row-cols-md-3}
\begin{sphinxuseclass}{sd-row-cols-lg-3}
\begin{sphinxuseclass}{sd-gx-3}
\begin{sphinxuseclass}{sd-gy-1}
\begin{sphinxuseclass}{sd-col}
\begin{sphinxuseclass}{sd-col-auto}
\begin{sphinxuseclass}{sd-d-flex-row}
\begin{sphinxuseclass}{sd-align-minor-center}
\sphinxAtStartPar
basics

\end{sphinxuseclass}
\end{sphinxuseclass}
\end{sphinxuseclass}
\end{sphinxuseclass}
\begin{sphinxuseclass}{sd-col}
\begin{sphinxuseclass}{sd-col-auto}
\begin{sphinxuseclass}{sd-d-flex-row}
\begin{sphinxuseclass}{sd-align-minor-center}
\sphinxAtStartPar
16 gen 2025

\end{sphinxuseclass}
\end{sphinxuseclass}
\end{sphinxuseclass}
\end{sphinxuseclass}
\begin{sphinxuseclass}{sd-col}
\begin{sphinxuseclass}{sd-col-auto}
\begin{sphinxuseclass}{sd-d-flex-row}
\begin{sphinxuseclass}{sd-align-minor-center}
\sphinxAtStartPar
2 min read

\end{sphinxuseclass}
\end{sphinxuseclass}
\end{sphinxuseclass}
\end{sphinxuseclass}
\end{sphinxuseclass}
\end{sphinxuseclass}
\end{sphinxuseclass}
\end{sphinxuseclass}
\end{sphinxuseclass}
\end{sphinxuseclass}
\end{sphinxuseclass}
\end{sphinxuseclass}
\end{sphinxuseclass}
\end{sphinxuseclass}
\end{sphinxuseclass}
\end{sphinxuseclass}
\end{sphinxuseclass}
\end{sphinxuseclass}
\end{sphinxuseclass}
\end{sphinxuseclass}
\end{sphinxuseclass}
\end{sphinxuseclass}
\end{sphinxuseclass}
\end{sphinxuseclass}
\end{sphinxuseclass}
\end{sphinxuseclass}

\section{Equazioni cardinali della dinamica}
\label{\detokenize{ch/dynamics-eom:equazioni-cardinali-della-dinamica}}\label{\detokenize{ch/dynamics-eom:classical-mechanics-dynamics-eom}}\label{\detokenize{ch/dynamics-eom::doc}}
\sphinxAtStartPar
Utilizzando i concetti di quantità di moto, momento della quantità di moto ed energia cinetica di un sistema, si possono scrivere le 3 equazioni cardinali della dinamica in una forma valida per \sphinxstylestrong{ogni sistema chiuso}. Nel caso siano soddisfatte alcune condizioni, e solo in questo caso, le equazioni cardinali della dinamica rappresentano dei principi di conservazione delle quantità dinamiche: osservando le espressioni delle equazioni cardinali, è facile intuire che la condizione da soddisfare per otenere un principio di conservazione è l’annullamento di tutti i termini ad eccezione della derivata temporale della quantità conservata.


\subsection{Equazioni cardinali}
\label{\detokenize{ch/dynamics-eom:equazioni-cardinali}}
\sphinxAtStartPar
\sphinxstylestrong{Bilancio della quantità di moto.} La derivata nel tempo della quantità di moto è uguale alla risultante delle forze esterne,
\begin{equation}\label{equation:ch/dynamics-eom:principle:q}
\begin{split}\dot{\mathbf{Q}} = \mathbf{R}^e \ .\end{split}
\end{equation}
\sphinxAtStartPar
\sphinxstylestrong{Bilancio del momento della quantità di moto rispetto a un polo \(H\).} La derivata nel tempo del momento della quantità di moto rispetto a un punto \(H\), a meno di «un termine di trasporto» è uguale alla risultate dei momenti esterni rispetto al polo \(H\)
\begin{equation}\label{equation:ch/dynamics-eom:principle:l}
\begin{split}\dot{\mathbf{L}}_H + \dot{\mathbf{x}}_H \times \mathbf{Q} = \mathbf{M}_H^e \ .\end{split}
\end{equation}
\sphinxAtStartPar
\sphinxstylestrong{Bilancio dell’energia cinetica.} La derivata nel tempo dell’energia cinetica è uguale alla potenza totale agente sul sistema, risultato della somma della potenza delle azioni esterne e della potenza delle azioni interne al sistema
\begin{equation}\label{equation:ch/dynamics-eom:principle:k}
\begin{split}\dot{K} = P^{tot} = P^e + P^i\end{split}
\end{equation}

\subsection{Principi di conservazione}
\label{\detokenize{ch/dynamics-eom:principi-di-conservazione}}
\sphinxAtStartPar
\sphinxstylestrong{Conservazione della quantità di moto, in presenza di forze esterne con risultante nulla.} Se la risultante delle forze esterne è nulla, \(\mathbf{R}^e = \mathbf{0}\), dal bilancio della quantità di moto segue immediatamente
\begin{equation*}
\begin{split}\dot{\mathbf{Q}} = \mathbf{0} \qquad \rightarrow \qquad \mathbf{Q} = \bar{\mathbf{Q}} = \text{cost.}\end{split}
\end{equation*}
\sphinxAtStartPar
\sphinxstylestrong{Conservazione del momento della quantità di moto, in presenza di momenti esterni con risultante nulla.} Se la scelta del polo \(H\) rende nullo il termine di trasporto, \(\dot{\mathbf{r}}_H \times \mathbf{Q} = \mathbf{0}\), se la risultante dei momenti esterni è nulla, \(\mathbf{M}^e_H = \mathbf{0}\), dal bilancio del momento della quantità di moto segue immediatamente
\begin{equation*}
\begin{split}\dot{\mathbf{L}}_H = \mathbf{0} \qquad \rightarrow \qquad \mathbf{L}_H = \bar{\mathbf{L}}_H = \text{cost.}\end{split}
\end{equation*}
\sphinxAtStartPar
\sphinxstylestrong{Conservazione dell’energia cinetica, in presenza di potenza totale nulla.} Se la potenza totale delle azioni sul sistema è nulla, \(P^{tot} = 0\), dal bilancio dell’energia cinetica segue immediatamente
\begin{equation*}
\begin{split}\dot{K} = 0 \qquad \rightarrow \qquad K = \bar{K} = \text{cost.}\end{split}
\end{equation*}
\sphinxAtStartPar
\sphinxstylestrong{Conservazione dell’energia meccanica, in assenza di azioni non conservative.} Ai tre principi di conservazione direttamente ottenibili dalle equazioni cardinali, si aggiunge il principio della conservazione dell’energia meccanica, somma dell’energia cinetica e dell’energia potenziale del sistema,
\begin{equation*}
\begin{split}E^{mech} = K + V \ ,\end{split}
\end{equation*}
\sphinxAtStartPar
in assenza di azioni non conservative. Se non ci sono forze non conervative, la potenza delle azioni agenti sul sistema può essere scritta come l’opposto della derivata nel tempo dell’energia potenziale del sistema,
\begin{equation*}
\begin{split}P^{tot} = -\dot{V}\end{split}
\end{equation*}
\sphinxAtStartPar
Dal bilancio dell’energia cinetica segue
\begin{equation*}
\begin{split}\dot{K} = - \dot{V} \qquad \rightarrow \qquad \dfrac{d}{dt}(K+V) = 0 \qquad \rightarrow \qquad \dot{E}^{mech = 0} \qquad \rightarrow \qquad E^{mech} = \bar{E}^{mech} = \text{const.}\end{split}
\end{equation*}
\sphinxstepscope

\begin{sphinxuseclass}{sd-container-fluid}
\begin{sphinxuseclass}{sd-sphinx-override}
\begin{sphinxuseclass}{sd-p-0}
\begin{sphinxuseclass}{sd-mt-2}
\begin{sphinxuseclass}{sd-mb-4}
\begin{sphinxuseclass}{sd-row}
\begin{sphinxuseclass}{sd-row-cols-2}
\begin{sphinxuseclass}{sd-gx-2}
\begin{sphinxuseclass}{sd-gy-1}
\begin{sphinxuseclass}{sd-col}
\begin{sphinxuseclass}{sd-d-flex-row}
\begin{sphinxuseclass}{sd-align-minor-center}
\begin{sphinxuseclass}{sd-container-fluid}
\begin{sphinxuseclass}{sd-sphinx-override}
\begin{sphinxuseclass}{sd-row}
\begin{sphinxuseclass}{sd-row-cols-2}
\begin{sphinxuseclass}{sd-row-cols-xs-2}
\begin{sphinxuseclass}{sd-row-cols-sm-3}
\begin{sphinxuseclass}{sd-row-cols-md-3}
\begin{sphinxuseclass}{sd-row-cols-lg-3}
\begin{sphinxuseclass}{sd-gx-3}
\begin{sphinxuseclass}{sd-gy-1}
\begin{sphinxuseclass}{sd-col}
\begin{sphinxuseclass}{sd-col-auto}
\begin{sphinxuseclass}{sd-d-flex-row}
\begin{sphinxuseclass}{sd-align-minor-center}
\sphinxAtStartPar
basics

\end{sphinxuseclass}
\end{sphinxuseclass}
\end{sphinxuseclass}
\end{sphinxuseclass}
\begin{sphinxuseclass}{sd-col}
\begin{sphinxuseclass}{sd-col-auto}
\begin{sphinxuseclass}{sd-d-flex-row}
\begin{sphinxuseclass}{sd-align-minor-center}
\sphinxAtStartPar
16 gen 2025

\end{sphinxuseclass}
\end{sphinxuseclass}
\end{sphinxuseclass}
\end{sphinxuseclass}
\begin{sphinxuseclass}{sd-col}
\begin{sphinxuseclass}{sd-col-auto}
\begin{sphinxuseclass}{sd-d-flex-row}
\begin{sphinxuseclass}{sd-align-minor-center}
\sphinxAtStartPar
0 min read

\end{sphinxuseclass}
\end{sphinxuseclass}
\end{sphinxuseclass}
\end{sphinxuseclass}
\end{sphinxuseclass}
\end{sphinxuseclass}
\end{sphinxuseclass}
\end{sphinxuseclass}
\end{sphinxuseclass}
\end{sphinxuseclass}
\end{sphinxuseclass}
\end{sphinxuseclass}
\end{sphinxuseclass}
\end{sphinxuseclass}
\end{sphinxuseclass}
\end{sphinxuseclass}
\end{sphinxuseclass}
\end{sphinxuseclass}
\end{sphinxuseclass}
\end{sphinxuseclass}
\end{sphinxuseclass}
\end{sphinxuseclass}
\end{sphinxuseclass}
\end{sphinxuseclass}
\end{sphinxuseclass}
\end{sphinxuseclass}

\subsection{Equazioni cardinali della dinamica: punto materiale}
\label{\detokenize{ch/dynamics-eom-point:equazioni-cardinali-della-dinamica-punto-materiale}}\label{\detokenize{ch/dynamics-eom-point:classical-mechanics-dynamics-eom-point}}\label{\detokenize{ch/dynamics-eom-point::doc}}
\sphinxAtStartPar
\sphinxstylestrong{Quantità dinamiche.}
\begin{equation*}
\begin{split}\begin{aligned}
  \mathbf{Q}_P & := m_P \mathbf{v}_P \\
  \mathbf{L}_{P,H} & := (\mathbf{r}_P - \mathbf{r}_H) \times \mathbf{Q} = m_P (\mathbf{r}_P - \mathbf{r}_H) \times \mathbf{v}_P \\
  K & := \frac{1}{2} m_P \mathbf{v}_P \cdot \mathbf{v}_P = \frac{1}{2} m_P |\mathbf{v}_P|^2
\end{aligned}\end{split}
\end{equation*}
\sphinxAtStartPar
\sphinxstylestrong{Bilancio della quantità di moto.} Il bilancio della quantità di moto di un punto materiale \(P\), \(\mathbf{Q}_P = m \mathbf{v}_P\) segue direttamente dal secondo principio della dinamica di Newton,
\begin{equation*}
\begin{split}\dot{\mathbf{Q}}_P = \mathbf{R}^e_P\end{split}
\end{equation*}
\sphinxAtStartPar
\sphinxstylestrong{Bilancio del momento della quantità di moto.} La derivata nel tempo del momento della quantità di moto viene calcolata usando la regola del prodotto,
\begin{equation*}
\begin{split}\begin{aligned}
\dot{\mathbf{L}}_{P,H} & = \dfrac{d}{dt} \left[ m_P (\mathbf{r}_P - \mathbf{r}_H) \times \mathbf{v}_P \right] = \\
& = m \left[ ( \dot{\mathbf{r}}_P - \dot{\mathbf{r}}_H ) \times \mathbf{v}_P + m_P (\mathbf{r}_P - \mathbf{r}_H) \times \dot{\mathbf{v}}_P \right] = \\
& = - m_P \dot{\mathbf{r}}_H \times \mathbf{v}_P + m_P (\mathbf{r}_P - \mathbf{r}_H) \times \dot{\mathbf{v}}_P = \\
& = - \dot{\mathbf{r}}_H \times \mathbf{Q} + \mathbf{M}_H^{ext} \ .
\end{aligned}\end{split}
\end{equation*}
\sphinxAtStartPar
\sphinxstylestrong{Bilancio dell’energia cinetica.}
\begin{equation*}
\begin{split}\begin{aligned}
\dot{K}_{P} & = \dfrac{d}{dt} \left( \frac{1}{2} m_P \mathbf{v}_P \cdot \mathbf{v}_P \right) = \\
            & = m_P \dot{\mathbf{v}}_P \cdot \mathbf{v}_P = \\
            & = \mathbf{R}^e \cdot \mathbf{v}_P = \\
            & = \mathbf{R}^{tot} \cdot \mathbf{v}_P = P^{tot} \ .
\end{aligned}\end{split}
\end{equation*}
\sphinxstepscope

\begin{sphinxuseclass}{sd-container-fluid}
\begin{sphinxuseclass}{sd-sphinx-override}
\begin{sphinxuseclass}{sd-p-0}
\begin{sphinxuseclass}{sd-mt-2}
\begin{sphinxuseclass}{sd-mb-4}
\begin{sphinxuseclass}{sd-row}
\begin{sphinxuseclass}{sd-row-cols-2}
\begin{sphinxuseclass}{sd-gx-2}
\begin{sphinxuseclass}{sd-gy-1}
\begin{sphinxuseclass}{sd-col}
\begin{sphinxuseclass}{sd-d-flex-row}
\begin{sphinxuseclass}{sd-align-minor-center}
\begin{sphinxuseclass}{sd-container-fluid}
\begin{sphinxuseclass}{sd-sphinx-override}
\begin{sphinxuseclass}{sd-row}
\begin{sphinxuseclass}{sd-row-cols-2}
\begin{sphinxuseclass}{sd-row-cols-xs-2}
\begin{sphinxuseclass}{sd-row-cols-sm-3}
\begin{sphinxuseclass}{sd-row-cols-md-3}
\begin{sphinxuseclass}{sd-row-cols-lg-3}
\begin{sphinxuseclass}{sd-gx-3}
\begin{sphinxuseclass}{sd-gy-1}
\begin{sphinxuseclass}{sd-col}
\begin{sphinxuseclass}{sd-col-auto}
\begin{sphinxuseclass}{sd-d-flex-row}
\begin{sphinxuseclass}{sd-align-minor-center}
\sphinxAtStartPar
basics

\end{sphinxuseclass}
\end{sphinxuseclass}
\end{sphinxuseclass}
\end{sphinxuseclass}
\begin{sphinxuseclass}{sd-col}
\begin{sphinxuseclass}{sd-col-auto}
\begin{sphinxuseclass}{sd-d-flex-row}
\begin{sphinxuseclass}{sd-align-minor-center}
\sphinxAtStartPar
16 gen 2025

\end{sphinxuseclass}
\end{sphinxuseclass}
\end{sphinxuseclass}
\end{sphinxuseclass}
\begin{sphinxuseclass}{sd-col}
\begin{sphinxuseclass}{sd-col-auto}
\begin{sphinxuseclass}{sd-d-flex-row}
\begin{sphinxuseclass}{sd-align-minor-center}
\sphinxAtStartPar
2 min read

\end{sphinxuseclass}
\end{sphinxuseclass}
\end{sphinxuseclass}
\end{sphinxuseclass}
\end{sphinxuseclass}
\end{sphinxuseclass}
\end{sphinxuseclass}
\end{sphinxuseclass}
\end{sphinxuseclass}
\end{sphinxuseclass}
\end{sphinxuseclass}
\end{sphinxuseclass}
\end{sphinxuseclass}
\end{sphinxuseclass}
\end{sphinxuseclass}
\end{sphinxuseclass}
\end{sphinxuseclass}
\end{sphinxuseclass}
\end{sphinxuseclass}
\end{sphinxuseclass}
\end{sphinxuseclass}
\end{sphinxuseclass}
\end{sphinxuseclass}
\end{sphinxuseclass}
\end{sphinxuseclass}
\end{sphinxuseclass}

\subsection{Equazioni cardinali della dinamica: sistemi discreti di punti materiali}
\label{\detokenize{ch/dynamics-eom-points:equazioni-cardinali-della-dinamica-sistemi-discreti-di-punti-materiali}}\label{\detokenize{ch/dynamics-eom-points:classical-mechanics-dynamics-eom-points}}\label{\detokenize{ch/dynamics-eom-points::doc}}
\sphinxAtStartPar
Partendo dalle equazioni dinamiche per un punto, si calcolano le equazioni dinamiche per un sistema di punti, sfruttando il terzo principio della dinamica di azione/reazione. Lo sviluppo delle equazioni permette di comprendere che la natura additiva delle grandezze dinamiche (quantità di moto, momento della quantità di moto, energia cinetica) segue direttamente dalla loro definizione.

\sphinxAtStartPar
\sphinxstylestrong{Bilancio della quantità di moto.}
E” possibile scrivere il bilancio della quantità di moto per ogni punto \(i\) del sistema, scrivendo la risultante delle forze esterne agente sul punto come la somma delle forze esterne all’intero sistema agenti sul punto e le forze interne scambiate con gli altri punti del sistema,
\begin{equation*}
\begin{split}\mathbf{R}_i^{ext,i} = \mathbf{F}_i^{ext} + \sum_{j \ne i} \mathbf{F}_{ij} \ .\end{split}
\end{equation*}
\sphinxAtStartPar
L’equazione di bilancio per la \(i\)\sphinxhyphen{}esima massa diventa quindi
\begin{equation*}
\begin{split}\dot{\mathbf{Q}}_i = \mathbf{R}_i^{ext,i} = \mathbf{F}_i^{ext} + \sum_{j \ne i} \mathbf{F}_{ij} \ .\end{split}
\end{equation*}
\sphinxAtStartPar
Sommando le equazioni di bilancio di tutte le masse, si ottiene
\begin{equation*}
\begin{split}\begin{aligned}
\sum_{i} \dot{\mathbf{Q}}_i & = \sum_i \mathbf{F}_{i}^{ext} + \sum_i \sum_{j \ne i} \mathbf{F}_{ij} = \\
                            & = \sum_i \mathbf{F}_{i}^{ext} + \sum_{\{i,j\}} \underbrace{\left( \mathbf{F}_{ij} + \mathbf{F}_{ji} \right)}_{=\mathbf{0}} 
\end{aligned}\end{split}
\end{equation*}
\sphinxAtStartPar
e definendo la quantità di moto di un sistema come la somma delle quantità di moto delle sue parti e la risultante delle forze esterne come somma delle forze esterne agenti sulle parti del sistema,
\begin{equation*}
\begin{split}\mathbf{Q} := \sum_i \mathbf{Q}\end{split}
\end{equation*}\begin{equation*}
\begin{split}\mathbf{R}^e := \sum_i \mathbf{F}_i^{ext}\end{split}
\end{equation*}
\sphinxAtStartPar
si ritrova la forma generale del bilancio della quantità di moto,
\begin{equation*}
\begin{split}\dot{\mathbf{Q}} = \mathbf{R}^e \ .\end{split}
\end{equation*}
\sphinxAtStartPar
\sphinxstylestrong{Bilancio del momento della quantità di moto.}
E” possibile scrivere il bilancio del momento della quantità di moto per ogni punto \(i\) del sistema, scrivendo la risultante dei momenti esterni agente sul punto come la somma dei momenti esterni all’intero sistema agenti sul punto e i momenti interni scambiati con gli altri punti del sistema,
\begin{equation*}
\begin{split}\mathbf{M}_{H,i}^{ext,i} = \mathbf{M}_{H,i}^{ext} + \sum_{j \ne i} \mathbf{M}_{H,ij} \ .\end{split}
\end{equation*}
\sphinxAtStartPar
Nel caso le parti del sistema interagiscano tramite forze, il momento rispetto al polo \(H\) generato dalla massa \(j\) sulla massa \(i\) vale
\begin{equation*}
\begin{split}\mathbf{M}_{H,ij} = (\mathbf{r}_i - \mathbf{r}_H) \times \mathbf{F}_{ij} \ .\end{split}
\end{equation*}
\sphinxAtStartPar
L’equazione di bilancio per la \(i\)\sphinxhyphen{}esima massa diventa quindi
\begin{equation*}
\begin{split}\dot{\mathbf{L}}_{H,i} + \dot{\mathbf{r}}_H \times \mathbf{Q}_i = \mathbf{M}_{H,i}^{ext,i} = \mathbf{M}_{H,i}^{ext} + \sum_{j \ne i} \mathbf{M}_{H,ij} \ .\end{split}
\end{equation*}
\sphinxAtStartPar
Sommando le equazioni di bilancio di tutte le masse, si ottiene
\begin{equation*}
\begin{split}\begin{aligned}
\sum_{i} \left( \dot{\mathbf{L}}_i + \dot{\mathbf{r}}_H \times \mathbf{Q}_i \right) & = \sum_i \mathbf{M}_{H,i}^{ext} + \sum_i \sum_{j \ne i} \mathbf{M}_{H,ij} = \\
                            & = \sum_i \mathbf{M}_{H,i}^{ext} + \sum_{\{i,j\}} \underbrace{\left( \mathbf{M}_{H,ij} + \mathbf{M}_{H,ji} \right)}_{=\mathbf{0}} 
\end{aligned}\end{split}
\end{equation*}
\sphinxAtStartPar
e riconoscendo la quantità di moto del sistema e definendo il momento della quantità di moto di un sistema come la somma del momento della quantità di moto delle sue parti e la risultante dei momenti esterni come somma dei momenti esterni agenti sulle parti del sistema,
\begin{equation*}
\begin{split}\mathbf{L}_H := \sum_i \mathbf{L}_{H,i}\end{split}
\end{equation*}\begin{equation*}
\begin{split}\mathbf{M}_H^e := \sum_i \mathbf{M}_{H,i}^{ext}\end{split}
\end{equation*}
\sphinxAtStartPar
si ritrova la forma generale del bilancio del momento della quantità di moto,
\begin{equation*}
\begin{split}\dot{\mathbf{L}}_{H} + \dot{\mathbf{r}}_H \times \mathbf{Q} = \mathbf{M}_H^e \ .\end{split}
\end{equation*}
\sphinxAtStartPar
\sphinxstylestrong{Bilancio dell’energia cinetica.}
E” possibile ricavare il bilancio dell’energia cinetica del sistema, moltiplicando scalarmente il bilancio della quantità di moto di ogni punto,
\begin{equation*}
\begin{split}\mathbf{v}_i \cdot m_i \dot{\mathbf{v}}_i = \mathbf{v}_i \cdot \left( \mathbf{F}_i^{e} + \sum_{j \ne i} \mathbf{F}_{ij} \right) \ ,\end{split}
\end{equation*}
\sphinxAtStartPar
riconoscendo nel primo termine la derivata nel tempo dell’energia cinetica dell”\(i\)\sphinxhyphen{}esimo punto,
\begin{equation*}
\begin{split}\dot{K}_i = \dfrac{d}{dt} \left( \frac{1}{2} m_i \mathbf{v}_i \cdot \mathbf{v}_i \right) = m_i \mathbf{v}_i \cdot \dot{\mathbf{v}}_i \ ,\end{split}
\end{equation*}
\sphinxAtStartPar
e sommando queste equazioni di bilancio per ottenere
\begin{equation*}
\begin{split}\begin{aligned}
  \sum_i \dot{K}_i = \sum_i \mathbf{v}_i \cdot  \mathbf{F}_i^{e} + \sum_i \mathbf{v}_i \cdot \sum_{j \ne i} \mathbf{F}_{ij} \ . 
\end{aligned}\end{split}
\end{equation*}
\sphinxAtStartPar
Definendo l’energia cinetica di un sistema come la somma dell’energia cinetica delle sue parti, e definendo la potenza delle forze esterne/interne agenti sul sistema come la somma della potenza di tutte le forze esterne/interni al sistema,
\begin{equation*}
\begin{split}K :=  \sum_i K_i\end{split}
\end{equation*}\begin{equation*}
\begin{split}P^e := \sum_i P^{ext}_i = \sum_i \mathbf{v}_i \cdot  \mathbf{F}_i^{e} \end{split}
\end{equation*}\begin{equation*}
\begin{split}P^i := \sum_i P^{int}_i = \sum_i \mathbf{v}_i \cdot \sum_{j \ne i} \mathbf{F}_{ij}\end{split}
\end{equation*}
\sphinxAtStartPar
si ritrova la forma generale del bilancio dell’energia cinetica,
\begin{equation*}
\begin{split}\dot{K} = P^e + P^i = P^{tot} \ .\end{split}
\end{equation*}
\sphinxstepscope

\begin{sphinxuseclass}{sd-container-fluid}
\begin{sphinxuseclass}{sd-sphinx-override}
\begin{sphinxuseclass}{sd-p-0}
\begin{sphinxuseclass}{sd-mt-2}
\begin{sphinxuseclass}{sd-mb-4}
\begin{sphinxuseclass}{sd-row}
\begin{sphinxuseclass}{sd-row-cols-2}
\begin{sphinxuseclass}{sd-gx-2}
\begin{sphinxuseclass}{sd-gy-1}
\begin{sphinxuseclass}{sd-col}
\begin{sphinxuseclass}{sd-d-flex-row}
\begin{sphinxuseclass}{sd-align-minor-center}
\begin{sphinxuseclass}{sd-container-fluid}
\begin{sphinxuseclass}{sd-sphinx-override}
\begin{sphinxuseclass}{sd-row}
\begin{sphinxuseclass}{sd-row-cols-2}
\begin{sphinxuseclass}{sd-row-cols-xs-2}
\begin{sphinxuseclass}{sd-row-cols-sm-3}
\begin{sphinxuseclass}{sd-row-cols-md-3}
\begin{sphinxuseclass}{sd-row-cols-lg-3}
\begin{sphinxuseclass}{sd-gx-3}
\begin{sphinxuseclass}{sd-gy-1}
\begin{sphinxuseclass}{sd-col}
\begin{sphinxuseclass}{sd-col-auto}
\begin{sphinxuseclass}{sd-d-flex-row}
\begin{sphinxuseclass}{sd-align-minor-center}
\sphinxAtStartPar
basics

\end{sphinxuseclass}
\end{sphinxuseclass}
\end{sphinxuseclass}
\end{sphinxuseclass}
\begin{sphinxuseclass}{sd-col}
\begin{sphinxuseclass}{sd-col-auto}
\begin{sphinxuseclass}{sd-d-flex-row}
\begin{sphinxuseclass}{sd-align-minor-center}
\sphinxAtStartPar
16 gen 2025

\end{sphinxuseclass}
\end{sphinxuseclass}
\end{sphinxuseclass}
\end{sphinxuseclass}
\begin{sphinxuseclass}{sd-col}
\begin{sphinxuseclass}{sd-col-auto}
\begin{sphinxuseclass}{sd-d-flex-row}
\begin{sphinxuseclass}{sd-align-minor-center}
\sphinxAtStartPar
0 min read

\end{sphinxuseclass}
\end{sphinxuseclass}
\end{sphinxuseclass}
\end{sphinxuseclass}
\end{sphinxuseclass}
\end{sphinxuseclass}
\end{sphinxuseclass}
\end{sphinxuseclass}
\end{sphinxuseclass}
\end{sphinxuseclass}
\end{sphinxuseclass}
\end{sphinxuseclass}
\end{sphinxuseclass}
\end{sphinxuseclass}
\end{sphinxuseclass}
\end{sphinxuseclass}
\end{sphinxuseclass}
\end{sphinxuseclass}
\end{sphinxuseclass}
\end{sphinxuseclass}
\end{sphinxuseclass}
\end{sphinxuseclass}
\end{sphinxuseclass}
\end{sphinxuseclass}
\end{sphinxuseclass}
\end{sphinxuseclass}

\subsection{Equazioni cardinali della dinamica: corpo rigido}
\label{\detokenize{ch/dynamics-eom-rigid:equazioni-cardinali-della-dinamica-corpo-rigido}}\label{\detokenize{ch/dynamics-eom-rigid:classical-mechanics-dynamics-eom-rigid}}\label{\detokenize{ch/dynamics-eom-rigid::doc}}
\sphinxstepscope

\begin{sphinxuseclass}{sd-container-fluid}
\begin{sphinxuseclass}{sd-sphinx-override}
\begin{sphinxuseclass}{sd-p-0}
\begin{sphinxuseclass}{sd-mt-2}
\begin{sphinxuseclass}{sd-mb-4}
\begin{sphinxuseclass}{sd-row}
\begin{sphinxuseclass}{sd-row-cols-2}
\begin{sphinxuseclass}{sd-gx-2}
\begin{sphinxuseclass}{sd-gy-1}
\begin{sphinxuseclass}{sd-col}
\begin{sphinxuseclass}{sd-d-flex-row}
\begin{sphinxuseclass}{sd-align-minor-center}
\begin{sphinxuseclass}{sd-container-fluid}
\begin{sphinxuseclass}{sd-sphinx-override}
\begin{sphinxuseclass}{sd-row}
\begin{sphinxuseclass}{sd-row-cols-2}
\begin{sphinxuseclass}{sd-row-cols-xs-2}
\begin{sphinxuseclass}{sd-row-cols-sm-3}
\begin{sphinxuseclass}{sd-row-cols-md-3}
\begin{sphinxuseclass}{sd-row-cols-lg-3}
\begin{sphinxuseclass}{sd-gx-3}
\begin{sphinxuseclass}{sd-gy-1}
\begin{sphinxuseclass}{sd-col}
\begin{sphinxuseclass}{sd-col-auto}
\begin{sphinxuseclass}{sd-d-flex-row}
\begin{sphinxuseclass}{sd-align-minor-center}
\sphinxAtStartPar
basics

\end{sphinxuseclass}
\end{sphinxuseclass}
\end{sphinxuseclass}
\end{sphinxuseclass}
\begin{sphinxuseclass}{sd-col}
\begin{sphinxuseclass}{sd-col-auto}
\begin{sphinxuseclass}{sd-d-flex-row}
\begin{sphinxuseclass}{sd-align-minor-center}
\sphinxAtStartPar
16 gen 2025

\end{sphinxuseclass}
\end{sphinxuseclass}
\end{sphinxuseclass}
\end{sphinxuseclass}
\begin{sphinxuseclass}{sd-col}
\begin{sphinxuseclass}{sd-col-auto}
\begin{sphinxuseclass}{sd-d-flex-row}
\begin{sphinxuseclass}{sd-align-minor-center}
\sphinxAtStartPar
0 min read

\end{sphinxuseclass}
\end{sphinxuseclass}
\end{sphinxuseclass}
\end{sphinxuseclass}
\end{sphinxuseclass}
\end{sphinxuseclass}
\end{sphinxuseclass}
\end{sphinxuseclass}
\end{sphinxuseclass}
\end{sphinxuseclass}
\end{sphinxuseclass}
\end{sphinxuseclass}
\end{sphinxuseclass}
\end{sphinxuseclass}
\end{sphinxuseclass}
\end{sphinxuseclass}
\end{sphinxuseclass}
\end{sphinxuseclass}
\end{sphinxuseclass}
\end{sphinxuseclass}
\end{sphinxuseclass}
\end{sphinxuseclass}
\end{sphinxuseclass}
\end{sphinxuseclass}
\end{sphinxuseclass}
\end{sphinxuseclass}

\subsection{Equazioni cardinali della dinamica: mezzi continui}
\label{\detokenize{ch/dynamics-eom-continuum:equazioni-cardinali-della-dinamica-mezzi-continui}}\label{\detokenize{ch/dynamics-eom-continuum:classical-mechanics-dynamics-eom-continuum}}\label{\detokenize{ch/dynamics-eom-continuum::doc}}
\sphinxstepscope

\begin{sphinxuseclass}{sd-container-fluid}
\begin{sphinxuseclass}{sd-sphinx-override}
\begin{sphinxuseclass}{sd-p-0}
\begin{sphinxuseclass}{sd-mt-2}
\begin{sphinxuseclass}{sd-mb-4}
\begin{sphinxuseclass}{sd-row}
\begin{sphinxuseclass}{sd-row-cols-2}
\begin{sphinxuseclass}{sd-gx-2}
\begin{sphinxuseclass}{sd-gy-1}
\begin{sphinxuseclass}{sd-col}
\begin{sphinxuseclass}{sd-d-flex-row}
\begin{sphinxuseclass}{sd-align-minor-center}
\begin{sphinxuseclass}{sd-container-fluid}
\begin{sphinxuseclass}{sd-sphinx-override}
\begin{sphinxuseclass}{sd-row}
\begin{sphinxuseclass}{sd-row-cols-2}
\begin{sphinxuseclass}{sd-row-cols-xs-2}
\begin{sphinxuseclass}{sd-row-cols-sm-3}
\begin{sphinxuseclass}{sd-row-cols-md-3}
\begin{sphinxuseclass}{sd-row-cols-lg-3}
\begin{sphinxuseclass}{sd-gx-3}
\begin{sphinxuseclass}{sd-gy-1}
\begin{sphinxuseclass}{sd-col}
\begin{sphinxuseclass}{sd-col-auto}
\begin{sphinxuseclass}{sd-d-flex-row}
\begin{sphinxuseclass}{sd-align-minor-center}
\sphinxAtStartPar
basics

\end{sphinxuseclass}
\end{sphinxuseclass}
\end{sphinxuseclass}
\end{sphinxuseclass}
\begin{sphinxuseclass}{sd-col}
\begin{sphinxuseclass}{sd-col-auto}
\begin{sphinxuseclass}{sd-d-flex-row}
\begin{sphinxuseclass}{sd-align-minor-center}
\sphinxAtStartPar
16 gen 2025

\end{sphinxuseclass}
\end{sphinxuseclass}
\end{sphinxuseclass}
\end{sphinxuseclass}
\begin{sphinxuseclass}{sd-col}
\begin{sphinxuseclass}{sd-col-auto}
\begin{sphinxuseclass}{sd-d-flex-row}
\begin{sphinxuseclass}{sd-align-minor-center}
\sphinxAtStartPar
1 min read

\end{sphinxuseclass}
\end{sphinxuseclass}
\end{sphinxuseclass}
\end{sphinxuseclass}
\end{sphinxuseclass}
\end{sphinxuseclass}
\end{sphinxuseclass}
\end{sphinxuseclass}
\end{sphinxuseclass}
\end{sphinxuseclass}
\end{sphinxuseclass}
\end{sphinxuseclass}
\end{sphinxuseclass}
\end{sphinxuseclass}
\end{sphinxuseclass}
\end{sphinxuseclass}
\end{sphinxuseclass}
\end{sphinxuseclass}
\end{sphinxuseclass}
\end{sphinxuseclass}
\end{sphinxuseclass}
\end{sphinxuseclass}
\end{sphinxuseclass}
\end{sphinxuseclass}
\end{sphinxuseclass}
\end{sphinxuseclass}

\section{Moti particolari}
\label{\detokenize{ch/dynamics-motions:moti-particolari}}\label{\detokenize{ch/dynamics-motions:classical-mechanics-dynamics-motions}}\label{\detokenize{ch/dynamics-motions::doc}}
\sphinxAtStartPar
In questa sezione verranno studiati alcuni moti particolari, interessanti e utili da analizzare per motivi didattici, storici, e applicativi.
\begin{itemize}
\item {} 
\sphinxAtStartPar
moto rettilineo uniforme

\item {} 
\sphinxAtStartPar
moto uniformemente accelerato

\item {} 
\sphinxAtStartPar
moto circolare uniforme

\item {} 
\sphinxAtStartPar
moti oscillatori e oscillatori smorzati:
\begin{itemize}
\item {} 
\sphinxAtStartPar
oscillazioni libere:
\begin{itemize}
\item {} 
\sphinxAtStartPar
sistema massa\sphinxhyphen{}molla(\sphinxhyphen{}smorzatore)

\item {} 
\sphinxAtStartPar
pendolo

\end{itemize}

\item {} 
\sphinxAtStartPar
oscillazioni forzate:
\begin{itemize}
\item {} 
\sphinxAtStartPar
primo passo verso l’analisi strutturale e non solo («ogni sistema fisico è un sistema di tanti oscillatori armonici»)

\item {} 
\sphinxAtStartPar
concetti di risposta in frequenza e risonanza. \sphinxstylestrong{todo} video e/o script su risposta in frequenza di strutture e strutture antisismiche, mass\sphinxhyphen{}damper,…

\end{itemize}

\end{itemize}

\item {} 
\sphinxAtStartPar
\sphinxstylestrong{Gravitazione}: partendo dalla legge di gravitazione universale fornita da Newton, si studia il moto dei corpi celesti in sistemi a due corpi, scoprendo che le loro traiettorie descrivono le sezioni coniche (cerchio, ellisse, parabola, iperbole), e dimostrando le leggi di Keplero

\item {} 
\sphinxAtStartPar
rotazione di un corpo attorno a un punto fisso, moti di Poinsot

\end{itemize}

\sphinxstepscope

\begin{sphinxuseclass}{sd-container-fluid}
\begin{sphinxuseclass}{sd-sphinx-override}
\begin{sphinxuseclass}{sd-p-0}
\begin{sphinxuseclass}{sd-mt-2}
\begin{sphinxuseclass}{sd-mb-4}
\begin{sphinxuseclass}{sd-row}
\begin{sphinxuseclass}{sd-row-cols-2}
\begin{sphinxuseclass}{sd-gx-2}
\begin{sphinxuseclass}{sd-gy-1}
\begin{sphinxuseclass}{sd-col}
\begin{sphinxuseclass}{sd-d-flex-row}
\begin{sphinxuseclass}{sd-align-minor-center}
\begin{sphinxuseclass}{sd-container-fluid}
\begin{sphinxuseclass}{sd-sphinx-override}
\begin{sphinxuseclass}{sd-row}
\begin{sphinxuseclass}{sd-row-cols-2}
\begin{sphinxuseclass}{sd-row-cols-xs-2}
\begin{sphinxuseclass}{sd-row-cols-sm-3}
\begin{sphinxuseclass}{sd-row-cols-md-3}
\begin{sphinxuseclass}{sd-row-cols-lg-3}
\begin{sphinxuseclass}{sd-gx-3}
\begin{sphinxuseclass}{sd-gy-1}
\begin{sphinxuseclass}{sd-col}
\begin{sphinxuseclass}{sd-col-auto}
\begin{sphinxuseclass}{sd-d-flex-row}
\begin{sphinxuseclass}{sd-align-minor-center}
\sphinxAtStartPar
basics

\end{sphinxuseclass}
\end{sphinxuseclass}
\end{sphinxuseclass}
\end{sphinxuseclass}
\begin{sphinxuseclass}{sd-col}
\begin{sphinxuseclass}{sd-col-auto}
\begin{sphinxuseclass}{sd-d-flex-row}
\begin{sphinxuseclass}{sd-align-minor-center}
\sphinxAtStartPar
16 gen 2025

\end{sphinxuseclass}
\end{sphinxuseclass}
\end{sphinxuseclass}
\end{sphinxuseclass}
\begin{sphinxuseclass}{sd-col}
\begin{sphinxuseclass}{sd-col-auto}
\begin{sphinxuseclass}{sd-d-flex-row}
\begin{sphinxuseclass}{sd-align-minor-center}
\sphinxAtStartPar
0 min read

\end{sphinxuseclass}
\end{sphinxuseclass}
\end{sphinxuseclass}
\end{sphinxuseclass}
\end{sphinxuseclass}
\end{sphinxuseclass}
\end{sphinxuseclass}
\end{sphinxuseclass}
\end{sphinxuseclass}
\end{sphinxuseclass}
\end{sphinxuseclass}
\end{sphinxuseclass}
\end{sphinxuseclass}
\end{sphinxuseclass}
\end{sphinxuseclass}
\end{sphinxuseclass}
\end{sphinxuseclass}
\end{sphinxuseclass}
\end{sphinxuseclass}
\end{sphinxuseclass}
\end{sphinxuseclass}
\end{sphinxuseclass}
\end{sphinxuseclass}
\end{sphinxuseclass}
\end{sphinxuseclass}
\end{sphinxuseclass}

\subsection{Gravitazione: problema dei due corpi}
\label{\detokenize{ch/dynamics-motions-gravitation-2bodies:gravitazione-problema-dei-due-corpi}}\label{\detokenize{ch/dynamics-motions-gravitation-2bodies:classical-mechanics-dynamics-motions-gravitation-2bodies}}\label{\detokenize{ch/dynamics-motions-gravitation-2bodies::doc}}
\sphinxstepscope


\part{Analytical Mechanics}

\sphinxstepscope




\chapter{Lagrangian Mechanics}
\label{\detokenize{ch/lagrange:lagrangian-mechanics}}\label{\detokenize{ch/lagrange:classical-mechanics-lagrange}}\label{\detokenize{ch/lagrange::doc}}
\sphinxAtStartPar
Classical mechanics can be re\sphinxhyphen{}formulated starting from variational principles, usually referred as \sphinxstylestrong{analytical mechanics}. Under some assumptions, that will be discussed during the derivation, analytical mechanics is equivalent to Newton mechanics.

\sphinxAtStartPar
Here, the equivalence of analytical mechanics and Newton mechanics is stressed, by means of the derivation of the principle of analytical mechanics starting from the equations of motions derived in Newtonian mechanics, relying on the conservation of mass and the three principles of Newton mechanics. The process is shown in the following sections for {\hyperref[\detokenize{ch/lagrange-point:classical-mechanics-lagrange-point}]{\sphinxcrossref{\DUrole{std,std-ref}{point systems}}}}, {\hyperref[\detokenize{ch/lagrange-points:classical-mechanics-lagrange-points}]{\sphinxcrossref{\DUrole{std,std-ref}{systems of points}}}}, {\hyperref[\detokenize{ch/lagrange-rigid-body:classical-mechanics-lagrange-rigid}]{\sphinxcrossref{\DUrole{std,std-ref}{extended rigid bodies}}}} and follows these steps:
\begin{itemize}
\item {} 
\sphinxAtStartPar
\sphinxstylestrong{strong form of equations.} Starting point is the dynamical equations of Newton mechanics, here also referred as the strong form of equations

\item {} 
\sphinxAtStartPar
\sphinxstylestrong{weak form of equations.} Strong form are recast in weak form, also referred as \sphinxstylestrong{D’Alembert approach} or \sphinxstylestrong{virtual work formulation}, multiplying strong form of equations for arbitrary test functions

\item {} 
\sphinxAtStartPar
\sphinxstylestrong{Lagrange equations.} A proper choice of test functions as a function of generalized coordinates, and some manipulation, leads to Lagrange equations. While the choice of test functions depends on the nature of the system, their expression always reads
\begin{equation*}
\begin{split}\dfrac{d}{dt}\left( \frac{\partial \mathscr{L}}{\partial \dot{q}^k} \right) - \frac{\partial \mathscr{L}}{\partial q^k} = Q_{q^k} \ ,\end{split}
\end{equation*}
\sphinxAtStartPar
being \(q^k(t)\) the generalized coordinates, \(\mathscr{L}\left(\dot{q}^k(t), q?k(t), t \right) = K\left(\dot{q}^k(t), q^k(t), t\right) + U(q^k(t), t)\) the Lagrangian function of the system, defined as the sum of the kinetic energy \(K\) and the potential function \(U = - V\), being \(V\) the potential energy \sphinxhyphen{} s.t. the conservative vector field reads \(\vec{F} = - \nabla V\), and \(Q_q\) the generalized force.

\item {} 
\sphinxAtStartPar
Lagrange equations can be interpreted as a result of a stationary principle of a functional, \(S\), defined \sphinxstylestrong{action functional}, as it can be shown with the tools of \sphinxhref{https://basics2022.github.io/bbooks-math-miscellanea/ch/calculus-variations/intro.html}{calculus of variations}. Here, \sphinxstylestrong{assuming \(Q_{q^{k}} = 0\)}, and multiplying by \(w^k(t)\), integrating over time from \(t_0\), \(t_1\), and assuming that \(w(t_0) = w(t_1) = 0\),
\begin{equation*}
\begin{split}\begin{aligned}
     0 & = \int_{t_0}^{t_1} w^k (t) \left[ \dfrac{d}{dt}\left( \frac{\partial \mathscr{L}}{\partial \dot{q}^k} \right) - \frac{\partial \mathscr{L}}{\partial q} \right] \, dt = \\
       & = w^k(t) \left.\left( \frac{\partial \mathscr{L}}{\partial \dot{q}^k} \right)\right|_{t_0}^{t_1} - \int_{t_0}^{t_1} \left[ \dot{w}^k(t) \, \frac{\partial \mathscr{L}}{\partial \dot{q}^k} + w^k(t) \, \frac{\partial \mathscr{L}}{\partial q^k} \right] \, dt \ . \\
   \end{aligned}\end{split}
\end{equation*}
\sphinxAtStartPar
If \(w^k(t)\) is equal to zero for \(t\) equal to \(t_0\) and \(t_1\), first term vanishes
\begin{equation*}
\begin{split}\begin{aligned}
       0 & = - \int_{t_0}^{t_1} \left[ \dot{w}^k(t) \, \frac{\partial \mathscr{L}}{\partial \dot{q}^k} + w^k(t) \, \frac{\partial \mathscr{L}}{\partial q^k} \right] \, dt \\
       & = - \frac{1}{\varepsilon} \int_{t_0}^{t_1} \varepsilon \left[ \dot{w}^k(t) \, \frac{\partial \mathscr{L}}{\partial \dot{q}^k}\left(\dot{q}^l(t), q^l(t), t \right) + w^k(t) \, \frac{\partial \mathscr{L}}{\partial q^k}\left(\dot{q}^l(t), q^l(t), t \right) \right] \, dt = \\
       & = - \lim_{\varepsilon \rightarrow 0} \left\{ \frac{1}{\varepsilon} \int_{t_0}^{t_1} \varepsilon \left[ \dot{w}^k(t) \, \frac{\partial \mathscr{L}}{\partial \dot{q}^k}\left(\dot{q}^l(t), q^l(t), t \right) + w^k(t) \, \frac{\partial \mathscr{L}}{\partial q^k}\left(\dot{q}^l(t), q^l(t), t \right) \right] \, dt \right\}= \\
       & = - \lim_{\varepsilon \rightarrow 0} \left\{ \frac{1}{\varepsilon} \int_{t_0}^{t_1} \left[ \mathscr{L}\left(\dot{q}^l(t)+\varepsilon \dot{w}^l(t), q^l(t) + \varepsilon w^l(t), t \right) - \mathscr{L}\left(\dot{q}^l(t), q^l(t), t \right) \right] \, dt + o(\varepsilon) \right\}= \\
       & = - \delta \int_{t_0}^{t_1} \mathscr{L}(\dot{q}^l(t), q^l(t), t) \, dt =: - \delta S[q^k(t)] \ ,
   \end{aligned}\end{split}
\end{equation*}
\sphinxAtStartPar
i.e. Lagrange equations are equivalent to the stationary condition of the action functional
\begin{equation*}
\begin{split}S[q^k(t)]:= \int_{t_0}^{t_1} \mathscr{L}\left(\dot{q}^l(t), q^l(t), t\right) \, dt \ .\end{split}
\end{equation*}
\end{itemize}



\sphinxstepscope




\section{Point}
\label{\detokenize{ch/lagrange-point:point}}\label{\detokenize{ch/lagrange-point:classical-mechanics-lagrange-point}}\label{\detokenize{ch/lagrange-point::doc}}
\sphinxAtStartPar
\sphinxstylestrong{Newton dynamical equations \sphinxhyphen{} strong form.} Dynamical equation governing the motion of a point \(P\) reads
\begin{equation*}
\begin{split}m \dot{\vec{v}}_P = \vec{R}^e \ ,\end{split}
\end{equation*}
\sphinxAtStartPar
being \(m\) the mass of the system, \(\vec{v}_P\) the velocity of point \(P\), \(\vec{a}_P = \dot{\vec{v}}_P\) its acceleration and \(\vec{R}^{e}\) the net external force acting on the system..

\sphinxAtStartPar
\sphinxstylestrong{Weak form.} Weak form of dynamical equations is derived with scalar multiplication of the strong form by an arbitrary test function \(\vec{w}\),
\begin{equation}\label{equation:ch/lagrange-point:eq:lagrange:point:weak}
\begin{split}\vec{0} = \vec{w} \cdot \left( m \dot{\vec{v}} - \vec{R}^e\right)  \qquad \forall \vec{w}\end{split}
\end{equation}
\sphinxAtStartPar
\sphinxstylestrong{Lagrange equations.} Lagrange equations are derived from a proper choice of the test function. The position of the point \(P\) is written as a function of the generalized coordinates \(q^k(t)\) and time \(t\)
\begin{equation*}
\begin{split}\vec{r}_P(t) = \vec{r}(q^k(t),t) \ ,\end{split}
\end{equation*}
\sphinxAtStartPar
so that its velocity can be written as
\begin{equation*}
\begin{split}\vec{v}_P(t) := \frac{d\vec{r}_P}{dt} = \dot{q}^k(t) \underbrace{\frac{\partial \vec{r}}{\partial q^k}}_{\frac{\partial \vec{v}}{\partial \dot{q}^k}}(q^l(t), t) + \frac{\partial \vec{r}}{\partial t}(q^l(t), t) = \vec{v}\left(\dot{q}^k(t), q^k(t), t \right) \ ,\end{split}
\end{equation*}
\sphinxAtStartPar
from which the relation between partial derivatives
\begin{equation}\label{equation:ch/lagrange-point:classical-mechanics:lagrange:point:mixed-der}
\begin{split}\dfrac{\partial \vec{r}}{\partial q^k} = \dfrac{\partial \vec{v}}{\partial \dot{q}^k} \ .\end{split}
\end{equation}
\sphinxAtStartPar
follows. Choosing the test function \(\vec{w}\) as
\begin{equation*}
\begin{split}\vec{w} = \dfrac{\partial \vec{r}}{\partial q^k} = \dfrac{\partial \vec{v}}{\partial \dot{q}^k} \ ,\end{split}
\end{equation*}
\sphinxAtStartPar
applying the rule of derivative of product, using Schwartz theorem to switch order of derivation, and exploiting relation \eqref{equation:ch/lagrange-point:classical-mechanics:lagrange:point:mixed-der} it’s possible to recast weak form \eqref{equation:ch/lagrange-point:eq:lagrange:point:weak} as
\begin{equation*}
\begin{split}\begin{aligned}
\vec{0} & = \frac{\partial \vec{v}}{\partial \dot{q}^k} \cdot \left( m \dot{\vec{v}} - \vec{R}^e \right) = \\
& = \frac{d}{dt} \left( \frac{\partial \vec{v}}{\partial \dot{q}^k} \cdot m \vec{v} \right) - \frac{d}{dt} \frac{\partial \vec{r}}{\partial q^k} \cdot m \vec{v} - \frac{\partial \vec{r}}{\partial q^k} \cdot ( \vec{R}^{e,c} + \vec{R}^{e,nc} ) \\
& = \frac{d}{dt} \left( \frac{\partial \vec{v}}{\partial \dot{q}^k} \cdot m \vec{v} \right) - \frac{\partial \vec{v}}{\partial q^k} \cdot m \vec{v} - \frac{\partial \vec{r}}{\partial q^k} \cdot ( \nabla U + \vec{R}^{e,nc} ) \\
& = \frac{d}{dt} \left( \frac{\partial K}{\partial \dot{q}^k} \right) - \frac{\partial K}{\partial q^k} - \frac{\partial U}{\partial q^k} - \underbrace{\frac{\partial \vec{r}}{\partial q^k} \cdot \vec{R}^{e,nc}}_{=: Q^k} \ . \\
\end{aligned}\end{split}
\end{equation*}
\sphinxAtStartPar
Introducing the \sphinxstylestrong{Lagrangian function}
\begin{equation*}
\begin{split}\mathscr{L}(\dot{q}^k(t), q^k(t), t) := K(\dot{q}^k(t), q^k(t), t) + U(q^k(t),t) \ ,\end{split}
\end{equation*}
\sphinxAtStartPar
and recalling that potential function \(U\) is not a function of velocity and thus of time derivatives of the generalized coordinates \(\dot{q}^k\), it’s possible to recast the dynamical equation as the \sphinxstylestrong{Lagrange equations}
\begin{equation*}
\begin{split}\frac{d}{dt}\left(\frac{\partial \mathscr{L}}{\partial \dot{q}^k} \right) - \frac{\partial \mathscr{L}}{\partial q^k} = Q^k \ ,\end{split}
\end{equation*}
\sphinxAtStartPar
being \(Q^k\) the \sphinxstylestrong{generalized force} not included in the gradient of the potential \(\nabla U\) \sphinxhyphen{} usually a non conservative contribution \sphinxhyphen{}, \(Q^k = \dfrac{\partial \vec{r}}{\partial q^k} \cdot \vec{R}^{e,nc}\).

\sphinxstepscope




\section{System of points}
\label{\detokenize{ch/lagrange-points:system-of-points}}\label{\detokenize{ch/lagrange-points:classical-mechanics-lagrange-points}}\label{\detokenize{ch/lagrange-points::doc}}
\sphinxAtStartPar
\sphinxstylestrong{Newton dynamical equations \sphinxhyphen{} strong form.}

\sphinxAtStartPar
\sphinxstylestrong{Weak form.}

\sphinxAtStartPar
\sphinxstylestrong{Lagrange equations.}

\sphinxstepscope


\section{Rigid Body}
\label{\detokenize{ch/lagrange-rigid-body:rigid-body}}\label{\detokenize{ch/lagrange-rigid-body:classical-mechanics-lagrange-rigid}}\label{\detokenize{ch/lagrange-rigid-body::doc}}
\sphinxAtStartPar
\sphinxstylestrong{Newton dynamical equations \sphinxhyphen{} strong form.} Dynamical equations governing the motion of a rigid body, referred to its center of mass \(G\) read
\begin{equation*}
\begin{split}\begin{cases}
  \dot{\vec{Q}} = \vec{R}^{e} \\
  \dot{\vec{\Gamma}}_G = \vec{M}^e_G \ ,
\end{cases}\end{split}
\end{equation*}
\sphinxAtStartPar
with momentum \(\vec{Q} = m \vec{v}_G\) and angular momentum \(\vec{\Gamma}_G = \mathbb{I}_G \cdot \vec{\omega}\).

\sphinxAtStartPar
\sphinxstylestrong{Weak form.} Weak form of dynamical equations is derived with scalar multiplication of the strong form by an arbitrary test functions \(\vec{w}_t\), \(\vec{w}_r\)
\begin{equation}\label{equation:ch/lagrange-rigid-body:eq:lagrange:rigid:weak}
\begin{split}\vec{0} = \vec{w}_t \cdot \left( m \dot{\vec{v}}_G - \vec{R}^e\right) + \vec{w}_r \cdot \left( \dot{\vec{\Gamma}}_G - \vec{M}^e_G \right) \qquad \forall \vec{w}_t, \, \vec{w}_r\end{split}
\end{equation}
\sphinxAtStartPar
\sphinxstylestrong{Lagrange equations.} Lagrange equations are derived from the weak form, with a proper choice of the weak test functions. The «translational part» is recasted after choosing
\begin{equation*}
\begin{split}\vec{w}_t = \frac{\partial \vec{r}}{\partial q^k} = \frac{\partial \vec{v}}{\partial \dot{q}^k} \ .\end{split}
\end{equation*}
\sphinxAtStartPar
Following the same steps show to derive {\hyperref[\detokenize{ch/lagrange-point:classical-mechanics-lagrange-point}]{\sphinxcrossref{\DUrole{std,std-ref}{Lagrange equations for a point system}}}}, the translational part becomes
\begin{equation*}
\begin{split}\dfrac{d}{dt}\frac{\partial K^{tr}}{\partial \dot{q}^k} - \frac{\partial K^{tr}}{\partial q^k} - \frac{\partial U^{tr}}{\partial q^{k}} = Q^{tr}_{k} \ ,\end{split}
\end{equation*}
\sphinxAtStartPar
being \(K^{tr} = \frac{1}{2} m |\vec{v}_G|^2\) the contribution to kinetic energy of the velocity of the center of mass \(G\) deriving from the momentum equation, \(U^{tr}\) the contribution to the potential energy \(U\) from the momentum equation, and \(Q^{tr}_{k}\) the contribution to the generalized force from the momentum equation.

\sphinxAtStartPar
The «rotational part» is recasted after choosing
\begin{equation*}
\begin{split}\vec{w}_r = \frac{\partial \vec{\theta}}{\partial q^k} = \frac{\partial \vec{\omega}}{\partial \dot{q}^k} \end{split}
\end{equation*}
\sphinxAtStartPar
Angular velocity \(\vec{\omega}\) can be written w.r.t the inertial \(\{ \hat{e}_i \}\) or the material reference frame \(\{ \hat{E}_i \}\),
\begin{equation*}
\begin{split}\vec{\omega} = \omega_i \hat{e}_i = \sigma_j \hat{E}_j \ ,\end{split}
\end{equation*}
\sphinxAtStartPar
and the inertia tensor as
\begin{equation*}
\begin{split}\mathbb{I}_G = I_{ij} \, \hat{E}_i \otimes \hat{E}_j \ ,\end{split}
\end{equation*}
\sphinxAtStartPar
being the components \(I_{ij}\) constant.
\begin{equation*}
\begin{split}0 = \frac{\partial \vec{\omega}}{\partial \dot{q}^k} \cdot \dfrac{d}{d t} \left( \mathbb{I}_G \cdot \vec{\omega} \right) - \frac{\partial \vec{\omega}}{\partial \dot{q}^k } \cdot \vec{M}^e_G = \dfrac{d}{dt}\left( \frac{\partial \vec{\omega}}{\partial \dot{q}^k} \cdot \mathbb{I}_G \cdot \vec{\omega} \right) - \dfrac{d}{dt} \frac{\partial \vec{\omega}}{\partial \dot{q}^k} \cdot \mathbb{I}_G \cdot \vec{\omega} - \frac{\partial \vec{\theta}}{\partial q^k} \cdot \vec{M}^e_G\end{split}
\end{equation*}
\sphinxAtStartPar
The first term becomes
\begin{equation*}
\begin{split}\dfrac{d}{dt}\left( \frac{\partial \vec{\omega}}{\partial \dot{q}^k} \cdot \mathbb{I}_G \cdot \vec{\omega} \right) = \dfrac{d}{dt} \left( \frac{\partial \sigma_a}{\partial \dot{q}^k} I_{ab} \sigma_b \right) = \dfrac{d}{dt} \dfrac{\partial}{\partial \dot{q}^k}\left( \dfrac{1}{2} \vec{\omega} \cdot \mathbb{I}_G \cdot \vec{\omega} \right) = \dfrac{d}{dt}\dfrac{\partial K^{rot}}{\partial \dot{q}^k}\end{split}
\end{equation*}
\sphinxAtStartPar
The second term becomes
\begin{equation*}
\begin{split}\begin{aligned}
 \dfrac{d}{dt} \frac{\partial \vec{\theta}}{\partial q^k} \cdot \mathbb{I}_G \cdot \vec{\omega} 
  & = \dfrac{\partial }{\partial q^k} \underbrace{\dfrac{d \vec{\theta}}{d t}}_{\vec{\omega}} \cdot \mathbb{I}_G \cdot \vec{\omega} = \\
  & = \dfrac{\partial \vec{\omega}}{\partial q^k} \cdot \mathbb{I}_G \cdot \vec{\omega} = \\
  & = \dfrac{\partial}{\partial q^k} \left( \sigma_a \hat{E}_a \right) \cdot \hat{E}_b \, I_{bc} \sigma_c = \\
  & = \dfrac{\partial \sigma_a}{\partial q^k} \underbrace{\hat{E}_a \cdot \hat{E}_b}_{= \delta_{ab}} \, I_{bc} \sigma_c 
    + \sigma_a \underbrace{\dfrac{\partial \hat{E}_a}{\partial q^k}  \cdot \hat{E}_b}_{= 0} \, I_{bc} \sigma_c = \\
  & = \dfrac{\partial}{\partial q^k} \left( \frac{1}{2} \sigma_a \, I_{ab} \sigma_b \right) = \\
  & = \dfrac{\partial }{\partial q^k} \left( \frac{1}{2} \vec{\omega} \cdot \mathbb{I}_G \cdot \vec{\omega} \right) 
    = \dfrac{\partial K^{rot}}{\partial q^k} \ .
\end{aligned}\end{split}
\end{equation*}
\sphinxAtStartPar
The third term can be written as the sum of the derivative of a potential function and a generalized force,
\begin{equation*}
\begin{split}\frac{\partial \vec{\theta}}{\partial q^k} \cdot \vec{M}_G^e = \frac{\partial U^{rot}}{\partial q^k} + Q^{rot}_{q^k}\end{split}
\end{equation*}
\sphinxAtStartPar
The rotational part of the wak form becomes
\begin{equation*}
\begin{split}\dfrac{d}{dt}\frac{\partial K^{rot}}{\partial \dot{q}^k} - \frac{\partial K^{rot}}{\partial q^k} - \frac{\partial U^{rot}}{\partial q^{k}} = Q^{rot}_{q^k} \ ,\end{split}
\end{equation*}
\sphinxAtStartPar
being \(K^{rot} = \frac{1}{2} \vec{\omega} \cdot \mathbb{I}_G \cdot \vec{\omega}\) the contribution to kinetic energy of the rotation aroung the center of mass \(G\) deriving from the angular momentum equation, \(U^{rot}\) the contribution to the potential energy \(U\) from the angular momentum equation, and \(Q^{rot}_{k}\) the contribution to the generalized force from the angular momentum equation.

\sphinxAtStartPar
Adding together the contributions of the momentum and the angular momentum equations, the Lagrange equation can be formally written with the same expression found for the system of points,
\begin{equation*}
\begin{split}\dfrac{d}{dt}\left(\frac{\partial \mathscr{L}}{\partial \dot{q}^k}\right) - \frac{\partial \mathscr{L}}{\partial q^k} = Q_{q^k} \ ,\end{split}
\end{equation*}
\sphinxAtStartPar
being \(\mathscr{L} = K + U\) the Lagrangian function of the system, and \(K = K^{tr} + K^{rot}\), \(U = U^{tr} + U^{rot}\), \(Q_{q^k} = Q_{q^k}^{tr} + Q_{q^k}^{rot}\) the kinetic energy the potential function and the generalized force of the system, defined as the sum of the contributions coming from the momentum and the angular momentum equations.



\sphinxstepscope




\chapter{Hamiltonian Mechanics}
\label{\detokenize{ch/hamilton:hamiltonian-mechanics}}\label{\detokenize{ch/hamilton:classical-mechanics-hamilton}}\label{\detokenize{ch/hamilton::doc}}
\sphinxAtStartPar
Riformulazione ulteriore della meccanica di Newton, a partire dalla meccanica di Lagrange.
Fornisce le basi per un approccio moderno anche in altre teorie della Fisica. \sphinxstylestrong{dots…}

\sphinxAtStartPar
Starting from Lagrange equations derived in {\hyperref[\detokenize{ch/lagrange:classical-mechanics-lagrange}]{\sphinxcrossref{\DUrole{std,std-ref}{Lagrangian mechanics}}}},
\begin{equation*}
\begin{split}\dfrac{d}{dt}\Big( \frac{\partial \mathscr{L}}{\partial \dot{q}} \Big) - \frac{\partial \mathscr{L}}{\partial q} = Q_q\end{split}
\end{equation*}
\sphinxAtStartPar
the \sphinxstylestrong{generalized moment} is defined as
\begin{equation*}
\begin{split}p_k := \frac{\partial \mathscr{L}}{\partial \dot{q}^k} \ ,\end{split}
\end{equation*}
\sphinxAtStartPar
and the \sphinxstylestrong{Hamiltonian function} as
\begin{equation*}
\begin{split}\mathscr{H}(q^k(t), p_k(t), t) := p_k \dot{q}^k - \mathscr{L}(\dot{q}^l(q^k, p_k, t), q^l(t), t) \ ,\end{split}
\end{equation*}
\sphinxAtStartPar
its differential reads
\begin{equation*}
\begin{split}\begin{aligned}
d\mathscr{H} & = dq^k \, \frac{\partial \mathscr{H}}{\partial q^k} + dp_k \, \frac{\partial \mathscr{H}}{\partial p_k} + dt \,  \frac{\partial \mathscr{H}}{\partial t} = \\
& = d p_k \, \dot{q}^k + \underbrace{ p_k \, d \dot{q}^k - d \dot{q}^k \, \frac{\partial \mathscr{L}}{\partial \dot{q}^k}}_{=0} - d q^k \, \frac{\partial \mathscr{L}}{\partial q^k} - dt \, \frac{\partial \mathscr{L}}{\partial t}
\end{aligned}\end{split}
\end{equation*}
\sphinxAtStartPar
and thus it follows
\begin{equation*}
\begin{split}\begin{cases}
 \dot{q}^k & = \dfrac{\partial \mathscr{H}}{\partial p_k} \\
 \dfrac{\partial \mathscr{H}}{\partial q^k} & = - \dfrac{\partial \mathscr{L}}{\partial q^k} \\
 \dfrac{\partial \mathscr{H}}{\partial t} & = - \dfrac{\partial\mathscr{L}}{\partial t} \ .
\end{cases}\end{split}
\end{equation*}
\sphinxAtStartPar
Recasting Lagrange equations as
\begin{equation*}
\begin{split}\frac{\partial \mathscr{L}}{\partial q^k} = - Q_{q^k} + \dfrac{d}{dt}\Big( \frac{\partial \mathscr{L}}{\partial \dot{q}^k} \Big) = -Q_{q^k} + \dot{p}_k\end{split}
\end{equation*}
\sphinxAtStartPar
\sphinxstylestrong{Hamilton equations} follow
\begin{equation*}
\begin{split}\begin{cases}
 \dot{q}^k & = \dfrac{\partial H}{\partial p_k} \\
 \dot{p}_k & =-\dfrac{\partial H}{\partial q^k} + Q_{q^k} \ .
\end{cases}\end{split}
\end{equation*}






\renewcommand{\indexname}{Indice}
\printindex
\end{document}